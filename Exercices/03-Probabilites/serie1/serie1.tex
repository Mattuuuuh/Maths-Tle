\documentclass[14pt]{beamer}
\usepackage[french]{babel}

\usetheme{CambridgeUS}
\usecolortheme{rose}
\beamertemplatenavigationsymbolsempty


\usepackage{libertinus}
\usepackage{amsmath,amsfonts,amsthm,amssymb,mathtools}
\usepackage{array}
\newcolumntype{P}[1]{>{\centering\arraybackslash}p{#1}}


\usepackage{stackengine}
\newcommand\xrowht[2][0]{\addstackgap[.5\dimexpr#2\relax]{\vphantom{#1}}}


% corps
\usepackage{calrsfs}
\newcommand{\C}{\mathcal{C}}
\newcommand{\R}{\mathbb{R}}
\newcommand{\Rnn}{\mathbb{R}^{2n}}
\newcommand{\Z}{\mathbb{Z}}
\newcommand{\N}{\mathbb{N}}
\newcommand{\Q}{\mathbb{Q}}

% domain
\newcommand{\D}{\mathcal{D}}


% date
\usepackage{advdate}
\AdvanceDate[0]

%plots
\usepackage{pgfplots, subcaption}
\definecolor{myg}{RGB}{56, 140, 70}
\definecolor{myb}{RGB}{45, 111, 177}
\definecolor{myr}{RGB}{199, 68, 64}

%boxes
\usepackage[most]{tcolorbox}
\usepackage{multicol}

%icomma
\usepackage{icomma}

%https://osl.ugr.es/CTAN/macros/latex/contrib/tcolorbox/tcolorbox.pdf
\newtcolorbox{mybox}[3][]
{
  colframe = #2!25,
  colback  = #2!10,
  coltitle = #2!20!black,  
  halign title=flush center, 
  title    = {#3},
  #1,
}

% BOX A BOX B
\newcommand{\boxAB}[2]{
		\begin{mybox}{red}{A}
		\begin{center}
			#1
		\end{center}
		\end{mybox}
		\begin{mybox}{green}{B}
		\begin{center}
			#2
		\end{center}
		\end{mybox}
}

%systèmes
\usepackage{systeme}

% trafficotage 
\usepackage[answerdelayed, lastexercise]{exercise}
\renewcommand{\ExerciseHeader}{
}
\renewcommand{\AnswerHeader}{
}

\newcommand{\framedelayed}[3][]{
	\begin{Exercise}
	\begin{frame}{\theExercise #1\vspace{-32pt}}
		#2
	\end{frame}
	\end{Exercise}
	\begin{Answer}
	\begin{frame}{\theExercise #1\vspace{-32pt}}
		#3
	\end{frame}
	\end{Answer}
}


\SetDate[01/10/2025]

\begin{document}
\pagestyle{fancy}
\fancyhead[L]{Tle STMG}
\fancyhead[C]{\textbf{Probabilités conditionnelles}}
\fancyhead[R]{\today}

\section*{Tableaux croisés}

\exe{}{
	Un vendeur de voitures possède un stock de 1 000 voitures dont les caractéristiques sont résumées dans le tableau ci-dessous.
	
	\begin{center}
	\begin{tabular}{|c|c|c|c|c|}
	\cline{2-5}
	\multicolumn{1}{c|}{}	&	Blanche	&	Noire	&	Rouge	&	TOTAL \\ \hline
	Française	&	150	&	$x$	&	400	&	750 \\ \hline
	Étrangère	&	100	&	50	&	100	&	250 \\ \hline
	TOTAL	&	250	&	250	&	500	&	1000 \\\hline
	\end{tabular}
	\end{center}
	
	\begin{enumerate}
		\item Indiquer ce que représente $x$ et déterminer sa valeur.
		\item Quel est le pourcentage de voitures noires parmi les voitures du stock ?
		\item Quel est le pourcentage de voitures noires étrangères parmi les voitures du stock ?
		\item Quel est le pourcentage de voitures blanches parmi les voitures françaises ?
		\item Quel est le pourcentage de voitures françaises parmi les voitures blanches ?
		\item Alice et Benoît jouent au jeu suivant.
			\begin{enumerate}[label=--]
				\item Alice choisit au harsard une voiture parmi les voitures françaises.
				Elle remporte 1 euro si ce n'est pas une voiture rouge.
				\item Benoît choisit au hasard une voiture parmi les voitures blanches.
				Il remporte 1 euro si c'est une voiture étrangère.
			\end{enumerate}
		Lequel des deux a le plus de chance de remporter 1 euro ?
	\end{enumerate}
}{exe:tableau0-1}{
	TODO
}


\exe{}{
	Une urne contient $49$ billes numérotées de $1$ à $49$.
	La moitié des billes paires sont bleues, les $\frac25$ des billes impaires sont jaunes.
	Compléter le tableau croisé d'effectifs.
	
	\begin{center}
	%\def\arraystretch{1.8}
	\setlength\tabcolsep{20pt}
	\tableaucroise{Paire & Impaire & Total}{Bleue & & &}{Jaune & & &}{Total &&&}
	\end{center}
	
	On choisit une bille uniformément au hasard et on dénote
		\begin{center}
			$I$ : \og La bille a un numéro impair. \fg
			\hspace{3cm}
			$B$ : \og La bille est bleue. \fg
		\end{center}
	Donner $P(I)$ et $P(I \sct B)$ à l'aide du tableau. Les événements $I$ et $B$ sont-ils indépendants ? corrélés positivement ? négativement ?
}{exe:proba6}{
	TODO
}

\newpage

\section*{Arbres de probabilité}

\exe{}{
	Compléter l'arbre sachant que 
		\begin{multicols}{3}
		\begin{enumerate}[label=$\bullet$]
			\item $P(A) = 0,3$
			\item $P(B \sct A) = 0,6$
			\item $P(B \sct \overline{A}) = 0,25$
		\end{enumerate}
		\end{multicols}
	\begin{center}
	\begin{tikzpicture}
		% depth 1
		\foreach \i in {-3, 3}
		\draw[-, thick, black] (0,0) node {$\bullet$} -- (\i,-1.5);
		% depth 2
		\foreach \i in {-3, 3} \foreach \j in {-1, 1}
			\draw[-, thick, black] (\i,-1.5) node {$\bullet$} -- (\i+\j,-3) node {$\bullet$};
			
		\draw (-3,-1.5) node[above left] {$A$};
		\draw (3,-1.5) node[above right] {$\overline{A}$};
			
		\draw (-4,-3) node[below] {$B\cap A$};
		\draw (2,-3) node[below] {$B\cap\overline{A}$};
		\draw (-2,-3) node[below] {$\overline{B}\cap A$};
		\draw (4,-3) node[below] {$\overline{B}\cap\overline{A}$};
	\end{tikzpicture}
	\end{center}
	Calculer $P(B\cap A)$ et $P(\overline{B}\cap\overline{A})$ en multipliant les probabilités des chemins racine-feuille correspondant.
}{exe:proba1}{
	TODO
}


\exe{}{
	Compléter l'arbre correspondant à une expérience aléatoire à deux épreuves d'issues $\{A ; B ; C ; D\}$ et répondre aux questions suivantes.
	\begin{center}
	\begin{tikzpicture}[scale=.9]
		% depth 1
		\draw[-, thick, black] (0,0) node {$\bullet$} -- (3,-2) node[midway, above right] {};
		\draw[-, thick, black] (0,0) node {$\bullet$} -- (-3,-2) node[midway, above left] {$0,7$};
		% depth 2
		\draw[-, thick, black] (-3,-2) node {$\bullet$} -- (-1,-4) node[midway, above right] {$\frac49$};
		\draw[-, thick, black] (-3,-2) node {$\bullet$} -- (-3,-4);
		\draw[-, thick, black] (-3,-2) node {$\bullet$} -- (-5,-4) node[midway, above left] {$\frac13$};
		
		\draw[-, thick, black] (3,-2) node {$\bullet$} -- (1,-4) node[midway, above left] {$\frac16$};
		\draw[-, thick, black] (3,-2) node {$\bullet$} -- (3,-4) node[pos=.6, left] {$\frac12$};
		\draw[-, thick, black] (3,-2) node {$\bullet$} -- (5,-4);
		
		\draw (1,-4) node {$\bullet$};
		\draw (3,-4) node {$\bullet$};
		\draw (5,-4) node {$\bullet$};
		\draw (1,-4) node[below] {$D$};
		\draw (3,-4) node[below] {$B$};
		\draw (5,-4) node[below] {$C$};
		
		\draw (-1,-4) node {$\bullet$};
		\draw (-3,-4) node {$\bullet$};
		\draw (-5,-4) node {$\bullet$};
		\draw (-1,-4) node[below] {$C$};
		\draw (-3,-4) node[below] {$B$};
		\draw (-5,-4) node[below] {$A$};
	\end{tikzpicture}
	\end{center}
	
	\begin{multicols}{2}
	\begin{enumerate}
		\item Calculer $P(D)$.
		\item Calculer $P(B)$.
		\item Calculer $P(D \cup B)$.
		\item Calculer $P(A\cup C)$.
	\end{enumerate}
	\end{multicols}
}{exe:proba3}{
	La somme des probabilités de chaque sous-branche est toujours $1$. 
	On complète donc l'arbre comme ci-dessous.
	
	\begin{center}
	\begin{tikzpicture}
		% depth 1
		\draw[-, thick, black] (0,0) node {$\bullet$} -- (3,-2) node[midway, above right] {$0,3$};
		\draw[-, thick, black] (0,0) node {$\bullet$} -- (-3,-2) node[midway, above left] {$0,7$};
		% depth 2
		\draw[-, thick, black] (-3,-2) node {$\bullet$} -- (-1,-4) node[midway, above right] {$\frac49$};
		\draw[-, thick, black] (-3,-2) node {$\bullet$} -- (-3,-4) node[pos=.6, left] {$\frac29$};
		\draw[-, thick, black] (-3,-2) node {$\bullet$} -- (-5,-4) node[midway, above left] {$\frac13$};
		
		\draw[-, thick, black] (3,-2) node {$\bullet$} -- (1,-4) node[midway, above left] {$\frac16$};
		\draw[-, thick, black] (3,-2) node {$\bullet$} -- (3,-4) node[pos=.6, left] {$\frac12$};
		\draw[-, thick, black] (3,-2) node {$\bullet$} -- (5,-4) node[midway, above right] {$\frac13$};
		
		\draw (1,-4) node {$\bullet$};
		\draw (3,-4) node {$\bullet$};
		\draw (5,-4) node {$\bullet$};
		\draw (1,-4) node[below] {$D$};
		\draw (3,-4) node[below] {$B$};
		\draw (5,-4) node[below] {$C$};
		
		\draw (-1,-4) node {$\bullet$};
		\draw (-3,-4) node {$\bullet$};
		\draw (-5,-4) node {$\bullet$};
		\draw (-1,-4) node[below] {$C$};
		\draw (-3,-4) node[below] {$B$};
		\draw (-5,-4) node[below] {$A$};
	\end{tikzpicture}
	\end{center}
	
	\begin{enumerate}
		\item 
			\begin{align*}
				P(D) &= 0,3 \times \dfrac16 \\ &= \dfrac{0,3}{6} \\ &= \dfrac{0,1}{2} = 0,05.
			\end{align*}
		\item 
			\begin{align*}
				P(B) &= 0,7 \times \dfrac29 + 0,3 \times \dfrac12 \approx 0,31.
			\end{align*}
		\item 
			Comme $D$ et $B$ sont deux issues distinctes de l'univers, on a
			\begin{align*}
				P(D \cup B) &= P(D) + P(B) \\ &\approx 0,05 + 0,31 = 0,36.
			\end{align*}
		\item On peut soit procéder comme ci-dessus, ou alors utiliser le fait que
			\[ P(A\cup C) = P(A) + P(C) = 1 - \left( P(D) + P(B) \right) = 1 - P(D \cup B), \]
		et donc
			\[ P(A\cup C) \approx 0,64. \]
	\end{enumerate}
}

\exe{}{
	On tire une boule dans une urne contenant $2$ boules rouges et $4$ boules vertes.
	\begin{enumerate}[label=---]
		\item Si la boule tirée est verte, on la met de côté et on retire une nouvelle boule
		\item Si la boule tirée est rouge, on la remet dans l'urne et on retire une nouvelle boule
	\end{enumerate}
	On considère les quatre événements suivants :
		\begin{multicols}{2}
		\begin{enumerate}[label=]
			\item v : \og la première boule tirée est verte \fg
			\item r : \og la première boule tirée est rouge \fg
			\item V : \og la deuxième boule tirée est verte \fg
			\item R : \og la deuxième boule tirée est rouge \fg
		\end{enumerate}
		\end{multicols}

	\begin{multicols}{2}
	\begin{enumerate}
		\item Donner $P(\text{v})$ et $P(\text{r})$.
		\item Donner $P(\text{V sachant v})$ et $P(\text{V sachant r})$.
		\item Calculer $P(\text{V $\cap$ v})$ et $P(\text{V $\cap$ r})$.
		\item Calculer $P(\text{V})$ puis $P(\text{R})$.
	\end{enumerate}
	\end{multicols}

	\begin{center}
	\begin{tikzpicture}[scale=1.2]
		% depth 1
		\foreach \i in {-3, 3}
		\draw[-, thick, black] (0,0) node {$\bullet$} -- (\i,-1.5);
		% depth 2
		\foreach \i in {-3, 3} \foreach \j in {-1, 1}
			\draw[-, thick, black] (\i,-1.5) node {$\bullet$} -- (\i+\j,-3) node {$\bullet$};
			
		\draw (-3,-1.5) node[above left] {v};
		\draw (3,-1.5) node[above right] {r};
			
		\draw (-4,-3) node[below] {V$\cap$v};
		\draw (2,-3) node[below] {V$\cap$r};
		\draw (-2,-3) node[below] {R$\cap$v};
		\draw (4,-3) node[below] {R$\cap$r};
	\end{tikzpicture}
	\end{center}
}{exe:proba2}{
	\, \\
	\begin{center}
	\begin{tikzpicture}
		% depth 1
		\foreach \i in {-3, 3}
		\draw[-, thick, black] (0,0) node {$\bullet$} -- (\i,-2);
		% depth 2
		\foreach \i in {-3, 3} \foreach \j in {-1, 1}
			\draw[-, thick, black] (\i,-2) node {$\bullet$} -- (\i+\j,-4) node {$\bullet$};
			
		\draw (-3,-2) node[above left] {v};
		\draw (3,-2) node[above right] {r};
			
		\draw (-4,-4) node[below] {V};
		\draw (2,-4) node[below] {V};
		\draw (-2,-4) node[below] {R};
		\draw (4,-4) node[below] {R};
		
		% sols
		\draw (1.5,0) node {$\frac13$};
		\draw (-1.5,0) node {$\frac23$};
		
		\draw (4,-2) node {$\frac13$};
		\draw (2,-2) node {$\frac23$};
		
		\draw (-2,-2) node {$\frac25$};
		\draw (-4,-2) node {$\frac35$};
	\end{tikzpicture}
	\end{center}
	La probabilité d'une issue est la somme des probabilités de chacun des chemins racine-feuille.
	Pour obtenir la probabilité d'un chemin racine-feuille, on multiplie les probabilités rencontrées en le parcourant.
	
		\begin{align*}
			P(V) &= \dfrac23 \times \dfrac35 + \dfrac13 \times \dfrac23 \\
				&= \dfrac25 + \dfrac29 \\
				&= \dfrac{18 + 10}{5 \times 9} = \dfrac{28}{45}
		\end{align*}
	
		\begin{align*}
			P(R) &= \dfrac23 \times \dfrac25 + \dfrac13 \times \dfrac13 \\
				&= \dfrac4{15} + \dfrac19 \\
				&= \dfrac{36 + 15}{15 \times 9} = \dfrac{51}{135} = \dfrac{17}{45}
		\end{align*}
		
		On aurait aussi pû utiliser le fait que $P(R) = 1 - P(V)$ pour ne pas augmenter la probabilité de faire une erreur de calcul.
}

%%%%%%%%%%%%

\newpage
\fancyhead[C]{\textbf{Solutions}}
\shipoutAnswer

\end{document}
