\documentclass[14pt]{beamer}
\usepackage[french]{babel}

\usetheme{CambridgeUS}
\usecolortheme{rose}
\beamertemplatenavigationsymbolsempty


\usepackage{libertinus}
\usepackage{amsmath,amsfonts,amsthm,amssymb,mathtools}
\usepackage{array}
\newcolumntype{P}[1]{>{\centering\arraybackslash}p{#1}}


\usepackage{stackengine}
\newcommand\xrowht[2][0]{\addstackgap[.5\dimexpr#2\relax]{\vphantom{#1}}}


% corps
\usepackage{calrsfs}
\newcommand{\C}{\mathcal{C}}
\newcommand{\R}{\mathbb{R}}
\newcommand{\Rnn}{\mathbb{R}^{2n}}
\newcommand{\Z}{\mathbb{Z}}
\newcommand{\N}{\mathbb{N}}
\newcommand{\Q}{\mathbb{Q}}

% domain
\newcommand{\D}{\mathcal{D}}


% date
\usepackage{advdate}
\AdvanceDate[0]

%plots
\usepackage{pgfplots, subcaption}
\definecolor{myg}{RGB}{56, 140, 70}
\definecolor{myb}{RGB}{45, 111, 177}
\definecolor{myr}{RGB}{199, 68, 64}

%boxes
\usepackage[most]{tcolorbox}
\usepackage{multicol}

%icomma
\usepackage{icomma}

%https://osl.ugr.es/CTAN/macros/latex/contrib/tcolorbox/tcolorbox.pdf
\newtcolorbox{mybox}[3][]
{
  colframe = #2!25,
  colback  = #2!10,
  coltitle = #2!20!black,  
  halign title=flush center, 
  title    = {#3},
  #1,
}

% BOX A BOX B
\newcommand{\boxAB}[2]{
		\begin{mybox}{red}{A}
		\begin{center}
			#1
		\end{center}
		\end{mybox}
		\begin{mybox}{green}{B}
		\begin{center}
			#2
		\end{center}
		\end{mybox}
}

%systèmes
\usepackage{systeme}

% trafficotage 
\usepackage[answerdelayed, lastexercise]{exercise}
\renewcommand{\ExerciseHeader}{
}
\renewcommand{\AnswerHeader}{
}

\newcommand{\framedelayed}[3][]{
	\begin{Exercise}
	\begin{frame}{\theExercise #1\vspace{-32pt}}
		#2
	\end{frame}
	\end{Exercise}
	\begin{Answer}
	\begin{frame}{\theExercise #1\vspace{-32pt}}
		#3
	\end{frame}
	\end{Answer}
}


\SetDate[21/01/2026]

\reversemarginpar
\setlength{\marginparsep}{.5cm}

\begin{document}
\pagestyle{fancy}
\fancyhead[L]{Tle STMG}
\fancyhead[C]{\textbf{Évaluation blanche \\ Fonction logarithme}}
\fancyhead[R]{\today}

%\null\vspace{-30pt}
Consignes particulières : 
\begin{itemize}[label=$\bullet$]
	\item 
	La calculatrice est {autorisée}.
	\item
	L'évaluation fait 1 page. La somme des points est \total{points}.
\end{itemize}

\marginpar{[pts]}
\hrule


\exe{4}{
	Exprimer les nombres suivants sont la forme $\log(n)$ où $n\in\N$ est un nombre entier non nul à l'aide des propriétés vues en classe.
	\begin{multicols}{2}
	\begin{enumerate}[label=\alph*)]
		\item $\log(5) + \log(6)$
		\item $2\log(7)$
		\item $\log(12)-\log(3)$
		\item $\log(10) - \log(\frac14)$
		\item $3\log(4) - 2\log(2)$
		\item $\log(2) - \log(9) + 2\log(3)$
	\end{enumerate}
	\end{multicols}
}{exe:2}{
	todo
}

\exe{4}{
	Résoudre les équations suivantes. Arrondir à $10^{-2}$.
	\begin{multicols}{2}
	\begin{enumerate}[label=\alph*)]
		\item $8^x = 123$
		\item $x^{63} = 4~812$
		\item $5^x \times 7^x = 28$
		\item $12^x = 854$
		\item $300 \times 1,02^x = 540$
		\item $300 \times 1,02^x = 130$
	\end{enumerate}
	\end{multicols}
}{exe:3}{
	todo
}



\exe{4}{
	On cherche à estimer le nombre de chiffres nécessaires pour écrire $N = 2^{320}$.
	\begin{enumerate}
		\item
		Donner un encadrement de $\log(N)$ à l'unité à partir de l'encadrement
			\[ 0,301 < \log(2) < 0,302 \]
		en justifiant.
		\item
		En déduire la valeur de $k\in\N$ entier naturel vérifiant $10^k < N < 10^{k+1}$.
		\item
		Conclure sur le nombre de chiffres nécessaires pour écrire $N$.
	\end{enumerate}
}{exe:4}{
	todo
}

\exe{4}{
	On estime qu'un placement de 2 000€ accumule 6\% d'intérêts par an : chaque année la valeur du placement augmente de 6\%.
	\begin{enumerate}
		\item
		Donner le coefficient multiplicateur associé à une augmentation de 6\%.
		\item
		Quelle sera la valeur du placement après 5 ans ?
		\item
		À partir de combien d'années le placement aura-t-il une valeur de 10 000€ ?
		Poser une équation et la résoudre à l'aide du logarithme.
	\end{enumerate}
}{exe:5}{
	todo
}

\exe{4}{
	En 2019, les recettes fiscales recouvrées par les Finances publiques s'élèvent à 464 milliards d'euros.
	Après quelques années, elles atteignent 544,4 milliards d'euros.
	
	Le Ministère de l'Économie annonce une augmentation annuelle moyenne de 5,4\%.
	\begin{enumerate}
		\item
		Calculer le coefficient multiplicateur global.
		\item
		Donner le coefficient multiplicateur moyen correspondant à une augmentation de 5,4\%.
		\item
		Sur combien d'année la moyenne a-t-elle été calculée ?
		Poser $n$ le nombre d'années, et trouver $n$ à l'aide de la formule du coefficient multiplicateur moyen vue en classe : $CM_{\text{moyen}} = CM_{\text{global}}^{1/n}$.
	\end{enumerate}
}{exe:evol-moy}{
	todo
}


%%%%%%%%%%%%

\newpage
\fancyhead[C]{\textbf{Solutions}}
\shipoutAnswer

\end{document}
