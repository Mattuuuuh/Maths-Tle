\documentclass[14pt]{beamer}
\usepackage[french]{babel}

\usetheme{CambridgeUS}
\usecolortheme{rose}
\beamertemplatenavigationsymbolsempty


\usepackage{libertinus}
\usepackage{amsmath,amsfonts,amsthm,amssymb,mathtools}
\usepackage{array}
\newcolumntype{P}[1]{>{\centering\arraybackslash}p{#1}}


\usepackage{stackengine}
\newcommand\xrowht[2][0]{\addstackgap[.5\dimexpr#2\relax]{\vphantom{#1}}}


% corps
\usepackage{calrsfs}
\newcommand{\C}{\mathcal{C}}
\newcommand{\R}{\mathbb{R}}
\newcommand{\Rnn}{\mathbb{R}^{2n}}
\newcommand{\Z}{\mathbb{Z}}
\newcommand{\N}{\mathbb{N}}
\newcommand{\Q}{\mathbb{Q}}

% domain
\newcommand{\D}{\mathcal{D}}


% date
\usepackage{advdate}
\AdvanceDate[0]

%plots
\usepackage{pgfplots, subcaption}
\definecolor{myg}{RGB}{56, 140, 70}
\definecolor{myb}{RGB}{45, 111, 177}
\definecolor{myr}{RGB}{199, 68, 64}

%boxes
\usepackage[most]{tcolorbox}
\usepackage{multicol}

%icomma
\usepackage{icomma}

%https://osl.ugr.es/CTAN/macros/latex/contrib/tcolorbox/tcolorbox.pdf
\newtcolorbox{mybox}[3][]
{
  colframe = #2!25,
  colback  = #2!10,
  coltitle = #2!20!black,  
  halign title=flush center, 
  title    = {#3},
  #1,
}

% BOX A BOX B
\newcommand{\boxAB}[2]{
		\begin{mybox}{red}{A}
		\begin{center}
			#1
		\end{center}
		\end{mybox}
		\begin{mybox}{green}{B}
		\begin{center}
			#2
		\end{center}
		\end{mybox}
}

%systèmes
\usepackage{systeme}

% trafficotage 
\usepackage[answerdelayed, lastexercise]{exercise}
\renewcommand{\ExerciseHeader}{
}
\renewcommand{\AnswerHeader}{
}

\newcommand{\framedelayed}[3][]{
	\begin{Exercise}
	\begin{frame}{\theExercise #1\vspace{-32pt}}
		#2
	\end{frame}
	\end{Exercise}
	\begin{Answer}
	\begin{frame}{\theExercise #1\vspace{-32pt}}
		#3
	\end{frame}
	\end{Answer}
}


\SetDate[27/11/2025]

\begin{document}
\pagestyle{fancy}
\fancyhead[L]{Tle STMG}
\fancyhead[C]{\textbf{Fonctions exponentielles 2}}
\fancyhead[R]{\today}


\exe{}{
	Sans calculatrice, exprimer les nombres suivants sous la forme $q^n$, où $q \in \N$ et $n\in\Z$ sont des entiers.
	\begin{multicols}{3}
	\begin{enumerate}[label=\roman*)]
		\item $10^3 \times 10^5$
		\item $\left(4^5\right)^2$
		\item $\dfrac{5^3}{5^3}$
		\item $1$
		\item $\dfrac{2^4}{2^7}$
		\item $\left(2^{-1}\right)^3$
		\item $\left(2^{3}\right)^{-1}$
		\item $\left(\dfrac{1}{7^{2,5}}\right)^6$
		\item $\dfrac{10^{12}}{10^{-12}}$
		\item $\dfrac{10^{-5,2}}{10^{6,8}}$
	\end{enumerate}
	\end{multicols}
}{exe:puissances}{
	\begin{multicols}{3}
	\begin{enumerate}[label=\roman*)]
		\item $10^8$
		\item $4^{10}$
		\item $5^0 = 1^1 = 4^0 = q^0$ pour n'importe quel $q\neq0$
		\item $1^1 = 2^0 = 100^0 = q^0$ pour n'importe quel $q\neq0$
		\item $2^{-3}$
		\item $2^{-3}$
		\item $2^{-3}$
		\item $7^{-15}$
		\item $10^{24}$
		\item $10^{-12}$
	\end{enumerate}
	\end{multicols}
}

\exe{}{
	Montrer qu'on a environ $2^{10} \approx 10^3$. 	
	\begin{enumerate}
		\item En déduire approximativement l'ordre de grandeur de $2^{20}$ et le nombre de chiffres nécessaires pour l'écrire.
		\item En déduire approximativement l'ordre de grandeur de $2^{35}$ et le nombre de chiffres nécessaires pour l'écrire.
	\end{enumerate}
}{exe:grandeur-binaire}{
	D'abord, $2^{10} = 1024 \approx 10^3$, comme requis.
	\begin{enumerate}
		\item
		Il suit que $2^{20} = \left( 2^{10} \right)^2 \approx \left(10^3\right)^2 = 10^6$.
		Il faut donc 7 chiffres pour écrire $2^{20}$ en base 10 (et il en faut 21 pour l'écrire en base 2).
		\item 
		Similairement, $2^{35} = 2^5 \times \left( 2^{10} \right)^3 \approx 32 \times \left(10^3\right)^3 = 32 \times 10^9 = 3,2 \times 10^{10}$.
		Il faut donc 11 chiffres pour écrire $2^{35}$ en base 10.
	\end{enumerate}
}

% déjà fait set 1
%\exe{, difficulty=1}{
%	Soit $S$ une suite géométrique de terme initial $S_0 = 4$ et de raison $q$ encore inconnue.
%	
%	%\begin{multicols}{2}
%	\begin{enumerate}
%		\item Calculer $q$ si $S_5 = 390,625$.
%		\item Calculer $q$ si $S_{10} = 4096$.
%		\item Calculer $q$ si $S_7 = 700$. Arrondir à $10^{-2}$ près.
%		\item Calculer $q$ si $S_{13} = 13~521$. Arrondir à $10^{-2}$ près.
%	\end{enumerate}
%	%end{multicols}
%
%}{exe:2a}{
%	todo
%}



\exe{, difficulty=2}{
	$g$ est une fonction exponentielle de base $q$ encore inconnue et telle que $g(0) = 350$.
	
	\begin{enumerate}
		%\item Montrer que $g(x)/g(0) = q^x$ pour tout $x\in\R$.
		\item Si $g(3,4) = 45~875~200$, que vaut $q$ ?
		\item Si $g\left(-\dfrac43\right) = 21,875$, que vaut $q$ ?
		\item Si $g(-4,1) = 15$, que vaut $q$ ? Arrondir au centième près.
		\item Si $g(\pi) = \pi$, que vaut $q$ ? Arrondir au centième près.
	\end{enumerate}
}{exe:3a}{
	todo
}


\exe{, difficulty=1}{
	Soit $S$ est une suite géométrique.
	Donner l'expression algébrique $S_n$ en sachant que
		\begin{align*}
			S_3 = \dfrac8{27}, && \text{ et } && S_4 = \dfrac{32}{81}.
		\end{align*}
}{exe:4a}{
	On se rappelle de la définition des suites géométriques : pour passer d'une terme à l'autre, on multiplie par la raison $q$.
	Par conséquent, on obtient
		\[ \dfrac8{27} \times q = \dfrac{32}{81}. \]
	On résoud en multipliant par $\dfrac{27}{8}$, l'inverse de $\dfrac{8}{27}$ : 
		\begin{align*}
			\dfrac{27}{8} \times \dfrac8{27} \times q &= \dfrac{27}{8}\times\dfrac{32}{81} \\
			q &= \dfrac{32 \times 27}{8 \times 81} \\
			q &= \dfrac{32}8 \times \dfrac{27}{81} \\
			q &= 4 \times \dfrac13 = \dfrac43,
		\end{align*}
	où on a utilisé que $81 = 27 \times 3$ pour simplifier la fraction $\frac{27}{81}$

	On a trouvé $q$, et on peut déduire le terme initial en remontant en arrière :
		\begin{align*}
			S(3) = q \times S(2) && \iff && S(2) = \dfrac1q  \times S(3).
		\end{align*}
	D'où 	
		\begin{align*}
			S(2) &= \dfrac34 \times \dfrac8{27} =\dfrac29, \\
			S(1) &= \dfrac34 \times \dfrac23 = \dfrac16, \\
			S(0) &= \dfrac34 \times \dfrac12 = \dfrac14.
		\end{align*}
		
	Un autre façon de faire est d'écrire 
		\[ \dfrac{8}{27} = S(3) = S(0) \times q^3 = S(0) \times \left(\dfrac43\right)^3 = S(0) \times \dfrac{64}{27}, \]
	de quoi on déduit immédiatement
		\[ S(0) = \dfrac8{27}\times\dfrac{27}{64} = \dfrac8{64} = \dfrac14. \]
}

\exe{, difficulty=2}{
	$g$ est une fonction exponentielle.
	Donner l'expression algébrique $g(x)$ en sachant que
		\begin{align*}
			g(2,5) = 204,8 && \text{ et } && g(3) = 819,2.
		\end{align*}
}{exe:5a}{
	La méthode de l'exercice précédent se généralise : au lieu de multiplier par $q = q^1$ pour passer de $S(3)$ à $S(4)$ (et donc d'avoir $\dfrac{S(4)}{S(3)} = q$), on multiplie par $q^{0,5}$ pour passer de $g(2,5)$ à $g(3)$.
	
	Ceci se démontre de la façon suivante :
		\begin{align*}
			g(2,5) &= g(0) \times q^{2,5} \\
			g(3) &=  g(0) \times q^{3},
		\end{align*}
	dont on déduit que
		\begin{align*}
			\dfrac{g(3)}{g(2,5)} &= \dfrac{g(0) \times q^{3}}{g(0) \times q^{2,5}} \\
									&= \dfrac{q^3}{q^{2,5}} \\
									&= q^3 \times q^{-2,5} \\
									&= q^{3-2,5} = q^{0,5}.
		\end{align*}
	On connaît le ratio, égal à $\dfrac{g(3)}{g(2,5)} = \dfrac{819,2}{204,8} = 4$.
	Pour obtenir $q$ à partir de $q^{0,5}$, on met à la puissance $\dfrac1{0,5} = 2$ :
		\[ q = q^1 = \left(q^{0,5}\right)^2 = 4^2 = 16. \]
	On conclut donc que $q=16$.

	Pour obtenir $g(0)$, on utilise une des deux équations déjà posées plus haut.
	Par exemple, $g(3) =  g(0) \times q^{3}$, qui implique
		\[ g(0) = \dfrac{g(3)}{q^3} = \dfrac{819,2}{16^3} = 0,2. \]
	
	Finalement, $g(x) = 0,2 \times 16^x$ pour tout $x\in\R$.
}

\exe{, difficulty=1}{
	Soit $S$ est une suite géométrique.
	Donner l'expression algébrique de $S$ en sachant que
		\begin{align*}
			S_4 = 768, && \text{ et }  && S_7 = 49152.
		\end{align*}
}{exe:6a}{
	Le même raisonnement qu'aux exercice précédent donne l'équation
		\[ S(7) = q^3 \times S(4), \]
	de laquelle on déduit
		\[ q^3 = \dfrac{49152}{768} = 64.\]
	Pour déduire $q$, on met à la puissance $\frac13$, qui donne
		\[ q = 64^{1/3} = 4. \]
	
	On cherche ensuite à connaître $S(0)$ :
		\[ S(0) = \dfrac{S(4)}{q^4} = \dfrac{768}{4^4} = 3. \]
	D'où $S(n) = 3 \times 4^n$ pour tout $n \in \N$.
}

\exe{, difficulty=2}{
	$g$ est une fonction exponentielle.
	Donner l'expression algébrique de $g$ en sachant que
		\begin{align*}
			g(-3) = \dfrac1{49}, && \text{ et }  && g(3) = 2401.
		\end{align*}
}{exe:7a}{
	On trouve d'abord $q$ en utilisant que
		\[ g(3) = g(-3) \times q^{6}, \]
	qui donne $q^6 = 2401\times49 = 117649$, et donc $q = 117649^{1/6} = 7$.
	
	On conclut à l'aide de
		\[ g(0) = \dfrac{g(3)}{q^3} = \dfrac{2401}{7^3} = 7. \]
	Il suit que $g(x) = 7 \times 7^x = 7^{x+1}$ pour tout $x\in\R$.
}

%%%%%%%%%%%%

\newpage
\fancyhead[C]{\textbf{Solutions}}
\shipoutAnswer

\end{document}
