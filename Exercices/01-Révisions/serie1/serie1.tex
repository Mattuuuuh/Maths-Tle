\documentclass[14pt]{beamer}
\usepackage[french]{babel}

\usetheme{CambridgeUS}
\usecolortheme{rose}
\beamertemplatenavigationsymbolsempty


\usepackage{libertinus}
\usepackage{amsmath,amsfonts,amsthm,amssymb,mathtools}
\usepackage{array}
\newcolumntype{P}[1]{>{\centering\arraybackslash}p{#1}}


\usepackage{stackengine}
\newcommand\xrowht[2][0]{\addstackgap[.5\dimexpr#2\relax]{\vphantom{#1}}}


% corps
\usepackage{calrsfs}
\newcommand{\C}{\mathcal{C}}
\newcommand{\R}{\mathbb{R}}
\newcommand{\Rnn}{\mathbb{R}^{2n}}
\newcommand{\Z}{\mathbb{Z}}
\newcommand{\N}{\mathbb{N}}
\newcommand{\Q}{\mathbb{Q}}

% domain
\newcommand{\D}{\mathcal{D}}


% date
\usepackage{advdate}
\AdvanceDate[0]

%plots
\usepackage{pgfplots, subcaption}
\definecolor{myg}{RGB}{56, 140, 70}
\definecolor{myb}{RGB}{45, 111, 177}
\definecolor{myr}{RGB}{199, 68, 64}

%boxes
\usepackage[most]{tcolorbox}
\usepackage{multicol}

%icomma
\usepackage{icomma}

%https://osl.ugr.es/CTAN/macros/latex/contrib/tcolorbox/tcolorbox.pdf
\newtcolorbox{mybox}[3][]
{
  colframe = #2!25,
  colback  = #2!10,
  coltitle = #2!20!black,  
  halign title=flush center, 
  title    = {#3},
  #1,
}

% BOX A BOX B
\newcommand{\boxAB}[2]{
		\begin{mybox}{red}{A}
		\begin{center}
			#1
		\end{center}
		\end{mybox}
		\begin{mybox}{green}{B}
		\begin{center}
			#2
		\end{center}
		\end{mybox}
}

%systèmes
\usepackage{systeme}

% trafficotage 
\usepackage[answerdelayed, lastexercise]{exercise}
\renewcommand{\ExerciseHeader}{
}
\renewcommand{\AnswerHeader}{
}

\newcommand{\framedelayed}[3][]{
	\begin{Exercise}
	\begin{frame}{\theExercise #1\vspace{-32pt}}
		#2
	\end{frame}
	\end{Exercise}
	\begin{Answer}
	\begin{frame}{\theExercise #1\vspace{-32pt}}
		#3
	\end{frame}
	\end{Answer}
}


\AdvanceDate[1]

\begin{document}
\pagestyle{fancy}
\fancyhead[L]{Terminale STMG 1}
\fancyhead[C]{\textbf{Révisions}}
\fancyhead[R]{\today}


\exe{}{
	Qu'est-ce que $\N$ ? Qu'est-ce que $\R$ ?
}{exe:1}{
	$\N = \bigset{0 ; 1 ; 2 ; 3; \dots}$ est l'ensemble des entiers naturels.
	
	$\R$ est l'ensemble des nombres réels. Ils sont continus, sans trou, contrairement aux entiers ou aux rationnels.
}

\exemulticols{}{
	Donner les coordonées des points $A, B, C$ du plan ci-contre.
	
	Y ajouter les points 
		\begin{align*}
			&O(0;0), \\
			&D(-2;1), \\
			&E(3;2), \et  \\ 
			&F(1;-2).
		\end{align*}
	\vfill\null
}{
	\centering
	\begin{tikzpicture}[>=stealth, scale=1]
		\begin{axis}[xmin = -2.9, xmax=4.9, xtick={ -3, ..., 5}, ymin=-2.9, ymax=4.9, ytick={-3, ..., 5}]
			\addplot[transparent, mark=*, mark size = 1] (0,0) node[below=8pt, left] {$O$};
			
			\addplot[BLUE_E, mark=*, mark size = 2] (2,3) node[above right] {$A$};
			\addplot[RED_E, mark=*, mark size = 2] (-1,2) node[above left] {$B$};
			\addplot[GREEN_E, mark=*, mark size = 2] (3,-1) node[below right] {$C$};
			
			\addplot[transparent, mark=*, mark size = 1] (-2,1) node[below] {$D(-2 ; 1)$};
			\addplot[transparent, mark=*, mark size = 1] (3,2) node[right] {$E(3 ; 2)$};
			\addplot[transparent, mark=*, mark size = 1] (1,-2) node[below] {$F(1; -2)$};
		\end{axis}
	\end{tikzpicture}
}{exe:2}{

	\centering
	\begin{tikzpicture}[>=stealth, scale=1]
		\begin{axis}[xmin = -2.9, xmax=4.9, xtick={ -3, ..., 5}, ymin=-2.9, ymax=4.9, ytick={-3, ..., 5}]
			\addplot[GREY, mark=*, mark size = 1] (0,0) node[below=8pt, left] {$O$};
			
			\addplot[BLUE_E, mark=*, mark size = 1] (2,3) node[above right] {$A(2 ; 3)$};
			\addplot[RED_E, mark=*, mark size = 1] (-1,2) node[above left] {$B(-1 ; 2)$};
			\addplot[GREEN_E, mark=*, mark size = 1] (3,-1) node[below right] {$C(3; -1)$};
			
			\addplot[BLUE, mark=*, mark size = 1] (-2,1) node[below] {$D(-2 ; 1)$};
			\addplot[RED, mark=*, mark size = 1] (3,2) node[right] {$E(3 ; 2)$};
			\addplot[GREEN, mark=*, mark size = 1] (1,-2) node[below] {$F(1; -2)$};
		\end{axis}
	\end{tikzpicture}
}

\begin{dfn*}[courbe représentative]
	\[ \C_f = \Bigset{ \bigl(x ; f(x)\bigr) \text{ où $x$ parcourt $\R$}}. \]
\end{dfn*}

\exe{}{
	On considère la fonction $f$ donnée algébriquement par 
		\[ f(x) = 3x -2. \]
	Donner les coordonnées des points de $\C_f$ en $x= -1 ; 0 ; 1 ;$ et 2.
	Placer ces points dans le repère en bas de page.
	Que dire de $\C_f$ ? et de $f$ ?
}{exe:3}{
	Les points sonts $\bigl( -1 ; f(-1) \bigr) = (-1 ; -5)$, $\bigl( 0 ; f(0) \bigr) = (0 ; -2)$, $\bigl( 1 ; f(1) \bigr) = (1 ; 1)$, $\bigl( 2 ; f(2) \bigr) = (2 ; 4)$.
	
	$\C_f$ est une droite, et $f$ est une fonction affine.
}

\exe{}{
	Procéder comme à l'exercice \ref{exe:3} pour la fonction $g(x) = -x + 2$.
	En quel $x$ se situe le point d'intersection de $\C_f$ et $\C_g$ ?
}{exe:4}{
	Les points sonts $\bigl( -1 ; g(-1) \bigr) = (-1 ; 3)$, $\bigl( 0 ; g(0) \bigr) = (0 ; 2)$, $\bigl( 1 ; g(1) \bigr) = (1 ; 1)$, $\bigl( 2 ; g(2) \bigr) = (2 ; 0)$.
	
	$\C_g$ est une droite, et $g$ est également une fonction affine.
	
	Le point d'intersection semble se situer en $x=1$.
	On peut vérifier que $f(1) = g(1) = 1$, ce qui démontre bien que $( 1 ; 1)$ appartient aux deux droites.
}

\exe{}{
	Résoudre $f(x) = g(x)$ pour $x$ et comparer la valeur obtenue avec celle de l'exercice \ref{exe:4}.
}{exe:5}{
	On pose d'abord $f(x) = g(x)$, et on procède ainsi : on substitue $f(x)$ et $g(x)$ par leur expression algébrique donnée par l'énoncé ; on isole $x$ d'un côté ; on isole les constantes de l'autre ; on divise pour trouver $x$.
	\begin{align*}
		f(x) &= g(x), \\
		3x - 2 &= -x + 2, \\
		4x &= 4, \\
		x &= 1.
	\end{align*}
	
	Poser $f(x) = g(x)$ et résoudre donne bien le (ou les) $x$, abscisse de point d'intersection entre $\C_f$ et $\C_g$.
}

\begin{center}
\begin{tikzpicture}[scale=1.5]
\begin{axis}[ytick distance=1, xtick distance=1, domain=-1.1:2.1]
	\addplot[transparent, very thick] expression{3*x-2} node[right] {$\C_f$};
	\addplot[transparent, very thick] expression{-x+2} node[pos=0,left] {$\C_g$};
\end{axis}
\end{tikzpicture}
\end{center}


%%%%%%%%%%%%

\newpage
\fancyhead[C]{\textbf{Solutions}}
\shipoutAnswer



\begin{center}
\begin{tikzpicture}[scale=1]
\begin{axis}[ytick distance=1, xtick distance=1, domain=-1.1:2.1]
	\addplot[BLUE_E, very thick] expression{3*x-2} node[right] {$\C_f$};
	\addplot[RED_E, very thick] expression{-x+2} node[pos=0,left] {$\C_g$};
\end{axis}
\end{tikzpicture}
\end{center}

\end{document}
