\documentclass[14pt]{beamer}
\usepackage[french]{babel}

\usetheme{CambridgeUS}
\usecolortheme{rose}
\beamertemplatenavigationsymbolsempty


\usepackage{libertinus}
\usepackage{amsmath,amsfonts,amsthm,amssymb,mathtools}
\usepackage{array}
\newcolumntype{P}[1]{>{\centering\arraybackslash}p{#1}}


\usepackage{stackengine}
\newcommand\xrowht[2][0]{\addstackgap[.5\dimexpr#2\relax]{\vphantom{#1}}}


% corps
\usepackage{calrsfs}
\newcommand{\C}{\mathcal{C}}
\newcommand{\R}{\mathbb{R}}
\newcommand{\Rnn}{\mathbb{R}^{2n}}
\newcommand{\Z}{\mathbb{Z}}
\newcommand{\N}{\mathbb{N}}
\newcommand{\Q}{\mathbb{Q}}

% domain
\newcommand{\D}{\mathcal{D}}


% date
\usepackage{advdate}
\AdvanceDate[0]

%plots
\usepackage{pgfplots, subcaption}
\definecolor{myg}{RGB}{56, 140, 70}
\definecolor{myb}{RGB}{45, 111, 177}
\definecolor{myr}{RGB}{199, 68, 64}

%boxes
\usepackage[most]{tcolorbox}
\usepackage{multicol}

%icomma
\usepackage{icomma}

%https://osl.ugr.es/CTAN/macros/latex/contrib/tcolorbox/tcolorbox.pdf
\newtcolorbox{mybox}[3][]
{
  colframe = #2!25,
  colback  = #2!10,
  coltitle = #2!20!black,  
  halign title=flush center, 
  title    = {#3},
  #1,
}

% BOX A BOX B
\newcommand{\boxAB}[2]{
		\begin{mybox}{red}{A}
		\begin{center}
			#1
		\end{center}
		\end{mybox}
		\begin{mybox}{green}{B}
		\begin{center}
			#2
		\end{center}
		\end{mybox}
}

%systèmes
\usepackage{systeme}

% trafficotage 
\usepackage[answerdelayed, lastexercise]{exercise}
\renewcommand{\ExerciseHeader}{
}
\renewcommand{\AnswerHeader}{
}

\newcommand{\framedelayed}[3][]{
	\begin{Exercise}
	\begin{frame}{\theExercise #1\vspace{-32pt}}
		#2
	\end{frame}
	\end{Exercise}
	\begin{Answer}
	\begin{frame}{\theExercise #1\vspace{-32pt}}
		#3
	\end{frame}
	\end{Answer}
}


\AdvanceDate[0]

\begin{document}
\pagestyle{fancy}
\fancyhead[L]{Tle STMG}
\fancyhead[C]{\textbf{Évaluation blanche — Suites, sommes, et complexités}}
\fancyhead[R]{\today}

Consignes particulières : 
\begin{itemize}[label=$\bullet$]
	\item 
	La calculatrice est {interdite}.
	\item
	Lorsqu'une réponse en fonction de $n$ est attendue, il est recommandé de prendre $n=1 ; 2 ; 3 ; ...$, pour ensuite généraliser à n'importe quel entier naturel $n\in\N$.
\end{itemize}

\hrule

\exe{}{
	Quelles sont les opérations arithmétiques élémentaires dans le modèle RAM ?
	
	Donner une opération mathématique qui n'est pas une opération élémentaire.
}{exe:1}{
	Les opérations arithmétiques élémentaires sont l'addition et la multiplication (ce qui inclut la soustration et la division).
}

\exe{}{
	Calculer les 4 premiers termes de la suite donnée algébriquement par 
		\[ u_n = 3n-2. \]
}{exe:1}{
	En remplaçant $n$ par 0 ; 1 ; 2 ; et 3, on trouve $u_0 = -2, u_1 = 1, u_2 = 4, u_3 = 7$.
}

\exe{}{
	Le professeur de Carl Freidrich Gauss\footnotemark lui demande de calculer la somme suivante.
		\[ 0+1 + 2 + 3 + \cdots + 199 + 200 \]
	Donner sa valeur.
}{exe:3}{
	D'après le cours, on calcule 
		\[ 201\times\dfrac{200}{2} = 201 \times 100 = 20~100. \]
}

\footnotetext{1777-1855, mathématicien. astronome, et physicien allemand.}

\exe{, difficulty=1}{
	En fonction de $n\in\N$, dire combien d'opérations élémentaires sont nécessaires pour calculer naïvement la valeur de $7^n$. Justifier.
	On ne s'intéresse qu'à l'ordre.
}{exe:complexite-pow3}{
	Comme
		\[ 7^n = \underbrace{7 \times 7 \times 7 \times \cdots \times 7}_{\text{$n$ fois}}, \]
	on compte $n-1$ opérations arithmétiques au total.
	$n-1 = \O(n)$, d'ordre linéaire.
}

\exe{, difficulty=2}{
	Dans cet exercice, on souhaite diminuer le nombre d'opérations nécessaires au calcul de $7^n$ de l'exercice \ref{exe:complexite-pow3}.
	On traite uniquement le cas $n=32$ et on se demande : combien d'opérations sont nécessaires pour calculer $7^{32}$ ?
	\begin{enumerate}
		\item
		Combien d'opérations sont nécessaires pour calculer $7^2$ ?
		\item 
		Supposons qu'on connaisse la valeur de $7^2$.
		Combien d'opérations sont nécessaires pour calculer $\bigl(7^2\bigr)^2$ ?
		\item
		Montrer que $\bigl(7^2\bigr)^2 = 7^4$.
		\item 
		Supposons qu'on connaisse la valeur de $7^4$.
		Combien d'opérations sont nécessaires pour calculer $\bigl(7^4\bigr)^2$ ?
		\item
		Montrer que $\bigl(7^4\bigr)^2 = 7^8$. 
		\item
		Répéter ce processus jusqu'à obtenir $7^{32}$.
		Combien d'opérations ont été effectuées au total ?	
	\end{enumerate}
}{bb}{
	\begin{enumerate}
		\item
		$7^2 = 7\times7$, donc une seule.
		\item 
		$\bigl(7^2\bigr)^2 = 7^2 \times 7^2$, donc une seule opération à nouveau.
		\item
		$\bigl(7^2\bigr)^2 = 7^{2 \times 2} = 7^4$.
		\item 
		$\bigl(7^4\bigr)^2 = 7^4 \times 7^4$, donc une seule opération à nouveau.
		\item
		$\bigl(7^4\bigr)^2 = 7^{4 \times 2} = 7^8$. 
		\item
		On calcule $7^2$, puis $7^4$, puis $7^8$, puis $7^{16}$, et enfin $7^{32}$ en 5 opérations au total, en mettant le résultat de chaque étape au carré pour passer à l'étape suivante.
	\end{enumerate}
}




\exe{}{
	Calculer les 5 premiers termes de la suite définie récursivement par
		\[ \begin{cases*} v_0 = 0, \\ v_{n+1} = v_n -5. \end{cases*} \]
	Donner une conjecture sur sa forme algébrique.
}{exe:suite-recursive}{
	L'expression récursive se lit de la façon suivante : pour obtenir le terme de rang $n+1$, il faut ajouter 12 au terme de rang $n$.
	Le terme initial (de rang 0) étant 0, on a : $v_0 = 0, v_1 = -5, v_2 = -10, v_3 = -15, v_4 = -20$.
	
	On conjecture que $v_n = -5n$ pour tout $n\in\N$.
}

\exe{}{
	Une procédure effectue boucle sur $k$ allant de 1 à $n$.
	Chaque exécution de la boucle effectue $\O(k)$ opérations élémentaires.
	
	Combien d'opération élémentaires sont effectuées au total par la procédure ?
}{exe:7}{
	D'après le cours, sommer sur un même ordre augmente d'un ordre : $\sum_{k=1}^n \O(k^a) = \O(n^{a+1})$.
	Au total, $\O(n^2)$ opérations élémentaires sont donc effectuées.
	
	Par exemple, si la boucle effectue $2k = \O(k)$ opérations à chaque exécution, au total seront effectuées
		\[ \sum_{k=1}^n 2k = 2 \sum_{k=1}^n k= n(n+1) = n^2 + n = \O(n^2) \]
	opérations, à nouveau d'après le cours.
}


%%%%%%%%%%%%

\newpage
\fancyhead[C]{\textbf{Solutions}}
\shipoutAnswer

\end{document}
