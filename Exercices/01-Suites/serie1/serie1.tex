\documentclass[14pt]{beamer}
\usepackage[french]{babel}

\usetheme{CambridgeUS}
\usecolortheme{rose}
\beamertemplatenavigationsymbolsempty


\usepackage{libertinus}
\usepackage{amsmath,amsfonts,amsthm,amssymb,mathtools}
\usepackage{array}
\newcolumntype{P}[1]{>{\centering\arraybackslash}p{#1}}


\usepackage{stackengine}
\newcommand\xrowht[2][0]{\addstackgap[.5\dimexpr#2\relax]{\vphantom{#1}}}


% corps
\usepackage{calrsfs}
\newcommand{\C}{\mathcal{C}}
\newcommand{\R}{\mathbb{R}}
\newcommand{\Rnn}{\mathbb{R}^{2n}}
\newcommand{\Z}{\mathbb{Z}}
\newcommand{\N}{\mathbb{N}}
\newcommand{\Q}{\mathbb{Q}}

% domain
\newcommand{\D}{\mathcal{D}}


% date
\usepackage{advdate}
\AdvanceDate[0]

%plots
\usepackage{pgfplots, subcaption}
\definecolor{myg}{RGB}{56, 140, 70}
\definecolor{myb}{RGB}{45, 111, 177}
\definecolor{myr}{RGB}{199, 68, 64}

%boxes
\usepackage[most]{tcolorbox}
\usepackage{multicol}

%icomma
\usepackage{icomma}

%https://osl.ugr.es/CTAN/macros/latex/contrib/tcolorbox/tcolorbox.pdf
\newtcolorbox{mybox}[3][]
{
  colframe = #2!25,
  colback  = #2!10,
  coltitle = #2!20!black,  
  halign title=flush center, 
  title    = {#3},
  #1,
}

% BOX A BOX B
\newcommand{\boxAB}[2]{
		\begin{mybox}{red}{A}
		\begin{center}
			#1
		\end{center}
		\end{mybox}
		\begin{mybox}{green}{B}
		\begin{center}
			#2
		\end{center}
		\end{mybox}
}

%systèmes
\usepackage{systeme}

% trafficotage 
\usepackage[answerdelayed, lastexercise]{exercise}
\renewcommand{\ExerciseHeader}{
}
\renewcommand{\AnswerHeader}{
}

\newcommand{\framedelayed}[3][]{
	\begin{Exercise}
	\begin{frame}{\theExercise #1\vspace{-32pt}}
		#2
	\end{frame}
	\end{Exercise}
	\begin{Answer}
	\begin{frame}{\theExercise #1\vspace{-32pt}}
		#3
	\end{frame}
	\end{Answer}
}


\AdvanceDate[0]

\begin{document}
\pagestyle{fancy}
\fancyhead[L]{Tle STMG}
\fancyhead[C]{\textbf{Suites, sommes, et complexités}}
\fancyhead[R]{\today}

\exe{}{
	Calculer les 5 premiers termes de la suite donnée algébriquement par 
		\[ u_n = 3n+2. \]
}{exe:1}{
	TODO
}


\exe{}{
	Calculer les 5 premiers termes de la suite
		\[ \begin{cases*} v_0 = 0, \\ v_{n+1} = v_n + 5. \end{cases*} \]
	Donner une conjecture sur sa forme algébrique.
}{exe:suite-recursive}{
	TODO
}

\exe{,difficulty=1}{
	On considère la série statistique $X=(0;1;2;\dots;19;20)$.
	
	Combien d'éléments $X$ contient-elle ? Quelle est la moyenne de $X$ ? Justifier sans calcul.
	
	En déduire que $1+2+\cdots+19+20 = 210$.
}{exe:moyenne}{
	Par symétrie autour de $10$, la moyenne de $X$ vaut $10$ : toute valeur $10+k$ se compense avec $10-k$ pour $k=1, \dots, 10$, et la valeur $10$ seule est moyenne et n'a donc aucun impact.
	
	La formule de la moyenne donne $\dfrac{1+2+\cdots+19+20}{21} = 10$, d'où le résultat.
}

\exe{}{
	Donner les valeurs des sommes suivantes.
	\begin{multicols}{3}
		\begin{enumerate}[label=\roman*)]
		\item
		$\sum\limits_{k=1}^{10} 0$
		
		\item
		$\sum\limits_{k=1}^{10} 1$
		
		\item
		$\sum\limits_{k=0}^{10} 1$
		
		\item
		$\sum\limits_{k=-5}^{11} 1$
		
		\item
		$\sum\limits_{k=4}^{8} 2$
		
		\item
		$\sum\limits_{k=0}^{n} 1$
		
		\item
		$\sum\limits_{k=1}^{10} k$
		
		\item
		$\sum\limits_{k=1}^{20} k$
		
		\item
		$\sum\limits_{k=11}^{20} k$
		\end{enumerate}
	\end{multicols}
}{exe:sommes}{
	\begin{multicols}{3}
		\begin{enumerate}[label=\roman*)]
		\item
		$\sum\limits_{k=1}^{10} 0 = 0$.
		
		\item
		$\sum\limits_{k=1}^{10} 1 = 10$.
		
		\item
		$\sum\limits_{k=0}^{10} 1 = 11$.
		
		\item
		$\sum\limits_{k=-5}^{11} 1 = (11 + 5 + 1) = 17$.
		
		\item
		$\sum\limits_{k=4}^{8} 2 = 2(8-4+1) = 10$.
		
		\item
		$\sum\limits_{k=0}^{n} 1 = n$.
		
		\item
		$\sum\limits_{k=1}^{10} k = \dfrac{10(11)}2 = 55$.
		
		\item
		$\sum\limits_{k=1}^{20} k = \dfrac{20(21)}2 = 210$.
		
		\item
		$\sum\limits_{k=11}^{20} k = \sum\limits_{k=1}^{20} k - \sum\limits_{k=1}^{10} k = 210-55 = 155$.
		\end{enumerate}
	\end{multicols}
}


\exe{, difficulty=1}{
	Montrer que 
		\[ 3\sum_{k=0}^{n} v_k = \sum_{k=0}^{n} 3v_k. \]
	En déduire la valeur de 
		\[ \sum_{k=0}^n 3k. \]
}{exe:distr-sum}{
	TODO
}

\exe{}{
	Donner les valeurs de 
		\begin{align*}
			\sum_{k=0}^n \bigl[ 3k + 2 \bigr], && \et && \sum_{k=0}^n \bigl[ 3k \bigr] + 2.
		\end{align*}
}{exe:sum-linear}{
	TODO
}

\exe{}{
	Combien de temps est-il nécessaire pour calculer $n^n$ ? On ne s'intéresse qu'à l'ordre.
}{exe:n-pow-n}{
	TODO
}

\exe{, difficulty=1}{
	En fonction de $n$, dire combien de temps et d'espace sont nécessaires pour calculer naïvement les sommes suivantes.
	On ne s'intéressera qu'aux ordres.
		\begin{multicols}{5}
		\begin{enumerate}
			\item $\sum\limits_{k=1}^n k^2$
			\item $\sum\limits_{k=1}^n \frac1k$
			\item $\sum\limits_{k=1}^{n^2} k$
			\item $\sum\limits_{k=1}^{n} \sum\limits_{\ell=1}^{k} 1$
			\item $\sum\limits_{k=1}^{n} k^k$
		\end{enumerate}
		\end{multicols}
}{exe:complexité-sommes}{
	TODO
}

\exe{}{
	Démontrer que la somme $1 + 2 + \cdots + n$ est calculable en temps $\O(1)$, indépendamment de $n$.
	
	À votre avis, un ordinateur prend-il vraiment le même temps pour calculer la somme pour $n=10$ et pour $n=131~072$ ?
}{exe:complexité-triangle}{
	TODO
}


%%%%%%%%%%%%

\newpage
\fancyhead[C]{\textbf{Solutions}}
\shipoutAnswer

\end{document}
