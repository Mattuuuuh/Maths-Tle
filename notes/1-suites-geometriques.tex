%!TEX encoding = UTF8
%!TEX root = 0-notes.tex

\chapter{Suites géométriques}

\section{Introduction}

\dfn{suite géométrique}{
	Une \emphindex{suite géométrique} $u$ est une suite vérifiant, pour tout $n\in\N$,
		\begin{align}\label{eq:suite-geom}
			u(n+1) = q \cdot u(n),
		\end{align}
	où $q \in \R$ est un réel fixé qui ne dépend pas de $n$ appelé la \emphindex{raison}.
	
	On lit :
		\begin{center}
		« Pour passer d'un terme au suivant, on multiplie par la raison $q$. »
		\end{center}
}{dfn:suite-géométrique}

\mprop{propriétés des puissances}{
	Soient $a, b, c \in \Z$. On a les relations suivantes.
		\begin{gather*}
			a^{b+c} = a^{b} \times a^{c} \\
			\left(a^b\right)^c = a^{b \times c} \\
			a^{c} \times b^c = \left( a \times b \right)^c
		\end{gather*}
	En particulier, si $a \neq 0$, on a
		\begin{align*}
			a^0 = 1, &
			&a^1 = a, &
			&a^{-1} = \dfrac1a. 
		\end{align*}
}{prop:prop-puissances}

\exe{}{
	Donner les 5 premiers termes de la suite géométrique $u$ de raison 2 et de terme initial $u(0) = 1$.
	Faire une conjecture sur l'expression algébrique de $u(n)$ pour tout $n\in\N$.
}{exe:geom1}{
	Pour passer d'un rang au suivant, on multiplie par la raison, ici $2$.
	Donc $u(1) = 2, u(2) = 4, u(3) = 8, u(4) = 16$.
	
	Il semblerait que $u(n) = 2^n$ pour tout $n\in\N$.
}

\exe{}{
	Montrer que la suite $u$ telle que $u(102) = 25, u(103) = 50, u(104) = 110$ ne peut pas être géométrique.
}{exe:nongeom1}{
	Si $u$ était géométrique, sa raison serait $2$ car c'est ce par quoi on multiplie pour passer du terme 102 au terme 103.
	Cependant, dans ce cas, on aurait $u(104) = 100$, ce qui est contradictoire avec la donnée de l'exercice. \Large\Lightning
}

\exe{}{
	Montrer que la suite $v$ telle que $u(65) = 32, u(67) = 64, u(68) = 128$ ne peut pas être géométrique.
}{exe:nongeom2}{
	Si $v$ était géométrique, sa raison serait $2$ car c'est ce par quoi on multiplie pour passer du terme 67 au terme 68.
	Cependant, dans ce cas, on aurait $u(66) = 32$, et $u(65) = 16$, ce qui est contradictoire avec la donnée de l'exercice. \Large\Lightning
}

\exe{, difficulty=1}{
	Si une suite $w$ vérifie $w(10) = 1, w(11) = 3, w(12) = 9$, est-elle nécessairement géométrique ?
}{exe:nongeom3}{
	$w$ pourrait être géométrique, auquel cas $w(n) = 3^{n-10}$ pour tout $n\in\N$, mais ce n'est pas nécessairement le cas.
	On peut librement définir $w(13) = -10 000$ pour casser le caractère géométrique de la suite. 
}

\exe{}{
	Soit $u$ une suite géométrique de raison $2$ telle que $u(3) = 24$.
	Donner le terme initial de $u$.
}{exe:terme-initial-geom}{
	Pour passer d'un rang au suivant, on doit multiplier par $2$.
	Pour revenir en arrière d'un rang, on multiplie donc par $\dfrac12$.
	Ainsi $u(2) = 24/2 = 12, u(1) = 12/2 = 6$, et $u(0) = 6/2 = 3$.
}


\ex{}{
	La suite $u$ définie par $u(n) = 5 \times 3^n$ pour tout $n\in\N$ est géométrique de raison 3 car
		\begin{align*}
			u(n+1) &= 5 \times 3^{n+1}, \\
					&= 5 \times 3^n \times 3, \\
					&= 3 \times \left( 5 \times 3^n \right), \\
					&= 3 u(n).
		\end{align*}
	Son terme initial est $u(0) = 5 \times 3^0 = 5 \times 1 = 5$.
}{ex:suite-geo}

\exe{}{
	Montrer que la suite $v(n) = 3 \times 2^n$ est géométrique. Donner sa raison et son terme initial.
}{exe:suite-geo}{
	Les propriétés des puissances donnent 
		\begin{align*}
			v(n+1) &= 3 \times 2^{n+1}, \\ 
					&= 3 \times 2 \times 2^n, \\
					&= 2 (3 \times 2^n), \\
					&= 2v(n).
		\end{align*}
	Il suit que $v$ est géométrique de raison 2. Son terme initial est $v(0) = 3$.
}

\section{Expression algébrique}

\ex{}{
	Considérons une suite $G$, géométrique de raison $4$ et de terme initial $3$.
	Alors, par définition,
		\[ G(0) = 3. \]
	En spécialisant l'équation \eqref{eq:suite-geom} pour $n=0$, on trouve ensuite
		\[ G(1) = G(0+1) = 4 \times G(0) = 4 \times 3 = 12. \]
	On peut déduire la suite des termes de façon analogue.
		\begin{align*}
			G(0) &= 3 \\
			G(1) &= 4 \times G(0) = 12 \\
			G(2) &= 4 \times G(1) = 48 \\
			G(3) &= 4 \times G(2) = 196 \\
			\vdots &\, \qquad \vdots
		\end{align*}
	On souhaite désormais connaître une formule valable pour tous les entiers naturels $n\in\N$ pour qu'on ait pas à reconstruire la suite depuis le début à chaque étude.
	Pour cela, on choisit de réécrire les termes de la suite de la façon suivante, en utilisant les propriétés de la puissance.
		\begin{align*}
			G(0) &= 3 = 3 \times 4^{0} \\
			G(1) &= 4 \times G(0) = 3 \times 4^{1} \\
			G(2) &= 4 \times G(1) = 3 \times 4^{2} \\
			G(3) &= 4 \times G(2) = 3 \times 4^{3} \\
			\vdots &\, \qquad \vdots \\
			G(n) &= 3 \times 4^n
		\end{align*}
}{}

\thm{de Maya}{
	Soit $v$ une suite géométrique de raison $q \in \R$ et de terme initial $v(0) \in \R$.
	Alors
		\[ v(n) = v(0) \times q^n. \]
}{thm:maya}

\exe{}{
	Pour chacune des suites données algébriquement pour tout $n\in\N$, donner leur raison et leur terme initial.
	\begin{multicols}{2}
	\begin{enumerate}
		\item $u(n) = 2 \times 3^n$
		\item $v(n) = 7 \times \left(\dfrac12 \right)^n$
		\item $\xi(n) = - 6^n$
		\item $a(n) = 11 \times 5^{2n}$
		\item $b(n) = 3 \times 5^{2n+3}$
		\item $c(n) = 10^{-n}$
	\end{enumerate}
	\end{multicols}
}{exe:param-geom}{
	On note $q$ la raison de chacune des suites géométrique.
	Pour une suite géométrique $u$ et par définition, $q = \dfrac{u(1)}{u(0)}$.
	Il suffit donc d'évaluer $u$ en 0 et en 1 pour connaître son terme initial et sa raison lorsqu'elle n'est pas exactement de la forme $u(0) \times q^n$.
	\begin{multicols}{2}
	\begin{enumerate}
		\item $u(0) = 2$ et $q = 3$.
		\item $v(0) = 7$ et $q=\dfrac12$.
		\item $\xi(0) = -1$ et $q=6$.
		\item $a(0) = 11$ et $q = \dfrac{a(1)}{a(0)} = \dfrac{11\times5^2}{11} = 25$.
		\item $b(0) = 3 \times 5^3 = 375$ et $q = \dfrac{b(1)}{b(0)} = \dfrac{3\times5^5}{3\times5^3} = 5^2 = 25$.
		\item $c(0) = 1$ et $q=c(1) = 10^{-1}= \dfrac1{10} = 0,1$.
	\end{enumerate}
	\end{multicols}
}

\exe{, difficulty=1}{
	Considérons la suite $v$ définie par la relation de récurrence suivante, pour tout $n\in\N$.
	\[
	\begin{cases}
		v(0) = 0, \\
		v(n+1) = 2v(n) + 1.
	\end{cases}
	\]
	\begin{enumerate}
		\item Montrer que la suite intermédiaire $w$ définie pour tout $n\in\N$ par
			\[ w(n) = v(n) + 1,\]
		est géométrique.
		\item En déduire que, pour tout $n\in\N$,
			\[ v(n) = 2^n - 1.\]
	\end{enumerate}
}{exe:arithmético-géométrique}{
	Toutes les égalités suivantes sont valables pour tout $n\in\N$.
	\begin{enumerate}
		\item On a $w(n+1) = v(n+1) + 1 = 2v(n) + 1 + 1 = 2\bigl(v(n)+1\bigr) = 2w(n)$.
		$w$ est donc géométrique de raison 2.
		\item De la question précédente, on déduit du terme initial $w(0) = v(0)+1=0+1=1$ que $w(n)=2^n$ grâce au théorème \ref{thm:maya}.
		D'où $v(n) = w(n)-1 = 2^n - 1$.
	\end{enumerate}
}

% à mettre plutôt dans le chapitre f° exp ? 
%\section{Variations}
%
%\mprop{variations}{
%	Soit $u$ une suite géométrique de raison $q > 0$ et de terme initial $u(0) > 0$.
%	On distingue alors les trois cas suivants.
%		\begin{enumerate}[label=---]
%			\item Si $0 < q < 1$, alors $u$ est décroissante et tend vers $0$ exponentiellement.
%			\item Si $q=1$, alors $u$ est constante : $u(n) = u(0)$ pour tout $n\in\N$.
%			\item Si $q>1$, alors $u$ est croissante et diverge exponentiellement vers $+ \infty$.
%		\end{enumerate}
%}{}

\section{Sommes téléscopiques}

\dfn{somme téléscopique}{
	Une \emphindex{somme téléscopique} est une somme de la forme
		\[ S_n = \sum\limits_{k=0} \bigl[v_{k+1} - v_k\bigr], \]
	pour une certaine suite $(v_n)$.
}{}

\thm{}{
		\[ S_n = \sum\limits_{k=0}^n \bigl[v_{k+1} - v_k\bigr] = v_{n+1} - v_0. \]
}{thm:somme-telescopique}

\thm{}{
	Pour tout $n\in\N$, on a l'égalité
		\[ 1 + q + q^2 + \cdots + q^n  = \dfrac{1-q^{n+1}}{1-q}. \]
}{thm:somme-geom}
\pf{}{
	Somme téléscopique.
}

\exe{, difficulty=2}{
	On souhaite étudier la série de Bâle\footnotemark $S_n = \sum\limits_{k=1}^n \dfrac{1}{k^2}$. 
	En particulier on se demande si $S_n$ diverge ou si elle est bornée supérieurement. 
	
	Montrer que $\dfrac1{k(k+1)} = \dfrac1k - \dfrac1{k+1}$ pour tout $k\geq1$
	et en déduire que, pour tout $n\in\N$,
		\[ \sum_{k=1}^n \dfrac1{k(k+1)} = 1 - \dfrac1{n+1}. \]
	Pour tout $k, n \in\N$, montrer que $k(k+1) \geq (k+1)^2$ et conclure que $S_n \leq 2-\dfrac1n \leq 2$.
}{exe:serie-bale}{
	TODO
}
\footnotetext{Notamment étudiée par Jacques Bernoulli (165?-1705), mathématicien et physicien suisse né à Bâle.}

\nt{
	Euler\footnotemark démontra (plus ou moins rigoureusement) en 1735 que
		\[ \sum_{k=1}^{\infty} \dfrac1k = \dfrac{\pi^2}6. \]
}
\footnotetext{Leonhard Euler (1707-1783), mathématicien et physicien suisse né à Bâle.}

% à ignorer car le log est en sujet d'étude et ça simplifie tout.

%\section{Problèmes de seuil}
%
%\str{
%	On décrit ici une stratégie pour résoudre un problème de seuil courant qui est facile à résoudre pour les suites arithmétiques mais très dur à résoudre pour les suites géométriques.
%	Étant donné une suite géométrique $v$ croissante et un seuil $M > 0$, on souhaite trouver le plus petit entier naturel $N\in\N$ vérifiant
%		\[ v(N) > M. \]
%	Considérons l'exemple suivant.
%		\begin{align*}
%			v(n) = 3 \times 5^n && \text{et} && M = 100~000.
%		\end{align*}
%	La stratégie est de considérer un intervalle dans lequel le $N$ recherché doit nécessairement appartenir, et de scinder cet intervalle à chaque étape.
%	Pour commencer, on prend
%		\[ I = [0 ; 10], \]
%	car $v(0) = 3 < 100~000 < v(10) \approx 2,9 \times 10^{7}$.
%	La borne supérieure $10$ a été choisie au hasard et suffisamment grande telle que $v(10)$ dépasse le seuil.
%	
%	On divise l'intervalle en deux parts égales en considérant son milieu, 
%		\[ m = \dfrac{0+10}2 = 5. \]
%	On évalue $v$ en $5$ pour décider si $N$ est plus grand ou plus petit que $5$.
%	En l'occurrence, 
%		\[ v(5) = 9375 < 100~000. \]
%	Le seuil recherché est donc nécessairement supérieur à $5$, et on peut définir
%		\[ I = [5 ; 10], \]
%	comme nouvel intervalle dans lequel $N$ appartient.
%	
%	On répète l'opération en calculant le milieu $m = \dfrac{5+10}2 = 7,5$, et 
%		\[ v(7,5) \approx 5,2 \times 10^5 > 100~000. \]
%	D'où $I = [5 ; 7,5]$.
%	Ici, on peut soit tester les $3$ valeurs qui restent ($5 ; 6 ; 7$) ou continuer pour trouver
%		\[ I = [ 6,25 ; 7,5 ], \]
%	dans lequel $N = 7$ est le seul entier possible.
%	
%	On vérifiera bien sûr que $v(6) < 100~000 < v(7)$, comme voulu.
%}{}
%
%\IncMargin{1em}
%\begin{algorithm}
%\SetKwInput{KwRes}{retourner}%
%\SetKwIF{Si}{SinonSi}{Sinon}{si}{alors}{sinon si}{sinon}{fin si}%
%\SetKwFor{Tq}{tant que}{faire}{fin tq}%
%\SetKwInOut{Input}{entrée}\SetKwInOut{Output}{sortie}
%	\Input{Suite géométrique $G$ de raison $q > 1$ tel que $G(0) > 0$. Seuil $M$. Intervalle $I=[a ; b]$ tel que $G(a) < M < G(b)$.}
%	\Output{Le plus petit entier naturel $N\in\N$ tel que $G(N) \geq M$.}
%	\BlankLine
%	\emph{L'intervalle $I=[a;b]$ de départ doit vérifier $G(a) < M < G(b)$, de telle sorte que l'entier $N$ recherché lui appartienne nécessairement car $G$ est croissante. On pourrait prendre $a=0$ et $b$ extrêmement grand en cas de doute.}\\
%	\Tq{ la longueur de l'intervalle $I=[a;b]$ est supérieure  ou égale à $1$}{
%		$m = \dfrac{a+b}2$ \emph{, le milieu de l'intervalle $I$}\\
%		\Si{ $G(m) \geq M$}{
%			$I = [a ; m]$
%		}
%		\Sinon{
%			$I = [m ; b]$
%		}
%	}
%	\KwRes{L'unique entier appartenant à l'intervalle $I$.}
%\caption{Problème de seuil.}\label{algo:seuil-geom}
%\label{alg:seul-geom}
%\end{algorithm}\DecMargin{1em} 
%
%\nt{
%	C'est au début du 17è siècle que John Neper crée les premières tables de valeurs de la fonction $\ln$ qui porte son nom : le logarithme népérien.
%	La notation $\log$ désigne une fonction qui transforme la multiplication en une addition, cette dernière opération étant beaucoup plus facile lorsqu'on manipule des grands nombres.
%	Une bonne table de valeurs permet alors, pour multiplier deux grands nombres $A$ et $B$, d'additionner leur logarithmes grâce à la relation
%		\[ \log(A \times B) = \log(A) + \log(B). \]
%	Pour retrouver le produit, on utilise alors la table de valeurs dans le sens inverse.
%	
%	Plus généralement, l'application du $\log$ convertit une équation exponentielle en une équation linéaire simple.
%}{}

% à mettre plutôt dans le chapitre f° exp ? 
%\subsection{Échelle logarithmique}
