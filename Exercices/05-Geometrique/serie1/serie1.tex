\documentclass[14pt]{beamer}
\usepackage[french]{babel}

\usetheme{CambridgeUS}
\usecolortheme{rose}
\beamertemplatenavigationsymbolsempty


\usepackage{libertinus}
\usepackage{amsmath,amsfonts,amsthm,amssymb,mathtools}
\usepackage{array}
\newcolumntype{P}[1]{>{\centering\arraybackslash}p{#1}}


\usepackage{stackengine}
\newcommand\xrowht[2][0]{\addstackgap[.5\dimexpr#2\relax]{\vphantom{#1}}}


% corps
\usepackage{calrsfs}
\newcommand{\C}{\mathcal{C}}
\newcommand{\R}{\mathbb{R}}
\newcommand{\Rnn}{\mathbb{R}^{2n}}
\newcommand{\Z}{\mathbb{Z}}
\newcommand{\N}{\mathbb{N}}
\newcommand{\Q}{\mathbb{Q}}

% domain
\newcommand{\D}{\mathcal{D}}


% date
\usepackage{advdate}
\AdvanceDate[0]

%plots
\usepackage{pgfplots, subcaption}
\definecolor{myg}{RGB}{56, 140, 70}
\definecolor{myb}{RGB}{45, 111, 177}
\definecolor{myr}{RGB}{199, 68, 64}

%boxes
\usepackage[most]{tcolorbox}
\usepackage{multicol}

%icomma
\usepackage{icomma}

%https://osl.ugr.es/CTAN/macros/latex/contrib/tcolorbox/tcolorbox.pdf
\newtcolorbox{mybox}[3][]
{
  colframe = #2!25,
  colback  = #2!10,
  coltitle = #2!20!black,  
  halign title=flush center, 
  title    = {#3},
  #1,
}

% BOX A BOX B
\newcommand{\boxAB}[2]{
		\begin{mybox}{red}{A}
		\begin{center}
			#1
		\end{center}
		\end{mybox}
		\begin{mybox}{green}{B}
		\begin{center}
			#2
		\end{center}
		\end{mybox}
}

%systèmes
\usepackage{systeme}

% trafficotage 
\usepackage[answerdelayed, lastexercise]{exercise}
\renewcommand{\ExerciseHeader}{
}
\renewcommand{\AnswerHeader}{
}

\newcommand{\framedelayed}[3][]{
	\begin{Exercise}
	\begin{frame}{\theExercise #1\vspace{-32pt}}
		#2
	\end{frame}
	\end{Exercise}
	\begin{Answer}
	\begin{frame}{\theExercise #1\vspace{-32pt}}
		#3
	\end{frame}
	\end{Answer}
}


\SetDate[16/10/2025]

\begin{document}
\pagestyle{fancy}
\fancyhead[L]{Tle STMG}
\fancyhead[C]{\textbf{Suites géométriques 1}}
\fancyhead[R]{\today}


\exe{}{
	Jean de Florette élève une population de lapin de la race Romarin.
	Il étudie le nombre de lapins pendant plusieurs années avant de remarquer que, d'une année à l'autre, la population augmente systématiquement de $50\%$.
	Il note les premières valeurs obtenues dans le tableau en commençant à l'année $0$ pour simplifier les choses.
	\begin{center}
	\begin{tabular}{|c|c|c|c|c|c|}\hline
	Année & 0 & 1 & 2 & 3 & 4 \\ \hline
	Population & 256 & 384 & & & \\ \hline
	\end{tabular}
	\end{center}
	
	\begin{enumerate}
		\item Vérifier les premières valeurs du tableau ci-dessus et le compléter.
	\end{enumerate}
	On note $P_n$ le nombre de lapins à l'année $n$.
	\begin{enumerate}[resume]
		\item Écrire l'expression algébrique de $P_n$.
		\item Donner $P_{20}$ en arrondissant à l'entier le plus proche.
		\item Trouver le plus petit entier naturel $N\in\N$ tel que
		\[ P(N) \geq 100~000.\]
	\end{enumerate}
}{exe:JeanFlorette}{
	\begin{center}
	\begin{tabular}{|c|c|c|c|c|c|}\hline
	Année & 0 & 1 & 2 & 3 & 4 \\ \hline
	Population & 256 & 384 & \color{red}576  & \color{red} 864 & \color{red} 1296 \\ \hline
	\end{tabular}
	\end{center}
	
	\begin{enumerate}[start=2]
		\item 
		La suite est géométrique de raison 1,5 et de terme initial $P_0 = 256$, donc
			\[ P_n = 256 \times (1,5)^n. \]
		\item 
		À l'aide de la calculatrice, $P_{20} = 256 \times (1,5)^{20} \approx 851~266$.
		\item 
		La suite $P_n$ est croissante, donc le $N$ recherché doit être inférieur à 20 (et supérieur à 4, d'après le tableau).
		
		On teste donc $P_{10} \approx 14~762$, qui implique que $N > 10$.
		
		Ensuite, $P_{15} \approx 112~101$, qui implique que $N \leq 15$.
		
		Enfin, $P_{14} \approx 74~734$, qui implique que $N > 14$, et qui conclut que $N=15$.
	\end{enumerate}
}




\exe{}{
	À l'âge de 17 ans une élève décide de placer 200€ en bourse qui lui rapportent 10\% d'intérêts chaque année.
	Chaque année, elle replace les intérêts gagnés.
	
	On souhaite étudier l'évolution de l'argent placé chaque année après ses 17 ans inclus.
	\begin{enumerate}
		\item Vérifier les premières valeurs du tableau suivantes et le compléter.
			\begin{center}
			\begin{tabular}{|c|c|c|c|c|c|c|}\hline
				Âge & 17 & 18 & 19 & 20 & 21 & 22 \\ \hline
				Argent placé (€) & 200 & 220 & 242 &  & & \\ \hline
				$n$ & 0&1&2&3&4&5 \\ \hline
			\end{tabular}
			\end{center}
	\end{enumerate}
	On appelle $A_n$ la quantité d'argent placé à l'âge $17+n$, où $n\in \{0 ; 1 ; 2; \dots \}$.
	\begin{enumerate}[resume]
		\item
		Décrire comment obtenir $A_{n+1}$ en connaissant $A_n$.
		\item Écrire l'expression algébrique de $A_n$.	
		\item Combien d'argent aura l'élève à l'âge de $50$ ans ? 
		\item Calculer $A(50)$ et interpréter le résultat.
		\item À quel âge la somme d'argent dépassera-t-elle $100 \  000$€ ?		
	\end{enumerate}
 }{exe:interets}{
 
	\begin{center}
	\begin{tabular}{|c|c|c|c|c|c|c|}\hline
		Âge & 17 & 18 & 19 & 20 & 21 & 22 \\ \hline
		Argent placé (€) & 200 & 220 & 242 & \color{red} $266,2$  & \color{red} $292,82$ &\color{red} $322,102$   \\ \hline
		$n$ & 0&1&2&3&4&5 \\ \hline
	\end{tabular}
	\end{center}
			
	\begin{enumerate}
		\item Un gain de $10\%$ correspond à un coefficient multiplicateur de $1+\dfrac{10}{100} = 1,1$.
		\item 
		Pour obtenir $A_{n+1}$, on multiplie $A_n$ par $1,1$. On a donc la relation de récurrence suivante.
			\[ A_{n+1} = 1,1 \times A_n, \]
		valable pour tout $n\in\N$ entier naturel.
		\item 
		À l'âge de $50 = 17+n$ ans, on doit calculer $A_n$ pour $n=50-17 = 34$, et donc $A(34)$.
		
		On utilise l'expression algébrique de $A_n$ qui est
			\[ A_n = 200 \times (1,1)^n, \]
		pour trouver $A(34) \approx 5109,5.$
		\item $A_{50} = 200 \times (1,1)^{50} \approx 23~478,2$.
		C'est l'argent économisé à l'âge de $17+50 = 67$ ans.
		\item 
		On résoud à l'aide d'un tableau de valeurs ou par dichotomie.
		Comme $A_{65} < 100~000 < A_{66}$, le plus petit $n$ pour lequel $A_n$ dépasse $100~000$ est $66$.
		Pour $n=66$, l'âge correspondant est $17+66 = 83$ ans.
	\end{enumerate}
 
 }
 
 
% \exe{}{
% 
% 	On reprend l'exercice \ref{exe:interets} en prenant en plus en compte l'inflation, qu'on suppose constante à $2\%$ par an.
% 	Cela signifie que les prix augmentent en moyenne de $2\%$ par an.
% 	
% 	Afin de rendre comparables les sommes d'argent dans le temps, on souhaite fixer les prix : au lieu de voir les prix comme augmentant, on voit l'euro comme se dépréciant.
% 
%	\begin{enumerate}
%		\item Calculer l'évolution réciproque de $+2\%$. 
%		C'est la diminution qu'on appliquera à l'euro chaque année.
%		\item Vérifier les premières valeurs du tableau suivantes et le compléter.
%			\begin{center}
%			\begin{tabular}{|c|c|c|c|c|c|c|}\hline
%				Année & 17 & 18 & 19 & 20 & 21 & 22 \\ \hline
%				Argent placé (€, prix fixes) & 200 & 215,69 & 232,60 &&&  \\ \hline
%			\end{tabular}
%			\end{center}
%		\item On appelle $B_n$ la quantité d'argent placé à prix fixes en l'an $2017+n$, où $n\in \{0 ; 1 ; 2; \dots \}$.
%		Décrire comment obtenir $B_{n+1}$ en connaissant $B_n$
%		.%, c'est-à-dire comment obtenir la quantité d'argent placé en l'an $2017+_{n+1}$ en connaissant la quantité en $2017+n$.
%		
%		\item Calculer $B(70)$ et interpréter le résultat.
%	\end{enumerate}
% }{exe:interets2}{
% 
%			\begin{center}
%			\begin{tabular}{|c|c|c|c|c|c|c|}\hline
%				Année & 17 & 18 & 19 & 20 & 21 & 22 \\ \hline
%				Argent placé (€, prix fixes) & 200 & 215,69 & 232,60 &  \color{red} $250,84$  &  \color{red} $270,5$  &  \color{red} $291,7$   \\ \hline
%			\end{tabular}
%			\end{center}
% 	\begin{enumerate}
%		\item 
%		Le coefficient multiplicateur de l'augmentation de $2\%$ est $\times1,02$.
%		La réciproque est donnée par la multiplication par $\dfrac1{1,02} \approx 0,98 = 1 - 0,02$.
%		C'est donc une diminution de $2\%$ à appliquer après les intérêts chaque année.
%		\item 
%			Pour passer d'un terme à l'autre, on multiplie par $\dfrac{1,1}{1,02}$.
%		\item 
%			\[ B_{n+1} = \dfrac{1,1}{1,02} \times B_n. \]
%		
%		\item 
%			L'expression algébrique de $B_n$ est 
%				\[ B_n = 200 \times \left(\dfrac{1,1}{1,02}\right)^n, \]
%			qui donne $B(70) = 39~491,7$.
%			C'est l'argent à prix fixes accumulé après $17+70 = 87$ ans.
%	\end{enumerate}
% 
% 
% }
 
 
 
\exe{}{
	Pour chacune des suites données algébriquement pour tout $n\in\N$, décider si elle est géométrique ou non.
	\begin{multicols}{2}
	\begin{enumerate}
		\item $a_n = 3^n$
		\item $f_n = 3n + 2$
		\item $b_n = \left(\dfrac25\right)^n$
		\item $c_n = 5 \times 2^n$
		\item $g_n = 3-n$
		\item $h_n =  \dfrac3{n+1}$
	\end{enumerate}
	\end{multicols}

}{exe:geom-or-not}{
	\begin{enumerate}
		\item 
		$a$ est géométrique car elle respecte le théorème du cours avec $a_0 = 1$ et $q=3$, car $3^n = 3^n \times 1$.
		On peut également utiliser la définition d'une suite géométrique et que 
			\[ a_{n+1} = 3^{n+1} = 3^{1} \times 3^{n} = 3 \times a_n. \]
		\item 
		Supposons que $f$ soit géométrique non nulle.
		Alors le ratio $\dfrac{f_{n+1}}{f_n} = q$, et est donc constant quelque soit $n$.
		
		En $n=0$, on calcule
			\[\dfrac{f_1}{f_0} = \dfrac52. \]
		En $n=1$, on calcule
			\[\dfrac{f_2}{f_1} = \dfrac85. \]
		Comme $\dfrac52 \neq \dfrac85$, la suite $f$ ne peut pas être géométrique (elle est arithmétique en fait).
		
		\item $b_n = \left(\dfrac25\right)^n \times 1$, donc elle est géométrique.
		\item $c_n = 5 \times 2^n$ est géométrique.
		\item On calcule deux ratios successifs.
			\begin{align*}
				\dfrac{g_1}{g_0} = \dfrac23, && \neq &&  \dfrac{g_2}{g_1} = \dfrac12,
			\end{align*}
		ce qui implique que $g$ ne peut pas être géométrique.
		\item On calcule deux ratios successifs.
			\begin{align*}
				\dfrac{h_1}{h_0} = \dfrac12, &&  \neq && \dfrac{h_2}{h_1} = \dfrac23,
			\end{align*}
		ce qui implique que $h$ ne peut pas être géométrique.
	\end{enumerate}

}


%%%%%%%%%%%%

\newpage
\fancyhead[C]{\textbf{Solutions}}
\shipoutAnswer

\end{document}
