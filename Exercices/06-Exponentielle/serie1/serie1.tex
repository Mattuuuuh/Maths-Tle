\documentclass[14pt]{beamer}
\usepackage[french]{babel}

\usetheme{CambridgeUS}
\usecolortheme{rose}
\beamertemplatenavigationsymbolsempty


\usepackage{libertinus}
\usepackage{amsmath,amsfonts,amsthm,amssymb,mathtools}
\usepackage{array}
\newcolumntype{P}[1]{>{\centering\arraybackslash}p{#1}}


\usepackage{stackengine}
\newcommand\xrowht[2][0]{\addstackgap[.5\dimexpr#2\relax]{\vphantom{#1}}}


% corps
\usepackage{calrsfs}
\newcommand{\C}{\mathcal{C}}
\newcommand{\R}{\mathbb{R}}
\newcommand{\Rnn}{\mathbb{R}^{2n}}
\newcommand{\Z}{\mathbb{Z}}
\newcommand{\N}{\mathbb{N}}
\newcommand{\Q}{\mathbb{Q}}

% domain
\newcommand{\D}{\mathcal{D}}


% date
\usepackage{advdate}
\AdvanceDate[0]

%plots
\usepackage{pgfplots, subcaption}
\definecolor{myg}{RGB}{56, 140, 70}
\definecolor{myb}{RGB}{45, 111, 177}
\definecolor{myr}{RGB}{199, 68, 64}

%boxes
\usepackage[most]{tcolorbox}
\usepackage{multicol}

%icomma
\usepackage{icomma}

%https://osl.ugr.es/CTAN/macros/latex/contrib/tcolorbox/tcolorbox.pdf
\newtcolorbox{mybox}[3][]
{
  colframe = #2!25,
  colback  = #2!10,
  coltitle = #2!20!black,  
  halign title=flush center, 
  title    = {#3},
  #1,
}

% BOX A BOX B
\newcommand{\boxAB}[2]{
		\begin{mybox}{red}{A}
		\begin{center}
			#1
		\end{center}
		\end{mybox}
		\begin{mybox}{green}{B}
		\begin{center}
			#2
		\end{center}
		\end{mybox}
}

%systèmes
\usepackage{systeme}

% trafficotage 
\usepackage[answerdelayed, lastexercise]{exercise}
\renewcommand{\ExerciseHeader}{
}
\renewcommand{\AnswerHeader}{
}

\newcommand{\framedelayed}[3][]{
	\begin{Exercise}
	\begin{frame}{\theExercise #1\vspace{-32pt}}
		#2
	\end{frame}
	\end{Exercise}
	\begin{Answer}
	\begin{frame}{\theExercise #1\vspace{-32pt}}
		#3
	\end{frame}
	\end{Answer}
}


\SetDate[21/11/2025]

\begin{document}
\pagestyle{fancy}
\fancyhead[L]{Tle STMG}
\fancyhead[C]{\textbf{Fonctions exponentielles}}
\fancyhead[R]{\today}

\exe{}{
	Une suite géométrique $u$ vérifie $u_0 = 1$ et $u_2 = 4$.
	Peut-on déduire la raison $q$ de la suite ?
}{exe:1}{
	On sait que $u_1 = q \times u_0$ et que $u_2 = q \times u_1 = q \times q \times u_0 = q^2 \times u_0$.
	Il suit que $4 = q^2$ et donc que $q = 2$ ou $q=-2$.
}


\exe{}{
	Une suite géométrique $u$ vérifie $u_0 = 2$ et $u_2 = 8$.
	Peut-on déduire la raison $q$ de la suite ?
}{exe:2}{
	De la même façon, $u_2 = q^2 \times u_0$, d'où $8 = 2 q^2$ et $q^2 = 4$.
	On trouve donc les deux mêmes solutions, $q=2$ ou $q=-2$.
}

\exe{}{
	Une suite géométrique $u$ vérifie $u_0 = 1$ et $u_2 = 15$.
	Peut-on déduire la raison $q$ de la suite ?
}{exe:3}{
	Similairement, $1 \times q^2 = 15$, et donc $q = \sqrt{15} \approx 3,87$ ou $q = - \sqrt{15}$.
	Remarquons que $15^{1/2} = \sqrt{15}$.
}

\exe{}{
	Une suite géométrique $u$ vérifie $u_0 = 2$ et $u_5 = 60$.
	Peut-on déduire la raison $q$ de la suite ?
}{exe:4}{
	Ici, $2q^5 = 60$ et donc $q^5 = 30$.
	Pour résoudre cela, on a besoin de la puissance fractionnaire : $q = 30^{1/5} \approx 1,97$.
}



\exe{}{
	Calculer les valeurs suivantes avec la calculatrice. Arrondir au centième.
	\begin{multicols}{4}
	\begin{enumerate}[label=(\alph*)]
		\item $2^3$
		\item $8^{1/3}$
		\item $3^4$
		\item $81^{1/4}$
		\item $8^{2/3}$
		\item $2^{-1}$
		\item $8^{-2/3}$
		\item $27^{-2/3}$
		\item $30^{1/5}$
		\item $30^{1/2}$
		\item $\dfrac{1}{30^{-1/2}}$
		\item $2^{\pi}$
	\end{enumerate}
	\end{multicols}
}{}{
	\begin{multicols}{4}
	\begin{enumerate}[label=(\alph*)]
		\item $2^3 = 8$
		\item $8^{1/3} = 2$
		\item $3^4 = 81$
		\item $81^{1/4} = 3$
		\item $8^{2/3} = 4$
		\item $2^{-1} = 0,5$
		\item $8^{-2/3} =  0,25$
		\item $27^{-2/3} \approx 0,11 $
		\item $30^{1/5} \approx 1,97$
		\item $30^{1/2} \approx 5,48$
		\item $\dfrac{1}{30^{-1/2}} \approx 5,48$
		\item $2^{\pi} \approx 8,82$
	\end{enumerate}
	\end{multicols}
}

\exe{}{
	Le nombre de visiteurs du Musée du Louvre au cours des années $2015$ à $2018$ sont donnés ci-dessous.
	\begin{center}
	\begin{tikzpicture}[scale=.8]
		% nodes
		\draw (0,0) ellipse (2cm and .5cm) node {$8,6$ millions};
		
		\draw (5,0) ellipse (2cm and .5cm) node {$7,3$ millions};
		
		\draw (10,0) ellipse (2cm and .5cm) node {$8,1$ millions};
		
		\draw (15,0) ellipse (2cm and .5cm) node {$10,2$ millions};
		
		% vertices
		\draw[->, thick, myg] (1cm,.6cm) arc (105:75:7) node[above, midway] {$\times q$};
		\draw[->, thick, myg] (6cm,.6cm) arc (105:75:7) node[above, midway] {$\times q$};
		\draw[->, thick, myg] (11cm,.6cm) arc (105:75:7) node[above, midway] {$\times q$};
		
		\draw[->, thick, myr] (1cm,-.5cm) arc (-105:-75:25) node[below, midway] {$\times CM$};
	\end{tikzpicture}
	\end{center}

	\begin{enumerate}
		\item 
		Trouver le coefficient multplicateur $CM$ et la raison $q$ telle que le schéma ci-dessus soit correct.
		Arrondir au centième.
		\item 
		En déduire le taux d'évolution moyen du nombre de visiteurs du Musée du Louvre.
	\end{enumerate}

}{exe:louvre1}{
	\begin{enumerate}
		\item
		On trouve $CM = \frac{10,2}{8,6} \approx 1,19$.
		Comme $q^3 = CM = 1,19$, il suit que $q = (1,19)^{1/3} \approx 1,06$.
		\item
		Le taux d'évolution moyen est le taux associé au coefficient multplicateur $q \approx 1,06$.
		C'est donc une augmentation de 6\%.
	\end{enumerate}
}

\exe{}{
	Le nombre de visiteurs du Musée du Louvre semble augmenter de $6\%$ chaque année.
	À l'année 0, on compte $8,6$ millions de visiteurs.

	\begin{enumerate}
		\item
		Écrire $V_n$, le nombre de millions de visiteurs du musée après $n$ années, où $n\in\N$ est un entier naturel.
		\item
		Écrire $V(x)$, le nombre de millions de visiteurs du musée à l'année $x$, où $x\in\R$ est un nombre réel (positif, négatif ou nul).
	\end{enumerate}
	Pour les questions suivantes, donner un nombre d'années non nécessairement entier, arrondi au centième.
	\begin{enumerate}[resume]
		\item 
		À partir de quand le nombre de visiteurs dépassera 13,3 millions ? 
		\item 
		À partir de quand le nombre de visiteurs dépassera 14,78 millions ?
		\item 
		Quand le nombre de visiteurs a-t-il dépassé 7 millions ?
	\end{enumerate}

}{exe:louvre2}{
	Le coefficient multiplicateur associé à une augmentation de 6\% est 1,06 : c'est la raison de la suite géométrique $V$.
	\begin{enumerate}
		\item
		$V_n = V_0 \times q^n = 8,6 \times 1,06^n$
		\item
		$V(x) = 8,6 \times 1,06^x$.
		\item
		On cherche le nombre $x\in\R$ vérifiant $V(x) = 13,3$.
		En explorant les valeurs, on trouve $x\approx 7,5$, soit 7 ans et demi.
		Le chapitre du logarithme permettra de résoudre cette équation rapidement sans tester des valeurs.
		\item
		On cherche le nombre $x\in\R$ vérifiant $V(x) = 14,78$.
		En explorant les valeurs, on trouve $x\approx 9,3$, soit 9 ans et 3 mois (et 15 jours).
		\item
		On cherche le nombre $x\in\R$ vérifiant $V(x) = 7$.
		Il faut explorer dans les valeurs négatives ici : $x\approx-3,5$ fonctionne, soit 3 ans et demi avant l'année 0.
	\end{enumerate}
		
}



%%%%%%%%%%%%

\newpage
\fancyhead[C]{\textbf{Solutions}}
\shipoutAnswer

\end{document}
