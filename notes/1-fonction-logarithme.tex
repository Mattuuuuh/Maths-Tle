%!TEX encoding = UTF8
%!TEX root = 0-notes.tex

\chapter{Fonction logarithme}
\label{chap:log}

\section{De la complexité d'ajouter deux nombres}


\section{Logarithme en base 10}

Le logarithme de base 10 compte le nombre de chiffres nécessaires à exprimer un nombre en base 2.

\section{Variations et problèmes de seuil}

\exe{1}{
	Résoudre algébriquement les problèmes de seuils de l'exercice \ref{exe:seuil-geom}.
}{exe:seuil-geom2}{
	TODO
}

\section{Complexités}

Le logarithme de base 2 compte le nombre de chiffres nécessaires à exprimer un nombre en base 2.
Comme tous les logarithmes sont multiples l'un de l'autre, ils sont de même ordre, et on notera $\log$ sans spécifier la base.

La complexité d'un algorithme se calcule en fonction de la taille de l'entrée.
Ainsi le programme \texttt{isPrime(N)} s'exécute en temps linéaire $\O(N)$.
Cependant, l'entrée $E$ n'est pas $N$ mais l'encodage binaire de $N$, de taille $E = \O(\log N)$.
L'algorithme est donc exponentiel d'ordre $\O(2^E)$.

\qs{}{
	P = NP ?
}