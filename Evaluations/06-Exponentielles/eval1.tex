\documentclass[14pt]{beamer}
\usepackage[french]{babel}

\usetheme{CambridgeUS}
\usecolortheme{rose}
\beamertemplatenavigationsymbolsempty


\usepackage{libertinus}
\usepackage{amsmath,amsfonts,amsthm,amssymb,mathtools}
\usepackage{array}
\newcolumntype{P}[1]{>{\centering\arraybackslash}p{#1}}


\usepackage{stackengine}
\newcommand\xrowht[2][0]{\addstackgap[.5\dimexpr#2\relax]{\vphantom{#1}}}


% corps
\usepackage{calrsfs}
\newcommand{\C}{\mathcal{C}}
\newcommand{\R}{\mathbb{R}}
\newcommand{\Rnn}{\mathbb{R}^{2n}}
\newcommand{\Z}{\mathbb{Z}}
\newcommand{\N}{\mathbb{N}}
\newcommand{\Q}{\mathbb{Q}}

% domain
\newcommand{\D}{\mathcal{D}}


% date
\usepackage{advdate}
\AdvanceDate[0]

%plots
\usepackage{pgfplots, subcaption}
\definecolor{myg}{RGB}{56, 140, 70}
\definecolor{myb}{RGB}{45, 111, 177}
\definecolor{myr}{RGB}{199, 68, 64}

%boxes
\usepackage[most]{tcolorbox}
\usepackage{multicol}

%icomma
\usepackage{icomma}

%https://osl.ugr.es/CTAN/macros/latex/contrib/tcolorbox/tcolorbox.pdf
\newtcolorbox{mybox}[3][]
{
  colframe = #2!25,
  colback  = #2!10,
  coltitle = #2!20!black,  
  halign title=flush center, 
  title    = {#3},
  #1,
}

% BOX A BOX B
\newcommand{\boxAB}[2]{
		\begin{mybox}{red}{A}
		\begin{center}
			#1
		\end{center}
		\end{mybox}
		\begin{mybox}{green}{B}
		\begin{center}
			#2
		\end{center}
		\end{mybox}
}

%systèmes
\usepackage{systeme}

% trafficotage 
\usepackage[answerdelayed, lastexercise]{exercise}
\renewcommand{\ExerciseHeader}{
}
\renewcommand{\AnswerHeader}{
}

\newcommand{\framedelayed}[3][]{
	\begin{Exercise}
	\begin{frame}{\theExercise #1\vspace{-32pt}}
		#2
	\end{frame}
	\end{Exercise}
	\begin{Answer}
	\begin{frame}{\theExercise #1\vspace{-32pt}}
		#3
	\end{frame}
	\end{Answer}
}


\SetDate[05/12/2025]

\reversemarginpar
\setlength{\marginparsep}{.5cm}

\begin{document}
\pagestyle{fancy}
\fancyhead[L]{Tle STMG}
\fancyhead[C]{\textbf{Évaluation blanche \\ Fonctions exponentielles}}
\fancyhead[R]{\today}

%\null\vspace{-30pt}
Consignes particulières : 
\begin{itemize}[label=$\bullet$]
	\item 
	La calculatrice est {autorisée}.
	\item
	L'évaluation fait 1 page. La somme des points est \total{points}.
\end{itemize}

\marginpar{[pts]}
\hrule




\exe{5}{
	Exprimer les nombre suivants sous la forme $10^x$ pour un réel $x\in\R$.
	
	\begin{multicols}{3}
	\begin{enumerate}
		\item $10^{1,14} \times 10^{2}$
		\item $\left(10^4\right)^6$
		\item $\dfrac{10^4}{10^{7}}$
		\item $\dfrac{10^{-3}}{10^{0}}$
		\item $\dfrac{10^2}{10^{2}}$
		\item $\dfrac{1}{10^{4}}$
		\item $\left(\dfrac1{10^2}\right)^6$
		\item $\dfrac{1}{10^{-4}}$
		\item $\dfrac{10^{-5}}{10^{6}}$
		\item $\dfrac{10^{12}}{10^{-12}}$
	\end{enumerate}
	\end{multicols}
}{exe:1}{
	\begin{multicols}{2}
	\begin{enumerate}
		\item $10^{3,14}$
		\item $10^{24}$
		\item $10^{-3}$
		\item $10^{-3}$
		\item $10^{0}$
		\item $10^{-4}$
		\item $10^{-12}$
		\item $10^{4}$
		\item $10^{-11}$
		\item $10^{24}$
	\end{enumerate}
	\end{multicols}
}


\exe{6, difficulty=2}{
	Soient $f, g, h$ des fonctions exponentielles. Déterminer la base $q$ et la valeur en $0$ de chaque fonction avec les informations suivantes.
		\begin{enumerate}[label=\roman*)]
			\item $f(4) = 8$ et $f(6) = 32$.
			\item $g(-3) = 15$ et $g(-4) = 5$.
			\item $g(-1) = 2$ et $g(1,5) = 54$.
		\end{enumerate}
}{exe:2}{
	Utilisons la relation
		\[ g(x_2) = g(x_1) \times q^{x_2 - x_1} \]
	pour d'abord trouver $q$, puis
		\[ g(x) = g(0) \times q^x \]
	pour déduire $g(0)$.
	\begin{enumerate}[label=\roman*)]
		\item 
		D'abord,
			\begin{align*}
				f(6) &= f(4) \times q^{6-4} \\
				32 &= 8 \times q^2 \\
				\dfrac{32}{8} &= q^2 \\
				q^2 &= 4 \\
				q &= 4^{1/2} = 2.
			\end{align*}
		Ensuite, 
			\begin{align*}
				f(4) &= f(0) \times q^{4} \\
				8 &= f(0) \times 2^4 \\
				f(0) &= \dfrac{8}{2^4} = \dfrac12 = 0,5.
			\end{align*}
		En conclusion, $f(x) = 0,5 \times 2^x$.
		\item
		D'abord,
			\begin{align*}
				g(-3) &= f(-4) \times q^{-3-(-4)} \\
				15 &= 5 \times q \\
				q &= \frac{15}5 = 3.
			\end{align*}
		Ensuite, 
			\begin{align*}
				g(-3) &= g(0) \times q^{-3} \\
				15 &= g(0) \times 3^{-3} \\
				g(0) &= \dfrac{15}{3^{-3}} = 405.
			\end{align*}
		En conclusion, $g(x) = 405 \times 3^x$.
		\item $g(-1) = 2$ et $g(1,5) = 54$.
		D'abord,
			\begin{align*}
				g(1,5) &= f(-1) \times q^{1,5-(-1)} \\
				54 &=2 \times q^{2,5} \\
				q^{2,5} &= \frac{54}2 \\
				q^{2,5} &= 27 \\
				q &= 27^{1/2,5} \approx 3,74
			\end{align*}
		Ensuite, 
			\begin{align*}
				g(-1) &= g(0) \times q^{-1} \\
				2 &= g(0) \times 3,74^{-1} \\
				g(0) &= \dfrac{15}{3,74^{-1}} = 7,46.
			\end{align*}
		En conclusion, $g(x) \approx 7,46 \times 3,74^x$.
	\end{enumerate}
}



\exe{3}{
	
	Le nombre de nuitées de touristes étrangers au land de Thuringe (Allemagne) au cours des années $2010$ à $2013$ sont donnés ci-dessous. \footnote{Source : \href{https://statistik.thueringen.de/analysen/Aufsatz-06a-2021.pdf}{https://statistik.thueringen.de/analysen/Aufsatz-06a-2021.pdf}, Abbildung 16, page 47.}
	
	\begin{center}
	\vspace{.5cm}
	\begin{tikzpicture}[scale=.8]
		% nodes
		\draw (0,0) ellipse (2cm and .5cm) node {567 826};
		
		\draw (5,0) ellipse (2cm and .5cm) node {565 645};
		
		\draw (10,0) ellipse (2cm and .5cm) node {593 444 };
		
		\draw (15,0) ellipse (2cm and .5cm) node {563 236};
		
		% vertices
		\draw[->, thick, myg] (1cm,.6cm) arc (105:75:7);
		\draw[->, thick, myg] (6cm,.6cm) arc (105:75:7);
		\draw[->, thick, myg] (11cm,.6cm) arc (105:75:7);
		
		\draw[->, thick, myr] (1cm,-.5cm) arc (-105:-75:25);
	\end{tikzpicture}
	\vspace{.5cm}
	\end{center}

	\begin{enumerate}
		\item Compléter le schéma en ajoutant les coefficients multiplicateurs.
		\item Calculer le coefficient multiplicateur moyen.
		\item En déduire le taux d'évolution annuel moyen.
	\end{enumerate}

}{exe:3}{
	\begin{center}
	\begin{tikzpicture}[scale=.8]
		% nodes
		\draw (0,0) ellipse (2cm and .5cm) node {567 826};
		
		\draw (5,0) ellipse (2cm and .5cm) node {565 645};
		
		\draw (10,0) ellipse (2cm and .5cm) node {593 444 };
		
		\draw (15,0) ellipse (2cm and .5cm) node {563 236};
		
		% vertices
		\draw[->, thick, myg] (1cm,.6cm) arc (105:75:7) node[midway, above]{$\approx\times0,996$};
		\draw[->, thick, myg] (6cm,.6cm) arc (105:75:7) node[midway, above]{$\approx\times1,049$};
		\draw[->, thick, myg] (11cm,.6cm) arc (105:75:7) node[midway, above]{$\approx\times0,949$};
		
		\draw[->, thick, myr] (1cm,-.5cm) arc (-105:-75:25) node[midway, below]{$\approx\times0,992$};
	\end{tikzpicture}
	\end{center}

	\begin{enumerate}
		\item[]
		\item Le coefficient global est en rouge : $0,992$.
		D'après le cours, le coefficient moyen est donné par $0,992^{1/3} \approx0,997$ car il y a trois évolutions.
		\item
		On calcule $0,997 - 1 = -0,003 = -0,3\%$.
		Le taux d'évolution annuel moyen est donc une diminution de $0,3\%$. 
	\end{enumerate}
}


\exe{6, difficulty=1}{
	Le \emph{nombre de reproduction de base} d'un virus est le nombre moyen de personnes saines qu'un personne contagieuse infecte. Il permet de décrire l'évolution d'une maladie à ses débuts (avant que l'immunité collective fasse effet).
	
	On estime que le nombre de reproduction de base d'un virus est de $3$ : chaque semaine, le nombre de malades est multiplié par $3$.
	On compte, au temps $0$, $M(0) = 18\ 347$ malades.

	\begin{enumerate}
		\item Écrire $M(n)$, le nombre de malades après $n$ semaines, où $n\in\N$ est un entier naturel.
		\item Écrire $M(t)$, le nombre de malades au temps $t$ (en semaines). $t$ peut être négatif.
		\item Combien de malades y aura-t-il après $2,5$ semaines ? Arrondir à $10^{-1}$.
		\item Combien de malades y avait-t-il il y a $3$ semaines ? Arrondir à $10^{-1}$.
		\item À partir de combien de semaines le nombre de malades dépassera $70$ millions ? Arrondir à  $10^{-1}$.
		\item En combien de semaines le premier malade a-t-il infecté $18\ 347$ personnes ? On cherche le plus petit $t$ tel que $M(t) \geq 1$. Arrondir à  $10^{-1}$.
	\end{enumerate}

}{exe:4}{
	\begin{enumerate}
		\item 
		$M$ est une suite géométrique de raison 3, donc
			\[ M(n) = M(0) \times q^n = 18~347\times3^n. \]
		\item 
		En remplaçant $n$ par $t$ dans l'expression ci-dessus, on obtient
			\[ M(t) = 18~347\times3^t. \]
		\item
		En calculant 
			\[M(2,5) = 18~347\times3^{2,5} \approx 286~001, \]
		on déduit qu'il y aura environ 286 001 malades après 2 semaines et demi.
		\item 
		En calculant 
			\[M(-3) = 18~347\times3^{-3} \approx 679, \]
		on déduit qu'il y avait environ 679 malades il y a 3 semaines.
		\item 
		On teste des valeurs. Déjà, $M(2,5)$ est trop petit.
		Ensuite,
			\[ M(8) = 120~374~667\]
		est trop grand (120 millions).
		De plus, 
			\[ M(7) = 40~124~889\]
		est trop petit (40 millions).
		Après 7 semaines et demi, 
			\[ M(7,5) \approx 70~000~000\]
		ce qui est la valeur recherchée.
		\item
		$M(-3)$ est trop grand, cherchons donc un $t$ inférieur à $-3$.
			\[ M(-10) \approx 0,3 \]
		qui est trop petit : le $t$ recherché se situe donc en $-10$ et $-3$.
			\[ M(-9) \approx 0,93 \]
		est très proche. En cherchant autour, on obtient
			\[ M(-8,9) \approx 1,04 \]
		qui nous suffit.
		Le premier malade a donc infecté les 18 347 personnes en 8,9 semaines.
	\end{enumerate}
}



%%%%%%%%%%%%

\newpage
\fancyhead[C]{\textbf{Solutions}}
\shipoutAnswer

\end{document}
