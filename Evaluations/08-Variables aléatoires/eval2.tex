\documentclass[14pt]{beamer}
\usepackage[french]{babel}

\usetheme{CambridgeUS}
\usecolortheme{rose}
\beamertemplatenavigationsymbolsempty


\usepackage{libertinus}
\usepackage{amsmath,amsfonts,amsthm,amssymb,mathtools}
\usepackage{array}
\newcolumntype{P}[1]{>{\centering\arraybackslash}p{#1}}


\usepackage{stackengine}
\newcommand\xrowht[2][0]{\addstackgap[.5\dimexpr#2\relax]{\vphantom{#1}}}


% corps
\usepackage{calrsfs}
\newcommand{\C}{\mathcal{C}}
\newcommand{\R}{\mathbb{R}}
\newcommand{\Rnn}{\mathbb{R}^{2n}}
\newcommand{\Z}{\mathbb{Z}}
\newcommand{\N}{\mathbb{N}}
\newcommand{\Q}{\mathbb{Q}}

% domain
\newcommand{\D}{\mathcal{D}}


% date
\usepackage{advdate}
\AdvanceDate[0]

%plots
\usepackage{pgfplots, subcaption}
\definecolor{myg}{RGB}{56, 140, 70}
\definecolor{myb}{RGB}{45, 111, 177}
\definecolor{myr}{RGB}{199, 68, 64}

%boxes
\usepackage[most]{tcolorbox}
\usepackage{multicol}

%icomma
\usepackage{icomma}

%https://osl.ugr.es/CTAN/macros/latex/contrib/tcolorbox/tcolorbox.pdf
\newtcolorbox{mybox}[3][]
{
  colframe = #2!25,
  colback  = #2!10,
  coltitle = #2!20!black,  
  halign title=flush center, 
  title    = {#3},
  #1,
}

% BOX A BOX B
\newcommand{\boxAB}[2]{
		\begin{mybox}{red}{A}
		\begin{center}
			#1
		\end{center}
		\end{mybox}
		\begin{mybox}{green}{B}
		\begin{center}
			#2
		\end{center}
		\end{mybox}
}

%systèmes
\usepackage{systeme}

% trafficotage 
\usepackage[answerdelayed, lastexercise]{exercise}
\renewcommand{\ExerciseHeader}{
}
\renewcommand{\AnswerHeader}{
}

\newcommand{\framedelayed}[3][]{
	\begin{Exercise}
	\begin{frame}{\theExercise #1\vspace{-32pt}}
		#2
	\end{frame}
	\end{Exercise}
	\begin{Answer}
	\begin{frame}{\theExercise #1\vspace{-32pt}}
		#3
	\end{frame}
	\end{Answer}
}


\SetDate[11/03/2026]

\reversemarginpar
\setlength{\marginparsep}{.5cm}

\begin{document}
\pagestyle{fancy}
\fancyhead[L]{Tle STMG}
\fancyhead[C]{\textbf{Évaluation — Variables aléatoires}}
\fancyhead[R]{\today}

%\null\vspace{-30pt}
Consignes particulières : 
\begin{itemize}[label=$\bullet$]
	\item 
	La calculatrice est {autorisée}.
	\item
	L'évaluation fait \pageref{lastpage} page. La somme des points est \total{points}.
\end{itemize}

\marginpar{[pts]}
\hrule


\exe{7}{
	Princesse lance un dé à $6$ faces équilibré puis regarde le numéro de la face du dessus (un entier entre 1 et 6).
	\begin{enumerate}[label=$\bullet$]
		\item Si le numéro obtenu est 1 ou 2, elle extrait au hasard une boule dans l'urne 1 qui contient 3 boules noires, 4 boules blanches et 3 boules rouges
		\item Sinon, elle extrait une boule dans l'urne 2 qui contient 3 boules noires et 2 boules blanches.
	\end{enumerate}
	En lisant les règles du jeu, Princesse décide de parier sur la boule noire : si elle tire une boule noire, elle gagne 20€, et sinon, elle perd 20€.
	Notons $X$ le gain de Princesse après un tirage.
	\begin{enumerate}
		\item
		Faire un arbre modélisant l'expérience aléatoire.
		\item
		Faire un tableau dans lequel à chaque issue de l'expérience on peut lire sa probabilité ainsi que la valeur de $X$ associée.
		\item
		Donner la loi de $X$ sous la forme d'un tableau dans lequel à chaque valeur de $X$ est associée sa probabilité.
		\item
		Quelle est l'espérance de gain de Princesse ?
		Comment interpréter ce résultat ?
	\end{enumerate}
}{exe:2}{

	L'arbre est construit en séparant les deux épreuves de l'expérience aléatoire : 
		\begin{enumerate}
			\item Jet de dé.
			\item Tirage dans urne.
		\end{enumerate}
	Le jet de dé admet 6 issues propres (1 à 6), mais il est utile de regrouper celles qui aboutissent à la même deuxième épreuve : un tirage dans la même urne.
	L'arbre suivante décrit donc deux issues à la première expérience.
	Par équiprobabilité (le dé est équilibré), la probabilité de tirer dans l'urne 1 est $\frac26$ et celle de tirer dans l'urne 2 est $\frac46$.
	
	Ensuite, pour chaque urne, et comme on ne s'intéresse qu'à la couleur de la boule tirée, chaque couleur est une issue.
	L'arbe est donc asymétrique car il y a 3 couleurs dans l'urne 1 et 2 dans l'urne 2.
	Les probabilités sont données par la fraction du nombre de boule d'une certaine couleur sur le nombre de boule dans l'urne en question.
	Par exemple, dans l'urne 1, la probabilité de choisir une boule noire est $\frac3{10}$.

	\begin{center}
	\begin{tikzpicture}
		% depth 1
		\foreach \i in {-3, 3}
		\draw[-, thick, black] (0,0) node {$\bullet$} -- (\i,-2);
		% depth 2
		\foreach \i in {-3, 3} \foreach \j in {-2, 2}
			\draw[-, thick, black] (\i,-2) node {$\bullet$} -- (\i+\j,-4) node {$\bullet$};
		\draw[-, thick, black] (-3,-2) -- (-3,-4) node {$\bullet$};
		
		\draw (-3,-2) node[above left] {Obtenir 1 ou 2};
		\draw (3,-2) node[above right] {Obtenir ni 1 ni 2};
			
		\draw (-5,-4) node[below] {Noire};
		\draw (-3,-4) node[below] {Blanche};
		\draw (-1,-4) node[below] {Rouge};
		\draw (1,-4) node[below] {Noire};
		\draw (5,-4) node[below] {Blanche};
		
		% sols
		\draw (-1.75,-.5) node {$\dfrac26$};
		\draw (1.75,-.5) node {$\dfrac46$};
		
		\draw (4.25,-2.5) node {$\dfrac25$};
		\draw (1.75,-2.5) node {$\dfrac35$};
		
		\draw (-1.75,-2.5) node {$\dfrac3{10}$};
		\draw (-2.5,-3.25) node {$\dfrac4{10}$};
		\draw (-4.25,-2.5) node {$\dfrac3{10}$};
	\end{tikzpicture}
	\end{center}
	
	Pour chaque issue (feuille de l'arbre), on inscrit dans un tableau la valeur de $X$ qui lui correspond (ici $+20$ ou $-20$) ainsi que sa probabilité : c'est le produit des probabilités de la racine vers la feuille.
	
	Par exemple, la probabilité de choisir une boule rouge est donnée par $\frac26\times\frac{3}{10} = \frac{6}{60} = \frac1{10}$.
	On rappelle que
		\[ \dfrac{a}{b} \times \dfrac{c}{d} = \dfrac{a\times c}{b\times d}. \]
	
	\begin{center}
	\begin{tabular}{|c|c|c|c|c|c|}\hline
		Valeur de $X$ & $20$ & -20 & -20 & $20$ & $-20$ \\ \hline
		Probabilité & $\dfrac1{10}$ & $\dfrac2{15}$ & $\dfrac1{10}$ & $\dfrac25$ & $\dfrac4{15}$ \\ \hline
	\end{tabular}
	\end{center}
	
	Pour écrire la loi de $X$, on ajoute les probabilités associées aux mêmes valeurs de $X$.
	Ici, les deux seules valeurs que prend $X$ sont $+20$ et $-20$.
	Ainsi, 
		\[P(X = -20) = \frac2{15}+\frac1{10}+\frac4{15} = \frac{4+3+8}{30} = \frac{15}{30} = \frac12, \]
	et 
		\[ P(X = +20) = \frac1{10} + \frac25 = \frac{1+4}{10} = \frac5{10} = \frac12.\]
	
	\begin{center}
	\begin{tabular}{|c|c|c|}\hline
		Valeur de $X$ & $+20$ & $-20$ \\ \hline
		Probabilité &  $\dfrac12$ & $\dfrac12$ \\ \hline
	\end{tabular}
	\end{center}
	
	L'espérence de $X$ est une moyenne pondérée (penser au calcul de la moyenne : somme des valeurs $\times$ coefficient) :
		\[ \E(X) = 20 \times \frac12 + (-20)\times\frac12 = 10-10 = 0. \]
	Après de nombreuses expériences, la moyenne de gain de Princesse est donc de 0 euros : le jeu est équilibré.
	Remarquons qu'obtenir $X=0$ n'est pas possible (soit Princesse perd 20€, soit elle gagne 20€) : l'espérance de $X$ traduit bien une moyenne au long terme sur plusieurs expériences, et non pas une valeur de $X$.
}

\exe{7}{
	\begin{enumerate}
		\item
		Créer un triangle de Pascal\footnotemark pour calculer $6\choose3$.
		\item
		Interpréter le nombre obtenu.
		\item
		Donner $340\choose1$ en justifiant à l'aide de l'interprétation du coefficient binomial.
		\item
		Étant donné que ${11\choose3} = 165$, déduire la valeur de ${11\choose8}$.
		Justifier à l'aide des interprétations des coefficients binomiaux.
	\end{enumerate}
}{exe:3}{
	En reprenant le triangle vu en exercice, rempli grâce aux propriétés
		\begin{align*}
			{n \choose k} &= {n-1 \choose k-1} + {n-1 \choose k}, \\
			{n \choose n } &= 1, \\
			{n \choose 0} &= 1,
		\end{align*}
	on obtient ${6 \choose 3} = 20$.
	
	Le nombre de façons d'obtenir 3 succès en 6 épreuves est égal à 20.
	C'est également le nombre de mots de 6 caractères sur l'alphabet \{ E ; S \} contenant exactement 3 S (par exemple SSSEEE, SSESEE, SSEESE, etc...).

	\begin{center}
	\begin{tabular}{>{$}c<{$}|*{8}{c}}
		\multicolumn{1}{l}{\thead{Nombre \\ d'épreuves}} &&&&&&&\\\cline{1-1} 
		1 &1&1\\
		2 &1&2&1\\
		3 &1&3&3&1\\
		4 &1&4&6&4&1\\
		5 &1&5&10&10&5&1\\
		6 &1&6&15&20&15&6&1&\\
		\multicolumn{1}{l}{} &0&1&2&3&4&5&6&7\\\cline{2-8}
		\multicolumn{1}{l}{} &\multicolumn{8}{c}{Nombre de succès}
	\end{tabular}
	\end{center}
	
	${340 \choose 1} = 340$ car c'est le nombre de façons d'obtenir exactement 1 succès parmi 340 épreuves : celui-ci arrive soit à la première épreuve, soit à la deuxième, etc..., soit à la 340ème.
	
	Comme ${11 \choose 3} = 165$, il existe 165 façons d'obtenir 3 succès parmi 11 épreuves.
	Cependant, ceci est égal à $11 \choose 8$, car obtenir 3 succès parmi 11 épreuves équivaut à obtenir 8 échecs parmis 11 épreuves ($3+8 = 11$).
	Il y a plusieurs façon de voir que compter les succès équivaut à compter les échecs :
		\begin{itemize}
			\item le triangle de Pascal est symétrique : on peut le lire de gauche à droite ou de droite à gauche ;
			\item en renversant l'arbre succès/échec, chaque succès devient un échec (et chaque tournant à gauche devient un tournant à droite) ;
			\item en échangeant les lettres S et E lors de l'énumération des possibilités, on voit que le nombre de mots à 11 caractères sur l'alphabet \{ S ; E \} avec 3 E est égal au nombre de mots avec 8 E ;
			\item et il en existe sûrement d'autres..!
		\end{itemize}
}

\footnotetext{Blaise Pascal (1623-1662), mathématicien, physicien et philosophe français.}
%\newpage


\exe{6}{
	Matthieu tape un énoncé sur son ordinateur. 
	Il estime que, pour chaque caractère, la probabilité qu'il appuie sur la bonne touche est $0,95$.
	
	\begin{enumerate}
		\item
		Donner la probabilité que Matthieu se trompe de touche pour un caractère quelconque.
		\item
		Donner la probabilité que Matthieu ne fasse aucune erreur lorsqu'il tape le mot \mbox{« apparemment »}.
		\item
		En déduire la probabilité qu'il fasse au moins une erreur en tapant « apparemment ».
		\item
		Calculer le nombre moyen d'erreurs commises en tapant l'énoncé de cet exercice.
		Celui-ci contient 515 caractères.
	\end{enumerate}
}{exe:4}{
	
	\begin{enumerate}
		\item
		D'après le texte, la probabilité d'un succès (appuyer sur la bonne touche) est égale à $p=0,95$.
		Celle d'un échec est donc $1-p = 0,05$.
		\item
		On compte $n=11$ lettres. La probabilité de ne faire aucune erreur est celle d'obtenir 11 succès parmi 11.
		Dans les notations du cours, $k=11$ et la probabilité de ne faire aucune erreur est donnée par
			\[ {n \choose k} \times p^k \times (1-p)^{n-k} = {11 \choose 11} \times 0,95^11 \times 0,05^0 = 0,95^11 \approx 0,57. \]
		\item
		L'événement contraire de « faire au moins une erreur » est « ne faire aucune erreur », dont la probabilité $0,57$ a été calculée à la question précédente.
		
		La probabilité de faire au moins une erreur est donc $1-0,57 = 0,43$.
		\item
		Avec les paramètres $n=515$ et $p=0,95$ l'espérance du nombre de succès est donné par
			\[ n \times p = 515 \times 0,95 = 489,25. \]
		Le nombre moyen de succès est donc $489,25$ et le nombre moyen d'erreurs commises est donc $515 - 489,25 = 25,75$.
		
		Remarquons qu'on aurait pû plutôt prendre $p=0,05$ en voyant un succès comme une erreur commise.
		Auquel cas nous aurions calculé $515 \times 0,05 = 25,75$ également !
	\end{enumerate}
}


%%%%%%%%%%%%

\label{lastpage}
\newpage
\fancyhead[C]{\textbf{Solutions}}
\shipoutAnswer

\end{document}
