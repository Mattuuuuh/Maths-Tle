\documentclass[14pt]{beamer}
\usepackage[french]{babel}

\usetheme{CambridgeUS}
\usecolortheme{rose}
\beamertemplatenavigationsymbolsempty


\usepackage{libertinus}
\usepackage{amsmath,amsfonts,amsthm,amssymb,mathtools}
\usepackage{array}
\newcolumntype{P}[1]{>{\centering\arraybackslash}p{#1}}


\usepackage{stackengine}
\newcommand\xrowht[2][0]{\addstackgap[.5\dimexpr#2\relax]{\vphantom{#1}}}


% corps
\usepackage{calrsfs}
\newcommand{\C}{\mathcal{C}}
\newcommand{\R}{\mathbb{R}}
\newcommand{\Rnn}{\mathbb{R}^{2n}}
\newcommand{\Z}{\mathbb{Z}}
\newcommand{\N}{\mathbb{N}}
\newcommand{\Q}{\mathbb{Q}}

% domain
\newcommand{\D}{\mathcal{D}}


% date
\usepackage{advdate}
\AdvanceDate[0]

%plots
\usepackage{pgfplots, subcaption}
\definecolor{myg}{RGB}{56, 140, 70}
\definecolor{myb}{RGB}{45, 111, 177}
\definecolor{myr}{RGB}{199, 68, 64}

%boxes
\usepackage[most]{tcolorbox}
\usepackage{multicol}

%icomma
\usepackage{icomma}

%https://osl.ugr.es/CTAN/macros/latex/contrib/tcolorbox/tcolorbox.pdf
\newtcolorbox{mybox}[3][]
{
  colframe = #2!25,
  colback  = #2!10,
  coltitle = #2!20!black,  
  halign title=flush center, 
  title    = {#3},
  #1,
}

% BOX A BOX B
\newcommand{\boxAB}[2]{
		\begin{mybox}{red}{A}
		\begin{center}
			#1
		\end{center}
		\end{mybox}
		\begin{mybox}{green}{B}
		\begin{center}
			#2
		\end{center}
		\end{mybox}
}

%systèmes
\usepackage{systeme}

% trafficotage 
\usepackage[answerdelayed, lastexercise]{exercise}
\renewcommand{\ExerciseHeader}{
}
\renewcommand{\AnswerHeader}{
}

\newcommand{\framedelayed}[3][]{
	\begin{Exercise}
	\begin{frame}{\theExercise #1\vspace{-32pt}}
		#2
	\end{frame}
	\end{Exercise}
	\begin{Answer}
	\begin{frame}{\theExercise #1\vspace{-32pt}}
		#3
	\end{frame}
	\end{Answer}
}


\SetDate[20/11/2025]

\reversemarginpar
\setlength{\marginparsep}{.5cm}

\begin{document}
\pagestyle{fancy}
\fancyhead[L]{Tle STMG}
\fancyhead[C]{\textbf{Évaluation \\ Suites arithmétiques et géométriques}}
\fancyhead[R]{\today}

%\null\vspace{-30pt}
Consignes particulières : 
\begin{itemize}[label=$\bullet$]
	\item 
	La calculatrice est {autorisée}.
	\item
	L'exercice \ref{exe:thms} peut être entièrement traité sur la feuille d'évaluation.
	Écrire son nom avant de rendre le sujet.
	\item
	L'évaluation fait 2 pages. La somme des points est \total{points}.
\end{itemize}

\marginpar{[pts]}
\hrule

\begin{thm}\label{thm:1}
	Soit $u$ une suite géométrique de raison $q$ et de terme initial $u_0$.
	Alors l'expression algébrique de $u$ est donnée par
		\[ u_n = \underline{\hspace{4cm}} \]
\end{thm}

\begin{thm}\label{thm:2}
	Soit $u$ une suite arithmétique de raison $a$ et de terme initial $u_0$.
	Alors l'expression algébrique de $u$ est donnée par
		\[ u_n = \underline{\hspace{4cm}} \]
\end{thm}

\exe{4}{
	Compléter les théorèmes \ref{thm:1} et \ref{thm:2} ci-dessus.
}{exe:thms}{
	Si $u$ est géométrique, le cours donne
		\[ u_n = u_0 \times q^n. \]
	Si $u$ est arithmétique, le cours donne
		\[ u_n = u_0 + a \times n. \]
}

\exe{6}{
	Soit $u$ une suite géométrique de raison 1,5 et de 4ème terme $u_3 = 10,8$.
	
	\begin{enumerate}
		\item Calculer $u_4$ et $u_5$.
		\item Calculer $u_2$ et $u_1$.
		\item Montrer que le terme initial est donné par $u_0 = 3,2$.
		\item En déduire l'expression algébrique de $u_n$.
		\item Calculer la somme 
			\[ u_0 + u_1 + u_2 + \cdots + u_{50}. \]
	\end{enumerate}
}{exe:suites-geom}{
	Comme $u$ est géométrique, on multiplie par sa raison 1,5 pour passer d'un terme au suivant.
	Pour remonter en arrière, on divise par la raison.
	\begin{enumerate}
		\item $u_4 = u_3 \times 1,5 = 16,2$ et $u_5 = u_4 \times 1,5 = 24,3$.
		\item $u_2 = \dfrac{u_3}{1,5} = 7,2$ et $u_1 = \dfrac{u_2}{1,5} = 4,8$.
		\item On a bien $u_0 = \dfrac{u_1}{1,5} = 3,2$.
		\item D'après le cours, $u_n = 3,2 \times 1,5^n$.
		\item $u_{50} = 3,2 \times 1,5^{50} \approx 2~040~388~801$.
	\end{enumerate}
}

\exe{5}{
	Après ingestion de 500mg de paracétamol, celui-ci est décomposé dans l'organisme au cours du temps.
	Chaque heure, $10\%$ de la quantité de médicament encore présente dans l'organisme est absorbée, et il n'en reste donc que $90\%$.
	\begin{center}
	\begin{tabular}{|c|c|c|c|c|c|}\hline
		Heure & 0 & 1 & 2 & 3 & 4 \\ \hline
		Quantité de médicament restante (mg) & 500 & 450 & & & \\ \hline
	\end{tabular}
	\end{center}

	\begin{enumerate}
		\item Vérifier les premières valeurs du tableau ci-dessus et le compléter.
		\item En notant $M_n$ la quantité de médicament restante à l'heure $n$ en mg, écrire l'expression algébrique de $M_n$.
		\item Donner $M_{24}$ en arrondissant à l'entier le plus proche.
		Que représente cette quantité ? Décrire avec des mots.
		\item Trouver le plus petit rang $N\in\N$ tel que
			\[ M_N \leq 250.\]
		Interpréter le résultat : que représente ce rang $N$ ?
	\end{enumerate}
}{exe:paracetamol}{
	\begin{center}
	\begin{tabular}{|c|c|c|c|c|c|}\hline
		Heure & 0 & 1 & 2 & 3 & 4 \\ \hline
		Quantité de médicament restante (mg) & 500 & 450 & 405 & 364,5 & 328,05 \\ \hline
	\end{tabular}
	\end{center}

	\begin{enumerate}
		\item
		On multiplie par $0,9$ pour passer d'un terme à l'autre du tableau : c'est la raison $q$ de la suite géométrique.
		\item 
		D'après le théorème \ref{thm:1},
			\[ M_n = M_0 \times q^n = 500 \times 0,9^n. \]
		\item 
			\[ M_{24} = 500 \times 0,9^{24} \approx 40 \]
		Il reste donc 40mg de médicament encore présents dans l'organisme après 24 heures.
		\item 
		On trouve $N = 7$, car $M_7 \approx 240 \leq 250$, et $M_6 \approx 265 > 250$.
		$N$ représente l'heure à laquelle la moitié du médicament aura été absorbée.
	\end{enumerate}
}



%\exe{4}{
%	Soit $u$ une suite arithmétique de raison 4 et de 5ème terme $u_4 = 17$.
%	
%	\begin{enumerate}
%		\item Calculer $u_5$ et $u_6$.
%		\item Calculer $u_3$ et $u_2$.
%		\item Montrer que le terme initial est donné par $u_0 = 1$.
%		\item En déduire l'expression algébrique de $u_n$.
%		\item Calculer la somme 
%			\[ u_0 + u_1 + u_2 + \cdots + u_{50}. \]
%	\end{enumerate}
%}{exe:suites-arithm}{
%	Comme $u$ est arithmétique, on ajoute sa raison 4 pour passer d'un terme au suivant.
%	Pour remonter en arrière, on soustrait la raison.
%	\begin{enumerate}
%		\item $u_5 = u_4 + 4 = 21$ et $u_6 = u_5 + 4 = 25$.
%		\item $u_3 = u_4 - 4 = 13$ et $u_2 = u_3 - 4 = 9$.
%		\item $u_1 = u_2 - 4 = 5$ et donc $u_0 = u_1 - 4 = 1$.
%		\item D'après le cours, $u_n = 4n + 1$.
%		\item D'après le cours, on calcule $u_{50} = 201$, et on en déduit que
%			\[ \text{(nb termes)}\times\text{(moyenne premier et dernier termes)} = 51 \times \dfrac{1+201}2 = 5~151. \]
%	\end{enumerate}
%}

\newpage

\exe{5}{
	On étudie la température d'un plat sorti du four.
	
	Au début de l'expérience, le plat est à 215°C.
	Toutes les 30 secondes, on mesure sa température.
	Les premiers résultats sont notés dans le tableau ci-dessous.
	
	\begin{center}
	\begin{tabular}{|c|c|}\hline
		{Temps écoulé  (en minutes)} & {Température (°C)} \\ \hline
		0 & 215 \\\hline
		0,5 & 210 \\\hline
		1 & 205 \\\hline
		1,5 & 200 \\\hline
		2 & 195 \\ \hline
	\end{tabular}
	\end{center}
	
	Soit $n$ un entier naturel.
	On note $u_n$ la température du plat après $n$ périodes de 30 secondes.
	Ainsi, $u_0 = 215 ; u_1 = 210 ; u_2 = 205 ; \dots$.
	\begin{enumerate}
		\item
		Justifier que les termes $u_0, u_1, u_2, u_3$ sont en progression arithmétique.
		Quelle est la raison ?
		\item 
		En supposant que la température continue d'évoluer de façon arithmétique, donner l'expression algébrique de $u_n$.
		\item 
		En combien de temps le plat atteindra-t-il une température de 25°C ? 
		Donner un résultat en minutes.
		
		\item
		Le modèle arithmétique de la température est-il réaliste ? 
	\end{enumerate}
}{exe:four}{
	\begin{enumerate}
		\item
		On calcule $u_1 - u_0 = -5, u_2 - u_1 = -5,$ et $u_3 - u_2 = -5$.
		Pour passer d'un terme à l'autre, on soustrait 5 : la progression est arithmétique de raison -5.
		\item 
		On a $u_n = 215 - 5n$ d'après le cours.
		\item
		On pose $u_n = 25$ et on résoud pour $n$.
		\begin{align*}
			u_n &= 25 \\
			215 - 5n&= 25 \\
			-5n &= -190 \\
			n &= \frac{-190}{-5} = \frac{190}5 \\
			n &= \frac{380}{10} = 38
		\end{align*}
		Après 38 périodes de 30 secondes, la température du plat est de 25°C.
		En minutes, cela fait $38 \times \frac12 = 19$ minutes.
		
		\item
		Le modèle n'est pas réaliste car la température modèle continue de diminuer indéfiniment.
		D'après celui-ci, en 50 minutes, la température du plat sera $u(100) = -500 + 215 = -385$, ce qui n'est pas une température possible.
	\end{enumerate}
}


%
%\exe{3}{
%	Considérons la suite $u$ définie algébriquement par
%		\[ u_n = n 3^n + 1. \]
%	\begin{enumerate}
%		\item
%		Calculer les trois premiers termes de $u$.
%		\item
%		La suite est-elle arithmétique ? géométrique ? Justifier.
%	\end{enumerate}
%}{exe:nature}{
%	todo
%}



%%%%%%%%%%%%

\newpage
\fancyhead[C]{\textbf{Solutions}}
\shipoutAnswer

\end{document}
