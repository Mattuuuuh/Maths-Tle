\documentclass[14pt]{beamer}
\usepackage[french]{babel}

\usetheme{CambridgeUS}
\usecolortheme{rose}
\beamertemplatenavigationsymbolsempty


\usepackage{libertinus}
\usepackage{amsmath,amsfonts,amsthm,amssymb,mathtools}
\usepackage{array}
\newcolumntype{P}[1]{>{\centering\arraybackslash}p{#1}}


\usepackage{stackengine}
\newcommand\xrowht[2][0]{\addstackgap[.5\dimexpr#2\relax]{\vphantom{#1}}}


% corps
\usepackage{calrsfs}
\newcommand{\C}{\mathcal{C}}
\newcommand{\R}{\mathbb{R}}
\newcommand{\Rnn}{\mathbb{R}^{2n}}
\newcommand{\Z}{\mathbb{Z}}
\newcommand{\N}{\mathbb{N}}
\newcommand{\Q}{\mathbb{Q}}

% domain
\newcommand{\D}{\mathcal{D}}


% date
\usepackage{advdate}
\AdvanceDate[0]

%plots
\usepackage{pgfplots, subcaption}
\definecolor{myg}{RGB}{56, 140, 70}
\definecolor{myb}{RGB}{45, 111, 177}
\definecolor{myr}{RGB}{199, 68, 64}

%boxes
\usepackage[most]{tcolorbox}
\usepackage{multicol}

%icomma
\usepackage{icomma}

%https://osl.ugr.es/CTAN/macros/latex/contrib/tcolorbox/tcolorbox.pdf
\newtcolorbox{mybox}[3][]
{
  colframe = #2!25,
  colback  = #2!10,
  coltitle = #2!20!black,  
  halign title=flush center, 
  title    = {#3},
  #1,
}

% BOX A BOX B
\newcommand{\boxAB}[2]{
		\begin{mybox}{red}{A}
		\begin{center}
			#1
		\end{center}
		\end{mybox}
		\begin{mybox}{green}{B}
		\begin{center}
			#2
		\end{center}
		\end{mybox}
}

%systèmes
\usepackage{systeme}

% trafficotage 
\usepackage[answerdelayed, lastexercise]{exercise}
\renewcommand{\ExerciseHeader}{
}
\renewcommand{\AnswerHeader}{
}

\newcommand{\framedelayed}[3][]{
	\begin{Exercise}
	\begin{frame}{\theExercise #1\vspace{-32pt}}
		#2
	\end{frame}
	\end{Exercise}
	\begin{Answer}
	\begin{frame}{\theExercise #1\vspace{-32pt}}
		#3
	\end{frame}
	\end{Answer}
}


\AdvanceDate[1]

\begin{document}
\pagestyle{fancy}
\fancyhead[L]{Tle STMG}
\fancyhead[C]{\textbf{Évaluation — Suites, sommes, et complexités}}
\fancyhead[R]{\today}

Consigne particulière : 
\begin{itemize}[label=$\bullet$]
	\item 
	La calculatrice est {interdite}.
\end{itemize}

\hrule

\exe{}{
	Quelles sont les opérations arithmétiques élémentaires dans le modèle RAM ?
	
	Donner une opération mathématique qui n'est pas une opération élémentaire.
}{aa}{}

\exe{}{
	Calculer les 4 premiers termes de la suite donnée algébriquement par 
		\[ u_n = 6n+1. \]
}{exe:1}{
	En remplaçant $n$ par 0 ; 1 ; 2 ; et 3, on trouve $u_0 = 1, u_1 = 7, u_2 = 13, u_3 = 19$.
}

\exe{}{
	Le professeur de Carl Freidrich Gauss\footnotemark lui demande de calculer la somme
		\[ 1 + 2 + 3 + \cdots + 100. \]
	Donner sa valeur.
}{}{}

\footnotetext{1777-1855, mathématicien. astronome, et physicien allemand.}

\exe{, difficulty=1}{
	En fonction de $n$, dire combien d'opérations élémentaires et d'espace sont nécessaires pour calculer naïvement la somme des puissances de trois,
		\[ S_n = 3^0 + 3^1 + 3^2 + \cdots + 3^n. \] 
	On ne s'intéresse qu'à l'ordre.
}{exe:complexite-pow3}{
}

\exe{, difficulty=1}{
	Dans cet exercice, on souhaite diminuer le nombre d'opérations nécessaires au calcul de la somme $S_n$ de l'exercice \ref{exe:complexite-pow3} en trouvant une expression algébrique.
	\begin{enumerate}
		\item
		Écrire $S_n$ sous la forme $S_n = \sum\limits_{\dots}^{\dots} \dots$.
		\item($\star$)
		Montrer que
			\[ 3 S_n = S_n - 1 + 3^{n+1}. \]
		\item
		En déduire que $S_n = \dfrac{3^{n+1}-1}{2}$.
		\item
		Conclure sur le nombre d'opérations élémentaires requises pour calculer $S_n$. 
		On ne s'intéresse qu'à l'ordre.
	\end{enumerate}
		
}{bb}{}




\exe{}{
	Calculer les 5 premiers termes de la suite définie récursivement par
		\[ \begin{cases*} v_0 = 0, \\ v_{n+1} = v_n + 12. \end{cases*} \]
	Donner une conjecture sur sa forme algébrique.
}{exe:suite-recursive}{
	L'expression récursive se lit de la façon suivante : pour obtenir le terme de rang $n+1$, il faut ajouter 12 au terme de rang $n$.
	Le terme initial (de rang 0) étant 0, on a : $v_0 = 0, v_1 = 12, v_2 = 24, v_3 = 36, v_4 = 48$.
	
	On conjecture que $v_n = 12n$ pour tout $n\in\N$.
}

\exe{}{
	Une procédure effectue boucle sur $k$ allant de 1 à $n$.
	Chaque exécution de la boucle effectue $\O(k^3)$ opérations élémentaires.
	
	Combien d'opération élémentaires sont effectuée au total par la procédure ?
}{cc}{}


%%%%%%%%%%%%

\newpage
\fancyhead[C]{\textbf{Solutions}}
\shipoutAnswer

\end{document}
