\documentclass[14pt]{beamer}
\usepackage[french]{babel}

\usetheme{CambridgeUS}
\usecolortheme{rose}
\beamertemplatenavigationsymbolsempty


\usepackage{libertinus}
\usepackage{amsmath,amsfonts,amsthm,amssymb,mathtools}
\usepackage{array}
\newcolumntype{P}[1]{>{\centering\arraybackslash}p{#1}}


\usepackage{stackengine}
\newcommand\xrowht[2][0]{\addstackgap[.5\dimexpr#2\relax]{\vphantom{#1}}}


% corps
\usepackage{calrsfs}
\newcommand{\C}{\mathcal{C}}
\newcommand{\R}{\mathbb{R}}
\newcommand{\Rnn}{\mathbb{R}^{2n}}
\newcommand{\Z}{\mathbb{Z}}
\newcommand{\N}{\mathbb{N}}
\newcommand{\Q}{\mathbb{Q}}

% domain
\newcommand{\D}{\mathcal{D}}


% date
\usepackage{advdate}
\AdvanceDate[0]

%plots
\usepackage{pgfplots, subcaption}
\definecolor{myg}{RGB}{56, 140, 70}
\definecolor{myb}{RGB}{45, 111, 177}
\definecolor{myr}{RGB}{199, 68, 64}

%boxes
\usepackage[most]{tcolorbox}
\usepackage{multicol}

%icomma
\usepackage{icomma}

%https://osl.ugr.es/CTAN/macros/latex/contrib/tcolorbox/tcolorbox.pdf
\newtcolorbox{mybox}[3][]
{
  colframe = #2!25,
  colback  = #2!10,
  coltitle = #2!20!black,  
  halign title=flush center, 
  title    = {#3},
  #1,
}

% BOX A BOX B
\newcommand{\boxAB}[2]{
		\begin{mybox}{red}{A}
		\begin{center}
			#1
		\end{center}
		\end{mybox}
		\begin{mybox}{green}{B}
		\begin{center}
			#2
		\end{center}
		\end{mybox}
}

%systèmes
\usepackage{systeme}

% trafficotage 
\usepackage[answerdelayed, lastexercise]{exercise}
\renewcommand{\ExerciseHeader}{
}
\renewcommand{\AnswerHeader}{
}

\newcommand{\framedelayed}[3][]{
	\begin{Exercise}
	\begin{frame}{\theExercise #1\vspace{-32pt}}
		#2
	\end{frame}
	\end{Exercise}
	\begin{Answer}
	\begin{frame}{\theExercise #1\vspace{-32pt}}
		#3
	\end{frame}
	\end{Answer}
}


\SetDate[05/11/2025]

\begin{document}
\pagestyle{fancy}
\fancyhead[L]{Tle STMG}
\fancyhead[C]{\textbf{Suites géométriques 2}}
\fancyhead[R]{\today}



\exe{}{
	Un ami vous propose de créer une entreprise dont le principe est le suivant :
	\begin{itemize}
		\item Vous recrutez des ``investisseurs'' en leurs proposant de verser 1 000€
		et en leur promettant qu'ils doubleront leur mise après une semaine. 
		\item Pour payer le premier investisseur après une semaine, vous recrutez deux nouveaux investisseurs.
		\item Vous poursuivez ainsi, en recrutant toujours deux nouveaux investisseurs pour en payer un.
	\end{itemize}
	On note $u_n$ le nombre de nouveaux investisseurs recrutés en semaine numéro $n$. On suppose que l'entreprise démarre en semaine 0 avec un seul investisseur.
	\begin{enumerate}
		\item Calculer $u_0$, $u_1$ et $u_2$.
		\item Quelle est la nature de la suite $u$ ? Justifier.
		\item Estimer le nombre d'investisseurs nécessaires pour que l'entreprise fonctionne sur une année entière (52 semaines). 
		Le plan de votre ami semble-t-il possible ? Expliquer.
	\end{enumerate}
	\textit{Ce type de montage se nomme « système de Ponzi ».}
}{exe:5}{
	\begin{enumerate}
		\item 
		$u_0 = 1, u_1 = 2, u_2 = 4$.
		\item 
		$u$ est géométrique de raison 2 car pour passer d'un terme au suivant, on multiplie par 2.
		\item 
		On calcule la somme avec le cours :
			\[ u_0 + u_1 + \cdots + u_{52} = \sum_{k=0}^{52} u_k = \sum_{k=0}^{52} 2^k = \dfrac{2^{53}-1}{2-1} = 2^{53} - 1. \] 
		Comme $2^{53} \approx 9\times10^{15}$ et qu'il existe moins de 9 milliards ($9\times10^{9}$) d'humains sur Terre, le plan n'a pas l'air très crédible.
	\end{enumerate}
}


\begin{multicols}{2}
\setlength\columnseprule{.1pt} 

\exe{}{
	Les suites suivantes données graphiquement peuvent-elles être géométriques ?
	
	\begin{center}
	\begin{tikzpicture}[>=stealth, scale=1]
		\begin{axis}[xmin = 0, xmax=4.2, xtick={ 0,1,2, 3, 4,5}, ymin=0, ymax=300, ytick={0, 30, ..., 300}, axis x line=middle, axis y line=middle, axis line style=->, ylabel={}, grid=both, extra x ticks = {0}]
			
			\addplot[black, thick, only marks, mark=star] coordinates {(0, 130) (1,140) (2,150) (3,160) (4,170)};
			
			\addplot[black, thick, only marks, mark=square] coordinates {(0,15) (1,30) (2,60) (3, 120) (4, 240)};
			
			\addplot[black, thick, only marks, mark=*] coordinates {(1,300) (2,200) (3,133.33) (4, 88.89)};
		\end{axis}
	
	\end{tikzpicture}
	\end{center}

}{exe:graph1}{
	On calcule les ratios successifs : s'ils ne sont pas constants, alors la suite n'est pas géométrique.
	Dans le cas contraire, on ne peut pas réellement conclure que la suite est géométrique, car nous n'avons qu'un échantillons restreint des images (et donc des ratios).
	
	Seule la suite $\star$ peut être écartée ici.
}

\exe{}{
	Donner le terme de rang $n \in \N$ des suites géométriques $\star$, $\bullet$, et $\square$ données graphiquement.

	\begin{center}
	\begin{tikzpicture}[>=stealth, scale=1]
		\begin{axis}[xmin = 0, xmax=4.2, xtick={ 0,1,2, 3, 4,5}, ymin=0, ymax=10000, ymode=log, log ticks with fixed point, axis x line=middle, axis y line=middle, axis line style=->, ylabel={}, grid=both, extra x ticks = {0}]
			
			\addplot[black, thick, only marks, mark=star] coordinates {(0, 1) (1,10) (2,100) (3,1000) (4,10000)};
			
			\addplot[black, thick, only marks, mark=square] coordinates {(0,10 000) (1,10 00) (2,100) (3,10) (4,1)};
			
			\addplot[black, thick, only marks, mark=*] coordinates {(0,3) (1,30) (2,300) (3, 3000)};
		\end{axis}
	
	\end{tikzpicture}
	\end{center}

}{exe:graph2}{
	Dans ce repère à échelle logarithmique, la première graduation des ordonnées est $1$.
	On obtient donc
		\begin{align*}
			\star_n = 1 \times 10^n, && \square_n = 10~000\times \left(\dfrac1{10}\right)^n, && \bullet_n = 3 \times 10^n
		\end{align*} 
}

\end{multicols}


\exe{, difficulty=1}{
	Lors de votre entretien d'embauche, une entreprise vous propose un salaire de départ de 35 000€ avec deux choix d'évolutions possibles pour ce salaire
	
	\begin{enumerate}[start=0,label={\bfseries Choix~\arabic*:},leftmargin=3cm]
		\item une augmentation annuelle de 2\% ;
		\item une augmentation annuelle de 7 00€.
	\end{enumerate}
	
	\begin{enumerate}
		\item Calculer le salaire (arrondi à l'euro près) à la 20ème année pour chacun des choix.
		\item Quel choix est, selon vous, le plus avantageux ? Expliquer.
		\item Selon les statistiques de l'OCDE, les français restent en moyenne 11 ans dans une même entreprise. En tenant compte de cette information, quel choix est probablement le plus avantageux ?
	\end{enumerate}
}{exe:exo9}{
	\begin{enumerate}
		\item
		Pour le premier choix : la suite est géométrique de raison $1,02$ et de terme initial 35 000.
		Ainsi, après 20 ans, le salaire est donné par
			\[ 35~000\times1,02^{20} \approx 52~008. \]
		Pour le second choix, la suite est arithmétique de raison 700 et de terme initial 35 000.
		Ainsi, après 20 ans, le salaire est donné par
			\[ 35~000 + 20 \times 700 = 49~000. \] 
		\item 
		Le premier choix semble donc plus avantageux après 20 ans d'ancienneté.
		\item
		On effectue les mêmes calculs sur 11 ans.
		Le premier choix donne
			\[ 35~000\times1,02^{11} \approx 43~518, \]
		et le deuxième choix
			\[ 35~000 + 11 \times 700 = 42~700. \] 
		Le premier choix est donc également plus avantageux.
	\end{enumerate}
}

\newpage
\exe{, difficulty=2}{
	Un élève choisit d'investir 1 000€ chaque année en bourse.
	À la fin de chaque année, l'argent investi génère 5\% d'intérêts. Ces intérêts sont réinvestis en bourse en plus des 1 000€.
	
	Notons $u_n$ l'argent placé (en comptant les intérêts) à l'année $n \in \N$.
	Ainsi, 
	\begin{itemize}
		\item $u_0 = 1 000$, car 1 000€ sont placés à l'année 0, sans intérêts.
		\item $u_1 = 2 050$, car 1 000€ sont ajoutés à l'année 1, en plus des 50€ d'intérêts sur l'année passée.
		\item $u_2 = 3 152,5$, car 1 000€ sont ajoutés à l'année 2, en plus des 102,5€ d'intérêts sur l'année passée.
	\end{itemize}
		
	\begin{enumerate}
		\item Décrire comment $u_{n+1}$ est calculé à partir de $u_n$.
	\end{enumerate}
	On suppose que la suite $u$ vérifie la relation suivante.
		\[
		\begin{cases}
		u_0 = 1~000, \\
		u_{n+1} = 1,05u_n + 1~000.
		\end{cases}
		\]
	
	\begin{enumerate}[resume]
		\item \label{q2}
		Calculer $u_0 ; u_1 ; u_2 ; u_3 ; u_4 ; u_5$ grâce à la relation de récurrence de $u$.
		\item
		Est-ce que $u$ est une suite arithmétique ? géométrique ?
	\end{enumerate}
	Le but des questions suivantes est de trouver une expression algébrique pour $u_n$ en trouvant une suite intermédiaire $w$ qui, elle, est géométrique.
	
	\begin{enumerate}[resume]
		\item 
		Résoudre l'équation suivante pour $x$.
			\[ x = 1,05 x + 1~000 \]
		\item 
		Définissons la suite intermédiaire $w$ par
			\[ w_n = u_n + 20~000. \]
		Calculer $w_0 ; w_1 ; w_2 ; w_3 ; w_4 ; w_5$ grâce à la question \ref{q2}.
		
		La suite $w$ est-elle arithmétique ? géométrique ? 
		Donner son terme initial et sa raison.
		\item 
		Donner l'expression algébrique de $w_n$ à l'aide du cours et conclure que
			\[ u_n = 1,05^n \times 21~000 - 20~000.\]
		\item
		Combien d'argent l'élève peut-il retirer après 60 ans d'investissements continus ?
		Combien d'argent brut a-t-il investi pendant ces 60 années (on ne compte pas les intérêts réinvestis) ?
		
		Décrire l'évolution de l'argent brut investi vers l'argent qu'il peut retirer après 60 ans (augmentation ou diminution, et de quel pourcentage).
	\end{enumerate}
}{exe:arithmetico-geom}{
	\begin{enumerate}
		\item 
		Pour passer d'un terme au suivant, on multiplie par $1,05$ (augmentation de 5\%), puis on ajoute 1 000.
		Ainsi, $u_{n+1} = 1,05 u_n + 1000$.
		\item 
		Avec la donnée initiale $u_0 = 1000$ et la relation de récurrence, on calcule
			\begin{align*}
				u_1 &= 2050, \\
				u_2 &= 3152,5 \\
				u_3 &= 4310,125 \\
				u_4 &= 5525,63125 \\
				u_5 &= 6801,9128125
			\end{align*}
		\item
		$u_1 - u_0 = 1050$ et $u_2 - u_1 = 1102,5$, donc $u$ n'est pas arithmétique.
		
		$\frac{u_1}{u_0} = 2,05$ et $\frac{u_2}{u_1} \approx 1,53$, donc $u$ n'est pas géométrique non plus.
		\item 
			On soustrait $x$ et $1000$ pour obtenir $-1000 = 0,05 x$.
			On divise par $0,05$ pour avoir $x = -20~000$.
		\item 
		Il suffit d'ajouter 20 000 aux termes calculés précédemment.
			\begin{align*}
				w_0 &= 21~000, \\
				w_1 &= 22~050, \\
				w_2 &= 23~152,5 \\
				w_3 &= 24~310,125 \\
				w_4 &= 25~525,63125 \\
				w_5 &= 26~801,9128125
			\end{align*}
		La suite $w$ n'est toujours pas arithmétique, mais les rapports successifs donnent
		$\frac{w_1}{w_0} = 1,05$, $\frac{w_2}{w_1} = 1,05$, etc...
		
		La suite $w$ est donc géométrique de raison $1,05$ et de terme initial $w_0 = 21~000$.
		\item 
		D'après le cours, $w_n = 21~000 \times 1,05^n$.
		Comme $w_n = u_n + 20~000$, on obtient bien
			\[ u_n = w_n - 20~000 =  1,05^n \times 21~000 - 20~000. \]
		\item
		À $n=60$, on calcule $u_{60} \approx 372~263$€ investis au total.
		
		Chaque année, l'investissement brut étant de 1 000€, l'investissement total après 60 ans est de 60 000€.
		
		Comme 
			\[ \dfrac{372~263}{60~000} \approx 6,2, \]
		l'évolution a un coefficient multiplicateur de $6,2$ : c'est une augmentation de $520\%$.
	\end{enumerate}
}



%%%%%%%%%%%%

\newpage
\fancyhead[C]{\textbf{Solutions}}
\shipoutAnswer

\end{document}
