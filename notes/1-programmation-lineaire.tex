%!TEX encoding = UTF8
%!TEX root = 0-notes.tex

\chapter{Programmation linéaire}

\section{Introduction}
	From DisOpt midterm 2019
	
	A factory produces two different products. To create one unit of product 1, it needs one unit of
	raw material A and one unit of raw material B. To create one unit of product 2, it needs one unit
	of raw material B and two units of raw material C. Raw material B needs preprocessing before it
	can be used, which takes one minute per unit. At most 20 hours of time is available per day for
	the preprocessing. Raw materials of capacity at most 1200 can be delivered to the factory per day.
	One unit of raw material A, B and C has size 4, 3 and 2 respectively.
	At most 130 units of the first and 100 units of the second product can be sold per day. The first
	product sells for 6 CHF per unit and the second one for 9 CHF per unit.
	Formulate the problem of maximizing turnover as a linear program in two variables.


\section{Théorie}

Dans la suite, on supposera que $a, b, c\in\R$ sont des réels fixés tels que $a$ et $b$ ne sont pas tous les deux nuls.

\dfn{équation cartésienne de droite}{
	Une \emph{équation cartésienne de droite} est une équation de la forme
		\begin{align}
			ax + by = c. \label{eq:cart}
		\end{align}
	L'ensemble des couples $(x ; y)$ vérifiant l'équation forme une droite :
		\begin{align}
			(d) = \bigset{ (x ; y) \tq ax + by = c }. \label{eq:droite}
		\end{align}
}{dfn:equation-cartesienne}

\exe{1}{
	Tracer la droite d'équation $2x + 3y = 0$ et le vecteur $\pvec23$ dans un même repère.
	Que remarque-t-on ?
}{exe:axbyz}{
	TODO
}

\exe{}{
	Tracer la droite d'équation $4x - y = 0$ et le vecteur $\pvec4{-1}$ dans un même repère.
	Que remarque-t-on ?
}{exe:axbyz2}{
	TODO
}

\notations{
	On dit de deux vecteurs $u, v$ qu'ils sont perpendiculaires dès que les droites supportées par $u$ sont perpendiculaires aux droites supportées par $v$.
	On pourra choisir les droites passant par l'origine données par $(OU)$ et $(OV)$ où $U = O + u$ et $V = O + v$.
}

\thm{}{
	Un couple (x ; y) vérifie $ax+by=0$ si et seulement si les vecteurs $\pvec{x}{y}$ et $\pvec{a}{b}$ sont perpendiculaires.
	\[ ax+by=0 \iff \pvec{x}{y} \perp \pvec{a}{b}. \]
}{thm:ab-orth}

\pf{}{
	On considère le triangle $OAX$ où $O(0;0)$, $A(a;b)$ et $X(x ; y)$.
	Les vecteurs $\pvec{x}{y}$ et $\pvec{a}{b}$ sont perpendiculaires si et seulement si les droites $(OA)$ et $(OX)$ le sont.
	
	D'après le théorème de Pythagore (et sa réciproque), on a
		\[ (OA) \perp (OX) \iff OA^2 + OX^2 = AB^2. \]
	Or $OA^2 = a^2 + b^2$, $OB^2 = x^2 + y^2$, et $AB^2 = (a-x)^2 + (b-y)^2$.
	L'égalité de Pythagore devient donc
		\begin{align*}
			(OA) \perp (OX) &\iff a^2 + b^2 + x^2 + y^2 = (a-x)^2 + (b-y)^2, \\
				&\iff a^2 + b^2 + x^2 + y^2 = a^2 + x^2 - 2ax + b^2 + y^2 - 2by, \\
				&\iff 0 = -2(ax + by), \\
				&\iff ax + by = 0.
		\end{align*}
}

\exe{}{
	Tracer les droites d'équations $2x - y = 0$, $2x - y = -2$, et $2x - y = 3$ dans un même repère.
	Que remarque-t-on ?
}{exe:parallélisme}{
	TODO
}

\exe{, difficulty = 1}{
	Montrer que les équations $ax+by =0$ et $ax+by=c$ sont parallèles, quel que soit $c\in\R$.	
}{exe:parallélisme2}{
	Si $b=0$, les droites sont verticales et parallèles.
	
	Sinon, leur équations réduites sont $y = \dfrac{-a}{b}x$ et $y = \dfrac{-a}{b}x + \dfrac{c}b$, de même coefficient directeur.
	Les droites sont donc parallèles dans ce cas aussi.
}

\thm{}{
	Considérons un point $P(x_P;y_P)$ vérifiant $ax_P + bx_P = c$.
	
	Alors la droite d'équation $ax+by = c$ est la droite perpendiculaire au vecteur $\pvec{a}{b}$ et passant par $P$.
}{thm:shift}

\pf{}{
	La droite passe bien sûr par $P$ par construction de celle-ci.
	
	De plus, elle est parallèle à $ax+by=0$ d'après l'exercice \ref{exe:parallélisme2}, et donc perpendiculaire à $\pvec{a}{b}$ d'après le théorème \ref{thm:ab-orth}.
	% mieux argumenter pour que ça soit généralisable plus facilement en 3d ?
}

\thm{}{
	Considérons la droite $(OA)$ où $A(a;b)$.
	
	Alors la forme linéaire $ax+by$ sur $(OA)$ est croissante dans la direction $\vec{OA}$, nulle en $O$, et décroissante dans la direction $\vec{AO}$.
	En particulier, pour n'importe quel $c\in\R$, il existe un point $P(x_P;y_P)$ de cette droite vérifiant
		\[ ax_P + by_P = c. \]
}{thm:allc}

\pf{}{
	On rappelle que la droite $(OA)$ est la droite passant par $O$ et supportée par le vecteur $\vec{OA} = \pvec{a}b$.
	Graphiquement, pour obtenir $(OA)$, on translate l'origine $O$ par tous les multiples du vecteur $\vec{OA}$.
	C'est donc l'ensemble des translatés $O + k\pvec{a}b = (ka ; kb)$, où $k$ parcourt $\R$.

	Le point $(x_P ; y_P) = (ka ; kb)$, parcourant $(OA)$ et passant par l'origine, vérifie
		\[ ax_P + by_P = k(a^2 + b^2). \]	
	Comme $a$ et $b$ ne sont pas tous les deux nuls, $a^2 + b^2 >0$. 
	L'expression $ax_P + by_P$ est ainsi une fonction affine croissante en $k$ : quand $k$ grandit, $ax_P + by_P$ aussi.
	
	Pour tout $c\in\R$, choisir $k = \dfrac{c}{a^2 + b^2}$ permet bien d'obtenir $ax_P + by_P = c$.
}

\exe{}{
	Tracer la droite $(OA)$ où $A(2, -1)$. Calculer les coordonées des points $(x;y)$ de $(OA)$ d'abscisse $-2, -1, 0, 1,$ et 2, et les placer sur la droite.
	Pour chacun des points $(x;y)$ obtenus, calculer $2x -y$.
}{exe:uscalaru}{
	TODO
}

\exe{}{
	\begin{multicols}{2}	
	On considère l'ensemble $P$ donné graphiquement ci-contre.
	Quel point $(x;y)$ de $P$ choisir pour maximiser $f(x,y) = 2x+y$ ? \\
	Et pour minimiser $f$ ?
	
	\includegraphics[page=1]{figures/fig-prog-lin.pdf}
	\end{multicols}
	
}{exe:max-on-convex}{
	TODO
}

\exe{}{
	Pour quel point $(x;y)$ de la droite $(OA)$, où $A(2, -1)$, a-t-on $2x - y = 5$ ? et $2x - y = -10$ ?
}{exe:uscalaru2}{
	$(x;y)$ est de la forme $kA = (2k ; -k)$ avec $k\in\R$.
	Le couple vérifie $2x-y=5$ si et seulement si $2(2k) -(-k) = 5 \iff 5k = 5 \iff k=1$.
	Le point recherché est donc $A(2; -1)$ lui-même.
	
	Le couple vérifie $2x-y=-10$ si et seulement si $2(2k) -(-k) = -10 \iff 5k = -10 \iff k=-2$.
	Le point recherché est donc $(-4; 2)$.
}

\thm{}{
	L'ensemble $\bigset{ ax+by \leq c }$ est un demi-plan de frontière la droite d'équation $ax+by=c$.
}{thm:demiplan}

\pf{}{
	En se restreignant à la droite $(OA)$ où $A(a;b)$, la forme linéaire $ax+by$ est une fonction affine.
	L'inégalité $ax+by \leq c$ scinde donc la droite en deux parties : les points la vérifiant, et les autres.
	En outre, chaque point de $(OA)$ vérifiant l'inéquation donne naissance à une droite perpendiculaire à $(OA)$ qui vérifie également l'inégalité, d'après le théorème \ref{thm:shift}.
}

\exe{}{
	Tracer, sans calcul, la droite d'équation $3x+y = 10$.
	Hachurer ensuite le demi-plan $3x+y \leq 10$.
}{exe:sanscalcul}{

}

\exe{}{
	Hachurer les demi-plans $2x - y \leq 5$, $x + y \leq 2$, $x\geq0$, et $y\geq-3$.
	Dans un repère vierge, hachurer l'intersection de ces quatre demi-plans.
}{exe:sanscalcul2}{

}

\dfn{polyèdre, polytope}{
	On appelle \emph{polyèdre} l'intersection de demi-plans.
	On dit que le polyèdre est \emph{borné} si les coordonnées de ses points sont bornées en valeur absolue par un certain nombre (possiblement très grand).
	Un polyèdre borné est appelé un \emph{polytope}.
}{dfn:polyhedre}

\exe{}{
	Le polygone $P$ suivant est-il un polytope ? Une démonstration rigoureuse n'est pas demandée : c'est l'objet de l'exercice \ref{exe:not-convex-proof}.
	
	\centering
	\includegraphics[page=2]{figures/fig-prog-lin.pdf}
	
}{exe:not-convex}{
	TODO
}

\exe{, difficulty = 2}{
	Montrer qu'un demi-plan est convexe : si $A, B$ sont deux points lui appartenant, alors le milieu du segment $[AB]$, $\dfrac{A+B}2$, lui appartient aussi.
	
	Montrer ensuite que l'intersection de deux convexes est convexe, et en déduire qu'un polyèdre est toujours convexe. 
	
	Montrer finalement que le polygone de l'exercice \ref{exe:not-convex} n'est pas convexe.
}{exe:not-convex-proof}{
	Pour tout demi-plan $ax+by \leq c$, et toute paire de points $(x_0,y_0), (x_1,y_1)$ lui appartenant,
		\begin{align*}
			ax_0+by_0 \leq c, && \text{ et } && ax_1 + by_1 \leq c.
		\end{align*}
	En multipliant par $1/2$ chaque inéquation et en les ajoutant ensemble, on trouve
		\[ a\dfrac{x_0 + x_1}2 + b \dfrac{y_0+y_1}2 \leq \dfrac12 c + \dfrac12c = c. \]
	Il suit que le milieu du segment appartient bien au demi-plan aussi.
	
	Soient désormais deux ensembles convexes, $C_1$ et $C_2$, d'intersection $C_1 \cap C_2$.
	Pour tous $A, B \in C_1 \cap C_2$, $A, B \in C_1$ et $A, B\in C_2$.
	Par convexité, le milieu $M = \dfrac{A+B}2$ appartient à $C_1$ et $C_2$, et donc également à leur intersection.
	Tout polyèdre, comme intersection de $m$ demi-plans convexes, est donc convexe.
	
	Le polyèdre de l'exercice \ref{exe:not-convex} n'est pas convexe : on peut trouver $A$ et $B$ deux points de $P$ tels que le milieu de $[AB]$ n'appartiennen pas à $P$.
	
	\centering
	\includegraphics[page=3]{figures/fig-prog-lin.pdf}
}

\exe{, difficulty = 2}{
	Soit $f(x,y) = ax+by$ une forme linéaire. 
	Montrer que si $[AB]$ est un segment de milieu $M$, alors
		\[ f(M) = \dfrac{f(A) + f(B)}2. \]
	En déduire, à l'aide de l'exercice \ref{exe:not-convex-proof}, que si $f$ atteint son \emph{unique} maximum en un point $C$ d'un polyèdre $P$, alors $C$ n'est pas le milieu d'un segment de $P$.
}{exe:max-at-vertex}{
	On a d'abord $f(x_M, y_M) = ax_M + by_M = a\dfrac{x_A+x_B}2 + b\dfrac{y_A+y_B}2 = \dfrac{f(A) + f(B)}2$.
	
	Supposons que $f$ atteigne son unique maximum en $C$, et supposons que $C$ est le milieu d'un intervalle $[AB]$ de $P$.
	Alors, par l'unicité du maximum, on a nécessairement $f(A) < f(C)$ et $f(B) < f(C)$.
	Or $f(C)$ est la moyenne de $f(A)$ et de $f(B)$, ce qui est contradictoire : 
		\[ f(C) = \dfrac{f(A)+f(B)}2 < \dfrac{f(C) + f(C)}2 = f(C). \qquad \text{\Large\Lightning} \]
}

\exe{}{
	Hachurer le polytope $P = \bigset{ 2\geq x\geq0, 2\leq y\leq3 }$ dans un repère.
	Tracer la droite $(OA)$ où $A(2;3)$. En déduire le point de $P$ maximisant $2x+3y$.
}{exe:nonborné}{
	TODO
}

\exe{}{
	Montrer que le polyèdre $P = \bigset{ x\leq0, -3\leq y\leq0 }$ n'est pas un polytope.
}{exe:nonborné}{
	$P$ n'est pas borné car on peut toujours trouver $(x,y)\in P$ avec $|x|$ aussi grand que souhaité.
}

\thm{programmation linéaire, admis}{
	Soit $P$ un polytope et $f$ une forme linéaire : $f(x,y) = ax+by$.
	
	Alors, le maximum de $f$ restreint à $P$ est atteint en un des sommets de $P$.
	
	De plus, les sommets de $P$ sont nécessairement à l'intersection de deux droites frontières de deux demi-plans définissant $P$.
}{thm:programmation-linéaire}

\pf{}{
Voir exercices \ref{exe:not-convex-proof} et \ref{exe:max-at-vertex} pour un début de preuve en dimension 2.
La démonstration générale requiert des outils légèrement plus puissants d'algèbre linéaire.
}


\nt{
	Ce théorème se généralise lorsque le nombre de variables augmente.
	En 3 dimensions, un polyèdre $P$ prend la forme d'une intersection de demi-espaces.
	Une forme linéaire $f(x,y,z) = ax+by+cz$ atteint aussi son maximum en un sommet de $P$, caractérisé par l'intersection de 3 plans.
}

\exe{}{
	Soit $P$ un polyèdre défini par l'intersection de $n$ demi-plans.
	Montrer que $P$ admet au plus $n^2$ sommets.
}{exe:nsquarevertices}{
	D'après le théorème \ref{thm:programmation-linéaire}, les sommets de $P$ sont nécessairement à l'intersection de deux droites frontières de deux demi-plans définissant $P$.
	Or il existe ${n \choose 2} = \dfrac{n(n-1)}2 \leq n^2$ paires de demi-plans.
}

\exe{}{
	Confectionner un algorithme renvoyant $\arg\max\limits_{(x;y)\in P} ax+by$ en temps $O(n^2)$ étant donné les $n$ demi-plans définissant $P$.
}{exe:algo-enumeration}{
	D'après le théorème \ref{thm:programmation-linéaire}, le maximum est atteint à l'intersection d'une paire de droites frontières de demi-plan.
	D'après l'exercice \ref{exe:nsquarevertices}, il en existe moins de $n^2$.
	L'algorithme énumère donc toutes les paires de droites, calcule leur intersection et l'image par $f(x,y) = ax+by$, ne gardant en mémoire que la plus grande valeur et le sommet la réalisant.
	Le calcul d'intersection prend un temps constant.
}

\exe{, difficulty=1}{
	En s'appuyant sur le théorème \ref{thm:programmation-linéaire}, montrer que le minimum de $f$ sur $P$ est également atteint en un des sommets de $P$.
}{exe:maxtomin}{
	Le minimum de $f(x,y) = ax+by$ est égal au maximum de $g(x,y) = -f(x,y) = (-a)x + (-b)y$, qui est également une forme linéaire.
}


\thm{complexité de la programmation linéaire, admis}{
	Soit $P$ un polytope, intersection de $m$ demi-espaces en dimension $n$, et $f$ une forme linéaire.
	
	Alors il existe un algorithme pour maximiser $f$ sur $P$ en temps polynomial en $m$ et $n$.
}{thm:programmation-linéaire}
\pf{}{ 
Voir l'algorithme de l'ellipsoïde. 
% algo simplexe n'est pas poly(m,n).
}
	
\section{Implémentation}

\texttt{python3 -m pip install gekko}
	
\begin{mintedbox}{python}
from gekko import GEKKO
m = GEKKO(remote=False)
x,y = m.Array(m.Var,2,lb=0)
m.Equations([6*x+4*y<=24,x+2*y<=6,-x+y<=1,y<=2])
m.Maximize(x+y)
m.solve(disp=False)
\end{mintedbox}
	