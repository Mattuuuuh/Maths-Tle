\documentclass[14pt]{beamer}
\usepackage[french]{babel}

\usetheme{CambridgeUS}
\usecolortheme{rose}
\beamertemplatenavigationsymbolsempty


\usepackage{libertinus}
\usepackage{amsmath,amsfonts,amsthm,amssymb,mathtools}
\usepackage{array}
\newcolumntype{P}[1]{>{\centering\arraybackslash}p{#1}}


\usepackage{stackengine}
\newcommand\xrowht[2][0]{\addstackgap[.5\dimexpr#2\relax]{\vphantom{#1}}}


% corps
\usepackage{calrsfs}
\newcommand{\C}{\mathcal{C}}
\newcommand{\R}{\mathbb{R}}
\newcommand{\Rnn}{\mathbb{R}^{2n}}
\newcommand{\Z}{\mathbb{Z}}
\newcommand{\N}{\mathbb{N}}
\newcommand{\Q}{\mathbb{Q}}

% domain
\newcommand{\D}{\mathcal{D}}


% date
\usepackage{advdate}
\AdvanceDate[0]

%plots
\usepackage{pgfplots, subcaption}
\definecolor{myg}{RGB}{56, 140, 70}
\definecolor{myb}{RGB}{45, 111, 177}
\definecolor{myr}{RGB}{199, 68, 64}

%boxes
\usepackage[most]{tcolorbox}
\usepackage{multicol}

%icomma
\usepackage{icomma}

%https://osl.ugr.es/CTAN/macros/latex/contrib/tcolorbox/tcolorbox.pdf
\newtcolorbox{mybox}[3][]
{
  colframe = #2!25,
  colback  = #2!10,
  coltitle = #2!20!black,  
  halign title=flush center, 
  title    = {#3},
  #1,
}

% BOX A BOX B
\newcommand{\boxAB}[2]{
		\begin{mybox}{red}{A}
		\begin{center}
			#1
		\end{center}
		\end{mybox}
		\begin{mybox}{green}{B}
		\begin{center}
			#2
		\end{center}
		\end{mybox}
}

%systèmes
\usepackage{systeme}

% trafficotage 
\usepackage[answerdelayed, lastexercise]{exercise}
\renewcommand{\ExerciseHeader}{
}
\renewcommand{\AnswerHeader}{
}

\newcommand{\framedelayed}[3][]{
	\begin{Exercise}
	\begin{frame}{\theExercise #1\vspace{-32pt}}
		#2
	\end{frame}
	\end{Exercise}
	\begin{Answer}
	\begin{frame}{\theExercise #1\vspace{-32pt}}
		#3
	\end{frame}
	\end{Answer}
}


\SetDate[19/12/2025]

\reversemarginpar
\setlength{\marginparsep}{.5cm}

\begin{document}
\pagestyle{fancy}
\fancyhead[L]{Tle STMG}
\fancyhead[C]{\textbf{Évaluation — Fonctions exponentielles}}
\fancyhead[R]{\today}

%\null\vspace{-30pt}
Consignes particulières : 
\begin{itemize}[label=$\bullet$]
	\item 
	La calculatrice est {autorisée}.
	\item
	L'évaluation fait 1 page. La somme des points est \total{points}.
\end{itemize}

\marginpar{[pts]}
\hrule


\exe{4}{
	Exprimer les nombre suivants sous la forme $10^x$ pour un entier $x\in\Z$.
	
	\begin{multicols}{4}
	\begin{enumerate}
		\item $10^{1,4} \times 10^{7,6}$
		\item $\left(10^{3,5}\right)^2$
		\item $\dfrac{10^{4,3}}{10^{12,3}}$
		\item $\dfrac{10^{11}}{10^{-3}}$
		\item $\dfrac{10^{0}}{10^{-12}}$
		\item $\dfrac{1}{10^{-6}}$
		\item $\dfrac{10^{32}}{10^{-16}}$
		\item $\left(\dfrac1{10^5}\right)^3$
	\end{enumerate}
	\end{multicols}

}{exe:pow}{
	todo
}


\exe{6, difficulty=2}{
	Soient $f, g, h$ trois fonctions exponentielles. Déterminer la \underline{base $q$} et la \underline{valeur en $0$} de chaque fonction avec les informations suivantes.
		\begin{enumerate}[label=\roman*)]
			\item $f(-2) = 8$ et $f(1) = 27$.
			\item $g(-2) = 125$ et $g(-0,5) = 216$.
			\item $h(4) = \dfrac{49}{3}$ et $h(5) = \dfrac{343}3$.
		\end{enumerate}
}{exe:2}{
	todo
}



\exe{4}{
	Les empreintes carbone par citoyen français au cours des années $2019$ à $2022$ sont données ci-dessous.
	Elles s'expriment en \emph{tCO$_2$eq}, tonne d'équivalent dioxyde de carbone.
	\begin{center}
	\vspace{.5cm}
	\begin{tikzpicture}[scale=.8]
		% nodes
		\draw (0,0) ellipse (2cm and .5cm) node {9,3 tCO$_2$eq};
		
		\draw (5,0) ellipse (2cm and .5cm) node {8,4 tCO$_2$eq};
		
		\draw (10,0) ellipse (2cm and .5cm) node {8,5 tCO$_2$eq};
		
		\draw (15,0) ellipse (2cm and .5cm) node {9,2 tCO$_2$eq};
		
		% vertices
		\draw[->, thick, myg] (1cm,.6cm) arc (105:75:7);
		\draw[->, thick, myg] (6cm,.6cm) arc (105:75:7);
		\draw[->, thick, myg] (11cm,.6cm) arc (105:75:7);
		
		\draw[->, thick, myr] (1cm,-.5cm) arc (-105:-75:25);
	\end{tikzpicture}
	\vspace{.5cm}
	\end{center}

	\begin{enumerate}
		\item Compléter le schéma en ajoutant les coefficients multiplicateurs. Arrondir à $10^{-3}$.
		\item Calculer le coefficient multiplicateur moyen. Arrondir à $10^{-3}$.
		\item En déduire le taux d'évolution moyen en arrondissant le pourcentage à $10^{-1}$.
	\end{enumerate}

}{exe:3}{
	todo
}


\exe{6, difficulty=1}{
	On estime que tous les milliers d'années après la mort d'un organisme, le nombre d'atomes de carbone 14 diminue de $11\%$.
	Ajourd'hui, au temps $0$, on mesure $2$ millions d'atomes de carbone 14.

	Répondre aux questions suivantes en arrondissant à $10^{-3}$.
	\begin{enumerate}
		\item Écrire $A(n)$, le nombre de millions d'atomes de carbone 14 après $n$ milliers d'années. $n\in\N$ est un entier naturel.
		\item Écrire $A(t)$, le nombre de millions d'atomes de carbone 14 au temps $t$ (en milliers d'années). Le nombre $t$ peut être négatif.
		\item Combien d'atomes de carbone 14 restera-t-il après 4 500 années ?
		\item Combien d'atomes de carbones 14 y avait-il il y a 10 000 ans ?
		\item À partir de combien d'années ne restera-t-il que 100 000 atomes de carbone 14 ?
	\end{enumerate}

}{exe:4}{
	todo
}



%%%%%%%%%%%%

\newpage
\fancyhead[C]{\textbf{Solutions}}
\shipoutAnswer

\end{document}
