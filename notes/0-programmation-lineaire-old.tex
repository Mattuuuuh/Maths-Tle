%!TEX encoding = UTF8
%!TEX root = 0-notes.tex

\chapter{Programmation linéaire}

\dfn{Équation cartésienne de droite}{
	Une \emph{équation cartésienne de droite} est une équation de la forme
		\begin{align}
			ax + by = c \label{eq:cart}
		\end{align}
	où $a, b, c\in\R$ sont des réels fixés tels que $a$ et $b$ ne sont pas tous les deux nuls.
	
	L'ensemble des couples $(x ; y)$ vérifiant l'équation forme une droite :
		\begin{align}
			(d) = \bigset{ (x ; y) \tq ax + by = c }. \label{eq:droite}
		\end{align}
}{dfn:equation-cartesienne}

\exe{}{
	Tracer la droite d'équation $x = 2$.
}{exe:graphe-droites}{
	TODO
}

\exe{}{
	Tracer la droite d'équation $5x = 7$.
}{exe:graphe-droites}{
	TODO
}

\exe{}{
	Tracer la droite d'équation $y = -3$.
}{exe:graphe-droites}{
	TODO
}

\dfn{Équation réduite de droite}{
	L'équation $y = mx + p$ est une l'\emph{équation réduite} d'une droite.
	
	Le coefficient directeur $m$ donne la pente de la droite, et l'ordonnée à l'origine $p$ donne l'ordonnée du point d'abscisse nulle lui appartenant.
}{dfn:equation-reduite}

\exe{}{
	Donner l'équation réduite puis tracer la droite d'équation $-3y = -3$.
}{exe:graphe-droites}{
	TODO
}

\exe{}{
	Donner l'équation réduite puis tracer la droite d'équation $-2x + y = 1$.
}{exe:graphe-droites}{
	TODO
}

\thm{}{
	Considérons deux droites
		\begin{align*}
			(d) : y= ax+b, && \text{ et } && (d') : y= a'x+b',
		\end{align*}
	où $a, a', b, b' \in \R$ sont les paramètres des droites.
	
	On distingue alors les cas suivants sur les points d'intersection entre les droites.
		\begin{enumerate}
			\item Si $a \neq a'$, les droites $(d)$ et $(d')$ sont sécantes en un unique point d'intersection.
			\item Sinon $a = a'$, les droites $(d)$ et $(d')$ sont parallèles et on distingue deux sous-cas.
				\begin{enumerate}[label=\roman*)]
					\item Si $b\neq b'$, il n'existe aucun point d'intersection ($(d) \cap (d') = \emptyset$).
					\item Si $b=b'$, alors $f=g$ et les droites sont confondues ($(d) \cap (d') = (d) = (d')$).
				\end{enumerate}
		\end{enumerate}
}{thm:parallélisme-réduit}

\exe{, difficulty = 1}{
	Démontrer le théorème \ref{thm:parallélisme-réduit}.
}{exe:parallélisme-réduit}{
	Si $a \neq a'$, il existe un unique point d'intersection des deux droites car $ax+b=a'x+b' \iff x = \dfrac{b'-b}{a-a'}$.
	Les droites sont donc sécantes non parallèles.
	
	Sinon, $a=a'$ et l'intersection $ax+b = ax+b' \iff b = b'$ est vide si $b\neq b'$, et infinie si $b=b'$.
	Les droites sont donc parallèles, non confondues dans le premier cas, et confondues dans le deuxième.
}

\mprop{}{
	Si $b=0$, la droite $(d)$ de l'équation \eqref{eq:droite} est verticale.
	
	Sinon, l'équation cartésienne \eqref{eq:cart} est équivalente à l'équation réduite
		\[ y = -\dfrac{a}b x + \dfrac{c}b. \]
}{prop:droite-non-verticale}

\exe{}{
	Démontrer la proposition \ref{prop:droite-non-verticale}.
}{exe:droite-non-verticale}{
	Si $b=0$, alors $a$ n'est pas nul par hypothèse sur les coefficients de l'équation cartésienne \ref{eq:cart}.
	Celle-ci s'écrit donc $ax = c \iff x = \dfrac{c}x$, et l'ensemble des points d'abscisse fixée forme bien une droite verticale.
	
	Sinon, $b\neq0$ et on manipule facilement l'équation pour obtenir le résultat requis.
}

\exe{}{
	Donner les équations réduites des droite d'équations $3x - 8y = 2$ et $3x - 8y = 3$.
	Que dire des droites ?
}{exe:équation-réduite}{
	On a d'abord $3x - 8y = 2 \iff y = \dfrac38 x - \dfrac14$ et ensuite $3x - 8y = 3 \iff y = \dfrac38 x - \dfrac38$.
	Les coefficients directeurs sont identiques et les droites sont donc parallèles. Les ordonnées à l'origine sont différentes, donc les droites ne sont pas confondues.
}

\cor{}{
	Les droites d'équations
		\begin{align*}
			(d): ax+by=c && \text{ et } && (d') : ax+by = c'
		\end{align*}
	sont parallèles. 
	De plus, elles sont confondues si et seulement si $c=c'$.
}{cor:parallélisme}

\exe{}{
	Démontrer le corollaire \ref{cor:parallélisme}.
}{exe:parallélisme}{
	On utilise la proposition \ref{prop:droite-non-verticale} qui impose de traiter les cas $b=0$ et $b\neq0$ séparément.

	Si $b=0$, alors les deux droites sont verticales et donc parallèles : l'une est $x = \dfrac{c}a$, et l'autre $x = \dfrac{c'}a$.
	Ces deux droites sont confondues si et seulement si $\dfrac{c}a = \dfrac{c'}a \iff c=c'$.
	
	Si $b\neq0$, le coefficient directeur des deux droites est $-\dfrac{a}b$ et elles sont donc parallèles.
	Elles sont confondues si et seulement si leur ordonnée à l'origine $\dfrac{c}b$ et $\dfrac{c'}b$ sont égales, si et seulement si $c=c'$.
}

\exe{}{
	Soient $a, b \in \R$ pas tous les deux nuls et $P(x_P ; y_P)$ un point quelconque.
	Montrer qu'il existe un $c\in\R$ tel que $P$ appartienne à la droite d'équation
		\[ ax + by = c. \]
}{exe:pavage-droites}{
	Posons $c = ax_P + by_P \in \R$.
	Par construction, $P$ appartient bien à la droite d'équation $ax + by = c$.
}

\thm{}{
	Soient $a, b \in \R$ pas tous les deux nuls. Alors l'union des droites
		\[ \bigcup_{c\in\R} \bigset{ ax + by = c } \]
	recouvre le plan par des droites deux à deux disjointes.
}{thm:pavage-droites}

\pf{}{
	Pour deux $c$ différents, les droites $ax+by=c$ sont disjointes d'après le corollaire \ref{cor:parallélisme}.
	
	L'union des droites est trivialement un sous-ensemble du plan.
	De plus, pour tout point du plan, l'exercice \ref{exe:pavage-droites} donne un $c\in\R$ tel que le point appartiennent à la droite $ax+by=c$ et donc à l'union des droites.
	
	En conclusion, l'union recouvre le plan tout entier.
}

\cor{}{
	Soient $a, b \in \R$ pas tous les deux nuls. 
	L'ensemble $\bigset{ ax + by \leq c }$ est un demi-plan de frontière la droite $ax+by=c$.
}{cor:demi-plan}

\pf{}{
	On écrit
		\[ \bigset{ ax + by \leq c } = \bigcup_{c' \leq c} \bigset{ ax + by = c'}, \]
	pour voir l'ensemble comme une union de droites.
}



