\documentclass[14pt]{beamer}
\usepackage[french]{babel}

\usetheme{CambridgeUS}
\usecolortheme{rose}
\beamertemplatenavigationsymbolsempty


\usepackage{libertinus}
\usepackage{amsmath,amsfonts,amsthm,amssymb,mathtools}
\usepackage{array}
\newcolumntype{P}[1]{>{\centering\arraybackslash}p{#1}}


\usepackage{stackengine}
\newcommand\xrowht[2][0]{\addstackgap[.5\dimexpr#2\relax]{\vphantom{#1}}}


% corps
\usepackage{calrsfs}
\newcommand{\C}{\mathcal{C}}
\newcommand{\R}{\mathbb{R}}
\newcommand{\Rnn}{\mathbb{R}^{2n}}
\newcommand{\Z}{\mathbb{Z}}
\newcommand{\N}{\mathbb{N}}
\newcommand{\Q}{\mathbb{Q}}

% domain
\newcommand{\D}{\mathcal{D}}


% date
\usepackage{advdate}
\AdvanceDate[0]

%plots
\usepackage{pgfplots, subcaption}
\definecolor{myg}{RGB}{56, 140, 70}
\definecolor{myb}{RGB}{45, 111, 177}
\definecolor{myr}{RGB}{199, 68, 64}

%boxes
\usepackage[most]{tcolorbox}
\usepackage{multicol}

%icomma
\usepackage{icomma}

%https://osl.ugr.es/CTAN/macros/latex/contrib/tcolorbox/tcolorbox.pdf
\newtcolorbox{mybox}[3][]
{
  colframe = #2!25,
  colback  = #2!10,
  coltitle = #2!20!black,  
  halign title=flush center, 
  title    = {#3},
  #1,
}

% BOX A BOX B
\newcommand{\boxAB}[2]{
		\begin{mybox}{red}{A}
		\begin{center}
			#1
		\end{center}
		\end{mybox}
		\begin{mybox}{green}{B}
		\begin{center}
			#2
		\end{center}
		\end{mybox}
}

%systèmes
\usepackage{systeme}

% trafficotage 
\usepackage[answerdelayed, lastexercise]{exercise}
\renewcommand{\ExerciseHeader}{
}
\renewcommand{\AnswerHeader}{
}

\newcommand{\framedelayed}[3][]{
	\begin{Exercise}
	\begin{frame}{\theExercise #1\vspace{-32pt}}
		#2
	\end{frame}
	\end{Exercise}
	\begin{Answer}
	\begin{frame}{\theExercise #1\vspace{-32pt}}
		#3
	\end{frame}
	\end{Answer}
}


\SetDate[28/01/2026]

\begin{document}
\pagestyle{fancy}
\fancyhead[L]{Tle STMG}
\fancyhead[C]{\textbf{Variables aléatoires}}
\fancyhead[R]{\today}


\exe{}{
	Un sac contient deux jetons indiscernables. L’un est marqué « 1 », l’autre « 2 ». On
	tire un jeton, on note son numéro puis on le remet dans le sac. On effectue de même un second
	tirage et on fait la somme des deux nombres obtenus. Cette somme définie une variable aléatoire $X$.
	
	On considère les événements
		\begin{center}
		$J_1$ : « le premier jeton vaut 1 » \hspace{2cm} $J_2$ : « le deuxième jeton vaut 2 ».
		\end{center}
	\begin{enumerate}
		\item
		Dessiner un arbre représentant la situation.
		\item
		En déduire l’ensemble des issues possibles de cette expérience. Quelles sont les valeurs prises par $X$ ?
		\item
		Décrire l’évènement $X = 3$ avec des mots puis calculer sa probabilité.
		\item
		Décrire l’évènement $X < 4$ avec des mots puis calculer sa probabilité.
	\end{enumerate}
}{exe:rv0}{

}
\exe{}{
	On lance un dé équilibré. Si le résultat est 1, 2 ou 3, on perd 5 euros. Si le résultat est 4 ou 5, on gagne 2 euros. Sinon, on gagne 10 euros. 
	Posons $X$ la variable aléatoire correspondant au gain	du joueur.
	\begin{enumerate}
		\item
		Quelles sont les valeurs prises par $X$ ?
		\item
		Déterminer la loi de $X$.
		\item
		Calculer $\E(X)$. Le jeu est-il équitable ?
	\end{enumerate}
}{exe:rv1}{

}
\exe{}{
	On choisit au hasard une carte dans un jeu de 52 cartes.
	Il contient 10 valeurs (1 à 10) et 3 têtes (valet, dame, roi) de quatre couleurs différentes.
	
	Si l’on obtient un 7, un 8 ou un 10, on perd 3 euros. 
	Si l’on obtient un valet, une dame ou un roi, on gagne 2 euros.
	Si l’on obtient un as, on gagne 5 euros. 
	
	Soit $X$ la variable aléatoire qui, à chaque tirage, associe le gain du joueur.
	\begin{enumerate}
		\item
		Quelles sont les valeurs prises par $X$.
		\item
		Déterminer la loi de $X$.
		\item
		Calculer $\E(X)$. Le jeu est-il équilibré ?
	\end{enumerate}
}{exe:rv2}{

}
\exe{}{
	Une urne contient 5 boules jaunes, 3 rouges et 4 vertes. On tire simultanément trois
	boules de l’urne. On appelle X la variable aléatoire qui à chaque tirage associe le nombre de boules
	rouges obtenues.
	\begin{enumerate}
		\item
		Quelles sont les valeurs prises par $X$ ?
		\item
		Décrire à l’aide d’une phrases les évènements $X = 2$ et $X < 1$.
	\end{enumerate}
}{exe:rv3}{

}
\exe{}{
	On note $X$ la variable aléatoire qui, à chaque jour, associe le nombre de voiture neuves
vendues par un concessionnaire. Sa loi de probabilité est donnée par le tableau suivant.
	
	\begin{center}
	\begin{tabular}{|c|c|c|c|c|}\hline
	Valeur $x_i$ & 0 & 1 & 2 & 3 \\ \hline
	$P(X = xi)$ & $0,45$ & $0,3$ & $0,15$ & $p$ \\ \hline
	\end{tabular}
	\end{center}
	\begin{enumerate}
		\item
		Donner la probabilité de l’évènement $X = 1$.
		\item
		Calculer à l’aide du tableau $P(X \leq 1)$.
		\item
		Déterminer la valeur de $p$ en supposant que $X \in \bigset{0, 1, 2, 3}$.
		\item
		Calculer $\E(X)$.
	\end{enumerate}
}{exe:rv4}{

}

%%%%%%%%%%%%

\newpage
\fancyhead[C]{\textbf{Solutions}}
\shipoutAnswer

\end{document}
