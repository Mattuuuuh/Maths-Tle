%%%%%%%%%%%%%%%%%%%%%%%%%%%%%%%%%
% PACKAGE IMPORTS
%%%%%%%%%%%%%%%%%%%%%%%%%%%%%%%%%


\usepackage[french]{babel}

\usepackage[tmargin=2cm,rmargin=1in,lmargin=1in,margin=0.85in,bmargin=2cm,footskip=.2in]{geometry}

%ams
\usepackage{amsmath,amsfonts,amsthm,amssymb,mathtools}

\usepackage{bookmark}
\usepackage{enumitem}
\usepackage[most,many,breakable]{tcolorbox}
\usepackage{varwidth,etoolbox}
\usepackage[makeroom]{cancel}
\usepackage{xcolor}
\usepackage{multicol,array}
\usepackage[ruled,vlined,linesnumbered]{algorithm2e}

\usepackage{pgfplots}
\pgfplotsset{compat=1.18}
\usepackage{caption, subcaption}

%virgules
\usepackage{icomma}
\pgfplotsset{/pgf/number format/use comma}

% for striked out implies sign (\centernot\implies)
\usepackage{centernot}

% roman numerals for \section
\renewcommand{\thesection}{\Roman{section}} 

% tikz
\usepackage{tikz}
% i wish external worked but idk it sucks
%\usetikzlibrary{external}
%\tikzexternalize[prefix=figures/]

% for function graph
\usetikzlibrary{positioning}
\usetikzlibrary{shapes.geometric}
\usetikzlibrary{positioning}
\tikzset{
dot/.style = {circle, fill=#1, minimum size=5pt,
              inner sep=0pt, outer sep=0pt},
dot/.default = black % size of the circle diameter
}

 % for braces
\usetikzlibrary{decorations.pathreplacing}
% for hashing area
\usetikzlibrary{patterns}

% tableaux var, signe
% source https://www.sqlpac.com/fr/documents/latex-package-tkz-tab-tikz-tableaux-de-signes-et-de-variations-de-fonctions.html
\usepackage{tkz-tab}

%%%%%%%%%%%%%%%%%%%%%%%%%%%%%%
% SELF MADE COLORS
%%%%%%%%%%%%%%%%%%%%%%%%%%%%%%

%!TEX encoding = UTF8
%!TEX root = 0-notes.tex

%%%%%%%%%%%%%%%%%%%%%%%%%%%%%%
% SELF MADE COLORS
%%%%%%%%%%%%%%%%%%%%%%%%%%%%%%


\definecolor{myg}{RGB}{56, 140, 70}
\definecolor{myb}{RGB}{45, 111, 177}
\definecolor{myr}{RGB}{199, 68, 64}
\definecolor{mytheorembg}{HTML}{F2F2F9}
\definecolor{mytheoremfr}{HTML}{00007B}
\definecolor{mylenmabg}{HTML}{FFFAF8}
\definecolor{mylenmafr}{HTML}{983b0f}
\definecolor{mypropbg}{HTML}{f2fbfc}
\definecolor{mypropfr}{HTML}{191971}
\definecolor{myexamplebg}{HTML}{F2FBF8}
\definecolor{myexamplefr}{HTML}{88D6D1}
\definecolor{myexampleti}{HTML}{2A7F7F}
\definecolor{mydefinitbg}{HTML}{E5E5FF}
\definecolor{mydefinitfr}{HTML}{3F3FA3}
\definecolor{notesgreen}{RGB}{0,162,0}
\definecolor{myp}{RGB}{197, 92, 212}
\definecolor{mygr}{HTML}{2C3338}
\definecolor{myred}{RGB}{127,0,0}
\definecolor{myyellow}{RGB}{169,121,69}
\definecolor{myexercisebg}{HTML}{F2FBF8}
\definecolor{myexercisefg}{HTML}{88D6D1}
\definecolor{doc}{RGB}{0,60,110}

% manim colors because they're beautiful
% https://docs.manim.community/en/stable/reference/manim.utils.color.manim_colors.html

\definecolor{BLACK}{HTML}{000000}\definecolor{BLUE}{HTML}{58C4DD}\definecolor{BLUE_A}{HTML}{C7E9F1}\definecolor{BLUE_B}{HTML}{9CDCEB}\definecolor{BLUE_C}{HTML}{58C4DD}\definecolor{BLUE_D}{HTML}{29ABCA}\definecolor{BLUE_E}{HTML}{236B8E}\definecolor{DARKER_GRAY}{HTML}{222222}\definecolor{DARKER_GREY}{HTML}{222222}\definecolor{DARK_BLUE}{HTML}{236B8E}\definecolor{DARK_BROWN}{HTML}{8B4513}\definecolor{DARK_GRAY}{HTML}{444444}\definecolor{DARK_GREY}{HTML}{444444}\definecolor{GOLD}{HTML}{F0AC5F}\definecolor{GOLD_A}{HTML}{F7C797}\definecolor{GOLD_B}{HTML}{F9B775}\definecolor{GOLD_C}{HTML}{F0AC5F}\definecolor{GOLD_D}{HTML}{E1A158}\definecolor{GOLD_E}{HTML}{C78D46}\definecolor{GRAY}{HTML}{888888}\definecolor{GRAY_A}{HTML}{DDDDDD}\definecolor{GRAY_B}{HTML}{BBBBBB}\definecolor{GRAY_BROWN}{HTML}{736357}\definecolor{GRAY_C}{HTML}{888888}\definecolor{GRAY_D}{HTML}{444444}\definecolor{GRAY_E}{HTML}{222222}\definecolor{GREEN}{HTML}{83C167}\definecolor{GREEN_A}{HTML}{C9E2AE}\definecolor{GREEN_B}{HTML}{A6CF8C}\definecolor{GREEN_C}{HTML}{83C167}\definecolor{GREEN_D}{HTML}{77B05D}\definecolor{GREEN_E}{HTML}{699C52}\definecolor{GREY}{HTML}{888888}\definecolor{GREY_A}{HTML}{DDDDDD}\definecolor{GREY_B}{HTML}{BBBBBB}\definecolor{GREY_BROWN}{HTML}{736357}\definecolor{GREY_C}{HTML}{888888}\definecolor{GREY_D}{HTML}{444444}\definecolor{GREY_E}{HTML}{222222}\definecolor{LIGHTER_GRAY}{HTML}{DDDDDD}\definecolor{LIGHTER_GREY}{HTML}{DDDDDD}\definecolor{LIGHT_BROWN}{HTML}{CD853F}\definecolor{LIGHT_GRAY}{HTML}{BBBBBB}\definecolor{LIGHT_GREY}{HTML}{BBBBBB}\definecolor{LIGHT_PINK}{HTML}{DC75CD}\definecolor{LOGO_BLACK}{HTML}{343434}\definecolor{LOGO_BLUE}{HTML}{525893}\definecolor{LOGO_GREEN}{HTML}{87C2A5}\definecolor{LOGO_RED}{HTML}{E07A5F}\definecolor{LOGO_WHITE}{HTML}{ECE7E2}\definecolor{MAROON}{HTML}{C55F73}\definecolor{MAROON_A}{HTML}{ECABC1}\definecolor{MAROON_B}{HTML}{EC92AB}\definecolor{MAROON_C}{HTML}{C55F73}\definecolor{MAROON_D}{HTML}{A24D61}\definecolor{MAROON_E}{HTML}{94424F}\definecolor{ORANGE}{HTML}{FF862F}\definecolor{PINK}{HTML}{D147BD}\definecolor{PURE_BLUE}{HTML}{0000FF}\definecolor{PURE_GREEN}{HTML}{00FF00}\definecolor{PURE_RED}{HTML}{FF0000}\definecolor{PURPLE}{HTML}{9A72AC}\definecolor{PURPLE_A}{HTML}{CAA3E8}\definecolor{PURPLE_B}{HTML}{B189C6}\definecolor{PURPLE_C}{HTML}{9A72AC}\definecolor{PURPLE_D}{HTML}{715582}\definecolor{PURPLE_E}{HTML}{644172}\definecolor{RED}{HTML}{FC6255}\definecolor{RED_A}{HTML}{F7A1A3}\definecolor{RED_B}{HTML}{FF8080}\definecolor{RED_C}{HTML}{FC6255}\definecolor{RED_D}{HTML}{E65A4C}\definecolor{RED_E}{HTML}{CF5044}\definecolor{TEAL}{HTML}{5CD0B3}\definecolor{TEAL_A}{HTML}{ACEAD7}\definecolor{TEAL_B}{HTML}{76DDC0}\definecolor{TEAL_C}{HTML}{5CD0B3}\definecolor{TEAL_D}{HTML}{55C1A7}\definecolor{TEAL_E}{HTML}{49A88F}\definecolor{WHITE}{HTML}{FFFFFF}\definecolor{YELLOW}{HTML}{FFFF00}\definecolor{YELLOW_A}{HTML}{FFF1B6}\definecolor{YELLOW_B}{HTML}{FFEA94}\definecolor{YELLOW_C}{HTML}{FFFF00}\definecolor{YELLOW_D}{HTML}{F4D345}\definecolor{YELLOW_E}{HTML}{E8C11C}

%%%%%%%%%%%%%%%%%%%%%%%%%%%%
% TCOLORBOX SETUPS
%%%%%%%%%%%%%%%%%%%%%%%%%%%%

%!TEX encoding = UTF8
%!TEX root = 0-notes.tex

%%%%%%%%%%%%%%%%%%%%%%%%%%%%
% CLEAN SETUPS
%%%%%%%%%%%%%%%%%%%%%%%%%%%%

\theoremstyle{definition}

\newtheorem{theorem}{Théorème}[chapter]
\newtheorem{corollaire}[theorem]{Corollaire}
\newtheorem{lemme}[theorem]{Lemme}
\newtheorem{proposition}[theorem]{Proposition}
\newtheorem{exercice}[theorem]{Exercice}
\newtheorem{exemple}[theorem]{Exemple}
\newtheorem{definition}[theorem]{Définition}
\newtheorem*{question}{Question}
\newtheorem*{preuve}{Preuve}
\newtheorem*{remarque}{Remarque}
\newtheorem*{strategie}{Stratégie}
\newtheorem*{methode}{Méthode}
\newtheorem*{notation}{Notation}
\newtheorem*{nomenclature}{Nomenclature}
\newtheorem{axiome}[theorem]{Axiome}

\newtheorem*{definition*}{Définition}
\newtheorem*{lemme*}{Lemme}
\newtheorem*{proposition*}{Proposition}
\newtheorem*{theorem*}{Théorème}
\newtheorem*{corollaire*}{Corollaire}

%%%%%%%%%%%%%%%%%%%%%%%%%%%%
% CLEAN SETUPS : MDFRAMED SURROUND
%%%%%%%%%%%%%%%%%%%%%%%%%%%%


\usepackage[framemethod=tikz]{mdframed}
% def
\surroundwithmdframed[
	hidealllines=true,
	leftline=true,
	innerleftmargin=10pt,
	innerrightmargin=10pt,
	innertopmargin=-4pt,
	nobreak=true,
]{definition}
% thm
\surroundwithmdframed[
	%hidealllines=true,
	leftline=true,
	innerleftmargin=10pt,
	innerrightmargin=10pt,
	innertopmargin=-4pt,
	nobreak=true,
]{theorem}
\surroundwithmdframed[
	%hidealllines=true,
	leftline=true,
	innerleftmargin=10pt,
	innerrightmargin=10pt,
	innertopmargin=-4pt,
	nobreak=true,
]{proposition}


% def
\surroundwithmdframed[
	hidealllines=true,
	leftline=true,
	innerleftmargin=10pt,
	innerrightmargin=10pt,
	innertopmargin=-4pt,
	nobreak=true,
]{definition*}
% thm
\surroundwithmdframed[
	%hidealllines=true,
	leftline=true,
	innerleftmargin=10pt,
	innerrightmargin=10pt,
	innertopmargin=-4pt,
	nobreak=true,
]{theorem*}
\surroundwithmdframed[
	%hidealllines=true,
	leftline=true,
	innerleftmargin=10pt,
	innerrightmargin=10pt,
	innertopmargin=-4pt,
	nobreak=true,
]{proposition*}


%%%%%%%%%%%%%%%%%%%%%%%%%%%%
% EXERCISES 
%%%%%%%%%%%%%%%%%%%%%%%%%%%%

\usepackage[answerdelayed, lastexercise]{exercise}
\renewcommand{\ExerciseHeader}{
	\textbf{
	\theExercise.
	\theExerciseDifficulty
	}
}
\renewcommand{\DifficultyMarker}{$\star$}
\renewcommand{\AnswerHeader}{
	% if exercise title is "1" then announce new chapter
	\if\ExerciseTitle1
		{ 
		%\newpage
		\hrule\vspace{1cm}
		\LARGE
		\textbf{Exercices du chapitre \thechapter}\newline\newline
		}
	\fi
	
	\centerline{\textbf{
	Exercice \ExerciseHeaderNB
	}}
}

%%%%%%%%%%%%%%%%%%%%%%%%%%%%%%%%%%%%%%%%%%%
% MINTED FOR PYTHON ALGORITHMS
%%%%%%%%%%%%%%%%%%%%%%%%%%%%%%%%%%%%%%%%%%%


\newcommand{\python}[1]{
\inputminted[
		linenos,
		gobble=0,
		breaklines=false, % otherwise it breaks for no apparent reason?
		breakafter=,,
		fontsize=\small,
		numbersep=8pt,
		tabsize=4, % tab ident = 4 spaces
		fontfamily=courier, %important pour les signes <, >
]{python}{python/#1.py}
}

\usepackage{tcolorbox}
\tcbuselibrary{minted,breakable,xparse,skins}
\definecolor{bg}{gray}{0.95}
\DeclareTCBListing{mintedbox}{O{}m!O{}}{%
  breakable=true,
  listing engine=minted,
  listing only,
  minted language=#2,
  minted style=default,
  minted options={%
    linenos,
    gobble=0,
    breaklines=false, % otherwise it breaks for no apparent reason?
    breakafter=,,
    fontsize=\small,
    numbersep=8pt,
    tabsize=4, % tab ident = 4 spaces
    fontfamily=courier, %important pour les signes <, >
    #1},
  boxsep=0pt,
  left skip=0pt,
  right skip=0pt,
  left=25pt,
  right=0pt,
  top=3pt,
  bottom=3pt,
  arc=5pt,
  leftrule=0pt,
  rightrule=0pt,
  bottomrule=2pt,
  toprule=2pt,
  colback=bg,
  colframe=orange!70,
  enhanced,
  overlay={%
    \begin{tcbclipinterior}
    \fill[orange!20!white] (frame.south west) rectangle ([xshift=20pt]frame.north west);
    \end{tcbclipinterior}},
  #3}


%%%%%%%%%%%%%%%%%%%%%%%%%%%%
% TCOLORBOX SETUPS
%%%%%%%%%%%%%%%%%%%%%%%%%%%%

\setlength{\parindent}{1cm}
%================================
% THEOREM BOX
%================================

\tcbuselibrary{theorems,skins,hooks}
\newtcbtheorem[number within=chapter]{Theorem}{Théorème}
{%
	enhanced,
	breakable,
	colback = mytheorembg,
	frame hidden,
	boxrule = 0sp,
	borderline west = {2pt}{0pt}{mytheoremfr},
	sharp corners,
	detach title,
	before upper = \tcbtitle\par\smallskip,
	coltitle = mytheoremfr,
	fonttitle = \bfseries\sffamily,
	description font = \mdseries,
	separator sign none,
	segmentation style={solid, mytheoremfr},
}
{th}


\tcbuselibrary{theorems,skins,hooks}
\newtcolorbox{Theoremcon}
{%
	enhanced
	,breakable
	,colback = mytheorembg
	,frame hidden
	,boxrule = 0sp
	,borderline west = {2pt}{0pt}{mytheoremfr}
	,sharp corners
	,description font = \mdseries
	,separator sign none
}

%================================
% Corollary
%================================
\tcbuselibrary{theorems,skins,hooks}
\newtcbtheorem[use counter=tcb@cnt@Theorem]{Corollary}{Corollaire}
{%
	enhanced
	,breakable
	,colback = myp!10
	,frame hidden
	,boxrule = 0sp
	,borderline west = {2pt}{0pt}{myp!85!black}
	,sharp corners
	,detach title
	,before upper = \tcbtitle\par\smallskip
	,coltitle = myp!85!black
	,fonttitle = \bfseries\sffamily
	,description font = \mdseries
	,separator sign none
	,segmentation style={solid, myp!85!black}
}
{th}

%================================
% LEMMA
%================================

\tcbuselibrary{theorems,skins,hooks}
\newtcbtheorem[use counter=tcb@cnt@Theorem]{Lemma}{Lemme}
{%
	enhanced,
	breakable,
	colback = mylenmabg,
	frame hidden,
	boxrule = 0sp,
	borderline west = {2pt}{0pt}{mylenmafr},
	sharp corners,
	detach title,
	before upper = \tcbtitle\par\smallskip,
	coltitle = mylenmafr,
	fonttitle = \bfseries\sffamily,
	description font = \mdseries,
	separator sign none,
	segmentation style={solid, mylenmafr},
}
{th}


%================================
% PROPOSITION
%================================

\tcbuselibrary{theorems,skins,hooks}
\newtcbtheorem[use counter=tcb@cnt@Theorem]{Prop}{Proposition}
{%
	enhanced,
	breakable,
	colback = mypropbg,
	frame hidden,
	boxrule = 0sp,
	borderline west = {2pt}{0pt}{mypropfr},
	sharp corners,
	detach title,
	before upper = \tcbtitle\par\smallskip,
	coltitle = mypropfr,
	fonttitle = \bfseries\sffamily,
	description font = \mdseries,
	separator sign none,
	segmentation style={solid, mypropfr},
}
{th}

%================================
% Exercise
%================================

\tcbuselibrary{theorems,skins,hooks}
\newtcbtheorem[use counter=tcb@cnt@Theorem]{Exe}{Exercice}
{%
	enhanced,
	breakable,
	colback = myexercisebg,
	frame hidden,
	boxrule = 0sp,
	borderline west = {2pt}{0pt}{myexercisefg},
	sharp corners,
	detach title,
	before upper = \tcbtitle\par\smallskip,
	coltitle = myexercisefg,
	fonttitle = \bfseries\sffamily,
	description font = \mdseries,
	separator sign none,
	segmentation style={solid, myexercisefg},
}
{th}

%================================
% EXAMPLE BOX
%================================

\newtcbtheorem[use counter=tcb@cnt@Theorem]{Example}{Exemple}
{%
	colback = myexamplebg
	,breakable
	,colframe = myexamplefr
	,coltitle = myexampleti
	,boxrule = 1pt
	,sharp corners
	,detach title
	,before upper=\tcbtitle\par\smallskip
	,fonttitle = \bfseries
	,description font = \mdseries
	,separator sign none
	,description delimiters parenthesis
}
{ex}

%================================
% DEFINITION BOX
%================================

\newtcbtheorem[use counter=tcb@cnt@Theorem]{Definition}{Définition}{enhanced,
	before skip=2mm,after skip=2mm, colback=red!5,colframe=red!80!black,boxrule=0.5mm,
	attach boxed title to top left={xshift=1cm,yshift*=1mm-\tcboxedtitleheight}, varwidth boxed title*=-3cm,
	boxed title style={frame code={
					\path[fill=tcbcolback]
					([yshift=-1mm,xshift=-1mm]frame.north west)
					arc[start angle=0,end angle=180,radius=1mm]
					([yshift=-1mm,xshift=1mm]frame.north east)
					arc[start angle=180,end angle=0,radius=1mm];
					\path[left color=tcbcolback!60!black,right color=tcbcolback!60!black,
						middle color=tcbcolback!80!black]
					([xshift=-2mm]frame.north west) -- ([xshift=2mm]frame.north east)
					[rounded corners=1mm]-- ([xshift=1mm,yshift=-1mm]frame.north east)
					-- (frame.south east) -- (frame.south west)
					-- ([xshift=-1mm,yshift=-1mm]frame.north west)
					[sharp corners]-- cycle;
				},interior engine=empty,
		},
	fonttitle=\bfseries,
	title={#2},#1}{def}

%================================
% Question BOX
%================================

\makeatletter
\newtcbtheorem[use counter=tcb@cnt@Theorem]{MyQuestion}{Question}{enhanced,
	breakable,
	colback=white,
	colframe=myb!80!black,
	attach boxed title to top left={yshift*=-\tcboxedtitleheight},
	fonttitle=\bfseries,
	title={#2},
	boxed title size=title,
	boxed title style={%
			sharp corners,
			rounded corners=northwest,
			colback=tcbcolframe,
			boxrule=0pt,
		},
	underlay boxed title={%
			\path[fill=tcbcolframe] (title.south west)--(title.south east)
			to[out=0, in=180] ([xshift=5mm]title.east)--
			(title.center-|frame.east)
			[rounded corners=\kvtcb@arc] |-
			(frame.north) -| cycle;
		},
	#1
}{def}
\makeatother




%================================
% NOTE BOX
%================================

\usetikzlibrary{arrows,calc,shadows.blur}
\tcbuselibrary{skins}
\newtcolorbox{Note}[1][]{%
	enhanced jigsaw,
	colback=gray!20!white,%
	colframe=gray!80!black,
	size=small,
	boxrule=1pt,
	title=\colorbox{white!100}{\textbf{ Remarque }},
	halign title=flush center,
	coltitle=black,
	breakable,
	drop shadow=black!50!white,
	attach boxed title to top left={xshift=1cm,yshift=-\tcboxedtitleheight/2,yshifttext=-\tcboxedtitleheight/2},
	minipage boxed title=2.6cm,
	boxed title style={%
			colback=white,
			size=fbox,
			boxrule=1pt,
			boxsep=2pt,
			underlay={%
					\coordinate (dotA) at ($(interior.west) + (-0.5pt,0)$);
					\coordinate (dotB) at ($(interior.east) + (0.5pt,0)$);
					\begin{scope}
						\clip (interior.north west) rectangle ([xshift=3ex]interior.east);
						\filldraw [white, blur shadow={shadow opacity=60, shadow yshift=-.75ex}, rounded corners=2pt] (interior.north west) rectangle (interior.south east);
					\end{scope}
					\begin{scope}[gray!80!black]
						\fill (dotA) circle (2pt);
						\fill (dotB) circle (2pt);
					\end{scope}
				},
		},
	#1,
}

%================================
% STRATÉGIE BOX
%================================

\usetikzlibrary{arrows,calc,shadows.blur}
\tcbuselibrary{skins}
\newtcolorbox{Strategy}[1][]{%
	enhanced jigsaw,
	colback=myb!20!white,%
	colframe=gray!80!black,
	size=small,
	boxrule=1pt,
	title=\colorbox{white!100}{\textbf{ Stratégie }},
	halign title=flush center,
	coltitle=black,
	breakable,
	drop shadow=black!50!white,
	attach boxed title to top left={xshift=1cm,yshift=-\tcboxedtitleheight/2,yshifttext=-\tcboxedtitleheight/2},
	minipage boxed title=2.5cm,
	boxed title style={%
			colback=white,
			size=fbox,
			boxrule=1pt,
			boxsep=2pt,
			underlay={%
					\coordinate (dotA) at ($(interior.west) + (-0.5pt,0)$);
					\coordinate (dotB) at ($(interior.east) + (0.5pt,0)$);
					\begin{scope}
						\clip (interior.north west) rectangle ([xshift=3ex]interior.east);
						\filldraw [white, blur shadow={shadow opacity=60, shadow yshift=-.75ex}, rounded corners=2pt] (interior.north west) rectangle (interior.south east);
					\end{scope}
					\begin{scope}[gray!80!black]
						\fill (dotA) circle (2pt);
						\fill (dotB) circle (2pt);
					\end{scope}
				},
		},
	#1,
}

%================================
% MÉTHODE BOX
%================================

\usetikzlibrary{arrows,calc,shadows.blur}
\tcbuselibrary{skins}
\newtcolorbox{Methode}[1][]{%
	enhanced jigsaw,
	colback=white,%
	colframe=gray!80!black,
	size=small,
	boxrule=1pt,
	title=\textbf{Méthode},
	halign title=flush center,
	coltitle=black,
	breakable,
	drop shadow=black!50!white,
	attach boxed title to top left={xshift=1cm,yshift=-\tcboxedtitleheight/2,yshifttext=-\tcboxedtitleheight/2},
	minipage boxed title=2.5cm,
	boxed title style={%
			colback=white,
			size=fbox,
			boxrule=1pt,
			boxsep=2pt,
			underlay={%
					\coordinate (dotA) at ($(interior.west) + (-0.5pt,0)$);
					\coordinate (dotB) at ($(interior.east) + (0.5pt,0)$);
					\begin{scope}
						\clip (interior.north west) rectangle ([xshift=3ex]interior.east);
						\filldraw [white, blur shadow={shadow opacity=60, shadow yshift=-.75ex}, rounded corners=2pt] (interior.north west) rectangle (interior.south east);
					\end{scope}
					\begin{scope}[gray!80!black]
						\fill (dotA) circle (2pt);
						\fill (dotB) circle (2pt);
					\end{scope}
				},
		},
	#1,
}  
%================================
% AXIOME BOX
%================================

\usetikzlibrary{arrows,calc,shadows.blur}
\tcbuselibrary{skins}
\newtcolorbox{Axiome}[1][]{%
	enhanced jigsaw,
	colback=white,%
	colframe=gray!80!black,
	size=small,
	boxrule=1pt,
	title=\textbf{Axiome},
	halign title=flush center,
	coltitle=black,
	breakable,
	drop shadow=black!50!white,
	attach boxed title to top left={xshift=1cm,yshift=-\tcboxedtitleheight/2,yshifttext=-\tcboxedtitleheight/2},
	minipage boxed title=2.5cm,
	boxed title style={%
			colback=white,
			size=fbox,
			boxrule=1pt,
			boxsep=2pt,
			underlay={%
					\coordinate (dotA) at ($(interior.west) + (-0.5pt,0)$);
					\coordinate (dotB) at ($(interior.east) + (0.5pt,0)$);
					\begin{scope}
						\clip (interior.north west) rectangle ([xshift=3ex]interior.east);
						\filldraw [white, blur shadow={shadow opacity=60, shadow yshift=-.75ex}, rounded corners=2pt] (interior.north west) rectangle (interior.south east);
					\end{scope}
					\begin{scope}[gray!80!black]
						\fill (dotA) circle (2pt);
						\fill (dotB) circle (2pt);
					\end{scope}
				},
		},
	#1,
}  

%%%%%%%%%%%%%%%%%%%%%%%%%%%%%%%%%%%%%%%%%%%
% TABLE OF CONTENTS
%%%%%%%%%%%%%%%%%%%%%%%%%%%%%%%%%%%%%%%%%%%

\usepackage{titletoc}
\contentsmargin{0cm}
\titlecontents{chapter}[3.7pc]
{\addvspace{30pt}%
	\begin{tikzpicture}[remember picture, overlay]%
		\draw[fill=doc!60,draw=doc!60] (-7,-.1) rectangle (0.2,.6);%
		\pgftext[left,x=-3.5cm,y=0.2cm]{\color{white}\Large\sc\bfseries Chapitre\ \thecontentslabel};%
	\end{tikzpicture}\qquad\color{doc!60}\large\sc\bfseries}%
{}
{}
{\;\titlerule\;\large\sc\bfseries Page \thecontentspage
	\begin{tikzpicture}[remember picture, overlay]
		\draw[fill=doc!60,draw=doc!60] (2pt,0) rectangle (4,0.1pt);
	\end{tikzpicture}}%
\titlecontents{section}[3.7pc]
{\addvspace{2pt}}
{\contentslabel[\thecontentslabel]{2pc}}
{}
{\hfill\small \thecontentspage}
[]
\titlecontents*{subsection}[3.7pc]
{\addvspace{-1pt}\small}
{}
{}
{\ --- \small\thecontentspage}
[ \textbullet\ ][]

\makeatletter
\renewcommand{\tableofcontents}{%
	\chapter*{%
	  \vspace*{-20\p@}%
	  \begin{tikzpicture}[remember picture, overlay]%
		  \pgftext[right,x=15cm,y=0.2cm]{\color{doc!60}\Huge\sc\bfseries \contentsname};%
		  \draw[fill=doc!60,draw=doc!60] (13,-.75) rectangle (20,1);%
		  \clip (13,-.75) rectangle (20,1);
		  \pgftext[right,x=15cm,y=0.2cm]{\color{white}\Huge\sc\bfseries \contentsname};%
	  \end{tikzpicture}}%
	\@starttoc{toc}}
\makeatother
  

\SetDate[16/10/2025]

\begin{document}
\pagestyle{fancy}
\fancyhead[L]{Tle STMG}
\fancyhead[C]{\textbf{Suites géométriques 2}}
\fancyhead[R]{\today}


\exe{, difficulty=1}{
	Lors de votre entretien d'embauche, une entreprise vous propose un salaire de départ de 35 000€ avec deux choix d'évolutions possibles pour ce salaire
	
	\begin{center}
	\begin{enumerate}[start=0,label={\bfseries Choix~\arabic*:},leftmargin=3cm]
	\item une augmentation annuelle de 2\% ;
	\item une augmentation annuelle de 7 00€.
	\end{enumerate}
	\end{center}
	\begin{enumerate}
	\item Calculer le salaire (arrondi à l'euro près) à la 20ème année pour chacun des choix.
	\item Quel choix est, selon vous, le plus avantageux ? Expliquer.
	\item Selon les statistiques de l'OCDE, les français restent en moyenne 11 ans dans une même entreprise. En tenant compte de cette information, quel choix est probablement le plus avantageux ?
	\end{enumerate}
}{exe:exo9}{
	todo
}
% ils s'en foutent de ça
%\exe{}{
%	Pour chacune des suites géométriques données algébriquement pour tout $n\in\N$, donner sa raison et son terme initial.
%	\begin{multicols}{3}
%	\begin{enumerate}
%		\item $u_n = 2 \times 3^n$
%		\item $v_n = 7 \times \left(\dfrac12 \right)^n$
%		\item $w_n = (-6)^n$
%		\item $\zeta_n = - 6^n$
%		\item $a_n = 11 \times 5^{2n}$
%		\item $b_n = 3 \times 5^{2n+3}$
%		\item $c_n = 10^{-n}$
%		\item $d_n = \dfrac{4}{7^n}$
%	\end{enumerate}
%	\end{multicols}
%}{exe:q-ti}{
%	Comme on sait que les suites sont géométriques, le terme initial est donné par $u_0$ et la raison par $\dfrac{u_1}{u_0}$.
%	Il suffit donc de savoir évaluer les suites en $0$ et $1$ pour conclure.
%	L'identité $a^0 = 1$ sera utile.
%
%	\begin{enumerate}
%		\item 
%			\begin{align*}
%				u_0 = 2 && q = 3
%			\end{align*}
%		\item 
%			\begin{align*}
%				v_0 = 7 && q = \dfrac12
%			\end{align*}
%		\item 
%			\begin{align*}
%				w_0 = 1 && q = -6
%			\end{align*}
%		\item 
%			\begin{align*}
%				\zeta_0 = -1 && q = 6
%			\end{align*}
%		\item 
%			\begin{align*}
%				a_0 = 11 && q = 25
%			\end{align*}
%		\item 
%			\begin{align*}
%				b_0 = 375 && q = 25
%			\end{align*}
%		\item 
%			\begin{align*}
%				c_0 = 1 && q = \dfrac{1}{10}
%			\end{align*}
%		\item 
%			\begin{align*}
%				d_0 = 4 && q = \dfrac17
%			\end{align*}
%	\end{enumerate}
%}

\begin{multicols}{2}
\setlength\columnseprule{.1pt} 

\exe{}{
	Les suites suivantes données graphiquement peuvent-elles être géométriques ?
	
	\begin{center}
	\begin{tikzpicture}[>=stealth, scale=1]
		\begin{axis}[xmin = 0, xmax=4.2, xtick={ 0,1,2, 3, 4,5}, ymin=0, ymax=300, ytick={0, 30, ..., 300}, axis x line=middle, axis y line=middle, axis line style=->, ylabel={}, grid=both, extra x ticks = {0}]
			
			\addplot[black, thick, only marks, mark=star] coordinates {(0, 130) (1,140) (2,150) (3,160) (4,170)};
			
			\addplot[black, thick, only marks, mark=square] coordinates {(0,15) (1,30) (2,60) (3, 120) (4, 240)};
			
			\addplot[black, thick, only marks, mark=*] coordinates {(1,300) (2,200) (3,133.33) (4, 88.89)};
		\end{axis}
	
	\end{tikzpicture}
	\end{center}

}{exe:graph1}{
	On calcule les ratios successifs : s'ils ne sont pas constants, alors la suite n'est pas géométrique.
	Dans le cas contraire, on ne peut pas réellement conclure que la suite est géométrique, car nous n'avons qu'un échantillons restreint des images (et donc des ratios).
	
	Seule la suite $\star$ peut être écartée ici.
}

\exe{}{
	Donner le terme de rang $n \in \N$ des suites géométriques $\star$, $\bullet$, et $\square$ données graphiquement.

	\begin{center}
	\begin{tikzpicture}[>=stealth, scale=1]
		\begin{axis}[xmin = 0, xmax=4.2, xtick={ 0,1,2, 3, 4,5}, ymin=0, ymax=10000, ymode=log, log ticks with fixed point, axis x line=middle, axis y line=middle, axis line style=->, ylabel={}, grid=both, extra x ticks = {0}]
			
			\addplot[black, thick, only marks, mark=star] coordinates {(0, 1) (1,10) (2,100) (3,1000) (4,10000)};
			
			\addplot[black, thick, only marks, mark=square] coordinates {(0,10 000) (1,10 00) (2,100) (3,10) (4,1)};
			
			\addplot[black, thick, only marks, mark=*] coordinates {(0,3) (1,30) (2,300) (3, 3000)};
		\end{axis}
	
	\end{tikzpicture}
	\end{center}

}{exe:graph2}{
	Dans ce repère à échelle logarithmique, la première graduation des ordonnées est $1$.
	On obtient donc
		\begin{align*}
			\star_n = 1 \times 10^n, && \square_n = 10~000\times \left(\dfrac1{10}\right)^n && \bullet_n = 3 \times 10^n
		\end{align*} 
}

\end{multicols}


% pas très bien compris, cet exo
%\exe{}{
%	Achille dispute une course avec une tortue. On suppose que les deux participants avancent à vitesse constante et que la tortue avance $10$ fois moins vite qu'Achille.
%	Celui-ci décide donc de lui laisser généreusement $10$ minutes d'avance.
%
%	En analysant la situation, Zénon décide de diviser la course en plusieurs étapes.
%	À chaque étape, Achille court jusqu'au point d'où a démarré la tortue à la dernière étape.
%	Il déduit que, comme la tortue avance pendant qu'Achille court, il ne pourra jamais la dépasser.
%
%	\begin{enumerate}
%		\item Vérifier les premières valeurs du tableau suivantes et le compléter.
%			\begin{center}
%			\begin{tabular}{|c|c|c|c|c|c|}\hline
%				Étape & 0 & 1 & 2 & 3 & 4 \\ \hline
%				Temps qu'Achille met pour finir l'étape (min) & 1 & 0,1 && & \\ \hline
%			\end{tabular}
%			\end{center}
%		\item Justifier du caractère géométrique de la suite $T_n$. Quelle est sa raison ?
%		\item Donner exactement $T(100)$ en écriture scientifique.
%		\item On considère la somme des temps de chaque étape pour comprendre quand Achille atteindra la tortue.
%			\[ S_n = \sum_{k=0}^n T_k = T_0 + T_1 + \dots + T_n. \]
%		Vérifier les premières valeurs du tableau suivantes et le compléter.
%			\begin{center}
%			\begin{tabular}{|c|c|c|c|c|c|}\hline
%				$n$ & 0 & 1 & 2 & 3 & 4 \\ \hline
%				$S_n$ & 1 & 1,1 & & & \\ \hline
%			\end{tabular}
%			\end{center}
%		\item Est-ce que $S_n$ peut être aussi grand qu'on le souhaite ? Par exemple, existe-t-il un rang $n$ pour lequel $S_n \geq 2$ ?
%		\item En combien de temps Achille arrive-t-il à dépasser la tortue ? Donner une valeur exacte sous forme de fraction.
%	\end{enumerate}
%}{exe:AchTor}{
%
%			\begin{center}
%			\begin{tabular}{|c|c|c|c|c|c|}\hline
%				Étape & 0 & 1 & 2 & 3 & 4 \\ \hline
%				Temps qu'Achille met pour finir l'étape (min) & 1 & 0,1 & \color{red} 0,01 & \color{red} 0,001 & \color{red} $10^{-4}$ \\ \hline
%			\end{tabular}
%			\end{center}
%			
%	\begin{enumerate}
%		\item Cf. tableau de l'énoncé.
%		\item D'une étape à l'autre, le temps pris par Achille pour rattraper la tortue est divisé par $10$ car celui-ci va $10$ fois plus vite.
%		\item $T(100) = 10^{-100}$.
%		\item On complète en utilisant que $S_2 = T_0 + T_1 + T_2$, puis $S(3) = T_0 + T_1 + T_2 + T(3)$, et idem pour $S(4)$.
%		\item 
%		$S_n$ ne dépasse jamais $1,2$ et donc ne peut pas être aussi grand que souhaité.
%		\item 
%		On cherche à étudier la limite de $S_n$ quand $n$ tend vers l'infini.
%		On souhaite donc déterminer $x = 1,1111\dots$ avec une infinité de $1$.
%		En multipliant par $3$, on remarque que
%			\[ 3x = 3,333\dots = \dfrac{10}3. \]
%		On en déduit que $x= \dfrac{10}9$, qu'on vérifiera avec une calculatrice par exemple.
%	\end{enumerate}
%
%			\begin{center}
%			\begin{tabular}{|c|c|c|c|c|c|}\hline
%				$n$ & 0 & 1 & 2 & 3 & 4 \\ \hline
%				$S_n$ & 1 & 1,1 & \color{red} $1,11$  & \color{red} $1,111$ &\color{red} $1,1111$ \\ \hline
%			\end{tabular}
%			\end{center}
%}

\exe{}{
	Un ami vous propose de créer une entreprise dont le principe est le suivant :
	\begin{itemize}
		\item Vous recrutez des ``investisseurs'' en leurs proposant de verser 1 000€
		et en leur promettant qu'ils doubleront leur mise après une semaine. 
		\item Pour payer le premier investisseur après une semaine, vous recrutez deux nouveaux investisseurs.
		\item Vous poursuivez ainsi, en recrutant toujours deux nouveaux investisseurs pour en payer un.
	\end{itemize}
	On note $u_n$ le nombre de nouveaux investisseurs recrutés en semaine numéro $n$. On suppose que l'entreprise démarre en semaine 0 avec un seul investisseur.
	\begin{enumerate}
		\item Calculer $u_0$, $u_1$ et $u_2$.
		\item Quelle est la nature de la suite $u$ ? Justifier.
		\item Estimer le nombre d'investisseurs nécessaires pour que l'entreprise fonctionne sur une année entière (52 semaines). 
		Le plan de votre ami semble-t-il possible ? Expliquer.
	\end{enumerate}
	\textit{Ce type de montage se nomme « système de Ponzi ».}
}{exe:5}{
	todo
}

\newpage
\exe{}{
	Un élève choisit d'investir 1 000€ chaque année en bourse.
	À la fin de chaque année, l'argent investi génère 5\% d'intérêts. Ces intérêts sont réinvestis en bourse en plus des 1 000€.
	
	Notons $u_n$ l'argent placé (en comptant les intérêts) à l'année $n \in \N$.
	Ainsi, 
	\begin{itemize}
		\item $u_0 = 1 000$, car 1 000€ sont placés à l'année 0, sans intérêts.
		\item $u_1 = 2 050$, car 1 000€ sont ajoutés à l'année 1, en plus des 50€ d'intérêts sur l'année passée.
		\item $u_2 = 3 152,5$, car 1 000€ sont ajoutés à l'année 2, en plus des 102,5€ d'intérêts sur l'année passée.
	\end{itemize}
		
	\begin{enumerate}
		\item Décrire comment $u_{n+1}$ est calculé à partir de $u_n$.
	\end{enumerate}
	On suppose que la suite $u$ vérifie la relation suivante.
		\[
		\begin{cases}
		u_0 = 1~000, \\
		u_{n+1} = 1,05u_n + 1~000.
		\end{cases}
		\]
	
	\begin{enumerate}[resume]
		\item \label{q2}
		Calculer $u_0 ; u_1 ; u_2 ; u_3 ; u_4 ; u_5$ grâce à la relation de récurrence de $u$.
		\item
		Est-ce que $u$ est une suite arithmétique ? géométrique ?
	\end{enumerate}
	Le but des questions suivantes est de trouver une expression algébrique pour $u_n$ en trouvant une suite intermédiaire $w$ qui, elle, est géométrique.
	
	\begin{enumerate}[resume]
		\item 
		Résoudre l'équation suivante pour $x$.
			\[ x = 1,05 x + 1~000 \]
		% trop dur pour eux
%		\item 
%		Définissons la suite intermédiaire $w$ par
%			\[ w_n = u_n + 20 000. \]
%		Montrer que
%			\[ w_{n+1} = 1,05 w_n. \]
%		Quelle est la nature de la suite $w$ ? Donner sa raison et son terme initial.
		\item 
		Définissons la suite intermédiaire $w$ par
			\[ w_n = u_n + 20~000. \]
		Calculer $w_0 ; w_1 ; w_2 ; w_3 ; w_4 ; w_5$ grâce à la question \ref{q2}.
		
		La suite $w$ est-elle arithmétique ? géométrique ? 
		Donner son terme initial et sa raison.
		\item 
		Donner l'expression algébrique de $w_n$ à l'aide du cours et conclure que
			\[ u_n = 1,05^n \times 21~000 - 20~000.\]
		\item
		Combien d'argent l'élève peut-il retirer après 60 ans d'investissements continus ?
		Combien d'argent brut a-t-il investi pendant ces 60 années (on ne compte pas les intérêts réinvestis) ?
		
		Décrire l'évolution de l'argent brut investi vers l'argent qu'il peut retirer après 60 ans (augmentation ou diminution, et de quel pourcentage).
	\end{enumerate}
}{exe:arithmetico-geom}{
	todo
}



%%%%%%%%%%%%

\newpage
\fancyhead[C]{\textbf{Solutions}}
\shipoutAnswer

\end{document}
