\documentclass[14pt]{beamer}
\usepackage[french]{babel}

\usetheme{CambridgeUS}
\usecolortheme{rose}
\beamertemplatenavigationsymbolsempty


\usepackage{libertinus}
\usepackage{amsmath,amsfonts,amsthm,amssymb,mathtools}
\usepackage{array}
\newcolumntype{P}[1]{>{\centering\arraybackslash}p{#1}}


\usepackage{stackengine}
\newcommand\xrowht[2][0]{\addstackgap[.5\dimexpr#2\relax]{\vphantom{#1}}}


% corps
\usepackage{calrsfs}
\newcommand{\C}{\mathcal{C}}
\newcommand{\R}{\mathbb{R}}
\newcommand{\Rnn}{\mathbb{R}^{2n}}
\newcommand{\Z}{\mathbb{Z}}
\newcommand{\N}{\mathbb{N}}
\newcommand{\Q}{\mathbb{Q}}

% domain
\newcommand{\D}{\mathcal{D}}


% date
\usepackage{advdate}
\AdvanceDate[0]

%plots
\usepackage{pgfplots, subcaption}
\definecolor{myg}{RGB}{56, 140, 70}
\definecolor{myb}{RGB}{45, 111, 177}
\definecolor{myr}{RGB}{199, 68, 64}

%boxes
\usepackage[most]{tcolorbox}
\usepackage{multicol}

%icomma
\usepackage{icomma}

%https://osl.ugr.es/CTAN/macros/latex/contrib/tcolorbox/tcolorbox.pdf
\newtcolorbox{mybox}[3][]
{
  colframe = #2!25,
  colback  = #2!10,
  coltitle = #2!20!black,  
  halign title=flush center, 
  title    = {#3},
  #1,
}

% BOX A BOX B
\newcommand{\boxAB}[2]{
		\begin{mybox}{red}{A}
		\begin{center}
			#1
		\end{center}
		\end{mybox}
		\begin{mybox}{green}{B}
		\begin{center}
			#2
		\end{center}
		\end{mybox}
}

%systèmes
\usepackage{systeme}

% trafficotage 
\usepackage[answerdelayed, lastexercise]{exercise}
\renewcommand{\ExerciseHeader}{
}
\renewcommand{\AnswerHeader}{
}

\newcommand{\framedelayed}[3][]{
	\begin{Exercise}
	\begin{frame}{\theExercise #1\vspace{-32pt}}
		#2
	\end{frame}
	\end{Exercise}
	\begin{Answer}
	\begin{frame}{\theExercise #1\vspace{-32pt}}
		#3
	\end{frame}
	\end{Answer}
}


\SetDate[18/02/2026]

\reversemarginpar
\setlength{\marginparsep}{.5cm}

\begin{document}
\pagestyle{fancy}
\fancyhead[L]{Tle STMG}
\fancyhead[C]{\textbf{Évaluation blanche — Variables aléatoires}}
\fancyhead[R]{\today}

%\null\vspace{-30pt}
Consignes particulières : 
\begin{itemize}[label=$\bullet$]
	\item 
	La calculatrice est {autorisée}.
	\item
	L'évaluation fait \pageref{lastpage} page. La somme des points est \total{points}.
\end{itemize}

\marginpar{[pts]}
\hrule


\exe{7}{
	Princesse lance un dé à $6$ faces équilibré puis regarde le numéro de la face du dessus (un entier entre 1 et 6).
	\begin{enumerate}[label=$\bullet$]
		\item Si le numéro obtenu est 1 ou 2, elle extrait au hasard une boule dans l'urne 1 qui contient 3 boules noires, 4 boules blanches et 3 boules rouges
		\item Sinon, elle extrait une boule dans l'urne 2 qui contient 3 boules noires et 2 boules blanches.
	\end{enumerate}
	En lisant les règles du jeu, Princesse décide de parier sur la boule noire : si elle tire une boule noire, elle gagne 20€, et sinon, elle perd 20€.
	Notons $X$ le gain de Princesse après un tirage.
	\begin{enumerate}
		\item
		Faire un arbre modélisant l'expérience aléatoire.
		\item
		Faire un tableau dans lequel à chaque issue de l'expérience on peut lire sa probabilité ainsi que la valeur de $X$ associée.
		\item
		Donner la loi de $X$ sous la forme d'un tableau dans lequel à chaque valeur de $X$ est associée sa probabilité.
		\item
		Quelle est l'espérance de gain de Princesse ?
		Comment interpréter ce résultat ?
	\end{enumerate}
}{exe:2}{
	todo
}

\exe{7}{
	\begin{enumerate}
		\item
		Créer un triangle de Pascal\footnotemark pour calculer $6\choose3$.
		\item
		Interpréter le nombre obtenu.
		\item
		Donner $340\choose1$ en justifiant à l'aide de l'interprétation du coefficient binomial.
		\item
		Étant donné que ${11\choose3} = 165$, déduire la valeur de ${11\choose8}$.
		Justifier à l'aide des interprétations des coefficients binomiaux.
	\end{enumerate}
}{exe:3}{
	todo
}

\footnotetext{Blaise Pascal (1623-1662), mathématicien, physicien et philosophe français.}
%\newpage


\exe{6}{
	Matthieu tape un énoncé sur son ordinateur. 
	Il estime que, pour chaque caractère, la probabilité qu'il appuie sur la bonne touche est $0,95$.
	
	\begin{enumerate}
		\item
		Donner la probabilité que Matthieu se trompe de touche pour un caractère quelconque.
		\item
		Donner la probabilité que Matthieu ne fasse aucune erreur lorsqu'il tape le mot \mbox{« apparemment »}.
		\item
		En déduire la probabilité qu'il fasse au moins une erreur en tapant « apparemment ».
		\item
		Calculer le nombre moyen d'erreurs commises en tapant l'énoncé de cet exercice.
		Celui-ci contient 515 caractères.
	\end{enumerate}
}{exe:4}{
	todo
}


%%%%%%%%%%%%

\label{lastpage}
\newpage
\fancyhead[C]{\textbf{Solutions}}
\shipoutAnswer

\end{document}
