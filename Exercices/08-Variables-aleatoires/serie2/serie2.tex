\documentclass[14pt]{beamer}
\usepackage[french]{babel}

\usetheme{CambridgeUS}
\usecolortheme{rose}
\beamertemplatenavigationsymbolsempty


\usepackage{libertinus}
\usepackage{amsmath,amsfonts,amsthm,amssymb,mathtools}
\usepackage{array}
\newcolumntype{P}[1]{>{\centering\arraybackslash}p{#1}}


\usepackage{stackengine}
\newcommand\xrowht[2][0]{\addstackgap[.5\dimexpr#2\relax]{\vphantom{#1}}}


% corps
\usepackage{calrsfs}
\newcommand{\C}{\mathcal{C}}
\newcommand{\R}{\mathbb{R}}
\newcommand{\Rnn}{\mathbb{R}^{2n}}
\newcommand{\Z}{\mathbb{Z}}
\newcommand{\N}{\mathbb{N}}
\newcommand{\Q}{\mathbb{Q}}

% domain
\newcommand{\D}{\mathcal{D}}


% date
\usepackage{advdate}
\AdvanceDate[0]

%plots
\usepackage{pgfplots, subcaption}
\definecolor{myg}{RGB}{56, 140, 70}
\definecolor{myb}{RGB}{45, 111, 177}
\definecolor{myr}{RGB}{199, 68, 64}

%boxes
\usepackage[most]{tcolorbox}
\usepackage{multicol}

%icomma
\usepackage{icomma}

%https://osl.ugr.es/CTAN/macros/latex/contrib/tcolorbox/tcolorbox.pdf
\newtcolorbox{mybox}[3][]
{
  colframe = #2!25,
  colback  = #2!10,
  coltitle = #2!20!black,  
  halign title=flush center, 
  title    = {#3},
  #1,
}

% BOX A BOX B
\newcommand{\boxAB}[2]{
		\begin{mybox}{red}{A}
		\begin{center}
			#1
		\end{center}
		\end{mybox}
		\begin{mybox}{green}{B}
		\begin{center}
			#2
		\end{center}
		\end{mybox}
}

%systèmes
\usepackage{systeme}

% trafficotage 
\usepackage[answerdelayed, lastexercise]{exercise}
\renewcommand{\ExerciseHeader}{
}
\renewcommand{\AnswerHeader}{
}

\newcommand{\framedelayed}[3][]{
	\begin{Exercise}
	\begin{frame}{\theExercise #1\vspace{-32pt}}
		#2
	\end{frame}
	\end{Exercise}
	\begin{Answer}
	\begin{frame}{\theExercise #1\vspace{-32pt}}
		#3
	\end{frame}
	\end{Answer}
}


\SetDate[11/02/2026]

\begin{document}
\pagestyle{fancy}
\fancyhead[L]{Tle STMG}
\fancyhead[C]{\textbf{Loi binomiale et coefficients binomiaux}}
\fancyhead[R]{\today}


\begin{figure}
	\begin{center}
	\begin{tikzpicture}[scale=.8]
		% depth 1
		\foreach \i in {-3, 3}
		\draw[-, thick, black] (0,0) node {$\bullet$} -- (\i,-2);
		
		\draw (-3,-2) node[above left] {$S$};
		\draw (3,-2) node[above right] {$E$};
%		\draw (-1.5,-.5) node {$\frac14$};
%		\draw (1.5,-.5) node {$\frac34$};
		% depth 2
		\foreach \i in {-3, 3} \foreach \j in {-1.5, 1.5}
			\draw[-, thick, black] (\i,-2) node {$\bullet$} -- (\i+\j,-4) node {$\bullet$};
			
		\draw (-4.5,-4) node[above left] {$S$};
		\draw (-1.5,-4) node[above right] {$E$};
		\draw (1.5,-4) node[above left] {$S$};
		\draw (4.5,-4) node[above right] {$E$};
%		\draw (-4.5,-3) node {$\frac14$};
%		\draw (-1.5,-3) node {$\frac34$};
%		\draw (1.5,-3) node {$\frac14$};
%		\draw (4.5,-3) node {$\frac34$};
		% depth 3
		\foreach \i in {-3, 3} \foreach \j in {-1.5, 1.5} \foreach \k in {-1, 1}
			\draw[-, thick, black] (\i+\j,-4) node {$\bullet$} -- (\i+\j+\k,-6) node {$\bullet$};

		\draw (-5.5,-6) node[below] {$S$};
		\draw (-3.5,-6) node[below] {$E$};
		\draw (-2.5,-6) node[below] {$S$};
		\draw (-.5,-6) node[below] {$E$};
		\draw (.5,-6) node[below] {$S$};
		\draw (2.5,-6) node[below] {$E$};
		\draw (3.5,-6) node[below] {$S$};
		\draw (5.5,-6) node[below] {$E$};
%		\draw (-5.5,-5) node {$\frac14$};
%		\draw (-3.5,-5) node {$\frac34$};
%		\draw (-2.5,-5) node {$\frac14$};
%		\draw (-.5,-5) node {$\frac34$};
%		\draw (.5,-5) node {$\frac14$};
%		\draw (2.5,-5) node {$\frac34$};
%		\draw (3.5,-5) node {$\frac14$};
%		\draw (5.5,-5) node {$\frac34$};
		
		% chemins bleu question b)
%		\draw[-, thick, BLUE_E, shift={(.1,0)}] (0,0) -- (-3, -2) -- (-1.5, -4) -- (-.5, -6);
%		\draw[-, thick, BLUE_E, shift={(.1,0)}] (0,0) -- (3, -2) -- (1.5, -4) -- (2.5, -6);
%		\draw[-, thick, BLUE_E, shift={(.1,0)}] (0,0) -- (3, -2) -- (4.5, -4) -- (3.5, -6);
%		
%		% chemin rouge question c)
%		\draw[-, thick, RED_E, shift={(-.1,0)}] (0,0) -- (3, -2) -- (4.5, -4) -- (5.5, -6);
	\end{tikzpicture}
	\end{center}
	\caption{Figure de l'exercice \ref{exe:binom0}.}
	\label{fig:1}
\end{figure}

\exe{}{
	On effectue une même épreuve trois fois et on note $S$ le succès et $E$ l'échec.
	L'expérience aléatoire est représentée par l'arbre figure \ref{fig:1}.
	\begin{enumerate}
		\item
		Combien y a-t-il de façons différentes de ne jamais obtenir de succès ?
		\item\label{q:2}
		Combien y a-t-il de façons différentes d'obtenir exactement 2 succès ? Compter les chemins racine-feuille correspondant.
	\end{enumerate}
	On suppose désormais que la probabilité d'un succès est $\frac15$.
	De plus, on suppose que les trois épreuves sont indépendantes : le résultat de l'une n'influe pas sur le résultat de l'autre.
	\begin{enumerate}[resume]
		\item
		Remplir l'arbre de probabilité figure \ref{fig:1} en ajoutant les probabilités aux branches.
		\item
		Quelle est la probabilité de ne jamais obtenir de succès ?
		\item
		Quelle est la probabilité d'obtenir exactement 2 succès ? Utiliser la question \ref{q:2}.
	\end{enumerate}
}{exe:binom0}{
	todo
}

\exe{}{
	Du \emph{triangle de Pascal} ci-dessous, on peut lire qu'il y a \underline{3} façons d'obtenir exactement 2 succès après 3 épreuves.
	On écrit alors ${3\choose2} = 3$.

	Compléter le triangle et en déduire la valeur de $5\choose2$ et de $6\choose4$
	\begin{center}
	\begin{tabular}{>{$}c<{$}|*{8}{c}}
		\multicolumn{1}{l}{\thead{Nombre \\ d'épreuves}} &&&&&&&\\\cline{1-1} 
		1 &1&1\\
		2 &1&2&1\\
		3 &1&&\underline{3}&1\\
		4 &1&4&&&1\\
		5 &1&&&&&1\\
		6 &1&&&&&&1&\\
		7 &1&&&&&&&1\\\hline
		\multicolumn{1}{l}{} &0&1&2&3&4&5&6&7\\\cline{2-8}
		\multicolumn{1}{l}{} &\multicolumn{8}{c}{Nombre de succès}
	\end{tabular}
	\end{center}
}{exe:Pascal}{

}

\exe{}{
	Donner la valeur de $7\choose4$ à l'aide de l'exercice \ref{exe:Pascal} et interpréter ce nombre.
}{exe:0}{

}

\exe{}{
	Adam et Bérat sont fatigués d'écrire pendant le cours.
	Pour se motiver, ils se promettent à chaque séance que celui qui s'endort en premier donnera 20€ à l'autre.
	
	On suppose qu'à chaque séance, un des deux s'endormira, et que la probabilité qu'Adam s'endorme en premier est de 60\%.
	
	Après une semaine, soit 3 séances, Adam et Bérat font les comptes.
	Ils notent $X$ l'argent dépensé par Adam.
	\begin{enumerate}
		\item
		Faire un arbre modélisant l'expérience et les différentes issues.
		\item
		Quelles sont les valeurs possibles de $X$ ?
		\item
		Démontrer que la probabilité qu'Adam perde 20€ au final est $0,432$.
		\item
		Calculer le nombre de fois, en moyenne, qu'Adam s'endort en premier après 15 séances.
		Quel gain cela représente-t-il pour Bérat ?
	\end{enumerate}
}{exe:AB}{

}

\exe{}{
	Des élèves de Terminale STMG passent un examen d'entrée en université donné sous la forme d'un questionnaire.
	Il y a 15 questions, chacune ayant quatre réponses possible dont une seule correcte. 
	
	Par dépit, les élèves choisissent de remplir le questionnaire au hasard.
	\begin{enumerate}
		\item
		Pour une question quelconque, donner la probabilité de répondre correctement en répondant au hasard.
		\item
		Donner la probabilité d'obtenir 15 bonnes réponses.
		\item
		Donner la probabilité d'obtenir exactement 1 bonne réponse.
		\item
		On calcule la moyenne des bonnes réponses de tous les élèves. 
		À quelle valeur cette moyenne est-elle très probablement très proche ?
		\item
		En sachant que ${15\choose7} = 6435$, calculer la probabilité d'obtenir exactement 7 bonnes réponses.
		\item
		Calculer la probabilité d'obtenir exactement 8 bonnes réponses.
	\end{enumerate}
}{exe:1}{

}




\exe{}{
	On pose un pion sur la case inférieure gauche d'un échiquier $4\times4$.
	On déplace le pion de case en case afin d'atteindre la case supérieure droite.
	Le pion peut soit se déplacer d'une case vers le haut, soit d'une case vers la droite. Il est donc impossible de faire une boucle.
	En combien de façons différentes peut-il atteindre la case supérieure droite ?
	
}{exe:binom2}{

}
	\begin{figure}[h!]
	\begin{subfigure}{.5\textwidth}
	\centering
	\begin{tikzpicture}
		\foreach \i in {0, ..., 4}{
			\draw (\i,0) -- (\i,4);
		}
		\foreach \j in {0, ..., 4}{
			\draw (0,\j) -- (4,\j);
		}
		
		\draw[<-] (.4,.5) arc[start angle=-90, end angle=-155, radius=.75cm] node[left] {Départ};
		\draw[<-] (3.6,3.5) arc[start angle=90, end angle=25, radius=.75cm] node[right] {Arrivée};
	\end{tikzpicture}
	\caption{Échiquier $4\times4$.}
	\label{fig:1a}
	\end{subfigure}
	\hfill
	\begin{subfigure}{.5\textwidth}
	\centering
	\begin{tikzpicture}
		\foreach \i in {0, ..., 4}{
			\draw (\i,0) -- (\i,4);
		}
		\foreach \j in {0, ..., 4}{
			\draw (0,\j) -- (4,\j);
		}
		\draw[thick] (.5,.5) -- (2.5, .5) -- (2.5, 1.5) -- (3.5, 1.5) -- (3.5, 3.5);
		\draw[thick, dashed] (.5,.5) -- (.5, 1.5) -- (1.5, 1.5) -- (2.5, 1.5) -- (2.5, 2.5) -- (2.5 ,3.5) -- (3.5, 3.5);
	\end{tikzpicture}
	\caption{Deux façons différentes d'atteindre l'arrivée.}
	\label{fig:1b}
	\end{subfigure}
	\end{figure}

%%%%%%%%%%%%

\newpage
\fancyhead[C]{\textbf{Solutions}}
\shipoutAnswer

\end{document}
