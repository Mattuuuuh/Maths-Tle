\documentclass[14pt]{beamer}
\usepackage[french]{babel}

\usetheme{CambridgeUS}
\usecolortheme{rose}
\beamertemplatenavigationsymbolsempty


\usepackage{libertinus}
\usepackage{amsmath,amsfonts,amsthm,amssymb,mathtools}
\usepackage{array}
\newcolumntype{P}[1]{>{\centering\arraybackslash}p{#1}}


\usepackage{stackengine}
\newcommand\xrowht[2][0]{\addstackgap[.5\dimexpr#2\relax]{\vphantom{#1}}}


% corps
\usepackage{calrsfs}
\newcommand{\C}{\mathcal{C}}
\newcommand{\R}{\mathbb{R}}
\newcommand{\Rnn}{\mathbb{R}^{2n}}
\newcommand{\Z}{\mathbb{Z}}
\newcommand{\N}{\mathbb{N}}
\newcommand{\Q}{\mathbb{Q}}

% domain
\newcommand{\D}{\mathcal{D}}


% date
\usepackage{advdate}
\AdvanceDate[0]

%plots
\usepackage{pgfplots, subcaption}
\definecolor{myg}{RGB}{56, 140, 70}
\definecolor{myb}{RGB}{45, 111, 177}
\definecolor{myr}{RGB}{199, 68, 64}

%boxes
\usepackage[most]{tcolorbox}
\usepackage{multicol}

%icomma
\usepackage{icomma}

%https://osl.ugr.es/CTAN/macros/latex/contrib/tcolorbox/tcolorbox.pdf
\newtcolorbox{mybox}[3][]
{
  colframe = #2!25,
  colback  = #2!10,
  coltitle = #2!20!black,  
  halign title=flush center, 
  title    = {#3},
  #1,
}

% BOX A BOX B
\newcommand{\boxAB}[2]{
		\begin{mybox}{red}{A}
		\begin{center}
			#1
		\end{center}
		\end{mybox}
		\begin{mybox}{green}{B}
		\begin{center}
			#2
		\end{center}
		\end{mybox}
}

%systèmes
\usepackage{systeme}

% trafficotage 
\usepackage[answerdelayed, lastexercise]{exercise}
\renewcommand{\ExerciseHeader}{
}
\renewcommand{\AnswerHeader}{
}

\newcommand{\framedelayed}[3][]{
	\begin{Exercise}
	\begin{frame}{\theExercise #1\vspace{-32pt}}
		#2
	\end{frame}
	\end{Exercise}
	\begin{Answer}
	\begin{frame}{\theExercise #1\vspace{-32pt}}
		#3
	\end{frame}
	\end{Answer}
}


\SetDate[16/11/2025]

\usepackage{chessboard, skak}

\begin{document}
\pagestyle{fancy}
\fancyhead[L]{Tle STMG}
\fancyhead[C]{\textbf{Évaluation --- Suites arithmétiques et géométriques}}
\fancyhead[R]{\today}


\exe{}{
	En Inde, le roi Belkib accorde la récompense de son choix au sage Sissa pour lui remercier d'avoir inventé le jeu d'échecs.
	L'échiquier, plateau du jeu d'échecs, reste inchangé depuis sa création jusqu'à aujourd'hui.
	
	\begin{multicols}{2}
	Sissa demande alors au roi de prendre le plateau du jeu d'échecs et, sur la première case, de poser un grain de riz.
	Ensuite, il lui demande de poser deux grains sur la deuxième case, puis quatre, puis huit, ... en doublant à chaque fois le nombre de grains de riz à déposer.
	
	
	\begin{center}
	\setchessboard{showmover=false}
	\newgame
	\chessboard[tinyboard, setfen=,]
	\\
	Exemple d'échiquier.
	\end{center}
	\end{multicols}
	
	Notons $G_n$ le nombre de grains de riz que le roi doit poser sur la $(n+1)$-ième case.
	Attention : le rang de la suite et le numéro de la case sont donc en décalage de 1.
	\begin{enumerate}
		\item
		De quelle nature est la suite $G$ ?
		Donner sa raison et son terme initial.
		
		\item
		À quel rang de la suite correspond la dernière case de l'échiquier ?
		
		\item
		Donner le nombre de grains de riz que le roi devra poser sur la dernière case de l'échiquier.
		
		\item
		On souhaite connaitre le nombre de grains de riz dont le roi a besoin pour satisfaire le sage Sissa.
		Écrire ce nombre sous forme de somme avec la notation $\Sigma$, puis donner sa valeur exacte à l'aide du cours.
	\end{enumerate}
	En 2022, 776 millions de tonnes de riz ont été produites, chaque grain riz pesant environ 0,04g.
	\begin{enumerate}[resume]
		\item	
		Calculer le nombre de grains de riz produits en 2022.
		\item
		Combien d'années de production comme celle de 2022 faudrait-il pour satisfaire le sage Sissa ?
	\end{enumerate}
	
	\emph{Indications : 1 million = $10^6$, et 1 tonne = $10^6$ grammes. }
	
}{exe:Sissa}{
	todo
}

\exe{}{
	On étudie la croissance d'une population de champignons.
	
	Au début de l'expérience, on dispose de 100 champignons.
	Toutes les dix minutes, on mesure l'évolution de leur nombre.
	On obtient les résultats suivants.
	
	\begin{multicols}{2}
	\begin{center}
	\begin{tabular}{|c|c|}\hline
		\thead{Temps écoulé \\ (en minutes)} & \thead{Nombre de \\ champignons} \\ \hline
		0 & 100 \\\hline
		10 & 125 \\\hline
		20 & 150 \\\hline
		30 & 175 \\ \hline
	\end{tabular}
	
	\begin{tikzpicture}[>=stealth, scale=1]
		\begin{axis}[xmin = -1, xmax=35, ymin=-10, ymax=180, axis x line=middle, axis y line=middle, axis line style=->, xlabel={temps écoulé (en minutes)}, ylabel={nombres de champignons}, grid=both, ytick distance=50, xtick distance = 10]
			\addplot[black, thick, only marks, mark=*] coordinates {(0, 100) (10, 125) (20, 150) (30,175)};
		\end{axis}
	\end{tikzpicture}
	\end{center}
	\end{multicols}
	
	Soit $n$ un entier naturel.
	On note $u_n$ le nombre de champignons après $n$ périodes de \textbf{dix} minutes.
	Ainsi, $u_0 = 100, u_1 = 125, u_2 = 150, \dots$.
	\begin{enumerate}
		\item Justifier que les termes $u_0, u_1, u_2, u_3$ sont en progression arithmétique.
		\item En supposant que la population de champignons continue d'évoluer selon le même rythme, montrer qu'elle aura quadruplé deux heures après le début de l'expérience.
	\end{enumerate}
}{exe:zero-2}{
	\begin{enumerate}
		\item
		On calcule $u_1 - u_0 = 25, u_2 - u_1 = 25,$ et $u_3 - u_2 = 25$.
		Pour passer d'un terme à l'autre, on ajoute 25 : la progression est arithmétique.
		\item 
		On a $u_n = 100 + 25n$ d'après le cours.
		Comme 2 heures sont $2 \times60 = 120$ minutes, il s'agit de calculer le terme $u(12)$, la suite étant indexée en dixaines de minutes.
		\begin{align*}
			u(12) &= 100 + 25\times12 \\
					&= 100 + 25 \times 4 \times 3 \\
					&= 100 + 100 \times 3 \\
					&= 400
		\end{align*}
		On obtient bien $u(12) = 400 = 4 \times u_0$, comme annoncé.
	\end{enumerate}
}


%%%%%%%%%%%%

\newpage
\fancyhead[C]{\textbf{Solutions}}
\shipoutAnswer

\end{document}
