\documentclass[14pt]{beamer}
\usepackage[french]{babel}

\usetheme{CambridgeUS}
\usecolortheme{rose}
\beamertemplatenavigationsymbolsempty


\usepackage{libertinus}
\usepackage{amsmath,amsfonts,amsthm,amssymb,mathtools}
\usepackage{array}
\newcolumntype{P}[1]{>{\centering\arraybackslash}p{#1}}


\usepackage{stackengine}
\newcommand\xrowht[2][0]{\addstackgap[.5\dimexpr#2\relax]{\vphantom{#1}}}


% corps
\usepackage{calrsfs}
\newcommand{\C}{\mathcal{C}}
\newcommand{\R}{\mathbb{R}}
\newcommand{\Rnn}{\mathbb{R}^{2n}}
\newcommand{\Z}{\mathbb{Z}}
\newcommand{\N}{\mathbb{N}}
\newcommand{\Q}{\mathbb{Q}}

% domain
\newcommand{\D}{\mathcal{D}}


% date
\usepackage{advdate}
\AdvanceDate[0]

%plots
\usepackage{pgfplots, subcaption}
\definecolor{myg}{RGB}{56, 140, 70}
\definecolor{myb}{RGB}{45, 111, 177}
\definecolor{myr}{RGB}{199, 68, 64}

%boxes
\usepackage[most]{tcolorbox}
\usepackage{multicol}

%icomma
\usepackage{icomma}

%https://osl.ugr.es/CTAN/macros/latex/contrib/tcolorbox/tcolorbox.pdf
\newtcolorbox{mybox}[3][]
{
  colframe = #2!25,
  colback  = #2!10,
  coltitle = #2!20!black,  
  halign title=flush center, 
  title    = {#3},
  #1,
}

% BOX A BOX B
\newcommand{\boxAB}[2]{
		\begin{mybox}{red}{A}
		\begin{center}
			#1
		\end{center}
		\end{mybox}
		\begin{mybox}{green}{B}
		\begin{center}
			#2
		\end{center}
		\end{mybox}
}

%systèmes
\usepackage{systeme}

% trafficotage 
\usepackage[answerdelayed, lastexercise]{exercise}
\renewcommand{\ExerciseHeader}{
}
\renewcommand{\AnswerHeader}{
}

\newcommand{\framedelayed}[3][]{
	\begin{Exercise}
	\begin{frame}{\theExercise #1\vspace{-32pt}}
		#2
	\end{frame}
	\end{Exercise}
	\begin{Answer}
	\begin{frame}{\theExercise #1\vspace{-32pt}}
		#3
	\end{frame}
	\end{Answer}
}

\SetDate[08/10/2025]

\begin{document}
\pagestyle{fancy}
\fancyhead[L]{Terminale STMG}
\fancyhead[C]{\textbf{Exercice à rendre : probabilités conditionnelles}}
\fancyhead[R]{\today}


\exe{}{
	Compléter l'arbre avec les données suivantes.
		\begin{multicols}{3}
		\begin{enumerate}[label=$\bullet$]
			\item $P(A) = 0,35$
			\item $P(B \sct A) = 0,45$
			\item $P(B \sct \overline{A}) = 0,7$
		\end{enumerate}
		\end{multicols}
	\begin{center}
	\begin{tikzpicture}
		% depth 1
		\foreach \i in {-3, 3}
		\draw[-, thick, black] (0,0) node {$\bullet$} -- (\i,-1.5);
		% depth 2
		\foreach \i in {-3, 3} \foreach \j in {-1, 1}
			\draw[-, thick, black] (\i,-1.5) node {$\bullet$} -- (\i+\j,-3) node {$\bullet$};
			
		\draw (-3,-1.5) node[above left] {$A$};
		\draw (3,-1.5) node[above right] {$\overline{A}$};
			
		\draw (-4,-3) node[below] {$B\cap A$};
		\draw (2,-3) node[below] {$B\cap\overline{A}$};
		\draw (-2,-3) node[below] {$\overline{B}\cap A$};
		\draw (4,-3) node[below] {$\overline{B}\cap\overline{A}$};
	\end{tikzpicture}
	\end{center}
	Calculer $P(B\cap A)$ et $P(\overline{B}\cap\overline{A})$ en multipliant les probabilités des chemins racine-feuille correspondant.
	\begin{flalign*}
		P(B\cap A) &= && \\ \\
		P(\overline{B}\cap\overline{A}) &= && \\
	\end{flalign*}
}{exe:1}{
	todo
}


\exe{}{
	Un test est mis au point pour détecter une maladie rare. Une étude est effectuée sur un échantillon représentatif de 5 000 personnes, et les résultats sont les suivants.
	\begin{enumerate}[label=\roman*)]
		\item 0,4\% des personnes sont malades.
		\item parmis les personnes malades, 99,9\% sont testées positives.
		\item parmis les personnes non malades, 94\% sont testées négatives.
	\end{enumerate}
	On choisit une personne uniformément au hasard dans la population et on considère les événements suivants.
	\begin{center}
		$M$ : \og La personne choisie est malade. \fg
		
		\vspace{10pt}
		
		$N$ : \og La personne choisie est testée négative. \fg
	\end{center}
		
	\begin{enumerate}
		\item
		À l'aide du texte, donner $P(M), P\bigl(\overline{N} \sct M\bigr)$, et $P\bigl(N \sct \overline{M} \bigr)$.
		\item
		Les événements $\overline{N}$ et $M$ sont-ils indépendants ? positivement corrélés ? négativement corrélés ?
		\item
		Créer un arbre de probabilités correspondant à l'expérience et le compléter.
		\item 
		Pour chaque événement suivant, le décrire avec des mots et donner sa probabilité.
		\begin{enumerate}[label=\roman*)]
			\item $\bigl(N \sct M\bigr)$ 
			\item $\bigl(\overline{N} \sct \overline{M} \bigr)$
		\end{enumerate}
		% c'est faux non ? ce serait P(malade sachant négatif) et P(sain sachant positif) non ?
		\item
		Calculer $P\bigl(\overline{N}\bigr)$ à l'aide l'arbre.
		\item
		Décrire avec des mots l'événement $\bigl(M \sct \overline{N}\bigr)$ et calculer sa probabilité à l'aide de la définition de la probabilité conditionnelle.
		\item
		Le test semble-t-il efficace pour détecter la maladie ? Le résultat du test semble-t-il fiable ? Justifier à l'aide des données et des probabilités calculées.
	\end{enumerate}
}{exe:2}{
	todo
}

\end{document}