\documentclass[14pt]{beamer}
\usepackage[french]{babel}

\usetheme{CambridgeUS}
\usecolortheme{rose}
\beamertemplatenavigationsymbolsempty


\usepackage{libertinus}
\usepackage{amsmath,amsfonts,amsthm,amssymb,mathtools}
\usepackage{array}
\newcolumntype{P}[1]{>{\centering\arraybackslash}p{#1}}


\usepackage{stackengine}
\newcommand\xrowht[2][0]{\addstackgap[.5\dimexpr#2\relax]{\vphantom{#1}}}


% corps
\usepackage{calrsfs}
\newcommand{\C}{\mathcal{C}}
\newcommand{\R}{\mathbb{R}}
\newcommand{\Rnn}{\mathbb{R}^{2n}}
\newcommand{\Z}{\mathbb{Z}}
\newcommand{\N}{\mathbb{N}}
\newcommand{\Q}{\mathbb{Q}}

% domain
\newcommand{\D}{\mathcal{D}}


% date
\usepackage{advdate}
\AdvanceDate[0]

%plots
\usepackage{pgfplots, subcaption}
\definecolor{myg}{RGB}{56, 140, 70}
\definecolor{myb}{RGB}{45, 111, 177}
\definecolor{myr}{RGB}{199, 68, 64}

%boxes
\usepackage[most]{tcolorbox}
\usepackage{multicol}

%icomma
\usepackage{icomma}

%https://osl.ugr.es/CTAN/macros/latex/contrib/tcolorbox/tcolorbox.pdf
\newtcolorbox{mybox}[3][]
{
  colframe = #2!25,
  colback  = #2!10,
  coltitle = #2!20!black,  
  halign title=flush center, 
  title    = {#3},
  #1,
}

% BOX A BOX B
\newcommand{\boxAB}[2]{
		\begin{mybox}{red}{A}
		\begin{center}
			#1
		\end{center}
		\end{mybox}
		\begin{mybox}{green}{B}
		\begin{center}
			#2
		\end{center}
		\end{mybox}
}

%systèmes
\usepackage{systeme}

% trafficotage 
\usepackage[answerdelayed, lastexercise]{exercise}
\renewcommand{\ExerciseHeader}{
}
\renewcommand{\AnswerHeader}{
}

\newcommand{\framedelayed}[3][]{
	\begin{Exercise}
	\begin{frame}{\theExercise #1\vspace{-32pt}}
		#2
	\end{frame}
	\end{Exercise}
	\begin{Answer}
	\begin{frame}{\theExercise #1\vspace{-32pt}}
		#3
	\end{frame}
	\end{Answer}
}


\SetDate[07/01/2026]

\begin{document}
\pagestyle{fancy}
\fancyhead[L]{Tle STMG}
\fancyhead[C]{\textbf{Fonction logarithme}}
\fancyhead[R]{\today}

\exe{}{
	Résoudre les équations suivantes sans calculatrice. Trouver $x\in\R$.
	\begin{multicols}{2}
	\begin{enumerate}[label=\alph*)]
		\item $10^x = 10$
		\item $5^x = 25$
		\item $2^x = 32$
		\item $10^x = 0,1$
		\item $10^x = 100~000$
		\item $10^x = 0,0001$
	\end{enumerate}
	\end{multicols}
}{exe:1}{
	todo
}

\exe{}{
	Estimer la valeur du nombre $x\in\R$ vérifiant $10^x = 30$.
}{exe:2}{
	todo
}

\exe{}{
	Exprimer les nombres suivants sont la forme $\log(n)$ où $n\in\N$ est un nombre entier non nul.
	\begin{multicols}{2}
	\begin{enumerate}[label=\alph*)]
		\item $\log(2) + \log(3)$
		\item $3\log(5)$
		\item $\log(8)-\log(4)$
		\item $\log(7) - \log(\frac13)$
		\item $3\log(5) - 2\log(5)$
		\item $\log(5) - \log(20) + 2\log(10)$
	\end{enumerate}
	\end{multicols}
}{exe:2}{
	todo
}

\exe{}{
	Résoudre les équations suivantes. Trouver $x\in\R$.
	\begin{multicols}{2}
	\begin{enumerate}[label=\alph*)]
		\item $8^x = 123$
		\item $x^{63} = 4~812$
		\item $5^x \times 7^x = 28$
		\item $12^x = 854$
	\end{enumerate}
	\end{multicols}
}{exe:3}{
	todo
}

\exe{}{
	Tracer la courbe représentative de la fonction logarithme dans le repère figure \ref{fig:1}.
}{exe:graph}{
	todo
}

\begin{figure}[h!]
	\centering
	\begin{tikzpicture}[>=stealth]
		\begin{axis}[xmin = 0, xmax=101, ymin=-2.5, ymax=2.5, axis x line=middle, axis y line=middle, axis line style=->, grid=both, clip=true, x=5pt, extra x ticks = {0}, ytick distance = 1]
			\addplot[no marks, BLUE_E, very thick, -, ] expression[domain=0:100, samples=1000]{log10(x)} node[pos=.5, above] {$\C_{\log}$};
		\end{axis}
	\end{tikzpicture}
	\caption{Repère de l'exercice \ref{exe:graph}.}
	\label{fig:1}
\end{figure}



\exe{}{
	À l'aide de la courbe figure \ref{fig:1}, résoudre approximativement les équations suivantes.
	Vérifier ses réponses à l'aide de la calculatrice.
		\begin{multicols}{2}
		\begin{enumerate}
			\item $\log(x) = 1,5$
			\item $\log(x) = 0,5$
			\item $10^y = 30$
			\item $10^y = 15$
		\end{enumerate}
		\end{multicols}
}{exe:graph}{
	todo
}

\begin{figure}[h!]
	\centering
	\begin{tabular}{|c|c|}\hline
		$x$ & $\log(x)$ \\\hline
		$0,105$ & \hspace{3cm} \\\hline
		$0,193$ & \\\hline
		$0,267$ & \\\hline
		$0,481$ & \\\hline
	\end{tabular}
	\hspace{10pt}
	\begin{tabular}{|c|c|}\hline
		$x$ & $\log(x)$ \\\hline
		$0,541$ & \hspace{3cm} \\\hline
		$0,612$ & \\\hline
		$0,723$ & \\\hline
		$0,865$ & \\\hline
	\end{tabular}
	\caption{Tableau de valeurs de la fonction $\log$, extraite d'une \emph{table de logarithmes}.}
	\label{fig:2}
\end{figure}

\newpage

\exe{}{
	Le but de l'exercice est de calculer le produit de deux nombres à l'aide d'une \emph{table de logarithmes}, dont est extraite la figure \ref{fig:2}.
	On propose, en guise d'exemple, de calculer le produit $2~671 \times 723$.
	\begin{enumerate}
		\item
		Remplir d'abord le tableau de valeurs figure \ref{fig:2} à l'aide de la calculatrice.
		Les valeurs doivent être données à $10^{-6}$ près.
		\item
		Montrer ensuite que
			\begin{align*}
				\log(2671) = \log(0,2671) + 4 && \et && \log(723) = \log(0,723) + 3.
			\end{align*}
		\item
		En déduire que
			\[ \log(2~671 \times 723) = \log(0,2671) + \log(0,723) + 7, \]
		et donc que 
			\[ 2~671 \times 723 = 10^{\log(0,2671) + \log(0,723)} \times 10^7. \]
		\item
		À l'aide du tableau de valeurs, calculer sans calculatrice la somme $\log(0,2671) + \log(0,723)$ à $10^{-6}$ près.
		\item
		À l'aide du tableau de valeurs et des deux dernières questions, donner une valeur approchée du produit $2~671\times723$.
	\end{enumerate}
}{exe:Napier}{
	todo
}


\exe{}{
	Un élève d'un lycée de 1 400 élèves  a répandu une rumeur.
	On estime que, chaque jour, le nombre d'élèves ayant connaissance de la rumeur augmente de $5\%$.
	Aujourd'hui, jour $0$, on compte $350$ élèves ayant appris la rumeur.
	
	Répondre au questions suivantes en arrondissant les nombres d'élèves à l'unité et les nombres de jours à $10^{-1}$.
	\begin{enumerate}
		\item Donner le coefficient multiplicateur associé à une augmentation de 5\%.
		\item Écrire $E(t)$, le nombre d'élèves ayant appris la rumeur au temps $t$ (en jours). $t$ peut être négatif.
		\item Combien d'élèves connaîtront la rumeur après $4,5$ jours ?
		\item Combien d'élèves connaissaient la rumeur il y a $3,5$ semaines ?
		\item À partir de combien de jours l'intégralité du lycée aura-t-il appris la rumeur ?
		Poser l'équation puis la résoudre à l'aide du logarithme.
		\item Il y a combien de jours la rumeur est-elle née ? On cherche à calculer le $t$ négatif tel que $E(t) = 1$ à l'aide du logarithme.
	\end{enumerate}

}{exe:expo}{

}

%%%%%%%%%%%%

\newpage
\fancyhead[C]{\textbf{Solutions}}
\shipoutAnswer

\end{document}
