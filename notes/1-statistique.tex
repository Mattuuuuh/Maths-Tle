%!TEX encoding = UTF8
%!TEX root = 0-notes.tex

\chapter{Statistiques à deux variables}

Dans tout ce chapitre on considère $n$ points dans l'espace, $(x_1, y_1), (x_2, y_2), \dots, (x_n, y_n)$, notés
	\[ (x, y) \in N, \]
et appelés \emphindex{nuage de points}.

Ces données sont souvent \emphindex{statistiques} et mesurent deux propriétés dont on étudie le lien : comment $y_i$ agit-il en fonction de $x_i$ ?
Par exemple, comment le salaire évolue-t-il en fonction de l'ancienneté ? en fonction du sexe ? en fonction de la durée de la formation ?

Si les données sont \emphindex{corrélées}, on pourra identifier une tendance dans le nuage de points.
Sinon, le nuage ne sera qu'un nuage : les données semblent \emphindex{décorrélées}.

\textbf{\warning La corrélation n'implique pas la causalité.}

\section{Ajustement affine et moindres carrés}

\begin{figure}
	\centering
	\includegraphics[page=1, scale=.9]{figures/fig-stats.pdf}
	\includegraphics[page=2, scale=.9]{figures/fig-stats.pdf}
	\caption{Exemple de nuage de points avec tendance linéaire, avec et sans ajustement affine.}
	\label{fig:stats-1}
\end{figure}



\begin{figure}
	\centering
	\includegraphics[page=4]{figures/fig-stats.pdf}
	\caption{Nombre de pompiers en Dakota du Sud en fonction du nombre de films dans lesquels Margot Robbie est apparue. La corrélation n'implique pas la causation. \href{https://www.tylervigen.com/spurious/correlation/5846_the-number-of-movies-margot-robbie-appeared-in_correlates-with_the-number-of-firefighters-in-south-dakota}{Source}.}
\end{figure}

\emphindex{interpolation affine}

\subsection{Cas $n=2$}

\begin{figure}
	\centering
	\includegraphics[page=14, scale=.9]{figures/fig-stats.pdf}
	\includegraphics[page=9, scale=.9]{figures/fig-stats.pdf}
	\caption{Quand $n$ est petit, l'extrapolation est inutile : deux tendances contraires sont réalistes sans départage possible.}
\end{figure}


\subsection{Cas $n>2$}

\nt{
	Lorsqu'il y a plus de deux points dans un nuage, il n'est en général pas possible de placer une droite passant par tous les points.
	On pourrait choisir deux points parmis le nuage et en déduire une droite, mais est-ce bien pertinent ?
}

Le but de l'interpolation étant d'approcher la tendance du nuage, il est important de mesurer l'erreur commise entre la valeur réelle et la valeur interpolée.
Pour cela, on choisit la distance au carré entre la valeur interpolée et la vraie valeur, soit $\bigl[ f(x) - y \bigr]^2$.
Sans hiérarchisation des points du nuage, l'erreur totale est alors donnée par la somme des erreurs individuelles.

\dfn{erreur d'interpolation}{
	L'\emphindex{erreur d'interpolation} commise par l'interpolation par $f$ d'un nuage $\{ (x, y) \}$ est donnée par
		\[ \sum_{(x,y) \in N} \bigl[ f(x) - y \bigr]^2. \]
}{}

\dfn{moindres carrés}{
	Le problème des \emphindex{moindres de carrés} demande de trouver les paramètres $a, b$ de la fonction affine $f(x)=ax+b$ minimisant l'erreur d'interpolation d'un nuage de points.
	
	\[ (a^\star , b^\star) = \arg\min_{a, b\in\R} \sum_{(x,y)\in N} \bigl[ ax+b - y \bigr]^2. \]
}{}



\dfn{extrapolation}{
	L'\emphindex{extrapolation} est l'utilisation d'un modèle pour prédire des valeurs encore inexistantes.
}{}

\ex{}{
	Le GIEC extrapole la température à la surface de la Terre dans le temps à partir d'un modèle d'interpolation basé sur les données collectées depuis plus de 100 ans.
	Un tel modèle est un général lisse, et ne prédit pas les valeurs aberrantes (été à 40°C, par exemple).
}{}

\ex{}{
	L'extrapolation n'est pas toujours pertinente. Dans la figure \ref{fig:stats-1}, en pointillés, l'extrapolation pour des anciennetés supérieures à 12 a un sens, mais celle pour  des anciennetés négatives n'en a pas.
}{}

\subsection{Non correlation et données aberrantes}

\begin{figure}
	\centering
	\includegraphics[page=3, scale=.9]{figures/fig-stats.pdf}
	\includegraphics[page=15, scale=.9]{figures/fig-stats.pdf}
	\caption{À gauche, exemple de nuage de points sans corrélation apparente. À droite, même nuage de points avec distinction des sexes : en rouge, les femmes ; en bleu, les hommes.}
\end{figure}

\begin{figure}
	\centering
	\includegraphics[page=5, scale=.9]{figures/fig-stats.pdf}
	\includegraphics[page=6, scale=.9]{figures/fig-stats.pdf}
	\includegraphics[page=7, scale=.9]{figures/fig-stats.pdf}
	\includegraphics[page=8, scale=.9]{figures/fig-stats.pdf}
	\caption{Salaire en fonction de l'ancienneté avec ajustements affines des moindres carrés. Trois valeurs aberrantes (en rouge) sont écartées.}
\end{figure}


Un point aberrant du nuage peut changer drastiquement l'ajustement affine de la série statistiques car son erreur d'interpolation domine celle des autres.
C'est le cas si $f(x)$ est beaucoup plus grand que $y$, ou beaucoup plus petit que celui-ci.


\section{Changements de variable}


\begin{figure}
	\centering
	\includegraphics[page=12, scale=.9]{figures/fig-stats.pdf}
	\includegraphics[page=13, scale=.9]{figures/fig-stats.pdf}
	\caption{Extrapolation affine incorrecte.}
\end{figure}

Pour décider d'un changement de variable adéquat, il est nécessaire de pouvoir reconnaître quelques courbes remarquables.



Des changements de variable plus généraux ($y=x^a$ avec $a$ quelconque) seront disponibles après l'étude du logarithme, chapitre \ref{chap:log}.



