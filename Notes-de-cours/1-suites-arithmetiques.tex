%!TEX encoding = UTF8
%!TEX root = 0-notes.tex

\chapter{Suites arithmétiques}

\section{Introduction}

\dfn{suite arithmétique}{
	Une \emphindex{suite arithmétique} $u$ est une suite vérifiant, pour tout $n\in\N$,
		\[ u(n+1) = u(n) + a, \]
	où $a\in\R$ est un nombre fixé qu'on appelle la \emphindex{raison}.
	
	On lit :
		\begin{center}
			« Pour passer d'un terme au suivant, on ajoute la raison $a$. »
		\end{center}
}{dfn:suite-arithmetique}

\exe{1}{
	Donner les 5 premiers termes de la suite arithmétique $u$ de raison 1 et de terme initial $u(0) = 0$.
	Faire une conjecture sur l'expression algébrique de $u(n)$ pour tout $n\in\N$.
}{exe:arithm1}{
	Pour passer d'un rang au suivant, on ajoute la raison, ici $1$.
	Donc $u(1) = 0+1=1, u(2) = 2, u(3) = 3, u(4) = 4$.
	
	Il semblerait que $u(n) = n$ pour tout $n\in\N$.
}

\exe{}{
	Donner les 5 premiers termes de la suite arithmétique $v$ de raison 1 et de terme initial $v(0) = 3$.
	Faire une conjecture sur l'expression algébrique de $u(n)$.
}{exe:arithm2}{
	Pour passer d'un rang au suivant, on ajoute la raison, ici $1$.
	Donc $v(1) = 3+1=4, v(2) = 5, v(3) = 6, v(4) = 7$.
	
	Il semblerait que $v(n) = n+1$ pour tout $n\in\N$.
}

\exe{}{
	Donner les 5 premiers termes de la suite arithmétique $w$ de raison -5 et de terme initial $w(0) = 5$.
	Faire une conjecture sur l'expression algébrique de $u(n)$.
}{exe:arithm3}{
	Pour passer d'un rang au suivant, on ajoute la raison, ici $-5$.
	Donc $w(1) = 5-5=0, w(2) = -5, w(3) = -10, w(4) = -15$.
	
	Il semblerait que $w(n) = -5(n-1) = -5n + 5$ pour tout $n\in\N$.
}

\exe{}{
	Montrer que la suite $u$ dont les trois premiers termes sont $u(0) = 3, u(1) = 10, u(2) = 20$ ne peut pas être arithmétique.
}{exe:nonarithm}{
	Pour passer d'un terme au suivant, on ajoute $7$ puis $10$.
	Si $u$ était arithmétique, on aurait ajouté la même quantité (sa raison). \Large\Lightning
}

\exe{}{
	Montrer que la suite $v$ telle que $v(41) = -10, v(42) = -20, v(44) = -30$ ne peut pas être arithmétique.
}{exe:nonarithm2}{
	Si $v$ était arithmétique, sa raison serait $-10$ car c'est ce qu'on ajoute pour passer du terme 41 au terme 42.
	Cependant, dans ce cas, on aurait $v(43) = -30$ et $v(44) = -40$, ce qui est contradictoire avec la donnée de l'exercice. \Large\Lightning
}

\exe{, difficulty=1}{
	Si une suite $w$ vérifie $w(10) = 1, w(11) = 3, w(12) = 5$, est-elle nécessairement arithmétique ?
}{exe:nonarithm3}{
	$w$ pourrait être arithmétique, auquel cas $w(n) = -9 + n$ pour tout $n\in\N$, mais ce n'est pas nécessairement le cas.
	On peut librement définir $w(13) = -100$ pour casser le caractère arithmétique de la suite. 
}

\exe{}{
	Soit $u$ une suite arithmétique de raison $7$ telle que $u(3) = 45$.
	Donner le terme initial de $u$.
}{exe:terme-initial}{
	Pour passer d'un rang au suivant, on doit ajouter $7$.
	Pour revenir en arrière d'un rang, on ajoute donc $-7$.
	Ainsi $u(2) = 45 - 7 = 38, u(1) = 38-7 = 31, u(0) = 31-7 = 24$.
}

\section{Expression algébrique}

\ex{}{
	La suite $u(n) = 2n$ est arithmétique de raison 2 car $u(n+1) = 2(n+1) = 2n + 2 = u(n) + 2$ pour tout $n\in\N$.
}{ex:suite-ar}

\exe{}{
	Montrer que la suite $u(n) = 7n+12$ est arithmétique. Donner sa raison et son terme initial.
}{exe:suite-ar}{
	La suite $u$ est arithmétique de raison 7 car $u(n+1) = 7(n+1) + 12 = 7n + 7 + 12 = (7n+12)+7 = u(n)+7$.
	Son terme initial est $u(0) = 12$.
}

\thm{}{
	Soit $u$ une suite arithmétique de raison $a$ et de terme initial $u(0)$.
	Alors, pour tout $n\in\N$,
		\[ u(n) = u(0) + n\cdot a. \]
}{thm:suite-arithmetique}

\pf{}{
	Pour passer du terme initial au suivant, $u(1)$, on ajoute $a$.
	Pour arriver à celui d'après, on ajoute à nouveau $a$.
	Donc pour arriver au $n$-ième terme, on doit ajoute $n$ fois $a$ : $u(n) = u(0) + n\cdot a$.
	Ceci est vrai pour n'importe quel $n\in\N$.
}

\exe{}{
	Donner l'expression algébrique de la suite $u$ définie récursivement comme suit.
		\[ \begin{cases*} u(0) = 3, \\ u(n+1) = u(n) - 1 & ($n\in\N$). \end{cases*}. \]
}{exe:arithm4}{
	$u$ est une suite arithmétique de terme initial $3$ et de raison $-1$.
	 D'après le théorème \ref{thm:suite-arithmetique}, $u(n) = 3-n$ pour tout $n\in\N$.
}

% ils savent, ça
% TODO: ajouter des exos de variations quand même

%\section{Variations}
%
%\mprop{variations}{
%	Soit $u$ une suite algébrique de raison $a$.
%	On distingue alors les trois cas suivants.
%		\begin{enumerate}[label=---]
%			\item Si $a > 0$, alors $u$ est croissante linéairement.
%			\item Si $a = 0$, alors $u$ est constante : $u(n) = u(0)$ pour tout $n\in\N$.
%			\item Si $a < 0$, alors $u$ est décroissante linéairement.
%		\end{enumerate}
%}{}

\section{Sommes arithmétiques}

\lem{}{
	\[ \sum_{k=a}^{b} k = (b-a+1)\dfrac{a+b}2. \]
}{lem:somme-arithm}

\exe{,difficulty=2}{
	Démontrer le lemme \ref{lem:somme-arithm}.
}{exe:lem-somme-arithm}{
	On se ramène au cas $a=0$ pour utiliser le théorème \ref{thm:triangle}.
		\begin{align*}
			\sum_{k=a}^b k = \sum_{k=0}^b k - \sum_{k=0}^{a-1}
								= \dfrac{b(b+1)}2 - \dfrac{a(a-1)}2
								= \dfrac{(a+b)(b-a+1)}2.
		\end{align*}
}

\thm{somme de série arithmétique}{
	Soit $u$ une suite arithmétique. Alors
		\begin{align*}
			\sum_{k=a}^b u(k) &= \text{(moyenne du premier et dernier terme) $\times$ (nombre de termes)} \\ &= \dfrac{u(a)+u(b)}2 \cdot (b-a+1).
		\end{align*}
}{thm:somme-arithm}

\exe{,difficulty=2}{
	Démontrer le théorème \ref{thm:somme-arithm}.
}{exe:somme-arithm}{
	Posons $u(n) = An +B$.
	Par linéarité de la somme, et d'après le lemme \ref{lem:somme-arithm},
		\begin{align*}
			\sum_{k=a}^b u(k) &= \sum_{k=a}^b \bigl[ Ak + B \bigr], \\
								&= A\sum_{k=a}^b k + B \sum_{k=a}^b 1, \\
								&= A \dfrac{(a+b)(b-a+1)}2 + B(b-a+1), \\
								&= (b-a+1) \dfrac{A(a+b) + 2B}2, \\
								&= (b-a+1) \dfrac{u(a) + u(b)}2.
		\end{align*}
}

\exe{}{
	Montrer que $\sum\limits_{k=a}^b u(k) = \sum\limits_{k=a-1}^{b-1} u(k+1)$.
}{exe:shiftk}{
	Expliciter la somme rend l'identité triviale.
}

\exe{, difficulty=2}{
	Posons $S_n = \sum\limits_{k=1}^{n+1} k^3$. 
	À l'aide de l'exercice \ref{exe:shiftk}, montrer qu'on a
		$S_n = \sum_{k=0}^{n} (k+1)^3.$
	En développant le cube et par linéarité de la somme, montrer que
		\[ S_n = S_n - (n+1)^3 + 3\sum_{k=0}^{n}k^2 + 3\sum_{k=0}^{n}k + n. \]
	En déduire la forme close
		\[ \sum_{k=0}^{n}k^2 = \dfrac{n(n+1)(2n+1)}6. \]
}{exe:somme-carres}{
	TODO
}

\exe{, difficulty=2}{
	En considérant la somme $S_n = \sum\limits_{k=1}^{n+1}k^4$, procéder comme à l'exercice \ref{exe:somme-carres} pour exprimer $\sum\limits_{k=1}^{n+1}k^3$ en fonction de $n$.
	En conclure que 
		\[ (1+2+\cdots+n)^2 = 1^3 + 2^3 + \dots + n^3 = \left[ \dfrac{n(n+1)}2 \right]^2. \]
}{exe:somme-cubes}{
	TODO
}

\exe{}{
	Au regard des exercices \ref{exe:somme-carres} et \ref{exe:somme-cubes}, déterminer combien d'opérations sont nécessaires pour calculer les sommes $1+2^2 + 3^2 + \cdots + n^2$ et $1+2^3 + 3^3 + \cdots + n^3$.
}{exe:complexité-somme}{
	Pour la première, il faut trois multiplications, une division, et deux additions, soit au total 6 opérations.
	Pour la seconde, il faut deux multiplications, une division, et une addition, soit au total 4 opérations.
}

% ça aussi, ils savent
% TODO: ajouter des exos de problèmes de seuil quand même

%\section{Problèmes de seuil}
