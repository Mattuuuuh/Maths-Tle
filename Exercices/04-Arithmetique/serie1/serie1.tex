%%%%%%%%%%%%%%%%%%%%%%%%%%%%%%%%%
% PACKAGE IMPORTS
%%%%%%%%%%%%%%%%%%%%%%%%%%%%%%%%%


\usepackage[french]{babel}

\usepackage[tmargin=2cm,rmargin=1in,lmargin=1in,margin=0.85in,bmargin=2cm,footskip=.2in]{geometry}

%ams
\usepackage{amsmath,amsfonts,amsthm,amssymb,mathtools}

\usepackage{bookmark}
\usepackage{enumitem}
\usepackage[most,many,breakable]{tcolorbox}
\usepackage{varwidth,etoolbox}
\usepackage[makeroom]{cancel}
\usepackage{xcolor}
\usepackage{multicol,array}
\usepackage[ruled,vlined,linesnumbered]{algorithm2e}

\usepackage{pgfplots}
\pgfplotsset{compat=1.18}
\usepackage{caption, subcaption}

%virgules
\usepackage{icomma}
\pgfplotsset{/pgf/number format/use comma}

% for striked out implies sign (\centernot\implies)
\usepackage{centernot}

% roman numerals for \section
\renewcommand{\thesection}{\Roman{section}} 

% tikz
\usepackage{tikz}
% i wish external worked but idk it sucks
%\usetikzlibrary{external}
%\tikzexternalize[prefix=figures/]

% for function graph
\usetikzlibrary{positioning}
\usetikzlibrary{shapes.geometric}
\usetikzlibrary{positioning}
\tikzset{
dot/.style = {circle, fill=#1, minimum size=5pt,
              inner sep=0pt, outer sep=0pt},
dot/.default = black % size of the circle diameter
}

 % for braces
\usetikzlibrary{decorations.pathreplacing}
% for hashing area
\usetikzlibrary{patterns}

% tableaux var, signe
% source https://www.sqlpac.com/fr/documents/latex-package-tkz-tab-tikz-tableaux-de-signes-et-de-variations-de-fonctions.html
\usepackage{tkz-tab}

%%%%%%%%%%%%%%%%%%%%%%%%%%%%%%
% SELF MADE COLORS
%%%%%%%%%%%%%%%%%%%%%%%%%%%%%%

%!TEX encoding = UTF8
%!TEX root = 0-notes.tex

%%%%%%%%%%%%%%%%%%%%%%%%%%%%%%
% SELF MADE COLORS
%%%%%%%%%%%%%%%%%%%%%%%%%%%%%%


\definecolor{myg}{RGB}{56, 140, 70}
\definecolor{myb}{RGB}{45, 111, 177}
\definecolor{myr}{RGB}{199, 68, 64}
\definecolor{mytheorembg}{HTML}{F2F2F9}
\definecolor{mytheoremfr}{HTML}{00007B}
\definecolor{mylenmabg}{HTML}{FFFAF8}
\definecolor{mylenmafr}{HTML}{983b0f}
\definecolor{mypropbg}{HTML}{f2fbfc}
\definecolor{mypropfr}{HTML}{191971}
\definecolor{myexamplebg}{HTML}{F2FBF8}
\definecolor{myexamplefr}{HTML}{88D6D1}
\definecolor{myexampleti}{HTML}{2A7F7F}
\definecolor{mydefinitbg}{HTML}{E5E5FF}
\definecolor{mydefinitfr}{HTML}{3F3FA3}
\definecolor{notesgreen}{RGB}{0,162,0}
\definecolor{myp}{RGB}{197, 92, 212}
\definecolor{mygr}{HTML}{2C3338}
\definecolor{myred}{RGB}{127,0,0}
\definecolor{myyellow}{RGB}{169,121,69}
\definecolor{myexercisebg}{HTML}{F2FBF8}
\definecolor{myexercisefg}{HTML}{88D6D1}
\definecolor{doc}{RGB}{0,60,110}

% manim colors because they're beautiful
% https://docs.manim.community/en/stable/reference/manim.utils.color.manim_colors.html

\definecolor{BLACK}{HTML}{000000}\definecolor{BLUE}{HTML}{58C4DD}\definecolor{BLUE_A}{HTML}{C7E9F1}\definecolor{BLUE_B}{HTML}{9CDCEB}\definecolor{BLUE_C}{HTML}{58C4DD}\definecolor{BLUE_D}{HTML}{29ABCA}\definecolor{BLUE_E}{HTML}{236B8E}\definecolor{DARKER_GRAY}{HTML}{222222}\definecolor{DARKER_GREY}{HTML}{222222}\definecolor{DARK_BLUE}{HTML}{236B8E}\definecolor{DARK_BROWN}{HTML}{8B4513}\definecolor{DARK_GRAY}{HTML}{444444}\definecolor{DARK_GREY}{HTML}{444444}\definecolor{GOLD}{HTML}{F0AC5F}\definecolor{GOLD_A}{HTML}{F7C797}\definecolor{GOLD_B}{HTML}{F9B775}\definecolor{GOLD_C}{HTML}{F0AC5F}\definecolor{GOLD_D}{HTML}{E1A158}\definecolor{GOLD_E}{HTML}{C78D46}\definecolor{GRAY}{HTML}{888888}\definecolor{GRAY_A}{HTML}{DDDDDD}\definecolor{GRAY_B}{HTML}{BBBBBB}\definecolor{GRAY_BROWN}{HTML}{736357}\definecolor{GRAY_C}{HTML}{888888}\definecolor{GRAY_D}{HTML}{444444}\definecolor{GRAY_E}{HTML}{222222}\definecolor{GREEN}{HTML}{83C167}\definecolor{GREEN_A}{HTML}{C9E2AE}\definecolor{GREEN_B}{HTML}{A6CF8C}\definecolor{GREEN_C}{HTML}{83C167}\definecolor{GREEN_D}{HTML}{77B05D}\definecolor{GREEN_E}{HTML}{699C52}\definecolor{GREY}{HTML}{888888}\definecolor{GREY_A}{HTML}{DDDDDD}\definecolor{GREY_B}{HTML}{BBBBBB}\definecolor{GREY_BROWN}{HTML}{736357}\definecolor{GREY_C}{HTML}{888888}\definecolor{GREY_D}{HTML}{444444}\definecolor{GREY_E}{HTML}{222222}\definecolor{LIGHTER_GRAY}{HTML}{DDDDDD}\definecolor{LIGHTER_GREY}{HTML}{DDDDDD}\definecolor{LIGHT_BROWN}{HTML}{CD853F}\definecolor{LIGHT_GRAY}{HTML}{BBBBBB}\definecolor{LIGHT_GREY}{HTML}{BBBBBB}\definecolor{LIGHT_PINK}{HTML}{DC75CD}\definecolor{LOGO_BLACK}{HTML}{343434}\definecolor{LOGO_BLUE}{HTML}{525893}\definecolor{LOGO_GREEN}{HTML}{87C2A5}\definecolor{LOGO_RED}{HTML}{E07A5F}\definecolor{LOGO_WHITE}{HTML}{ECE7E2}\definecolor{MAROON}{HTML}{C55F73}\definecolor{MAROON_A}{HTML}{ECABC1}\definecolor{MAROON_B}{HTML}{EC92AB}\definecolor{MAROON_C}{HTML}{C55F73}\definecolor{MAROON_D}{HTML}{A24D61}\definecolor{MAROON_E}{HTML}{94424F}\definecolor{ORANGE}{HTML}{FF862F}\definecolor{PINK}{HTML}{D147BD}\definecolor{PURE_BLUE}{HTML}{0000FF}\definecolor{PURE_GREEN}{HTML}{00FF00}\definecolor{PURE_RED}{HTML}{FF0000}\definecolor{PURPLE}{HTML}{9A72AC}\definecolor{PURPLE_A}{HTML}{CAA3E8}\definecolor{PURPLE_B}{HTML}{B189C6}\definecolor{PURPLE_C}{HTML}{9A72AC}\definecolor{PURPLE_D}{HTML}{715582}\definecolor{PURPLE_E}{HTML}{644172}\definecolor{RED}{HTML}{FC6255}\definecolor{RED_A}{HTML}{F7A1A3}\definecolor{RED_B}{HTML}{FF8080}\definecolor{RED_C}{HTML}{FC6255}\definecolor{RED_D}{HTML}{E65A4C}\definecolor{RED_E}{HTML}{CF5044}\definecolor{TEAL}{HTML}{5CD0B3}\definecolor{TEAL_A}{HTML}{ACEAD7}\definecolor{TEAL_B}{HTML}{76DDC0}\definecolor{TEAL_C}{HTML}{5CD0B3}\definecolor{TEAL_D}{HTML}{55C1A7}\definecolor{TEAL_E}{HTML}{49A88F}\definecolor{WHITE}{HTML}{FFFFFF}\definecolor{YELLOW}{HTML}{FFFF00}\definecolor{YELLOW_A}{HTML}{FFF1B6}\definecolor{YELLOW_B}{HTML}{FFEA94}\definecolor{YELLOW_C}{HTML}{FFFF00}\definecolor{YELLOW_D}{HTML}{F4D345}\definecolor{YELLOW_E}{HTML}{E8C11C}

%%%%%%%%%%%%%%%%%%%%%%%%%%%%
% TCOLORBOX SETUPS
%%%%%%%%%%%%%%%%%%%%%%%%%%%%

%!TEX encoding = UTF8
%!TEX root = 0-notes.tex

%%%%%%%%%%%%%%%%%%%%%%%%%%%%
% CLEAN SETUPS
%%%%%%%%%%%%%%%%%%%%%%%%%%%%

\theoremstyle{definition}

\newtheorem{theorem}{Théorème}[chapter]
\newtheorem{corollaire}[theorem]{Corollaire}
\newtheorem{lemme}[theorem]{Lemme}
\newtheorem{proposition}[theorem]{Proposition}
\newtheorem{exercice}[theorem]{Exercice}
\newtheorem{exemple}[theorem]{Exemple}
\newtheorem{definition}[theorem]{Définition}
\newtheorem*{question}{Question}
\newtheorem*{preuve}{Preuve}
\newtheorem*{remarque}{Remarque}
\newtheorem*{strategie}{Stratégie}
\newtheorem*{methode}{Méthode}
\newtheorem*{notation}{Notation}
\newtheorem*{nomenclature}{Nomenclature}
\newtheorem{axiome}[theorem]{Axiome}

\newtheorem*{definition*}{Définition}
\newtheorem*{lemme*}{Lemme}
\newtheorem*{proposition*}{Proposition}
\newtheorem*{theorem*}{Théorème}
\newtheorem*{corollaire*}{Corollaire}

%%%%%%%%%%%%%%%%%%%%%%%%%%%%
% CLEAN SETUPS : MDFRAMED SURROUND
%%%%%%%%%%%%%%%%%%%%%%%%%%%%


\usepackage[framemethod=tikz]{mdframed}
% def
\surroundwithmdframed[
	hidealllines=true,
	leftline=true,
	innerleftmargin=10pt,
	innerrightmargin=10pt,
	innertopmargin=-4pt,
	nobreak=true,
]{definition}
% thm
\surroundwithmdframed[
	%hidealllines=true,
	leftline=true,
	innerleftmargin=10pt,
	innerrightmargin=10pt,
	innertopmargin=-4pt,
	nobreak=true,
]{theorem}
\surroundwithmdframed[
	%hidealllines=true,
	leftline=true,
	innerleftmargin=10pt,
	innerrightmargin=10pt,
	innertopmargin=-4pt,
	nobreak=true,
]{proposition}


% def
\surroundwithmdframed[
	hidealllines=true,
	leftline=true,
	innerleftmargin=10pt,
	innerrightmargin=10pt,
	innertopmargin=-4pt,
	nobreak=true,
]{definition*}
% thm
\surroundwithmdframed[
	%hidealllines=true,
	leftline=true,
	innerleftmargin=10pt,
	innerrightmargin=10pt,
	innertopmargin=-4pt,
	nobreak=true,
]{theorem*}
\surroundwithmdframed[
	%hidealllines=true,
	leftline=true,
	innerleftmargin=10pt,
	innerrightmargin=10pt,
	innertopmargin=-4pt,
	nobreak=true,
]{proposition*}


%%%%%%%%%%%%%%%%%%%%%%%%%%%%
% EXERCISES 
%%%%%%%%%%%%%%%%%%%%%%%%%%%%

\usepackage[answerdelayed, lastexercise]{exercise}
\renewcommand{\ExerciseHeader}{
	\textbf{
	\theExercise.
	\theExerciseDifficulty
	}
}
\renewcommand{\DifficultyMarker}{$\star$}
\renewcommand{\AnswerHeader}{
	% if exercise title is "1" then announce new chapter
	\if\ExerciseTitle1
		{ 
		%\newpage
		\hrule\vspace{1cm}
		\LARGE
		\textbf{Exercices du chapitre \thechapter}\newline\newline
		}
	\fi
	
	\centerline{\textbf{
	Exercice \ExerciseHeaderNB
	}}
}

%%%%%%%%%%%%%%%%%%%%%%%%%%%%%%%%%%%%%%%%%%%
% MINTED FOR PYTHON ALGORITHMS
%%%%%%%%%%%%%%%%%%%%%%%%%%%%%%%%%%%%%%%%%%%


\newcommand{\python}[1]{
\inputminted[
		linenos,
		gobble=0,
		breaklines=false, % otherwise it breaks for no apparent reason?
		breakafter=,,
		fontsize=\small,
		numbersep=8pt,
		tabsize=4, % tab ident = 4 spaces
		fontfamily=courier, %important pour les signes <, >
]{python}{python/#1.py}
}

\usepackage{tcolorbox}
\tcbuselibrary{minted,breakable,xparse,skins}
\definecolor{bg}{gray}{0.95}
\DeclareTCBListing{mintedbox}{O{}m!O{}}{%
  breakable=true,
  listing engine=minted,
  listing only,
  minted language=#2,
  minted style=default,
  minted options={%
    linenos,
    gobble=0,
    breaklines=false, % otherwise it breaks for no apparent reason?
    breakafter=,,
    fontsize=\small,
    numbersep=8pt,
    tabsize=4, % tab ident = 4 spaces
    fontfamily=courier, %important pour les signes <, >
    #1},
  boxsep=0pt,
  left skip=0pt,
  right skip=0pt,
  left=25pt,
  right=0pt,
  top=3pt,
  bottom=3pt,
  arc=5pt,
  leftrule=0pt,
  rightrule=0pt,
  bottomrule=2pt,
  toprule=2pt,
  colback=bg,
  colframe=orange!70,
  enhanced,
  overlay={%
    \begin{tcbclipinterior}
    \fill[orange!20!white] (frame.south west) rectangle ([xshift=20pt]frame.north west);
    \end{tcbclipinterior}},
  #3}


%%%%%%%%%%%%%%%%%%%%%%%%%%%%
% TCOLORBOX SETUPS
%%%%%%%%%%%%%%%%%%%%%%%%%%%%

\setlength{\parindent}{1cm}
%================================
% THEOREM BOX
%================================

\tcbuselibrary{theorems,skins,hooks}
\newtcbtheorem[number within=chapter]{Theorem}{Théorème}
{%
	enhanced,
	breakable,
	colback = mytheorembg,
	frame hidden,
	boxrule = 0sp,
	borderline west = {2pt}{0pt}{mytheoremfr},
	sharp corners,
	detach title,
	before upper = \tcbtitle\par\smallskip,
	coltitle = mytheoremfr,
	fonttitle = \bfseries\sffamily,
	description font = \mdseries,
	separator sign none,
	segmentation style={solid, mytheoremfr},
}
{th}


\tcbuselibrary{theorems,skins,hooks}
\newtcolorbox{Theoremcon}
{%
	enhanced
	,breakable
	,colback = mytheorembg
	,frame hidden
	,boxrule = 0sp
	,borderline west = {2pt}{0pt}{mytheoremfr}
	,sharp corners
	,description font = \mdseries
	,separator sign none
}

%================================
% Corollary
%================================
\tcbuselibrary{theorems,skins,hooks}
\newtcbtheorem[use counter=tcb@cnt@Theorem]{Corollary}{Corollaire}
{%
	enhanced
	,breakable
	,colback = myp!10
	,frame hidden
	,boxrule = 0sp
	,borderline west = {2pt}{0pt}{myp!85!black}
	,sharp corners
	,detach title
	,before upper = \tcbtitle\par\smallskip
	,coltitle = myp!85!black
	,fonttitle = \bfseries\sffamily
	,description font = \mdseries
	,separator sign none
	,segmentation style={solid, myp!85!black}
}
{th}

%================================
% LEMMA
%================================

\tcbuselibrary{theorems,skins,hooks}
\newtcbtheorem[use counter=tcb@cnt@Theorem]{Lemma}{Lemme}
{%
	enhanced,
	breakable,
	colback = mylenmabg,
	frame hidden,
	boxrule = 0sp,
	borderline west = {2pt}{0pt}{mylenmafr},
	sharp corners,
	detach title,
	before upper = \tcbtitle\par\smallskip,
	coltitle = mylenmafr,
	fonttitle = \bfseries\sffamily,
	description font = \mdseries,
	separator sign none,
	segmentation style={solid, mylenmafr},
}
{th}


%================================
% PROPOSITION
%================================

\tcbuselibrary{theorems,skins,hooks}
\newtcbtheorem[use counter=tcb@cnt@Theorem]{Prop}{Proposition}
{%
	enhanced,
	breakable,
	colback = mypropbg,
	frame hidden,
	boxrule = 0sp,
	borderline west = {2pt}{0pt}{mypropfr},
	sharp corners,
	detach title,
	before upper = \tcbtitle\par\smallskip,
	coltitle = mypropfr,
	fonttitle = \bfseries\sffamily,
	description font = \mdseries,
	separator sign none,
	segmentation style={solid, mypropfr},
}
{th}

%================================
% Exercise
%================================

\tcbuselibrary{theorems,skins,hooks}
\newtcbtheorem[use counter=tcb@cnt@Theorem]{Exe}{Exercice}
{%
	enhanced,
	breakable,
	colback = myexercisebg,
	frame hidden,
	boxrule = 0sp,
	borderline west = {2pt}{0pt}{myexercisefg},
	sharp corners,
	detach title,
	before upper = \tcbtitle\par\smallskip,
	coltitle = myexercisefg,
	fonttitle = \bfseries\sffamily,
	description font = \mdseries,
	separator sign none,
	segmentation style={solid, myexercisefg},
}
{th}

%================================
% EXAMPLE BOX
%================================

\newtcbtheorem[use counter=tcb@cnt@Theorem]{Example}{Exemple}
{%
	colback = myexamplebg
	,breakable
	,colframe = myexamplefr
	,coltitle = myexampleti
	,boxrule = 1pt
	,sharp corners
	,detach title
	,before upper=\tcbtitle\par\smallskip
	,fonttitle = \bfseries
	,description font = \mdseries
	,separator sign none
	,description delimiters parenthesis
}
{ex}

%================================
% DEFINITION BOX
%================================

\newtcbtheorem[use counter=tcb@cnt@Theorem]{Definition}{Définition}{enhanced,
	before skip=2mm,after skip=2mm, colback=red!5,colframe=red!80!black,boxrule=0.5mm,
	attach boxed title to top left={xshift=1cm,yshift*=1mm-\tcboxedtitleheight}, varwidth boxed title*=-3cm,
	boxed title style={frame code={
					\path[fill=tcbcolback]
					([yshift=-1mm,xshift=-1mm]frame.north west)
					arc[start angle=0,end angle=180,radius=1mm]
					([yshift=-1mm,xshift=1mm]frame.north east)
					arc[start angle=180,end angle=0,radius=1mm];
					\path[left color=tcbcolback!60!black,right color=tcbcolback!60!black,
						middle color=tcbcolback!80!black]
					([xshift=-2mm]frame.north west) -- ([xshift=2mm]frame.north east)
					[rounded corners=1mm]-- ([xshift=1mm,yshift=-1mm]frame.north east)
					-- (frame.south east) -- (frame.south west)
					-- ([xshift=-1mm,yshift=-1mm]frame.north west)
					[sharp corners]-- cycle;
				},interior engine=empty,
		},
	fonttitle=\bfseries,
	title={#2},#1}{def}

%================================
% Question BOX
%================================

\makeatletter
\newtcbtheorem[use counter=tcb@cnt@Theorem]{MyQuestion}{Question}{enhanced,
	breakable,
	colback=white,
	colframe=myb!80!black,
	attach boxed title to top left={yshift*=-\tcboxedtitleheight},
	fonttitle=\bfseries,
	title={#2},
	boxed title size=title,
	boxed title style={%
			sharp corners,
			rounded corners=northwest,
			colback=tcbcolframe,
			boxrule=0pt,
		},
	underlay boxed title={%
			\path[fill=tcbcolframe] (title.south west)--(title.south east)
			to[out=0, in=180] ([xshift=5mm]title.east)--
			(title.center-|frame.east)
			[rounded corners=\kvtcb@arc] |-
			(frame.north) -| cycle;
		},
	#1
}{def}
\makeatother




%================================
% NOTE BOX
%================================

\usetikzlibrary{arrows,calc,shadows.blur}
\tcbuselibrary{skins}
\newtcolorbox{Note}[1][]{%
	enhanced jigsaw,
	colback=gray!20!white,%
	colframe=gray!80!black,
	size=small,
	boxrule=1pt,
	title=\colorbox{white!100}{\textbf{ Remarque }},
	halign title=flush center,
	coltitle=black,
	breakable,
	drop shadow=black!50!white,
	attach boxed title to top left={xshift=1cm,yshift=-\tcboxedtitleheight/2,yshifttext=-\tcboxedtitleheight/2},
	minipage boxed title=2.6cm,
	boxed title style={%
			colback=white,
			size=fbox,
			boxrule=1pt,
			boxsep=2pt,
			underlay={%
					\coordinate (dotA) at ($(interior.west) + (-0.5pt,0)$);
					\coordinate (dotB) at ($(interior.east) + (0.5pt,0)$);
					\begin{scope}
						\clip (interior.north west) rectangle ([xshift=3ex]interior.east);
						\filldraw [white, blur shadow={shadow opacity=60, shadow yshift=-.75ex}, rounded corners=2pt] (interior.north west) rectangle (interior.south east);
					\end{scope}
					\begin{scope}[gray!80!black]
						\fill (dotA) circle (2pt);
						\fill (dotB) circle (2pt);
					\end{scope}
				},
		},
	#1,
}

%================================
% STRATÉGIE BOX
%================================

\usetikzlibrary{arrows,calc,shadows.blur}
\tcbuselibrary{skins}
\newtcolorbox{Strategy}[1][]{%
	enhanced jigsaw,
	colback=myb!20!white,%
	colframe=gray!80!black,
	size=small,
	boxrule=1pt,
	title=\colorbox{white!100}{\textbf{ Stratégie }},
	halign title=flush center,
	coltitle=black,
	breakable,
	drop shadow=black!50!white,
	attach boxed title to top left={xshift=1cm,yshift=-\tcboxedtitleheight/2,yshifttext=-\tcboxedtitleheight/2},
	minipage boxed title=2.5cm,
	boxed title style={%
			colback=white,
			size=fbox,
			boxrule=1pt,
			boxsep=2pt,
			underlay={%
					\coordinate (dotA) at ($(interior.west) + (-0.5pt,0)$);
					\coordinate (dotB) at ($(interior.east) + (0.5pt,0)$);
					\begin{scope}
						\clip (interior.north west) rectangle ([xshift=3ex]interior.east);
						\filldraw [white, blur shadow={shadow opacity=60, shadow yshift=-.75ex}, rounded corners=2pt] (interior.north west) rectangle (interior.south east);
					\end{scope}
					\begin{scope}[gray!80!black]
						\fill (dotA) circle (2pt);
						\fill (dotB) circle (2pt);
					\end{scope}
				},
		},
	#1,
}

%================================
% MÉTHODE BOX
%================================

\usetikzlibrary{arrows,calc,shadows.blur}
\tcbuselibrary{skins}
\newtcolorbox{Methode}[1][]{%
	enhanced jigsaw,
	colback=white,%
	colframe=gray!80!black,
	size=small,
	boxrule=1pt,
	title=\textbf{Méthode},
	halign title=flush center,
	coltitle=black,
	breakable,
	drop shadow=black!50!white,
	attach boxed title to top left={xshift=1cm,yshift=-\tcboxedtitleheight/2,yshifttext=-\tcboxedtitleheight/2},
	minipage boxed title=2.5cm,
	boxed title style={%
			colback=white,
			size=fbox,
			boxrule=1pt,
			boxsep=2pt,
			underlay={%
					\coordinate (dotA) at ($(interior.west) + (-0.5pt,0)$);
					\coordinate (dotB) at ($(interior.east) + (0.5pt,0)$);
					\begin{scope}
						\clip (interior.north west) rectangle ([xshift=3ex]interior.east);
						\filldraw [white, blur shadow={shadow opacity=60, shadow yshift=-.75ex}, rounded corners=2pt] (interior.north west) rectangle (interior.south east);
					\end{scope}
					\begin{scope}[gray!80!black]
						\fill (dotA) circle (2pt);
						\fill (dotB) circle (2pt);
					\end{scope}
				},
		},
	#1,
}  
%================================
% AXIOME BOX
%================================

\usetikzlibrary{arrows,calc,shadows.blur}
\tcbuselibrary{skins}
\newtcolorbox{Axiome}[1][]{%
	enhanced jigsaw,
	colback=white,%
	colframe=gray!80!black,
	size=small,
	boxrule=1pt,
	title=\textbf{Axiome},
	halign title=flush center,
	coltitle=black,
	breakable,
	drop shadow=black!50!white,
	attach boxed title to top left={xshift=1cm,yshift=-\tcboxedtitleheight/2,yshifttext=-\tcboxedtitleheight/2},
	minipage boxed title=2.5cm,
	boxed title style={%
			colback=white,
			size=fbox,
			boxrule=1pt,
			boxsep=2pt,
			underlay={%
					\coordinate (dotA) at ($(interior.west) + (-0.5pt,0)$);
					\coordinate (dotB) at ($(interior.east) + (0.5pt,0)$);
					\begin{scope}
						\clip (interior.north west) rectangle ([xshift=3ex]interior.east);
						\filldraw [white, blur shadow={shadow opacity=60, shadow yshift=-.75ex}, rounded corners=2pt] (interior.north west) rectangle (interior.south east);
					\end{scope}
					\begin{scope}[gray!80!black]
						\fill (dotA) circle (2pt);
						\fill (dotB) circle (2pt);
					\end{scope}
				},
		},
	#1,
}  

%%%%%%%%%%%%%%%%%%%%%%%%%%%%%%%%%%%%%%%%%%%
% TABLE OF CONTENTS
%%%%%%%%%%%%%%%%%%%%%%%%%%%%%%%%%%%%%%%%%%%

\usepackage{titletoc}
\contentsmargin{0cm}
\titlecontents{chapter}[3.7pc]
{\addvspace{30pt}%
	\begin{tikzpicture}[remember picture, overlay]%
		\draw[fill=doc!60,draw=doc!60] (-7,-.1) rectangle (0.2,.6);%
		\pgftext[left,x=-3.5cm,y=0.2cm]{\color{white}\Large\sc\bfseries Chapitre\ \thecontentslabel};%
	\end{tikzpicture}\qquad\color{doc!60}\large\sc\bfseries}%
{}
{}
{\;\titlerule\;\large\sc\bfseries Page \thecontentspage
	\begin{tikzpicture}[remember picture, overlay]
		\draw[fill=doc!60,draw=doc!60] (2pt,0) rectangle (4,0.1pt);
	\end{tikzpicture}}%
\titlecontents{section}[3.7pc]
{\addvspace{2pt}}
{\contentslabel[\thecontentslabel]{2pc}}
{}
{\hfill\small \thecontentspage}
[]
\titlecontents*{subsection}[3.7pc]
{\addvspace{-1pt}\small}
{}
{}
{\ --- \small\thecontentspage}
[ \textbullet\ ][]

\makeatletter
\renewcommand{\tableofcontents}{%
	\chapter*{%
	  \vspace*{-20\p@}%
	  \begin{tikzpicture}[remember picture, overlay]%
		  \pgftext[right,x=15cm,y=0.2cm]{\color{doc!60}\Huge\sc\bfseries \contentsname};%
		  \draw[fill=doc!60,draw=doc!60] (13,-.75) rectangle (20,1);%
		  \clip (13,-.75) rectangle (20,1);
		  \pgftext[right,x=15cm,y=0.2cm]{\color{white}\Huge\sc\bfseries \contentsname};%
	  \end{tikzpicture}}%
	\@starttoc{toc}}
\makeatother
  

\SetDate[09/10/2025]

\begin{document}
\pagestyle{fancy}
\fancyhead[L]{Tle STMG}
\fancyhead[C]{\textbf{Suites arithmétiques}}
\fancyhead[R]{\today}


\exe{}{
	Donner les 5 premiers termes de la suite arithmétique $u$ de raison 2 et de terme initial $u_0 = 0$.
	Faire une conjecture sur l'expression algébrique de $u_n$ pour tout $n\in\N$.
}{exe:arithm1}{
	Pour passer d'un rang au suivant, on ajoute la raison, ici $1$.
	Donc $u_1 = 0+2=1, u_2 = 4, u_3 = 6, u_4 = 8$.
	
	Il semblerait que $u_n = 2n$ pour tout $n\in\N$.
}

\exe{}{
	Donner les 5 premiers termes de la suite arithmétique $v$ de raison 2 et de terme initial $v_0 = 3$.
	Faire une conjecture sur l'expression algébrique de $u_n$.
}{exe:arithm2}{
	Pour passer d'un rang au suivant, on ajoute la raison, ici $1$.
	Donc $v_1 = 3+2=5, v_2 = 7, v_3 = 9, v_4 = 11$.
	
	Il semblerait que $v_n = 2n+3$ pour tout $n\in\N$.
}

\exe{}{
	Montrer que la suite $u$ dont les trois premiers termes sont $u_0 = 3, u_1 = 10, u_2 = 20$ ne peut pas être arithmétique.
}{exe:nonarithm}{
	Pour passer d'un terme au suivant, on ajoute $7$ puis $10$.
	Si $u$ était arithmétique, on aurait ajouté la même quantité (sa raison). \Large\Lightning
}

\exe{}{
	Soit $u$ une suite arithmétique de raison $7$ telle que $u_3 = 45$.
	Donner le terme initial de $u$.
}{exe:terme-initial}{
	Pour passer d'un rang au suivant, on doit ajouter $7$.
	Pour revenir en arrière d'un rang, on ajoute donc $-7$.
	Ainsi $u_2 = 45 - 7 = 38, u_1 = 38-7 = 31, u_0 = 31-7 = 24$.
}

\exe{}{
	On considère trois suites arithmétiques.
	Compléter le nuage de points et déterminer leur expression algébrique.
	\begin{center}
	\begin{tikzpicture}[>=stealth, scale=1.1]
		\begin{axis}[xmin = 0, xmax=5.5, ymin=-2.8, ymax=4.8, axis x line=middle, axis y line=middle, axis line style=->, xlabel={rang $n$}, x label style={anchor=north}, grid=both, ytick distance=1, xtick distance = 1, x=2cm]
			\addplot[black, thick, only marks, mark=*] coordinates {(0,1) (4,3)}
				node[pos=0, left=25pt] {$v$};
			
			\addplot[black, thick, only marks, mark=star] coordinates {(0,4.5) (2,2.5)}
				node[pos=0, left=25pt] {$u$};
			
			\addplot[black, thick, only marks, mark=square] coordinates {(0,-1.5) (1,-1.5)}
				node[pos=0, left=25pt] {$w$};
		\end{axis}
	\end{tikzpicture}
	\end{center}
	
	\begin{align*}
		u_n = &&&& v_n = &&&& w_n = 
	\end{align*}
}{exe:graph}{
	On complète les nuages de points en utilisant que les points sont les valeurs entières d'une droite, et qu'ils sont donc alignés.
	On lit le terme initial en $n=0$. 
	On lit la raison en calculant la différence d'ordonnée des points d'un terme à l'autre (deux termes suffisent donc).
	\begin{center}
	\begin{tikzpicture}[>=stealth, scale=1.1]
		\begin{axis}[xmin = 0, xmax=5.5, ymin=-2.8, ymax=4.8, axis x line=middle, axis y line=middle, axis line style=->, xlabel={rang $n$}, x label style={anchor=north}, grid=both, ytick distance=1, xtick distance = 1, x=2cm]
			\addplot[black, thick, only marks, mark=*] coordinates {(0,1) (1,1.5) (2,2) (3,2.5) (4,3) (5,3.5)}
				node[pos=0, left=25pt] {$v$};
			
			\addplot[black, thick, only marks, mark=star] coordinates {(0,4.5) (1, 3.5) (2,2.5) (3,1.5) (4,0.5)}
				node[pos=0, left=25pt] {$u$};
			
			\addplot[black, thick, only marks, mark=square] coordinates {(0,-1.5) (1,-1.5) (2,-1.5) (3,-1.5) (4,-1.5) (5,-1.5)}
				node[pos=0, left=25pt] {$w$};
		\end{axis}
	\end{tikzpicture}
	\end{center}
	
	\begin{align*}
		u_n = 4,5 - n  &&&& v_n = 1 + 0,5n &&&& w_n = -1,5
	\end{align*}
}

\exe{}{
	Un professeur hésite entre louer un appartement ou l'acheter.
	\begin{itemize}
		\item
		D'une part, le loyer mensuel de l'appartement est de $700$€, avec un coût initial (la caution) fixé à deux loyers, soit $1\ 400$€.
		\item
		D'autre part, à l'achat, cet appartement coûte $140~000$€.
	\end{itemize}
	
	Dans la première alternative, lorsque l'appartement est loué, on dénote par $u_n$ le montant total payé en euros par le locataire après $n$ mois (et ceci pour tout $n\in\N$).
	Ainsi, par exemple, après $1$ mois, un seul loyer a été payé en plus de la caution initiale. D'où $u_1 = 2~100$.
	\begin{enumerate}
		\item Donner le terme initial $u_0$.
		\item Donner $u_2$, le montant payé par le locataire après $2$ mois.
		\item Justifier du caractère arithmétique de la suite $u$. Quelle est sa raison ?
		\item Pour tout $n\in\N$, donner $u_n$ en fonction de $n$ sans justifier.
		\item À partir de combien de mois le locataire aura-t-il payé plus que le coût de l'appartement à l'achat ? Convertir le résultat en années.
	\end{enumerate}
}{exe:loyer-achat}{
	\begin{enumerate}
		\item $u_0 = 1~400$.
		\item $u_2 = u_1 + 700 = 2~100 + 700 = 2~800$.
		\item D'un mois à l'autre, on ajoute 700 au montant total payé.
		La raison est donc de 700.
		\item D'après le cours, $u_n = 1400 + 700n = 700(2+n)$.
		\item 
		On pose et on résoud $u_n = 140~000$.
			\begin{align*}
				u_n &= 140~000 \\
				1400 + 700n &= 140~000 \\
				700(2+n) &= 140~000 \\
				2+n &=200 \\
				n &= 198
			\end{align*}
		Par conséquent, au bout de 198 mois, le locataire aura payé le coût de l'appartement à l'achat.
		En divisant par 12, on convertit le résultat en $\frac{198}{12} = \frac{99}{6} = \frac{33}2 = 16,5$ années.
	\end{enumerate}
}

\newpage

\exe{}{
	Le mathématicien français Abraham de Moivre, alors âgé de plus que 80 ans, étudia dans les années 1750 la durée de son sommeil ; il remarqua que celle-ci augmentait de façon inquiétante.
	Il prit donc note chaque jour de son heure de coucher et de réveil et compara avec une valeur initiale : la nuit $0$, il dormit $8$ heures.
	
	Au fil des jours, il compta que son temps de sommeil augmentait chaque nuit de $15$ minutes, soit $\frac14$ d'heure.
	En notant $v_n$ la durée de son sommeil en heures lors de sa $n$-ième nuit, il put alors calculer la nuit de la mort : celle où il dormirait $24$ heures.
	\begin{enumerate}
		\item Donner le terme initial $v_0$.
		\item Donner $v_4$, la durée de son sommeil en heures lors de la $4$-ème nuit.
		\item Pourquoi la suite $v$ est-elle arithmétique ? Quelle est sa raison ?
		\item Pour tout $n\in\N$, donner $v_n$ en fonction de $n$ sans justifier.
		\item Quelle nuit Abraham de Moivre mourut-t-il ?
		\item Combien de temps, au total, Abraham de Moivre a-t-il dormi à partir de la nuit initiale de 8h et jusqu'à la nuit de sa mort ?
		Exprimer le résultat en jours.
	\end{enumerate}
}{exe:Moivre}{
	\begin{enumerate}
		\item $v_0 = 8$, car on compte en heures.
		\item $v_4 = 8 + \frac14 \times 4 = 9$.
		\item D'une nuit à l'autre, de Moivre dort $\frac14$ d'heure en plus.
		La suite est donc arithmétique de raison $\frac14$.
		\item D'après le cours, $v_n = 8 + \frac14 n$.
		\item 
		On pose et on résoud $v_n = 24$.
			\begin{align*}
				v_n &= 24 \\
				8 + \frac14 n &= 24 \\
				\frac14n &= 16 \\
				n &= 4\times16 = 48
			\end{align*}
		On conclut que de Moivre meurt la 48-ème nuit après le début de son étude.
		\item
		On somme les termes de la suite de $n=0$ à $n=48$.
			\[
				\sum_{k=0}^{48} (8 + \frac14 n) = \dfrac{8+24}2 \times 49 = 784
			\]
		De Moivre a donc dormi 784 heures au total, soit 32 jours et 16 heures.
	\end{enumerate}
}

\exe{}{
	Parmis les suites suivantes, lesquelles sont arithmétiques ? 
	Si la suite est arithmétique, donner son terme initial et sa raison.
	Sinon, justifier qu'elle n'est pas arithmétique.
	\begin{multicols}{2}
	\begin{enumerate}[itemsep=5pt]
		\item $u_n = 3n - 4$
		\item $w_n = n^2 + 3$
		\item $v_n = 13$
		\item $f_n = \dfrac{1}{2n+1}$
	\end{enumerate}
	\end{multicols}
}{exe:arithm}{
	\begin{enumerate}[itemsep=5pt]
		\item 
		$u$ est de la forme $an+b$ avec $a=3$, sa raison, et $b=-4$, son terme initial.
		\item 
		$w$ ne semble pas arithmétique.
		Montrons que ses trois premiers termes ne forment pas une suite arithmétique.
		Ses trois premiers termes sont $w_0 = 3, f_1 = 4,$ et $f_2 = 7$.
		
		Comme $f_1 - f_0 = 1$ et que $f_2 - f_1 =3 \neq 1$, la suite n'est pas arithmétique.
		\item
		$v$ est de la forme $an+b$ avec $a=0$, sa raison, et $b=13$, son terme initial.
		\item 
		$f$ ne semble pas arithmétique.
		Montrons que ses trois premiers termes ne forment pas une suite arithmétique.
		Ses trois premiers termes sont $f_0 = 1, f_1 = \frac13,$ et $f_2 = \frac15$.
		
		Comme $f_1 - f_0 = -\frac23$ et que $f_2 - f_1 = -\frac{2}{15} \neq - \frac23$, la suite n'est pas arithmétique.
	\end{enumerate}
}


%\exe{}{
%	On étudie la croissance d'une population de champignons.
%	
%	Au début de l'expérience, on dispose de 100 champignons.
%	Toutes les dix minutes, on mesure l'évolution de leur nombre.
%	On obtient les résultats suivants.
%	
%	\begin{multicols}{2}
%	\begin{center}
%	\begin{tabular}{|c|c|}\hline
%		\thead{Temps écoulé \\ (en minutes)} & \thead{Nombre de \\ champignons} \\ \hline
%		0 & 100 \\\hline
%		10 & 125 \\\hline
%		20 & 150 \\\hline
%		30 & 175 \\ \hline
%	\end{tabular}
%	
%	\begin{tikzpicture}[>=stealth, scale=1]
%		\begin{axis}[xmin = -1, xmax=35, ymin=-10, ymax=180, axis x line=middle, axis y line=middle, axis line style=->, xlabel={temps écoulé (en minutes)}, ylabel={nombres de champignons}, grid=both, ytick distance=50, xtick distance = 10]
%			\addplot[black, thick, only marks, mark=*] coordinates {(0, 100) (10, 125) (20, 150) (30,175)};
%		\end{axis}
%	\end{tikzpicture}
%	\end{center}
%	\end{multicols}
%	
%	Soit $n$ un entier naturel.
%	On note $u_n$ le nombre de champignons après $n$ périodes de \textbf{dix} minutes.
%	Ainsi, $u_0 = 100, u_1 = 125, u_2 = 150, \dots$.
%	\begin{enumerate}
%		\item Justifier que les termes $u_0, u_1, u_2, u_3$ sont en progression arithmétique.
%		\item En supposant que la population de champignons continue d'évoluer selon le même rythme, montrer qu'elle aura quadruplé deux heures après le début de l'expérience.
%	\end{enumerate}
%}{exe:zero-2}{
%	\begin{enumerate}
%		\item
%		On calcule $u_1 - u_0 = 25, u_2 - u_1 = 25,$ et $u_3 - u_2 = 25$.
%		Pour passer d'un terme à l'autre, on ajoute 25 : la progression est arithmétique.
%		\item 
%		On a $u_n = 100 + 25n$ d'après le cours.
%		Comme 2 heures sont $2 \times60 = 120$ minutes, il s'agit de calculer le terme $u(12)$, la suite étant indexée en dixaines de minutes.
%		\begin{align*}
%			u(12) &= 100 + 25\times12 \\
%					&= 100 + 25 \times 4 \times 3 \\
%					&= 100 + 100 \times 3 \\
%					&= 400
%		\end{align*}
%		On obtient bien $u(12) = 400 = 4 \times u_0$, comme annoncé.
%	\end{enumerate}
%}



% © Benjamin Clamme
\exe{}{
	Milon de Crotone était un athlète de la Grèce antique (né aux alentours
	de 550 avant Jésus Christ). Un mythe à son propos explique que, pour son entraînement, il
	décida de soulever un jeune veau tous les jours. Le veau grandissant, la charge soulevée par
	Milon augmentait progressivement. Lorsque le veau devint adulte, Milon pouvait toujours le
	soulever, par son entraînement il avait acquis, petit à petit, une force Herculéenne.
	\begin{enumerate}
		\item
		On suppose que le veau de Milon pesait 40 kg à sa naissance et gagnait 500g chaque jour.
		Justifier du caractère arithmétique de l'évolution du poids du veau.
		\item 
		Milon s’entraîne ainsi durant une année entière. 
		Quel charge est-il capable de soulever à l’issue de son entraînement ?
		\item
		Quelle charge cumulée (la somme des charges journalières) Milon a-t-il soulevé durant cette année d’entraînement ?
	\end{enumerate}
}{exe:somme-arithm}{
	\begin{enumerate}
		\item
		Pour passer d'un poids au suivant, on ajoute 500g, un nombre fixe (la raison).
		\item 
		Nommons $u_n$ le poids du veau au jour $n\in\N$ en kilogrammes.
		D'après le texte, $u_0 = 40$.
		D'après la question précédente, $u$ est arithmétique de raison $0,5$ (car on exprime tout en kilogrammes pour ne pas mélanger les unités).
		
		D'après le cours, on a $u_n = 0,5 n + 40$.
		Il suit que $u_{365} = 0,5 \times 365 + 40 = 222,5$, le poids du veau en kilogrammes après 365 jours.
		\item
		On calcule la somme des 366 premiers termes (0 à 365) à l'aide du cours.
			\[ \sum_{k=0}^{365}u_k = \dfrac{40 + 222,5}{2} \times 366 = 48037,5. \]
	\end{enumerate}
}


%%%%%%%%%%%%

\newpage
\fancyhead[C]{\textbf{Solutions}}
\shipoutAnswer

\end{document}
