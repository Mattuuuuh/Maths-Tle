\documentclass[14pt]{beamer}
\usepackage[french]{babel}

\usetheme{CambridgeUS}
\usecolortheme{rose}
\beamertemplatenavigationsymbolsempty


\usepackage{libertinus}
\usepackage{amsmath,amsfonts,amsthm,amssymb,mathtools}
\usepackage{array}
\newcolumntype{P}[1]{>{\centering\arraybackslash}p{#1}}


\usepackage{stackengine}
\newcommand\xrowht[2][0]{\addstackgap[.5\dimexpr#2\relax]{\vphantom{#1}}}


% corps
\usepackage{calrsfs}
\newcommand{\C}{\mathcal{C}}
\newcommand{\R}{\mathbb{R}}
\newcommand{\Rnn}{\mathbb{R}^{2n}}
\newcommand{\Z}{\mathbb{Z}}
\newcommand{\N}{\mathbb{N}}
\newcommand{\Q}{\mathbb{Q}}

% domain
\newcommand{\D}{\mathcal{D}}


% date
\usepackage{advdate}
\AdvanceDate[0]

%plots
\usepackage{pgfplots, subcaption}
\definecolor{myg}{RGB}{56, 140, 70}
\definecolor{myb}{RGB}{45, 111, 177}
\definecolor{myr}{RGB}{199, 68, 64}

%boxes
\usepackage[most]{tcolorbox}
\usepackage{multicol}

%icomma
\usepackage{icomma}

%https://osl.ugr.es/CTAN/macros/latex/contrib/tcolorbox/tcolorbox.pdf
\newtcolorbox{mybox}[3][]
{
  colframe = #2!25,
  colback  = #2!10,
  coltitle = #2!20!black,  
  halign title=flush center, 
  title    = {#3},
  #1,
}

% BOX A BOX B
\newcommand{\boxAB}[2]{
		\begin{mybox}{red}{A}
		\begin{center}
			#1
		\end{center}
		\end{mybox}
		\begin{mybox}{green}{B}
		\begin{center}
			#2
		\end{center}
		\end{mybox}
}

%systèmes
\usepackage{systeme}

% trafficotage 
\usepackage[answerdelayed, lastexercise]{exercise}
\renewcommand{\ExerciseHeader}{
}
\renewcommand{\AnswerHeader}{
}

\newcommand{\framedelayed}[3][]{
	\begin{Exercise}
	\begin{frame}{\theExercise #1\vspace{-32pt}}
		#2
	\end{frame}
	\end{Exercise}
	\begin{Answer}
	\begin{frame}{\theExercise #1\vspace{-32pt}}
		#3
	\end{frame}
	\end{Answer}
}


\SetDate[16/10/2025]

\begin{document}
\pagestyle{fancy}
\fancyhead[L]{Tle STMG}
\fancyhead[C]{\textbf{Suites géométriques}}
\fancyhead[R]{\today}


\exe{}{
	Achille dispute une course avec une tortue. On suppose que les deux participants avancent à vitesse constante et que la tortue avance $10$ fois moins vite qu'Achille.
	Celui-ci décide donc de lui laisser généreusement $10$ minutes d'avance.

	En analysant la situation, Zénon décide de diviser la course en plusieurs étapes.
	À chaque étape, Achille court jusqu'au point d'où a démarré la tortue à la dernière étape.
	Il déduit que, comme la tortue avance pendant qu'Achille court, il ne pourra jamais la dépasser.

	\begin{enumerate}
		\item Vérifier les premières valeurs du tableau suivantes et le compléter.
			\begin{center}
			\begin{tabular}{|c|c|c|c|c|c|}\hline
				Étape & 0 & 1 & 2 & 3 & 4 \\ \hline
				Temps qu'Achille met pour finir l'étape (min) & 1 & 0,1 && & \\ \hline
			\end{tabular}
			\end{center}
		\item En déduire l'évolution du temps que prend Achille d'une étape à l'autre :
		en notant $T_n$ ce temps à l'étape $n=0, 1, 2, 3,\dots$, montrer que
			\[ T_{n+1} = \dfrac1{10} T_n. \]
		\item Donner exactement $T(100)$ en écriture scientifique.
		\item On considère la somme des temps de chaque étape pour comprendre quand Achille atteindra la tortue.
			\[ S_n = \sum_{k=0}^n T_k = T_0 + T_1 + \dots + T_n. \]
		Vérifier les premières valeurs du tableau suivantes et le compléter.
			\begin{center}
			\begin{tabular}{|c|c|c|c|c|c|}\hline
				$n$ & 0 & 1 & 2 & 3 & 4 \\ \hline
				$S_n$ & 1 & 1,1 & & & \\ \hline
			\end{tabular}
			\end{center}
		\item Est-ce que $S_n$ peut être aussi grand qu'on le souhaite ? Par exemple, existe-t-il un rang $n$ pour lequel $S_n \geq 2$ ?
		\item En combien de temps Achille arrive-t-il à dépasser la tortue ? Donner une valeur exacte sous forme de fraction.
	\end{enumerate}
}{exe:AchTor}{

			\begin{center}
			\begin{tabular}{|c|c|c|c|c|c|}\hline
				Étape & 0 & 1 & 2 & 3 & 4 \\ \hline
				Temps qu'Achille met pour finir l'étape (min) & 1 & 0,1 & \color{red} 0,01 & \color{red} 0,001 & \color{red} $10^{-4}$ \\ \hline
			\end{tabular}
			\end{center}
			
	\begin{enumerate}
		\item Cf. tableau de l'énoncé.
		\item D'une étape à l'autre, le temps pris par Achille pour rattraper la tortue est divisé par $10$ car celui-ci va $10$ fois plus vite.
		\item $T(100) = 10^{-100}$.
		\item On complète en utilisant que $S_2 = T_0 + T_1 + T_2$, puis $S(3) = T_0 + T_1 + T_2 + T(3)$, et idem pour $S(4)$.
		\item 
		$S_n$ ne dépasse jamais $1,2$ et donc ne peut pas être aussi grand que souhaité.
		\item 
		On cherche à étudier la limite de $S_n$ quand $n$ tend vers l'infini.
		On souhaite donc déterminer $x = 1,1111\dots$ avec une infinité de $1$.
		En multipliant par $3$, on remarque que
			\[ 3x = 3,333\dots = \dfrac{10}3. \]
		On en déduit que $x= \dfrac{10}9$, qu'on vérifiera avec une calculatrice par exemple.
	\end{enumerate}

			\begin{center}
			\begin{tabular}{|c|c|c|c|c|c|}\hline
				$n$ & 0 & 1 & 2 & 3 & 4 \\ \hline
				$S_n$ & 1 & 1,1 & \color{red} $1,11$  & \color{red} $1,111$ &\color{red} $1,1111$ \\ \hline
			\end{tabular}
			\end{center}
}


\exe{}{

	À l'âge de $17$ ans une élève décide de placer $200$€ en bourse qui lui rapportent $10\%$ d'intérêts chaque année.
	Chaque année, elle replace les intérêts gagnés.
	
	On souhaite étudier l'évolution de l'argent placé chaque année après ses $17$ ans inclus.
	\begin{enumerate}
		\item Vérifier les premières valeurs du tableau suivantes et le compléter.
			\begin{center}
			\begin{tabular}{|c|c|c|c|c|c|c|}\hline
				Âge & 17 & 18 & 19 & 20 & 21 & 22 \\ \hline
				Argent placé (€) & 200 & 220 & 242 &  & & \\ \hline
			\end{tabular}
			\end{center}
		\item On appelle $A_n$ la quantité d'argent placé à l'âge $17+n$, où $n\in \{0 ; 1 ; 2; \dots \}$.
		Décrire comment obtenir $A_{n+1}$ en connaissant $A_n$
		.%, c'est-à-dire comment obtenir la quantité d'argent placé en l'an $2017+_{n+1}$ en connaissant la quantité en $2017+n$.
		
		\item Combien d'argent aura l'élève à l'âge de $50$ ans ? 
		\item Calculer $A(50)$ et interpréter le résultat.
		\item À quel âge la somme d'argent dépassera-t-elle $100 \  000$€ ?		
	\end{enumerate}
 }{exe:interets}{
 
			\begin{center}
			\begin{tabular}{|c|c|c|c|c|c|c|}\hline
				Âge & 17 & 18 & 19 & 20 & 21 & 22 \\ \hline
				Argent placé (€) & 200 & 220 & 242 & \color{red} $266,2$  & \color{red} $292,82$ &\color{red} $322,102$   \\ \hline
			\end{tabular}
			\end{center}
			
	\begin{enumerate}
		\item Un gain de $10\%$ correspond à un coefficient multiplicateur de $1+\dfrac{10}{100} = 1,1$.
		\item 
		Pour obtenir $A_{n+1}$, on multiplie $A_n$ par $1,1$. On a donc la relation de récurrence suivante.
			\[ A_{n+1} = 1,1 \times A_n, \]
		valable pour tout $n\in\N$ entier naturel.
		\item 
		À l'âge de $50 = 17+n$ ans, on doit calculer $A_n$ pour $n=50-17 = 34$, et donc $A(34)$.
		
		On utilise l'expression algébrique de $A_n$ qui est
			\[ A_n = 200 \times (1,1)^n, \]
		pour trouver $A(34) \approx 5109,5.$
		\item $A(50) = 200 \times (1,1)^{50} \approx 23~478,2$.
		C'est l'argent économisé à l'âge de $17+50 = 67$ ans.
		\item 
		On résoud à l'aide d'un tableau de valeurs ou par dichotomie.
		Comme $A(65) < 100~000 < A(66)$, le plus petit $n$ pour lequel $A_n$ dépasse $100~000$ est $66$.
		Pour $n=66$, l'âge correspondant est $17+66 = 83$ ans.
	\end{enumerate}
 
 }
 
 
 \exe{}{
 
 	On reprend l'exercice \ref{ex:2} en prenant en plus en compte l'inflation, qu'on suppose constante à $2\%$ par an.
 	Cela signifie que les prix augmentent en moyenne de $2\%$ par an.
 	
 	Afin de rendre comparables les sommes d'argent dans le temps, on souhaite fixer les prix : au lieu de voir les prix comme augmentant, on voit l'euro comme se dépréciant.
 
	\begin{enumerate}
		\item Calculer l'évolution réciproque de $+2\%$. 
		C'est la diminution qu'on appliquera à l'euro chaque année.
		\item Vérifier les premières valeurs du tableau suivantes et le compléter.
			\begin{center}
			\begin{tabular}{|c|c|c|c|c|c|c|}\hline
				Année & 17 & 18 & 19 & 20 & 21 & 22 \\ \hline
				Argent placé (€, prix fixes) & 200 & 215,69 & 232,60 &&&  \\ \hline
			\end{tabular}
			\end{center}
		\item On appelle $B_n$ la quantité d'argent placé à prix fixes en l'an $2017+n$, où $n\in \{0 ; 1 ; 2; \dots \}$.
		Décrire comment obtenir $B_{n+1}$ en connaissant $B_n$
		.%, c'est-à-dire comment obtenir la quantité d'argent placé en l'an $2017+_{n+1}$ en connaissant la quantité en $2017+n$.
		
		\item Calculer $B(70)$ et interpréter le résultat.
	\end{enumerate}
 }{exe:interets2}{
 
			\begin{center}
			\begin{tabular}{|c|c|c|c|c|c|c|}\hline
				Année & 17 & 18 & 19 & 20 & 21 & 22 \\ \hline
				Argent placé (€, prix fixes) & 200 & 215,69 & 232,60 &  \color{red} $250,84$  &  \color{red} $270,5$  &  \color{red} $291,7$   \\ \hline
			\end{tabular}
			\end{center}
 	\begin{enumerate}
		\item 
		Le coefficient multiplicateur de l'augmentation de $2\%$ est $\times1,02$.
		La réciproque est donnée par la multiplication par $\dfrac1{1,02} \approx 0,98 = 1 - 0,02$.
		C'est donc une diminution de $2\%$ à appliquer après les intérêts chaque année.
		\item 
			Pour passer d'un terme à l'autre, on multiplie par $\dfrac{1,1}{1,02}$.
		\item 
			\[ B_{n+1} = \dfrac{1,1}{1,02} \times B_n. \]
		
		\item 
			L'expression algébrique de $B_n$ est 
				\[ B_n = 200 \times \left(\dfrac{1,1}{1,02}\right)^n, \]
			qui donne $B(70) = 39~491,7$.
			C'est l'argent à prix fixes accumulé après $17+70 = 87$ ans.
	\end{enumerate}
 
 
 }
 
 
 
\exe{}{
	Pour chacune des suites données algébriquement pour tout $n\in\N$, décider si elle est géométrique ou non.
	\begin{multicols}{2}
	\begin{enumerate}
		\item $a_n = 3^n$
		\item $f_n = 3n + 2$
		\item $b_n = \left(\dfrac25\right)^n$
		\item $c_n = 5 \times 2^n$
		\item $g_n = 3-n$
		\item $h_n =  \dfrac3{n+1}$
	\end{enumerate}
	\end{multicols}

}{exe:geom-or-not}{
	\begin{enumerate}
		\item 
		$a$ est géométrique car elle respecte le théorème du cours avec $a_0 = 1$ et $q=3$, car $3^n = 3^n \times 1$.
		On peut également utiliser la définition d'une suite géométrique et que 
			\[ a_{n+1} = 3^{n+1} = 3^{1} \times 3^{n} = 3 \times a_n. \]
		\item 
		Supposons que $f$ soit géométrique non nulle.
		Alors le ratio $\dfrac{f_{n+1}}{f_n} = q$, et est donc constant quelque soit $n$.
		
		En $n=0$, on calcule
			\[\dfrac{f_1}{f_0} = \dfrac52. \]
		En $n=1$, on calcule
			\[\dfrac{f_2}{f_1} = \dfrac85. \]
		Comme $\dfrac52 \neq \dfrac85$, la suite $f$ ne peut pas être géométrique (elle est arithmétique en fait).
		
		\item $b_n = \left(\dfrac25\right)^n \times 1$, donc elle est géométrique.
		\item $c_n = 5 \times 2^n$ est géométrique.
		\item On calcule deux ratios successifs.
			\begin{align*}
				\dfrac{g_1}{g_0} = \dfrac23, && \neq &&  \dfrac{g_2}{g_1} = \dfrac12,
			\end{align*}
		ce qui implique que $g$ ne peut pas être géométrique.
		\item On calcule deux ratios successifs.
			\begin{align*}
				\dfrac{h_1}{h_0} = \dfrac12, &&  \neq && \dfrac{h_2}{h_1} = \dfrac23,
			\end{align*}
		ce qui implique que $h$ ne peut pas être géométrique.
	\end{enumerate}

}

\exe{}{
	Pour chacune des suites géométriques données algébriquement pour tout $n\in\N$, donner sa raison et son terme initial.
	\begin{multicols}{2}
	\begin{enumerate}
		\item $u_n = 2 \times 3^n$
		\item $v_n = 7 \times \left(\dfrac12 \right)^n$
		\item $w_n = (-6)^n$
		\item $\zeta_n = - 6^n$
		\item $a_n = 11 \times 5^{2n}$
		\item $b_n = 3 \times 5^{2n+3}$
		\item $c_n = 10^{-n}$
		\item $d_n = \dfrac{4}{7^n}$
	\end{enumerate}
	\end{multicols}
}{exe:q-ti}{
	Comme on sait que les suites sont géométriques, le terme initial est donné par $u_0$ et la raison par $\dfrac{u_1}{u_0}$.
	Il suffit donc de savoir évaluer les suites en $0$ et $1$ pour conclure.
	L'identité $a^0 = 1$ du théorème sera utile.

	\begin{enumerate}
		\item 
			\begin{align*}
				u_0 = 2 && q = 3
			\end{align*}
		\item 
			\begin{align*}
				v_0 = 7 && q = \dfrac12
			\end{align*}
		\item 
			\begin{align*}
				w_0 = 1 && q = -6
			\end{align*}
		\item 
			\begin{align*}
				\zeta_0 = -1 && q = 6
			\end{align*}
		\item 
			\begin{align*}
				a_0 = 11 && q = 25
			\end{align*}
		\item 
			\begin{align*}
				b_0 = 375 && q = 25
			\end{align*}
		\item 
			\begin{align*}
				c_0 = 1 && q = \dfrac{1}{10}
			\end{align*}
		\item 
			\begin{align*}
				d_0 = 4 && q = \dfrac17
			\end{align*}
	\end{enumerate}
}

% à mettre dans chapitre log
%\exe{
%	Trouver le plus petit entier naturel $n\in\N$ vérifiant les inéquations suivantes.
%	\begin{multicols}{2}
%	\begin{enumerate}
%		\item $2 \times 3^n > 100~000$
%		\item $7 \times \left(\dfrac32 \right)^n > 50~000$
%		\item $7 \times \left(\dfrac32 \right)^n > 500~000$
%		\item $3 \times \left( \dfrac43 \right)^n > 1~000$
%		\item $3 \times \left( \dfrac43 \right)^n > 10~000$
%		\item $3 \times \left( \dfrac43 \right)^n > 100~000$
%	\end{enumerate}
%	\end{multicols}
%}{
%	On obtient par dichotomie ou avec un tableau de valeurs les rangs suivants.
%	
%	\begin{multicols}{2}
%		\begin{enumerate}
%			\item $n=10$
%			\item $n=22$
%			\item $n=28$
%			\item $n=21$
%			\item $n=29$
%			\item $n=37$
%		\end{enumerate}
%	\end{multicols}
%
%}

\exe{}{
	Les suites suivantes données graphiquement peuvent-elles être géométriques ?
	
	\begin{center}
	\begin{tikzpicture}[>=stealth, scale=1.5]
		\begin{axis}[xmin = 0, xmax=4.2, xtick={ 0,1,2, 3, 4,5}, ymin=0, ymax=300, ytick={0, 30, ..., 300}, axis x line=middle, axis y line=middle, axis line style=->, ylabel={}, grid=both]
			
			\addplot[black, thick, only marks, mark=star] coordinates {(0, 130) (1,140) (2,150) (3,160) (4,170)};
			
			\addplot[black, thick, only marks, mark=square] coordinates {(0,15) (1,30) (2,60) (3, 120) (4, 240)};
			
			\addplot[black, thick, only marks, mark=*] coordinates {(1,300) (2,200) (3,133.33) (4, 88.89)};
		\end{axis}
	
	\end{tikzpicture}
	\end{center}

}{exe:graph1}{
	On calcule les ratios successifs : s'ils ne sont pas constants, alors la suite n'est pas géométrique.
	Dans le cas contraire, on ne peut pas réellement conclure que la suite est géométrique, car nous n'avons qu'un échantillons restreint des images (et donc des ratios).
	
	Seule la suite $\star$ peut être écartée ici.
}

\exe{}{
	Donner le terme de rang $n \in \N$ des suites géométriques $\star$, $\bullet$, et $\square$ données graphiquement.

	\begin{center}
	\begin{tikzpicture}[>=stealth, scale=1.5]
		\begin{axis}[xmin = 0, xmax=4.2, xtick={ 0,1,2, 3, 4,5}, ymin=0, ymax=10000, ymode=log, log ticks with fixed point, axis x line=middle, axis y line=middle, axis line style=->, ylabel={}, grid=both]
			
			\addplot[black, thick, only marks, mark=star] coordinates {(0, 1) (1,10) (2,100) (3,1000) (4,10000)};
			
			\addplot[black, thick, only marks, mark=square] coordinates {(0,10 000) (1,10 00) (2,100) (3,10) (4,1)};
			
			\addplot[black, thick, only marks, mark=*] coordinates {(0,3) (1,30) (2,300) (3, 3000)};
		\end{axis}
	
	\end{tikzpicture}
	\end{center}

}{exe:graph2}{
	Dans ce repère à échelle logarithmique, la première graduation des ordonnées est $1$.
	On obtient donc
		\begin{align*}
			\star_n = 1 \times 10^n, && \square_n = 10~000\times \left(\dfrac1{10}\right)^n && \bullet_n = 3 \times 10^n
		\end{align*} 
}

\exe{}{
    Considérons la suite $v$ définie par la relation de récurrence suivante, pour tout $n\in\N$.
    \[
    \begin{cases}
        v_0 = 0, \\
        v_{n+1} = 3v_n + 2.
    \end{cases}
    \]
    \begin{enumerate}
        \item Montrer que la suite intermédiaire $w$ définie pour tout $n\in\N$ par
            \[ w_n = v_n + 1,\]
        est géométrique.
        \item En déduire que, pour tout $n\in\N$,
            \[ v_n = 3^n - 1.\]
    \end{enumerate}
}{exe:arithmetico-geom}{
	todo
}


\exe{}{
	Jean de Florette élève une population de lapin de la race Romarin.
	Il étudie le nombre de lapins pendant plusieurs années avant de remarquer que, d'une année à l'autre, la population augmente systématiquement de $50\%$.
	Il note les premières valeurs obtenues dans le tableau en commençant à l'année $0$ pour simplifier les choses.
	\begin{center}
	\begin{tabular}{|c|c|c|c|c|c|}\hline
	Année & 0 & 1 & 2 & 3 & 4 \\ \hline
	Population & 256 & 384 & & & \\ \hline
	\end{tabular}
	\end{center}
	
	\begin{enumerate}
		\item Vérifier les premières valeurs du tableau ci-dessus et le compléter.
		\item En notant $P_n$ le nombre de lapins à l'année $n$, écrire l'expression algébrique de $P_n$.
		\item Donner $P(20)$ en arrondissant à l'entier le plus proche.
		\item Trouver le plus petit entier naturel $N\in\N$ tel que
		\[ P(N) \geq 100~000.\]
		\item Somme ?
	\end{enumerate}
}{exe:JeanFlorette}{
	todo
}

\exe{, difficulty=1}{
	Lors de votre entretien d'embauche, une entreprise vous propose un salaire de départ de 35 000€ avec deux choix d'évolutions possibles pour ce salaire
	
	\begin{center}
	\begin{enumerate}[start=0,label={\bfseries Choix~\arabic*:},leftmargin=3cm]
	\item une augmentation annuelle de 2\% ;
	\item une augmentation annuelle de 7 00€.
	\end{enumerate}
	\end{center}
	\begin{enumerate}
	\item Calculer le salaire (arrondi à l'euro près) à la 20ème année pour chacun des choix.
	\item Quel choix est, selon vous, le plus avantageux ? Expliquer.
	\item Selon les statistiques de l'OCDE, les français restent en moyenne 11 ans dans une même entreprise. En tenant compte de cette information, quel choix est probablement le plus avantageux ?
	\end{enumerate}
}{exe:exo9}{
	todo
}

\exe{}{
	Un ami vous propose de créer une entreprise dont le principe est le suivant :
	\begin{itemize}
		\item Vous recrutez des ``investisseurs'' en leurs proposant de verser 1 000€
		et en leur promettant qu'ils doubleront leur mise après une semaine. 
		\item Pour payer le premier investisseur après une semaine, vous recrutez deux nouveaux investisseurs.
		\item Vous poursuivez ainsi, en recrutant toujours deux nouveaux investisseurs pour en payer un.
	\end{itemize}
	On note $(u_n)$ le nombre de nouveaux investisseurs recrutés en semaine numéro $n$. On suppose que l'entreprise démarre en semaine 0 avec un seul investisseur.
	\begin{enumerate}
		\item Calculer $u_0$, $u_1$ et $u_2$.
		\item Quelle est la nature de la suite $(u_n)$ ? (Arithmétique, géométrique, autre).
		\item Estimer le nombre d'investisseurs nécessaires pour que l'entreprise fonctionne sur une année entière (1 an = 52 semaines). 
		Le plan de votre ami semble-t-il possible ? Expliquer.
	\end{enumerate}
	\textit{Ce type de montage se nomme « système de Ponzi ».}
}{exe:5}{
	todo
}


%%%%%%%%%%%%

\newpage
\fancyhead[C]{\textbf{Solutions}}
\shipoutAnswer

\end{document}
