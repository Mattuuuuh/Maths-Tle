\documentclass[14pt]{beamer}
\usepackage[french]{babel}

\usetheme{CambridgeUS}
\usecolortheme{rose}
\beamertemplatenavigationsymbolsempty


\usepackage{libertinus}
\usepackage{amsmath,amsfonts,amsthm,amssymb,mathtools}
\usepackage{array}
\newcolumntype{P}[1]{>{\centering\arraybackslash}p{#1}}


\usepackage{stackengine}
\newcommand\xrowht[2][0]{\addstackgap[.5\dimexpr#2\relax]{\vphantom{#1}}}


% corps
\usepackage{calrsfs}
\newcommand{\C}{\mathcal{C}}
\newcommand{\R}{\mathbb{R}}
\newcommand{\Rnn}{\mathbb{R}^{2n}}
\newcommand{\Z}{\mathbb{Z}}
\newcommand{\N}{\mathbb{N}}
\newcommand{\Q}{\mathbb{Q}}

% domain
\newcommand{\D}{\mathcal{D}}


% date
\usepackage{advdate}
\AdvanceDate[0]

%plots
\usepackage{pgfplots, subcaption}
\definecolor{myg}{RGB}{56, 140, 70}
\definecolor{myb}{RGB}{45, 111, 177}
\definecolor{myr}{RGB}{199, 68, 64}

%boxes
\usepackage[most]{tcolorbox}
\usepackage{multicol}

%icomma
\usepackage{icomma}

%https://osl.ugr.es/CTAN/macros/latex/contrib/tcolorbox/tcolorbox.pdf
\newtcolorbox{mybox}[3][]
{
  colframe = #2!25,
  colback  = #2!10,
  coltitle = #2!20!black,  
  halign title=flush center, 
  title    = {#3},
  #1,
}

% BOX A BOX B
\newcommand{\boxAB}[2]{
		\begin{mybox}{red}{A}
		\begin{center}
			#1
		\end{center}
		\end{mybox}
		\begin{mybox}{green}{B}
		\begin{center}
			#2
		\end{center}
		\end{mybox}
}

%systèmes
\usepackage{systeme}

% trafficotage 
\usepackage[answerdelayed, lastexercise]{exercise}
\renewcommand{\ExerciseHeader}{
}
\renewcommand{\AnswerHeader}{
}

\newcommand{\framedelayed}[3][]{
	\begin{Exercise}
	\begin{frame}{\theExercise #1\vspace{-32pt}}
		#2
	\end{frame}
	\end{Exercise}
	\begin{Answer}
	\begin{frame}{\theExercise #1\vspace{-32pt}}
		#3
	\end{frame}
	\end{Answer}
}


\SetDate[07/11/2025]

\usepackage{chessboard, skak}
\reversemarginpar
\setlength{\marginparsep}{.5cm}

\begin{document}
\pagestyle{fancy}
\fancyhead[L]{Tle STMG}
\fancyhead[C]{\textbf{Évaluation blanche \\ Suites arithmétiques et géométriques}}
\fancyhead[R]{\today}

%\null\vspace{-30pt}
Consignes particulières : 
\begin{itemize}[label=$\bullet$]
	\item 
	La calculatrice est {autorisée}.
	\item
	L'évaluation fait 2 pages. La somme des points est \total{points}.
\end{itemize}

\marginpar{[pts]}
\hrule

\exe{4}{
	Soit $u$ une suite géométrique de raison 1,5 et de 4ème terme $u_3 = 10,8$.
	
	\begin{enumerate}
		\item Calculer $u_4$ et $u_5$.
		\item Calculer $u_2$ et $u_1$.
		\item Montrer que le terme initial est donné par $u_0 = 3,2$.
		\item En déduire l'expression algébrique de $u_n$.
		\item Calculer $u_{50}$ en arrondissant à l'unité.
	\end{enumerate}
}{exe:suites-geom}{
	Comme $u$ est géométrique, on multiplie par sa raison 1,5 pour passer d'un terme au suivant.
	Pour remonter en arrière, on divise par la raison.
	\begin{enumerate}
		\item $u_4 = u_3 \times 1,5 = 16,2$ et $u_5 = u_4 \times 1,5 = 24,3$.
		\item $u_2 = \dfrac{u_3}{1,5} = 7,2$ et $u_1 = \dfrac{u_2}{1,5} = 4,8$.
		\item On a bien $u_0 = \dfrac{u_1}{1,5} = 3,2$.
		\item D'après le cours, $u_n = 3,2 \times 1,5^n$.
		\item $u_{50} = 3,2 \times 1,5^{50} \approx 2~040~388~801$.
	\end{enumerate}
}

\exe{6}{
	En Inde, le roi Belkib accorde la récompense de son choix au sage Sissa pour lui remercier d'avoir inventé le jeu d'échecs.
	L'échiquier, plateau du jeu d'échecs, reste inchangé depuis sa création jusqu'à aujourd'hui.
	
	\begin{multicols}{2}
	Sissa demande alors au roi de prendre le plateau du jeu d'échecs et, sur la première case, de poser un grain de riz.
	Ensuite, il lui demande de poser deux grains sur la deuxième case, puis quatre, puis huit, ... en doublant à chaque fois le nombre de grains de riz à déposer.
	
	
	\begin{center}
	\setchessboard{showmover=false}
	\newgame
	\chessboard[tinyboard, setfen=,]
	\\
	Exemple d'échiquier.
	\end{center}
	\end{multicols}
	
	Notons $G_n$ le nombre de grains de riz que le roi doit poser sur la $(n+1)$-ième case.
	Attention : le rang de la suite et le numéro de la case sont en décalage de 1.
	\begin{enumerate}
		\item
		De quelle nature est la suite $G$ ?
		Donner sa raison et son terme initial.
		
		\item
		À quel rang de la suite correspond la dernière case de l'échiquier ?
		
		\item
		Donner le nombre de grains de riz que le roi devra poser sur la dernière case de l'échiquier.
		Arrondir à la puissance de 10 la plus proche.
		
		\item
		On souhaite connaitre le nombre de grains de riz dont le roi a besoin pour satisfaire le sage Sissa.
		Écrire ce nombre sous forme de somme avec la notation $\Sigma$, puis donner sa valeur sous la forme $2^N - 1$, à l'aide du cours.
	\end{enumerate}
	En 2022, 776 millions de tonnes de riz ont été produites, chaque grain riz pesant environ 0,04g.
	\begin{enumerate}[resume]
		\item	
		Montrer que $1,94 \times 10^{16}$ grains de riz sont produits en 2022.
		\item
		Combien d'années de production comme celle de 2022 faudrait-il pour satisfaire le sage Sissa ?
	\end{enumerate}
	
	\emph{Indications : 1 million = $10^6$, et 1 tonne = $10^6$ grammes. }
	
}{exe:Sissa}{
	\begin{enumerate}
		\item
		$G$ est géométrique de raison 2 et de terme initial 1.
		
		\item
		Il y a 64 case, donc cela correspond au rang 63 (car on commence au rang 0, d'où le décalage).
		
		\item
		D'après le cours, $G_n = 2^n$.
		Ainsi $G_{63} = 2^{63} \approx 9 \times 10^{18}$, qu'on arrondit à $10^{19}$.
		
		\item
		On cherche à calculer
			\[ \sum_{k=0}^{63} 2^k = \dfrac{2^{64}-1}{2-1} = 2^{64} - 1, \]
		d'après le cours.
		\item
		D'après l'indication, $776 \times 10^6 \times 10^6 = 776 \times 10^{12}$ grammes de riz sont produits.
		Par proportionnalité, on divise par $0,04$ pour obtenir le nombre de grains de riz, soit
			\[ \dfrac{776 \times 10^{12}}{0,04} = 19400 \times 10^{12} = 1,94 \times 10^{16}. \]
			
		\item
		On divise $2^{64} - 1$ par $1,94 \times 10^{16}$, ce qui donne 950 années !
	\end{enumerate}
}

\exe{4}{
	Soit $u$ une suite arithmétique de raison 4 et de 5ème terme $u_4 = 17$.
	
	\begin{enumerate}
		\item Calculer $u_5$ et $u_6$.
		\item Calculer $u_3$ et $u_2$.
		\item Montrer que le terme initial est donné par $u_0 = 1$.
		\item En déduire l'expression algébrique de $u_n$.
		\item Calculer la somme 
			\[ u_0 + u_1 + u_2 + \cdots + u_{50}. \]
	\end{enumerate}
}{exe:suites-arithm}{
	Comme $u$ est arithmétique, on ajoute sa raison 4 pour passer d'un terme au suivant.
	Pour remonter en arrière, on soustrait la raison.
	\begin{enumerate}
		\item $u_5 = u_4 + 4 = 21$ et $u_6 = u_5 + 4 = 25$.
		\item $u_3 = u_4 - 4 = 13$ et $u_2 = u_3 - 4 = 9$.
		\item $u_1 = u_2 - 4 = 5$ et donc $u_0 = u_1 - 4 = 1$.
		\item D'après le cours, $u_n = 4n + 1$.
		\item D'après le cours, on calcule $u_{50} = 201$, et on en déduit que
			\[ \text{(nb termes)}\times\text{(moyenne premier et dernier termes)} = 51 \times \dfrac{1+201}2 = 5~151. \]
	\end{enumerate}
}

\newpage

\exe{3}{
	On étudie la croissance d'une population de champignons.
	
	Au début de l'expérience, on dispose de 100 champignons.
	Toutes les dix minutes, on mesure l'évolution de leur nombre.
	On obtient les résultats suivants.
	
	\begin{multicols}{2}
	\begin{center}
	\begin{tabular}{|c|c|}\hline
		\thead{Temps écoulé \\ (en minutes)} & \thead{Nombre de \\ champignons} \\ \hline
		0 & 100 \\\hline
		10 & 125 \\\hline
		20 & 150 \\\hline
		30 & 175 \\ \hline
	\end{tabular}
	
	\begin{tikzpicture}[>=stealth, scale=1]
		\begin{axis}[xmin = -1, xmax=35, ymin=-10, ymax=180, axis x line=middle, axis y line=middle, axis line style=->, xlabel={temps écoulé (en minutes)}, ylabel={nombres de champignons}, grid=both, ytick distance=50, xtick distance = 10]
			\addplot[black, thick, only marks, mark=*] coordinates {(0, 100) (10, 125) (20, 150) (30,175)};
		\end{axis}
	\end{tikzpicture}
	\end{center}
	\end{multicols}
	
	Soit $n$ un entier naturel.
	On note $u_n$ le nombre de champignons après $n$ périodes de \textbf{dix} minutes.
	Ainsi, $u_0 = 100, u_1 = 125, u_2 = 150, \dots$.
	\begin{enumerate}
		\item Justifier que les termes $u_0, u_1, u_2, u_3$ sont en progression arithmétique.
		\item En supposant que la population de champignons continue d'évoluer selon le même rythme, montrer qu'elle aura quadruplé deux heures après le début de l'expérience.
	\end{enumerate}
}{exe:zero-2}{
	\begin{enumerate}
		\item
		On calcule $u_1 - u_0 = 25, u_2 - u_1 = 25,$ et $u_3 - u_2 = 25$.
		Pour passer d'un terme à l'autre, on ajoute 25 : la progression est arithmétique.
		\item 
		On a $u_n = 100 + 25n$ d'après le cours.
		Comme 2 heures sont $2 \times60 = 120$ minutes, il s'agit de calculer le terme $u(12)$, la suite étant indexée en dixaines de minutes.
		\begin{align*}
			u(12) &= 100 + 25\times12 \\
					&= 100 + 25 \times 4 \times 3 \\
					&= 100 + 100 \times 3 \\
					&= 400
		\end{align*}
		On obtient bien $u(12) = 400 = 4 \times u_0$, comme annoncé.
	\end{enumerate}
}

\exe{3}{
	Considérons la suite $u$ définie algébriquement par
		\[ u_n = n 3^n + 1. \]
	\begin{enumerate}
		\item
		Calculer les trois premiers termes de $u$.
		\item
		La suite est-elle arithmétique ? géométrique ? Justifier.
	\end{enumerate}
}{exe:nature}{
	\begin{enumerate}
		\item
			\begin{align*}
				u_0 &= 0 \times 3^0 + 1 = 1, \\
				u_1 &= 1 \times 3^1 + 1 = 3 + 1 = 4, \\
				u_2 &= 2\times3^2 + 1 = 2\times9 + 1 = 19.
			\end{align*}
		\item
		Les différences successives sont $u_1 - u_0 = 3$ et $u_2 - u_1 = 15$, la suite n'est donc pas arithmétique.
		
		Les ratios successifs sont $\frac{u_1}{u_0} = 4$ et $\frac{u_2}{u_1} = 4,75$, la suite n'est donc pas géométrique non plus.
	\end{enumerate}
}



%%%%%%%%%%%%

\newpage
\fancyhead[C]{\textbf{Solutions}}
\shipoutAnswer

\end{document}
