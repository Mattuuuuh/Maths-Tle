\documentclass[14pt]{beamer}
\usepackage[french]{babel}

\usetheme{CambridgeUS}
\usecolortheme{rose}
\beamertemplatenavigationsymbolsempty


\usepackage{libertinus}
\usepackage{amsmath,amsfonts,amsthm,amssymb,mathtools}
\usepackage{array}
\newcolumntype{P}[1]{>{\centering\arraybackslash}p{#1}}


\usepackage{stackengine}
\newcommand\xrowht[2][0]{\addstackgap[.5\dimexpr#2\relax]{\vphantom{#1}}}


% corps
\usepackage{calrsfs}
\newcommand{\C}{\mathcal{C}}
\newcommand{\R}{\mathbb{R}}
\newcommand{\Rnn}{\mathbb{R}^{2n}}
\newcommand{\Z}{\mathbb{Z}}
\newcommand{\N}{\mathbb{N}}
\newcommand{\Q}{\mathbb{Q}}

% domain
\newcommand{\D}{\mathcal{D}}


% date
\usepackage{advdate}
\AdvanceDate[0]

%plots
\usepackage{pgfplots, subcaption}
\definecolor{myg}{RGB}{56, 140, 70}
\definecolor{myb}{RGB}{45, 111, 177}
\definecolor{myr}{RGB}{199, 68, 64}

%boxes
\usepackage[most]{tcolorbox}
\usepackage{multicol}

%icomma
\usepackage{icomma}

%https://osl.ugr.es/CTAN/macros/latex/contrib/tcolorbox/tcolorbox.pdf
\newtcolorbox{mybox}[3][]
{
  colframe = #2!25,
  colback  = #2!10,
  coltitle = #2!20!black,  
  halign title=flush center, 
  title    = {#3},
  #1,
}

% BOX A BOX B
\newcommand{\boxAB}[2]{
		\begin{mybox}{red}{A}
		\begin{center}
			#1
		\end{center}
		\end{mybox}
		\begin{mybox}{green}{B}
		\begin{center}
			#2
		\end{center}
		\end{mybox}
}

%systèmes
\usepackage{systeme}

% trafficotage 
\usepackage[answerdelayed, lastexercise]{exercise}
\renewcommand{\ExerciseHeader}{
}
\renewcommand{\AnswerHeader}{
}

\newcommand{\framedelayed}[3][]{
	\begin{Exercise}
	\begin{frame}{\theExercise #1\vspace{-32pt}}
		#2
	\end{frame}
	\end{Exercise}
	\begin{Answer}
	\begin{frame}{\theExercise #1\vspace{-32pt}}
		#3
	\end{frame}
	\end{Answer}
}


\SetDate[09/10/2025]

\begin{document}
\pagestyle{fancy}
\fancyhead[L]{Tle STMG}
\fancyhead[C]{\textbf{Suites arithmétiques}}
\fancyhead[R]{\today}

\exe{}{
	On considère trois suites arithmétiques.
	Compléter le nuage de points et déterminer leur expression algébrique.
	\begin{center}
	\begin{tikzpicture}[>=stealth, scale=1.1]
		\begin{axis}[xmin = 0, xmax=5.5, ymin=-2.8, ymax=4.8, axis x line=middle, axis y line=middle, axis line style=->, xlabel={rang $n$}, x label style={anchor=north}, grid=both, ytick distance=1, xtick distance = 1, x=2cm]
			\addplot[black, thick, only marks, mark=*] coordinates {(0,1) (4,3)}
				node[pos=0, left=25pt] {$v$};
			
			\addplot[black, thick, only marks, mark=star] coordinates {(0,4.5) (2,2.5)}
				node[pos=0, left=25pt] {$u$};
			
			\addplot[black, thick, only marks, mark=square] coordinates {(0,-1.5) (1,-1.5)}
				node[pos=0, left=25pt] {$w$};
		\end{axis}
	\end{tikzpicture}
	\end{center}
	
	\begin{align*}
		u_n = &&&& v_n = &&&& w_n = 
	\end{align*}
}{exe:graph}{
	\begin{center}
	\begin{tikzpicture}[>=stealth, scale=1.1]
		\begin{axis}[xmin = 0, xmax=5.5, ymin=-2.8, ymax=4.8, axis x line=middle, axis y line=middle, axis line style=->, xlabel={rang $n$}, x label style={anchor=north}, grid=both, ytick distance=1, xtick distance = 1, x=2cm]
			\addplot[black, thick, only marks, mark=*] coordinates {(0,1) (1,1.5) (2,2) (3,2.5) (4,3) (5,3.5)}
				node[pos=0, left=25pt] {$v$};
			
			\addplot[black, thick, only marks, mark=star] coordinates {(0,4.5) (1, 3.5) (2,2.5) (3,1.5) (4,0.5)}
				node[pos=0, left=25pt] {$u$};
			
			\addplot[black, thick, only marks, mark=square] coordinates {(0,-1.5) (1,-1.5) (2,-1.5) (3,-1.5) (4,-1.5) (5,-1.5)}
				node[pos=0, left=25pt] {$w$};
		\end{axis}
	\end{tikzpicture}
	\end{center}
	
	\begin{align*}
		u_n = 4,5 - n  &&&& v_n = 1 + 0,5n &&&& w_n = -1,5
	\end{align*}
}

\exe{}{
	Un professeur hésite entre louer un appartement ou l'acheter.
	\begin{itemize}
		\item
		D'une part, le loyer mensuel de l'appartement est de $700$€, avec un coût initial (la caution) fixé à deux loyers, soit $1\ 400$€.
		\item
		D'autre part, à l'achat, cet appartement coûte $140~000$€.
	\end{itemize}
	
	Dans la première alternative, lorsque l'appartement est loué, on dénote par $u_n$ le montant total payé en euros par le locataire après $n$ mois (et ceci pour tout $n\in\N$).
	Ainsi, par exemple, après $1$ mois, un seul loyer a été payé en plus de la caution initiale. D'où $u_1 = 2~100$.
	\begin{enumerate}
		\item Donner le terme initial $u_0$.
		\item Donner $u_2$, le montant payé par le locataire après $2$ mois.
		\item Justifier du caractère arithmétique de la suite $u$. Quelle est sa raison ?
		\item Pour tout $n\in\N$, donner $u_n$ en fonction de $n$ sans justifier.
		\item À partir de combien de mois le locataire aura-t-il payé plus que le coût de l'appartement à l'achat ? Convertir le résultat en années.
	\end{enumerate}
}{exe:loyer-achat}{
	\begin{enumerate}
		\item $u_0 = 1~400$.
		\item $u_2 = u_1 + 700 = 2~100 + 700 = 2~800$.
		\item D'un mois à l'autre, on ajoute 700 au montant total payé.
		La raison est donc de 700.
		\item D'après le cours, $u_n = 1400 + 700n = 700(2+n)$.
		\item 
		On pose et on résoud $u_n = 140~000$.
			\begin{align*}
				u_n &= 140~000 \\
				1400 + 700n &= 140~000 \\
				700(2+n) &= 140~000 \\
				2+n &=200 \\
				n &= 198
			\end{align*}
		Par conséquent, au bout de 198 mois, le locataire aura payé le coût de l'appartement à l'achat.
		En divisant par 12, on convertit le résultat en $\frac{198}{12} = \frac{99}{6} = \frac{33}2 = 16,5$ années.
	\end{enumerate}
}

\exe{}{
	Le mathématicien français Abraham de Moivre, alors âgé de plus que 80 ans, étudia dans les années 1750 la durée de son sommeil ; il remarqua que celle-ci augmentait de façon inquiétante.
	Il prit donc note chaque jour de son heure de coucher et de réveil et compara avec une valeur initiale : la nuit $0$, il dormit $8$ heures.
	
	Au fil des jours, il compta que son temps de sommeil augmentait chaque nuit de $15$ minutes, soit $\frac14$ d'heure.
	En notant $v_n$ la durée de son sommeil en heures lors de sa $n$-ième nuit, il put alors calculer la nuit de la mort : celle où il dormirait $24$ heures.
	\begin{enumerate}
		\item Donner le terme initial $v_0$.
		\item Donner $v(4)$, la durée de son sommeil en heures lors de la $4$-ème nuit.
		\item Pourquoi la suite $v$ est-elle arithmétique ? Quelle est sa raison ?
		\item Pour tout $n\in\N$, donner $v_n$ en fonction de $n$ sans justifier.
		\item Quelle nuit Abraham de Moivre mourut-t-il ?
	\end{enumerate}
}{exe:Moivre}{
	\begin{enumerate}
		\item $v_0 = 8$, car on compte en heures.
		\item $v(4) = 8 + \frac14 \times 4 = 9$.
		\item D'une nuit à l'autre, de Moivre dort $\frac14$ d'heure en plus.
		La suite est donc arithmétique de raison $\frac14$.
		\item D'après le cours, $v_n = 8 + \frac14 n$.
		\item 
		On pose et on résoud $v_n = 24$.
			\begin{align*}
				v_n &= 24 \\
				8 + \frac14 n &= 24 \\
				\frac14n &= 16 \\
				n &= 4\times16 = 48
			\end{align*}
		On conclut que de Moivre meurt la 48-ème nuit après le début de son étude.
	\end{enumerate}
}

\exe{}{
	Parmis les suites suivantes, lesquelles sont arithmétiques ? 
	Si la suite est arithmétique, donner son terme initial et sa raison.
	Sinon, justifier qu'elle n'est pas arithmétique.
	\begin{multicols}{2}
	\begin{enumerate}[itemsep=5pt]
		\item $u_n = 3n - 4$
		\item $w_n = n^2 + 3$
		\item $v_n = 13$
		\item $f_n = \dfrac{1}{2n+1}$
	\end{enumerate}
	\end{multicols}
}{exe:arithm}{
	\begin{enumerate}[itemsep=5pt]
		\item 
		$u$ est de la forme $an+b$ avec $a=3$, sa raison, et $b=-4$, son terme initial.
		\item 
		$w$ ne semble pas arithmétique.
		Montrons que ses trois premiers termes ne forment pas une suite arithmétique.
		Ses trois premiers termes sont $w_0 = 3, f_1 = 4,$ et $f_2 = 7$.
		
		Comme $f_1 - f_0 = 1$ et que $f_2 - f_1 =3 \neq 1$, la suite n'est pas arithmétique.
		\item
		$v$ est de la forme $an+b$ avec $a=0$, sa raison, et $b=13$, son terme initial.
		\item 
		$f$ ne semble pas arithmétique.
		Montrons que ses trois premiers termes ne forment pas une suite arithmétique.
		Ses trois premiers termes sont $f_0 = 1, f_1 = \frac13,$ et $f_2 = \frac15$.
		
		Comme $f_1 - f_0 = -\frac23$ et que $f_2 - f_1 = -\frac{2}{15} \neq - \frac23$, la suite n'est pas arithmétique.
	\end{enumerate}
}


\exe{}{
	On étudie la croissance d'une population de champignons.
	
	Au début de l'expérience, on dispose de 100 champignons.
	Toutes les dix minutes, on mesure l'évolution de leur nombre.
	On obtient les résultats suivants.
	
	\begin{multicols}{2}
	\begin{center}
	\begin{tabular}{|c|c|}\hline
		\thead{Temps écoulé \\ (en minutes)} & \thead{Nombre de \\ champignons} \\ \hline
		0 & 100 \\\hline
		10 & 125 \\\hline
		20 & 150 \\\hline
		30 & 175 \\ \hline
	\end{tabular}
	
	\begin{tikzpicture}[>=stealth, scale=1]
		\begin{axis}[xmin = -1, xmax=35, ymin=-10, ymax=180, axis x line=middle, axis y line=middle, axis line style=->, xlabel={temps écoulé (en minutes)}, ylabel={nombres de champignons}, grid=both, ytick distance=50, xtick distance = 10]
			\addplot[black, thick, only marks, mark=*] coordinates {(0, 100) (10, 125) (20, 150) (30,175)};
		\end{axis}
	\end{tikzpicture}
	\end{center}
	\end{multicols}
	
	Soit $n$ un entier naturel.
	On note $u_n$ le nombre de champignons après $n$ périodes de \textbf{dix} minutes.
	Ainsi, $u_0 = 100, u_1 = 125, u_2 = 150, \dots$.
	\begin{enumerate}
		\item Justifier que les termes $u_0, u_1, u_2, u_3$ sont en progression arithmétique.
		\item En supposant que la population de champignons continue d'évoluer selon le même rythme, montrer qu'elle aura quadruplé deux heures après le début de l'expérience.
	\end{enumerate}
}{exe:zero-2}{
	\begin{enumerate}
		\item
		On calcule $u_1 - u_0 = 25, u_2 - u_1 = 25,$ et $u_3 - u_2 = 25$.
		Pour passer d'un terme à l'autre, on ajoute 25 : la progression est arithmétique.
		\item 
		On a $u_n = 100 + 25n$ d'après le cours.
		Comme 2 heures sont $2 \times60 = 120$ minutes, il s'agit de calculer le terme $u(12)$, la suite étant indexée en dixaines de minutes.
		\begin{align*}
			u(12) &= 100 + 25\times12 \\
					&= 100 + 25 \times 4 \times 3 \\
					&= 100 + 100 \times 3 \\
					&= 400
		\end{align*}
		On obtient bien $u(12) = 400 = 4 \times u_0$, comme annoncé.
	\end{enumerate}
}

\exe{}{
	Exe sur la somme
}{exe:somme-arithm}{
	todo
}


%%%%%%%%%%%%

\newpage
\fancyhead[C]{\textbf{Solutions}}
\shipoutAnswer

\end{document}
