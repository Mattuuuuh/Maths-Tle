\documentclass[14pt]{beamer}
\usepackage[french]{babel}

\usetheme{CambridgeUS}
\usecolortheme{rose}
\beamertemplatenavigationsymbolsempty


\usepackage{libertinus}
\usepackage{amsmath,amsfonts,amsthm,amssymb,mathtools}
\usepackage{array}
\newcolumntype{P}[1]{>{\centering\arraybackslash}p{#1}}


\usepackage{stackengine}
\newcommand\xrowht[2][0]{\addstackgap[.5\dimexpr#2\relax]{\vphantom{#1}}}


% corps
\usepackage{calrsfs}
\newcommand{\C}{\mathcal{C}}
\newcommand{\R}{\mathbb{R}}
\newcommand{\Rnn}{\mathbb{R}^{2n}}
\newcommand{\Z}{\mathbb{Z}}
\newcommand{\N}{\mathbb{N}}
\newcommand{\Q}{\mathbb{Q}}

% domain
\newcommand{\D}{\mathcal{D}}


% date
\usepackage{advdate}
\AdvanceDate[0]

%plots
\usepackage{pgfplots, subcaption}
\definecolor{myg}{RGB}{56, 140, 70}
\definecolor{myb}{RGB}{45, 111, 177}
\definecolor{myr}{RGB}{199, 68, 64}

%boxes
\usepackage[most]{tcolorbox}
\usepackage{multicol}

%icomma
\usepackage{icomma}

%https://osl.ugr.es/CTAN/macros/latex/contrib/tcolorbox/tcolorbox.pdf
\newtcolorbox{mybox}[3][]
{
  colframe = #2!25,
  colback  = #2!10,
  coltitle = #2!20!black,  
  halign title=flush center, 
  title    = {#3},
  #1,
}

% BOX A BOX B
\newcommand{\boxAB}[2]{
		\begin{mybox}{red}{A}
		\begin{center}
			#1
		\end{center}
		\end{mybox}
		\begin{mybox}{green}{B}
		\begin{center}
			#2
		\end{center}
		\end{mybox}
}

%systèmes
\usepackage{systeme}

% trafficotage 
\usepackage[answerdelayed, lastexercise]{exercise}
\renewcommand{\ExerciseHeader}{
}
\renewcommand{\AnswerHeader}{
}

\newcommand{\framedelayed}[3][]{
	\begin{Exercise}
	\begin{frame}{\theExercise #1\vspace{-32pt}}
		#2
	\end{frame}
	\end{Exercise}
	\begin{Answer}
	\begin{frame}{\theExercise #1\vspace{-32pt}}
		#3
	\end{frame}
	\end{Answer}
}


\SetDate[28/01/2026]

\begin{document}
\pagestyle{fancy}
\fancyhead[L]{Tle STMG}
\fancyhead[C]{\textbf{Coefficients binomiaux}}
\fancyhead[R]{\today}


\exe{}{
	On jette une pièce 4 fois de suite. Chaque lancer aboutit à « pile » ou « face », et on note l'ordre des résulats.
	De combien de façons peut-on obtenir
	\begin{enumerate}
		\item jamais « pile » ?
		\item exactement une fois « pile » ?
		\item exactement deux fois « pile » ?
	\end{enumerate}
	En déduire le nombre de façons d'obtenir
	\begin{enumerate}[label=\alph*)]
		\item au moins une fois « face » ;
		\item exactement trois fois « pile » ;
		\item moins de deux fois « pile » (y compris deux).
	\end{enumerate}
}{exe:binom0}{

}

\exe{}{
	Compléter le \emph{triangle de Pascal} et en déduire la valeur de $5\choose2$ et de $6\choose4$.
	\begin{center}
	\begin{tabular}{>{$}l<{$}|*{7}{c}}
		\multicolumn{1}{l}{$n$} &&&&&&&\\\cline{1-1} 
		0 &1&&&&&&\\
		1 &1&1&&&&&\\
		2 &1&2&1&&&&\\
		3 &1&3&3&1&&&\\
		4 &1&4&6&4&1&&\\
		5 &1&5&10&10&5&1&\\
		6 &1&6&15&20&15&6&1\\\hline
		\multicolumn{1}{l}{} &0&1&2&3&4&5&6\\\cline{2-8}
		\multicolumn{1}{l}{} &\multicolumn{7}{c}{$k$}
	\end{tabular}
	\end{center}
}{exe:Pascal}{

}

%\exe{}{
%	Interpréter le nombre $8\choose4$.
%}{exe:0}{
%
%}


\exe{}{
	On pose un pion sur la case inférieure gauche d'un échiquier $4\times4$.
	On déplace le pion de case en case afin d'atteindre la case supérieure droite.
	Le pion peut soit se déplacer d'une case vers le haut, soit d'une case vers la droite. Il est donc impossible de faire une boucle.
	En combien de façons différentes peut-il atteindre la case supérieure droite ?
	
}{exe:binom2}{

}
	\begin{figure}[h!]
	\begin{subfigure}{.5\textwidth}
	\centering
	\begin{tikzpicture}
		\foreach \i in {0, ..., 4}{
			\draw (\i,0) -- (\i,4);
		}
		\foreach \j in {0, ..., 4}{
			\draw (0,\j) -- (4,\j);
		}
		
		\draw[<-] (.4,.5) arc[start angle=-90, end angle=-155, radius=.75cm] node[left] {Départ};
		\draw[<-] (3.6,3.5) arc[start angle=90, end angle=25, radius=.75cm] node[right] {Arrivée};
	\end{tikzpicture}
	\caption{Échiquier $4\times4$.}
	\label{fig:1a}
	\end{subfigure}
	\hfill
	\begin{subfigure}{.5\textwidth}
	\centering
	\begin{tikzpicture}
		\foreach \i in {0, ..., 4}{
			\draw (\i,0) -- (\i,4);
		}
		\foreach \j in {0, ..., 4}{
			\draw (0,\j) -- (4,\j);
		}
		\draw[thick] (.5,.5) -- (2.5, .5) -- (2.5, 1.5) -- (3.5, 1.5) -- (3.5, 3.5);
		\draw[thick, dashed] (.5,.5) -- (.5, 1.5) -- (1.5, 1.5) -- (2.5, 1.5) -- (2.5, 2.5) -- (2.5 ,3.5) -- (3.5, 3.5);
	\end{tikzpicture}
	\caption{Deux façons différentes d'atteindre l'arrivée.}
	\label{fig:1b}
	\end{subfigure}
	\end{figure}

%%%%%%%%%%%%

\newpage
\fancyhead[C]{\textbf{Solutions}}
\shipoutAnswer

\end{document}
