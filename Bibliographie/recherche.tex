\documentclass[a4paper,12pt, notitlepage]{article}
\usepackage{amssymb}
\usepackage{amsmath}
\usepackage{amsthm}

\theoremstyle{definition}
\newtheorem{idee}{Idées}

\newcommand{\N}{\mathbb{N}}
\newcommand{\Q}{\mathbb{Q}}
\newcommand{\R}{\mathbb{R}}

\newcommand{\Var}{\text{Var}}

\begin{document}

\section{Thèmes d'étude}

\subsection{Optimisation linéaire}

\begin{idee}
	Modélisation de problème comme problème d'optimisation linéaire $\max_{Ax \leq b}{c^T x}$
	
	Manipulation d'équations linéaires pour obtenir une borne supérieure au max.
	
	Régionnement du plan : $a^T x \leq b$ scinde le plan en deux.
	Intersection de demi-plans, polyèdre $Ax \leq b$ et sommets comme intersection de deux équations.
	
	Le nombre de sommets d'un polyèdre est au plus ${m \choose 2} = \mathcal{O}(m^2)$.
	
	Démonstation que $Ax \leq b$ est convexe.
	
	Démonstration que si $x$ est moyenne de deux points du polyèdre $Ax \leq b$, alors $c^Tx$ n'est pas maximal.
	Conclusion que, si le polyèdre est borné, un de ses sommets réalise le maximum.
	
	Application : identification des sommets, maximum qui correspond à la solution duale trouvé.
	
	Pseudo-algorithme général pour résolution d'optimisation linéaire en 2 dimensions.
	Complexité en temps.
	
	Discussion sur l'optimisation discrète.	
\end{idee}

\begin{idee}
	From DisOpt midterm 2019
	
	A factory produces two different products. To create one unit of product 1, it needs one unit of
	raw material A and one unit of raw material B. To create one unit of product 2, it needs one unit
	of raw material B and two units of raw material C. Raw material B needs preprocessing before it
	can be used, which takes one minute per unit. At most 20 hours of time is available per day for
	the preprocessing. Raw materials of capacity at most 1200 can be delivered to the factory per day.
	One unit of raw material A, B and C has size 4, 3 and 2 respectively.
	At most 130 units of the first and 100 units of the second product can be sold per day. The first
	product sells for 6 CHF per unit and the second one for 9 CHF per unit.
	Formulate the problem of maximizing turnover as a linear program in two variables.

\end{idee}

\begin{idee}
	Optimisation discrète avec théorie des nombres.
	Algorithme d'Euclide étendu pour trouver la combinaison linéaire donnant le gcd.
	
	Résolution de 
		\[ ax+ by = c \]
	avec $x,y$ entiers. Par exemple problème de Laura avec rémunérations à l'heure.

	Montrer que si $2|a$ et $3|a$ alors $6|a$.
\end{idee}

\subsection{Monte Carlo}

\begin{idee}
	Approximation de $\pi$ en prenant $x,y \sim U([0,1]^2)$ et $P(x^2 + y^2 <= 1) = \pi/4$, espérance de $z = 1_{|\{x^2 + y^2 <= 1\}}$.

	Loi géométrique $f_X(k) = p(1-p)^{k-1}, k\in\N^*$ donnée comme temps d'arrêt de Bernoulli.
	
	Approximation de l'espérance de $X$ à l'aide de Monte-Carlo.
	
	Déduire que 
		\[ \sum_k k q^{k-1} \approx \dfrac1{p^2}. \]
		
	Démonstration de somme géométrique
		\[ \sum_k x^k = \dfrac1{1-x}. \]
	Dérivation de la somme infinie admise. Dérivation de $\dfrac1{1-x}$ guidée : 
		\[ \dfrac1{1-x-h} - \dfrac1{1-x} = \dfrac{h}{(1-x-h)(1-x)}. \]
	En déduire que
		\[ \sum_k k q^{k-1} = \dfrac1{p^2}. \]
	
\end{idee}

\begin{idee}
	p = P($k$ piles parmis $n$ lancers) = ${n \choose k} (1/2)^k$.
	
	Avec $N$ lancers, donner la fréquence $f$ de $B=k$.
	$E(f) = p$, et $\Var(f) = p(1-p)/n$.
	
	Montrer que $x(1-x) \leq 1/4$ pour tout $x\in[0,1]$.
	
	En déduire que $\sigma(f) \leq 1/(2\sqrt{n})$.
	
	L'inégalité de Chebyshev donne
		\[ P( |f - p| \geq \ell ) \leq \dfrac{\Var(f)}{\ell^2} \leq \dfrac1{4n\ell^2}. \]
	En déduire que $|f-p| \leq 2^{-12}$ avec probabilité $95\%$ dès que $n \geq 2^28$.
	
	Conclure que $|2^{12} f - {n \choose k}| < 1/2$ avec probabilité $95\%$.
\end{idee}

\subsection{Marche aléatoire}

\begin{idee}
	Marche aléatoire $S_n = \sum_i X_i$ simple sur $\mathbb{Z}$, où $P(X_i = +1) = p, P(X_i = -1) = q = 1-p$.
	
	Sampling de $N$ marches pour $n$ grand fixé et estimation de la probabilité
		\[ r_k = P(S_n \geq k \text{ pour un certain $n\in\N$ }). \]
	Graphe de Monte Carlo, stabilisation de la valeur de $r$ quand $N$ augmente.
	
	Justification que 
		\[ r_k = q r_{k-1} + p r_{k+1}. \]
		
	Justification que $r_M = r_1^M$ puis que
		\[ r_1 = p + q r_1^2. \]
	Résolution de $x = p + qx^2, x \in (0,1)$ par $x = \dfrac{1-\sqrt{1-4pq}}{2q} = \dfrac{p}{q}$ pour $p<1/2$.
	
	En déduire que $P(\max(S_n) = k) = (p/q)^k - (p/q)^{k+1} = (p/q)^k (1-p/q)$. 
	Conclure que $\sum_k (p/q)^k = \dfrac{1}{1-p/q}$ si $p<q$.
	
	Étendre à $p \in\R-\{1\}$ et en déduire Fibonacci en prenant $p/q = \phi$ et $\overline{\phi}$.
\end{idee}


\subsection{Théorie des graphes}

\begin{idee}
	BFS pour le calcul de distance. Complexité, implémentation.
	
	Trouver la façon optimale de changer de monnaie (graphe orienté simple).
	
	Identifier un cycle d'échange de monnaies permettant de gagner de l'argent (graphe orienté avec source et sink).
\end{idee}

\begin{idee}
	Scheduling. Mais c'est de l'optimisation linéaire non ?
\end{idee}

\begin{idee}
	Échiquier $n \times n$ avec départ en bas à gauche et arrivée en haut à droite.
	
\end{idee}

\subsection{Suites récurrentes}

\begin{idee}
	Suite $u_{n+1} = f(u_n)$ pour $f : K \rightarrow K$ $\alpha$-Lipschitz ($\alpha < 1$).
	Montrer que $f(K) \subseteq K$ et donc que $u$ reste dans $K$.
	Tests avec $u_0 \not\in K$ qui diverge.
	
	On suppose que $K$ contient un point fixe (sinon il faut $u$ Cauchy et $K$ fermé).
	
	Représentation de la suite sur le graphe $y=f(x)$ vs $y=x$.
	
	Application à l'approximation de $\sqrt{2}$ avec $f(x) =1 + 1/(1+x)$.
	Montrer que $K=[4/3,2]$ fonctionne. Prendre $u_0$ quelconque.
	
	Vitesse de convergence :
		\[ |v_{n+1} - L| = |f(v_n) - f(L)| \leq \alpha |v_n - L|, \]
	d'où
		\[ |v_n - L| \leq C \alpha^n. \]
	En déduire le nombre d'étapes pour calculer $L$ à $m$ chiffres après la virgules (en base 10).
	
	Conclure que $\sqrt{2} = 1 + 1/(1+1/(1+\sqrt{2}))$ etc... (fraction continue).
	
	Montrer que $g(x) = 2/x$ est une involution et donc ne convergera jamais vers $\sqrt{2}$. 
	Quelle hypothèse n'est pas vérifiée ? $g'$ n'est pas Lipschitz !
	
	Montrer que si $L = a + 1/(b+L)$ alors $L$ est un quadratic surd.
	
	Montrer que $\sqrt2 \approx p_n/q_n$ et $p_{n+1} = p_n + 2q_n$ et $q_{n+1} = p_n + q_n$.
	Donner $p_{100}, q_{100}$ et en déduire que $p_{100}/q_{100}$ est une meilleure approximation de $\sqrt2$ que sa valeur numérique donnée par \texttt{math.sqrt(2)}.
	
	Montrer que $p_n$ et $q_n$ sont toujours premiers entre eux si $p_0$ et $q_0$ le sont.
	
	Pour $\alpha\in\R-\Q$, est-il toujours possible d'obtenir un nombre rationnel $r\in\Q$ tel que $|\alpha - r| < 10^{-n}$ pour n'importe quel $n$ ?
\end{idee}

\begin{idee}
	Représentations des nombres en différentes bases.
	
	En déduire que 
		\[ \sum_{k=0}^{n} (b-1)b^k = b^{n+1} - 1, \]
	et donc que 
		\[ \sum_{k=0}^n b^k = \dfrac{b^{n+1} - 1}{b-1}. \]
	
	Le nombre de chiffre pour représenter $N$ est environ $\log_b(N)$. 
	
	Algorithme pour représenter $N$ en base $b$.
	
	Exponentiation rapide complète avec complexité.
\end{idee}

\subsection{Étude de sommes et applications}

\begin{idee}	
	Série harmonique $H_n$ : $H_N \geq \dfrac12 \log_2N - 1/2$ pour $N$ puissance de $2$.
	Et donc pour tout $N$ en prenant $N/2$ au membre de droite (ou la puissance de 2 inférieure).
	
	Application : empilement d'objets en équilibre à l'infini.
	
	Série de Bâle : $\sum_k 1/k^2$. Étude de $\sum_k 1/(k(k+1))$ en somme téléscopique puis borne $\sum_k 1/k^2 \leq 2-1/N$.
	
	Déduire que $\sum_k 1/k^s < \infty$ pour tout $s \geq 2$.
	
	Montrer qu'à partir d'un certain rang, les $n$ premiers chiffres de la somme ne changent plus.	
	À partir de quel rang peut-on connaître les $n$ premiers chiffres de la somme ?
	
	Somme triangulaire à partir de somme téléscopique de carrés.
	Idem pour somme des cubes, puis des bécarres.
	Montrer que $(\sum_k k)^2 = \sum_k k^3$.

	Application : encadrement de l'aire sous la courbe de $x^2, x^3$ par exemple par sommes de Riemann.
\end{idee}

\subsection{Jeux d'argent}


\begin{idee}
	Considérons la suite $v$ définie par la relation de récurrence suivante, pour tout $n\in\N$.
    \[
    \begin{cases}
        v(0) = 0, \\
        v(n+1) = 2v(n) + 1.
    \end{cases}
    \]
    \begin{enumerate}
        \item Montrer que la suite intermédiaire $w$ définie pour tout $n\in\N$ par
            \[ w(n) = v(n) + 1,\]
        est géométrique.
        \item En déduire que, pour tout $n\in\N$,
            \[ v(n) = 2^n - 1.\]
    \end{enumerate}

	Généralisation avec des suites arithmético-géométriques.
	
	Conclusion stratégie de gain jeu d'argent où on gagne le double.
	Calcul du temps d'arrêt, modélisation, calcul de l'argent nécessaire en moyenne.
	
	Temps d'arrêt : loi géométrique.
	
	Comparaison Binom(n,p) vs nBern(p) : même espérance de gain mais la probabilité de gain non nul $P(X>0)$ est bien plus grande dans la première situation.
\end{idee}

\subsection{Interpolation}

\begin{idee}
	Interpolation de Lagrange.
	Unicité de la solution à degré minimal.
	
	Lien avec les systèmes linéaires de Vandermonde.
\end{idee}

\end{document}