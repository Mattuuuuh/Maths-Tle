%!TEX encoding = UTF8
%!TEX root = 0-notes.tex

\chapter{Statistiques à deux variables}

Dans tout ce chapitre on considère $n$ points dans l'espace, notés 
	\[ (x_1, y_1), (x_2, y_2), \dots, (x_n, y_n), \]
et appelés \emphindex{nuage de points}.


\begin{figure}
	\centering
	\includegraphics[page=1]{figures/fig-stats.pdf}
	\caption{Exemple de nuage de points avec tendance linéaire.}
\end{figure}

Ces données sont souvent \emphindex{statistiques} et mesurent deux propriétés dont on étudie le lien : comment $y_i$ agit-il en fonction de $x_i$ ?
Par exemple, comment le salaire évolue-t-il en fonction de l'ancienneté ? en fonction du sexe ? en fonction de la durée de la formation ?

Si les données sont \emphindex{corrélées}, on pourra identifier une tendance dans le nuage de points.
Sinon, le nuage ne sera qu'un nuage : les données semblent \emphindex{décorrélées}.

\textbf{\warning La corrélation n'implique pas la causalité.}

\section{Ajustement affine et moindres carrés}

\begin{figure}
	\centering
	\includegraphics[page=2]{figures/fig-stats.pdf}
	\caption{Exemple de nuage de points avec ajustement affine}
\end{figure}

\emphindex{interpolation affine}

\subsection{Cas $n=2$}


\subsection{Cas $n>2$}

\nt{
	Lorsqu'il y a plus de deux points dans un nuage, il n'est en général pas possible de placer une droite passant par tous les points.
	On pourrait choisir deux points parmis le nuage et en déduire une droite, mais est-ce bien pertinent ?
}

Le but de l'interpolation étant d'approcher la tendance du nuage, il est important de mesurer l'erreur commise entre la valeur réelle et la valeur interpolée.
Pour cela, on choisit la distance au carré entre la valeur interpolée et la vraie valeur, soit $\bigl[ f(x) - y \bigr]^2$.
Sans hiérarchisation des points du nuage, l'erreur totale est alors donnée par la somme des erreurs individuelles.

\dfn{erreur d'interpolation}{
	L'\emphindex{erreur d'interpolation} commise par l'interpolation par $f$ d'un nuage $\{ (x, y) \}$ est donnée par
		\[ \sum_{(x,y)} \bigl[ f(x) - y \bigr]^2. \]
}{}

\dfn{moindres carrés}{
	Le problème des \emphindex{moindres de carrés} demande de trouver les paramètres $a, b$ de la fonction affine $f(x)=ax+b$ minimisant l'erreur d'interpolation d'un nuage de points.
	
	\[ (a^\star , b^\star) = \arg\min_{a, b\in\R} \sum_{(x,y)} \bigl[ ax+b - y \bigr]^2. \]
}{}



\dfn{extrapolation}{
	L'\emphindex{extrapolation} est l'utilisation d'un modèle pour prédire des valeurs encore inexistantes.
}{}

\ex{}{
	Le GIEC extrapole la température à la surface de la Terre dans le temps à partir d'un modèle d'interpolation basé sur les données collectées depuis plus de 100 ans.
	Un tel modèle est un général lisse, et ne prédit pas les valeurs aberrantes (été à 40°C, par exemple).
}{}

\subsection{Non correlation et données aberrantes}



Un point aberrant du nuage peut changer drastiquement l'ajustement affine de la série statistiques.
C'est le cas du salaire du CEO par exemple :).



\section{Changements de variable}

Pour décider d'un changement de variable adéquat, il est nécessaire de pouvoir reconnaître quelques courbes remarquables.



Des changements de variable plus généraux ($y=x^a$ avec $a$ quelconque) seront disponibles après l'étude du logarithme, chapitre \ref{chap:log}.



