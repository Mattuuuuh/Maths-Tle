\documentclass[14pt]{beamer}
\usepackage[french]{babel}

\usetheme{CambridgeUS}
\usecolortheme{rose}
\beamertemplatenavigationsymbolsempty


\usepackage{libertinus}
\usepackage{amsmath,amsfonts,amsthm,amssymb,mathtools}
\usepackage{array}
\newcolumntype{P}[1]{>{\centering\arraybackslash}p{#1}}


\usepackage{stackengine}
\newcommand\xrowht[2][0]{\addstackgap[.5\dimexpr#2\relax]{\vphantom{#1}}}


% corps
\usepackage{calrsfs}
\newcommand{\C}{\mathcal{C}}
\newcommand{\R}{\mathbb{R}}
\newcommand{\Rnn}{\mathbb{R}^{2n}}
\newcommand{\Z}{\mathbb{Z}}
\newcommand{\N}{\mathbb{N}}
\newcommand{\Q}{\mathbb{Q}}

% domain
\newcommand{\D}{\mathcal{D}}


% date
\usepackage{advdate}
\AdvanceDate[0]

%plots
\usepackage{pgfplots, subcaption}
\definecolor{myg}{RGB}{56, 140, 70}
\definecolor{myb}{RGB}{45, 111, 177}
\definecolor{myr}{RGB}{199, 68, 64}

%boxes
\usepackage[most]{tcolorbox}
\usepackage{multicol}

%icomma
\usepackage{icomma}

%https://osl.ugr.es/CTAN/macros/latex/contrib/tcolorbox/tcolorbox.pdf
\newtcolorbox{mybox}[3][]
{
  colframe = #2!25,
  colback  = #2!10,
  coltitle = #2!20!black,  
  halign title=flush center, 
  title    = {#3},
  #1,
}

% BOX A BOX B
\newcommand{\boxAB}[2]{
		\begin{mybox}{red}{A}
		\begin{center}
			#1
		\end{center}
		\end{mybox}
		\begin{mybox}{green}{B}
		\begin{center}
			#2
		\end{center}
		\end{mybox}
}

%systèmes
\usepackage{systeme}

% trafficotage 
\usepackage[answerdelayed, lastexercise]{exercise}
\renewcommand{\ExerciseHeader}{
}
\renewcommand{\AnswerHeader}{
}

\newcommand{\framedelayed}[3][]{
	\begin{Exercise}
	\begin{frame}{\theExercise #1\vspace{-32pt}}
		#2
	\end{frame}
	\end{Exercise}
	\begin{Answer}
	\begin{frame}{\theExercise #1\vspace{-32pt}}
		#3
	\end{frame}
	\end{Answer}
}

\AdvanceDate[3]

\begin{document}

\section{Automatismes}

\begin{frame}

\centering \huge
Automatismes

\large
Calculatrice interdite

\end{frame}

\subsection{Propriétés des puissances}

\begin{frame}{1\vspace{-32pt}}
	Écrire le nombre suivant sous forme $a^b$, où $a, b \in \Z$ sont des entiers relatifs.
	\boxAB{
		$\dfrac{13^{15}}{13^{9}}$
	}{
		$\dfrac{11^{17}}{11^9}$.
	}
\end{frame}

\begin{frame}{2\vspace{-32pt}}
	Écrire le nombre suivant sous forme $a^b$, où $a, b \in \Z$ sont des entiers relatifs.
	\boxAB{
		$\bigl(2^6\bigr)^{-5}$
	}{
		$\bigl(2^7\bigr)^{-4}$
	}
\end{frame}

\subsection{Longueur d'intervalle}

\begin{frame}{3\vspace{-32pt}}
	Donner l'amplitude de l'encadrement suivant sous forme de puissance de $10$.
	\boxAB{
		$318,615  < 318,621702 < 318,625$
	}{
		$610,3075 < 610,3081702 <  610,3085$
	}
\end{frame}

\subsection{Oracle}

\begin{frame}{4\vspace{-32pt}}	
	On considère une fonction $f$ croissante sur $\R$, et $x_0$ l'unique nombre vérifiant $f(x_0) = 0$.
	
	On sait en outre que $x_0$ est encadré par
		\[ 6 < x_0 < 8. \]
	\boxAB{
		Donner le signe de $f(8)$.
	}{
		Donner le signe de $f(6)$.
	}
	%Réponse attendue $\in$ \{ positif ; négatif ; nul \}.
\end{frame}

\end{document}